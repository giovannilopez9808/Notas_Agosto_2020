\documentclass[12pt,letterpaper]{report}
\usepackage{graphicx}
\usepackage{scrextend}
\usepackage{vmargin}
\usepackage{graphicx}
\usepackage{multirow}
\usepackage[utf8]{inputenc}
\usepackage[spanish]{babel}
\usepackage{multicol}
\usepackage{enumerate}
\usepackage{hyperref}
\usepackage{float}
\usepackage{amsmath, amsthm, amssymb, amsfonts}
\usepackage[usenames]{color}
\definecolor{urlcolor}{rgb}{0,.145,.698}
    \definecolor{linkcolor}{rgb}{.71,0.21,0.01}
    \definecolor{citecolor}{rgb}{.12,.54,.11}
    \definecolor{def}{rgb}{0.00,0.27,0.87}
    \hypersetup{
      breaklinks=true,  % so long urls are correctly broken across lines
      colorlinks=true,
      urlcolor=urlcolor,
      linkcolor=linkcolor,
      citecolor=citecolor,
      }
\parindent=0mm
\pagestyle{empty}
\definecolor{miorange}{rgb}{0.91, 0.43, 0.0}
\begin{document}
\setmargins{2.5cm}      
{1.5cm}                     
{2cm}  
{24cm}                    
{10pt}                          
{1cm}                          
{0pt}                             
{2cm}
\begin{titlepage}
\begin{center}
\includegraphics[scale=0.40]{../../Logos/uanl.png} 
\hspace{2.5cm}
\includegraphics[scale=0.40]{../../Logos/fcfm.png}
\end{center}
\vspace{2cm}
\begin{center}
\textbf{
UNIVERSIDAD AUTÓNOMA DE NUEVO LEÓN\\
FACULTAD DE CIENCIAS
    FÍSICO MATEMÁTICAS}\\
\vspace*{2cm}
\begin{large}
\vspace{1cm}
\large{\textbf{Tópicos de Mécanica Cuántica}}\\
\textbf{Tarea 6}\\
Dr. Carlos Luna Criado\\
\end{large}
\vspace{3.5cm}
\begin{minipage}{0.6\linewidth}
\vspace{0.5cm}
\changefontsizes{14pt}
Nombre:\\                                                                                                                                                                                                                                                           
Giovanni Gamaliel López Padilla\\
\end{minipage}
\begin{minipage}{0.2\linewidth}
\changefontsizes{14pt} 
Matricula:\\                                                                                                                        
1837522
\end{minipage}
\end{center}
\vspace{4cm}
\begin{flushright}
\today
\end{flushright}
\end{titlepage}
Razone si los siguientes pares tienen estructura de grupo o no:
\begin{itemize}
    \item $(\mathbb{R},+)$\\
    Sean $a,b,c \in \mathbb{R}$, entonces:
    \begin{itemize}
        \item Cerradura\\
        Proponiendo la operación $a+b$, esto nos da como resultado un elemento, el cual esta dentro del conjunto $\mathbb{R}$
        \item Elemento neutro\\
        El elemento neutro en este conjunto con la operación $+$ es el número 0, el cual es elemento del conjunto $\mathbb{R}$
        \item Elemento inverso\\
        El elemento inverso para cada número en este conjunto es el mismo múmero pero de signo opuesto, para obtener que $a+(-a)=(-a)+a=0$
        \item Propiedad asociativa\\
        Se tiene que:
        \begin{align*}
            a+(b+c)=(a+b)+c
        \end{align*}
    \end{itemize}
    por todo lo anterior, se tiene que el conjunto $\mathbb{R}$ con la operación +, forman un grupo.
    \item $(\mathbb{Z},\times)$\\
    Sea $a,b,c \in \mathbb{Z}$, entonces
    \begin{itemize}
        \item Cerradura
        Usando la operación $\times$, se tiene que $a\times b \in \mathbb{Z}$.
        \item Elemento neutro
        El elemento neutro de este conjunto con esta operación sería el 1, ya que $a\times 1 = 1\times a = a$
        \item Elemento inverso
        El elemento inverso para este conjunto no existe, ya que para que esto sea necesario este deberia ser $1/a$ y al no estar dentro del conjunto $\mathbb{Z}$, no existe.
    \end{itemize}
    Por lo tanto, el conjunto $\mathbb{Z}$ con la operación $\times$ no es un grupo
    \item El conjunto de matrices de $n\times n$ construidas con los números reales y determinante no nulo, junto a la operación producto de matrices.\\
    Sean $a,b,c$ elemento de $M_{n\times n}$, entonces
    \begin{itemize}
        \item Cerradura\\
        Calculando el determinante de ab, se tiene que:
        \begin{align*}
            det(ab)=det(a)det(b)
        \end{align*}
        como $det(a),det(b)\neq 0$ entonces $det(a)det(b)\neq 0$, por lo tanto $ab \in M_{n\times n}$
        \item Elemento neutro\\
        El elemento neutro en este conjunto sería la matriz identidad ya que: $a\mathbb{I}=\mathbb{I}a=a$, como $det(\mathbb{I})=1$, entonces $\mathbb{I}\in M_{n\times n}$
        \item Elemento inverso\\
        El elemento inverso de cada elemento del conjunto sería $a^{-1}$, el cual contiene numeros reales en sus elementos y 
        \begin{equation*}
            det(a^{-1})=1/det(a)\neq 0,    
        \end{equation*}
        es por ello que $a^{-1}\in M_{n\times n}$
        \item Propiedad asociativa\\
        Se tiene que se cumple que 
        \begin{equation*}
            a(bc)=(ab)c
        \end{equation*}
    \end{itemize}
    por lo tanto el conjunto $M_{n\times n}$ es un conjunto con la operación de multiplicación.
    \item $(V,\cdot$ siendo $V$ el conjunto de vectores de un espacio vectorial y $\cdot$ el producto escalar o producto interno
    \begin{itemize}
        \item Cerradura\\
        Como la operación del producto interno nos llevaria al grupo de los reales, entonces no cumpliria la condición de cerradura, por lo tanto
        el conjunto V no es un grupo bajo el producto interno.
    \end{itemize}
    \item $(V,\times)$ siendo V el conjunto de vectores de un espacio vectorial y $\times$ el prodcuto vectorial.\\
    Sea $a,b,c \in V$
    \begin{itemize}
        \item Propiedad asociativa\\
        En este conjunto, la propiedad asociativa no se llega a cumplir ya que, 
        \begin{align*}
            a\times (b\times c) &= (a\times b) \times c\\
            a\times \hat{d} |b||c|sin(\theta_{d}) &= |a||b|sin(\theta_{f}) \hat{f} \times c\\
            |a||b||c| sin(\theta_{d})sin(\theta_{g}) &= |a||b||c| sin(\theta_f) sin(\theta_h)
        \end{align*}
    como los argumentos de las funciones seno no necesariamente son iguales, entonces las dos expresions son diferentes, es por esto que el conjunto V no forma un grupo con el producto vectorial.
    \end{itemize}
\end{itemize}
\end{document}