\documentclass[12pt,letterpaper]{report}
\usepackage{scrextend}
\usepackage{vmargin}
\usepackage{graphicx}
\usepackage{multirow}
\usepackage[utf8]{inputenc}
\usepackage[spanish]{babel}
\usepackage{multicol}
\usepackage{enumerate}
\usepackage{float}
\usepackage{amsmath, amsthm, amssymb, amsfonts}
\usepackage[usenames]{color}
\parindent=0mm
\spanishdecimal{.}
\pagestyle{empty}
\definecolor{miorange}{rgb}{0.91, 0.43, 0.0}
\begin{document}
\setmargins{2.5cm}      
{1.5cm}                     
{2cm}  
{24cm}                    
{10pt}                          
{1cm}                          
{0pt}                             
{2cm}
\begin{titlepage}
\begin{center}
\includegraphics[scale=0.40]{../../Logos/uanl.png} 
\hspace{2.5cm}
\includegraphics[scale=0.40]{../../Logos/fcfm.png}
\end{center}
\vspace{2cm}
\begin{center}
\textbf{
UNIVERSIDAD AUTÓNOMA DE NUEVO LEÓN\\
FACULTAD DE CIENCIAS
FÍSICO MATEMÁTICAS}\\
\vspace*{2cm}
\begin{large}
\vspace{1cm}
\large{\textbf{Aplicaciones de la Mecánica Cuántica}}\\
\textbf{Cálculo de la longitud de De Broglie}\\
Carlos Luna Criado\\
\end{large}
\vspace{3.5cm}
\begin{minipage}{0.6\linewidth}
\vspace{0.5cm}
\changefontsizes{14pt}
Nombre:\\
Giovanni Gamaliel López Padilla\\
\end{minipage}
\begin{minipage}{0.2\linewidth}
\changefontsizes{14pt}
Matricula:\\
1837522
\end{minipage}
\end{center}
\vspace{4cm}
\begin{flushright}
\today
\end{flushright}
\end{titlepage}
\begin{enumerate}
    \item \textbf{Calcule la longitud de onda de De Broglie de una pelota de béisbol que se mueve a una velocidad v=10 m/s y que tiene una masa m=1.0 kg.}\\
    Se tiene que: v=10 m/s y m=1kg, introduciendo estos valores en la siguiente ecuación podremos obtener el valor de la onda de De Broglie.\\
    \begin{align*}
        \lambda &= \frac{h}{m_0 v}\\ 
                &= \frac{6.6x10^{-34} Js}{(1 kg)(10 m/s)} \\
                &= 6.6x10^{-35} m 
    \end{align*}
    por lo que la longitud de onda de De Broglie para este caso es de $\lambda = 6.6x10^{-35}m $
    \item \textbf{Calcule la longitud de onda de un electrón cuya energía cinética es 100 eV.}\\
    Al ser un electrón, en este caso su masa viene dada por $m_e$, y obteniendo su energía en terminos de J, se tiene que:
    \begin{align*}
        E   &= 100eV \\
            &= 100eV \left(\frac{1.6x10^{-19}J}{1eV} \right) \\
            &= 1.6x10^{-17}J\\
    \end{align*}
    por lo que aplicando la ecuación
    \begin{equation*}
        \lambda = \frac{hc}{E}
    \end{equation*}
    se obtiene que 
    \begin{align*}
        \lambda &= \frac{hc}{E} \\
                &= \frac{(6.6x10^{-34}Js)(3x10^8 m/s)}{1.6x10^{-17}J }\\
                & =12.375x 10^{-9} m \\
                &= 12.375 nm
    \end{align*}
    por lo que la longitud de De Boglie para este caso es $\lambda= 12.375$nm.
    \item \textbf{Compare los dos valores obtenidos en los apartados anteriores y acorde a esta compración razone en qué condiciones se pueden detectar el comportamiento ondulatorio de la materia.}\\
    Los valores en el problema 1 y 2 tienen una diferencia grande, en el cual sabemos que en el problema dos se menciona que se trata de un electrón, por lo que este presenta un comportamiento ondulatorio, en cambio 
    la pelota de beísbol no, por lo que se obtiene que la energía que obtiene el cuerpo debe ser un tamaño por el orden de los nanometros para que su energía sea la necesaria para producir un movimiento. De otra manera, esto lo podemos calcular con la onda de broglie, 
    si este valor es mayor o igual al tamaño de los átomos esta puede ser percibida y ser medida para comprobar que se trata de un movimiento ondulatorio.
\end{enumerate}
\textbf{Datos:}\\
$h=6.6x10^{-34}Js$\\
$m_e=9.1x10^{-31}kg$\\
$1 eV=1.6x10^{-19}j$
\end{document}