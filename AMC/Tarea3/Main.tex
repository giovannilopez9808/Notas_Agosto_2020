\documentclass[12pt,letterpaper]{report}
\usepackage{graphicx}
\usepackage{scrextend}
\usepackage{vmargin}
\usepackage{graphicx}
\usepackage{multirow}
\usepackage[utf8]{inputenc}
\usepackage[spanish]{babel}
\usepackage{multicol}
\usepackage{enumerate}
\usepackage{float}
\usepackage{amsmath, amsthm, amssymb, amsfonts}
\usepackage[usenames]{color}
\parindent=0mm
\pagestyle{empty}
\definecolor{miorange}{rgb}{0.91, 0.43, 0.0}
\begin{document}
\setmargins{2.5cm}      
{1.5cm}                     
{2cm}  
{24cm}                    
{10pt}                          
{1cm}                          
{0pt}                             
{2cm}
\begin{titlepage}
\begin{center}
\includegraphics[scale=0.40]{../../Logos/uanl.png} 
\hspace{2.5cm}
\includegraphics[scale=0.40]{../../Logos/fcfm.png}
\end{center}
\vspace{2cm}
\begin{center}
\textbf{
UNIVERSIDAD AUTÓNOMA DE NUEVO LEÓN\\
FACULTAD DE CIENCIAS
    FÍSICO MATEMÁTICAS}\\
\vspace*{2cm}
\begin{large}
\vspace{1cm}
\large{\textbf{Tópicos de Mécanica Cuántica}}\\
\textbf{Tarea 3}\\
Dr. Carlos Luna Criado\\
\end{large}
\vspace{3.5cm}
\begin{minipage}{0.6\linewidth}
\vspace{0.5cm}
\changefontsizes{14pt}
Nombre:\\
Giovanni Gamaliel López Padilla\\
\end{minipage}
\begin{minipage}{0.2\linewidth}
\changefontsizes{14pt}
Matricula:\\
1837522
\end{minipage}
\end{center}
\vspace{4cm}
\begin{flushright}
\today
\end{flushright}
\end{titlepage}
Sea el vector 
\begin{equation*}
    \left| \psi \right\rangle = \left(\begin{matrix}
        x+3i \\ x\\ -2x
    \end{matrix} \right)
\end{equation*}
donde $x\in \Re $. Encuentre $x$ tal que la norma del vector es igual a 1.\\
Calculando $\left\langle \psi | \psi \right\rangle $:
\begin{align*}
    \left\langle \psi | \psi \right\rangle &= \left(\begin{matrix}
        x-3i & x & -2x
    \end{matrix} \right)\left(\begin{matrix}
        x+3i \\ x \\ -2x
    \end{matrix} \right)\\
    & = x^2+9+x^2+4x^2 \\
    & = 6x^2+9 
\end{align*}
como $\left\langle \psi | \psi \right\rangle=1$, entonces:
\begin{align*}
    6x^2+9&=1\\
    6x^2+8&=0\\
    \left(\sqrt{6}x-i2\sqrt{2}\right)\left(\sqrt{6}x+i2\sqrt{2}\right)&=0 \\
    \left(\sqrt{3}x-i2\right)\left(\sqrt{3}x+i2\right)&=0 \\
    x=\frac{2i}{\sqrt{3}} \qquad x=\frac{-2i}{\sqrt{3}}
\end{align*}
Como los valores de x son complejos, entonces no existe x tal que $\left\langle \psi | \psi \right\rangle =1$
\end{document}
