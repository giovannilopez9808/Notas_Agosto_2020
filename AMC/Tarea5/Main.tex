\documentclass[12pt,letterpaper]{report}
\usepackage{graphicx}
\usepackage{scrextend}
\usepackage{vmargin}
\usepackage{graphicx}
\usepackage{multirow}
\usepackage[utf8]{inputenc}
\usepackage[spanish]{babel}
\usepackage{multicol}
\usepackage{enumerate}
\usepackage{float}
\usepackage{amsmath, amsthm, amssymb, amsfonts}
\usepackage[usenames]{color}
\parindent=0mm
\pagestyle{empty}
\definecolor{miorange}{rgb}{0.91, 0.43, 0.0}
\begin{document}
\setmargins{2.5cm}      
{1.5cm}                     
{2cm}  
{24cm}                    
{10pt}                          
{1cm}                          
{0pt}                             
{2cm}
\begin{titlepage}
\begin{center}
\includegraphics[scale=0.40]{../../Logos/uanl.png} 
\hspace{2.5cm}
\includegraphics[scale=0.40]{../../Logos/fcfm.png}
\end{center}
\vspace{2cm}
\begin{center}
\textbf{
UNIVERSIDAD AUTÓNOMA DE NUEVO LEÓN\\
FACULTAD DE CIENCIAS
    FÍSICO MATEMÁTICAS}\\
\vspace*{2cm}
\begin{large}
\vspace{1cm}
\large{\textbf{Tópicos de Mécanica Cuántica}}\\
\textbf{Tarea 5}\\
Dr. Carlos Luna Criado\\
\end{large}
\vspace{3.5cm}
\begin{minipage}{0.6\linewidth}
\vspace{0.5cm}
\changefontsizes{14pt}
Nombre:\\                                                                                                                                                                                                                                                           
Giovanni Gamaliel López Padilla\\
\end{minipage}
\begin{minipage}{0.2\linewidth}
\changefontsizes{14pt} 
Matricula:\\                                                                                                                        
1837522
\end{minipage}
\end{center}
\vspace{4cm}
\begin{flushright}
\today
\end{flushright}
\end{titlepage}
Un operador importante en computación cuántica es el operador "puerta de Hadamard" $(\hat{H})$, que representa una de las puestas lógicas cuánticas más comúnmente empleadas.\\
La puerta lógica cuántica de Hadamard viene representada con la matriz:
\begin{equation*}
    H=\frac{1}{\sqrt{2}}\left(\begin{matrix}1 & 1 \\ 
    1 & -1\end{matrix}\right)
\end{equation*}
\begin{enumerate}
    \item ¿Es esta matriz hermitica?\\
    Una matriz es hermitica si $H=H^\dagger$, por lo que calculando $H^\dagger = (H^T)^*$, se tiene que:
    \begin{equation*}
        H^T= \frac{1}{\sqrt{2}} \left(\begin{matrix}
            1 & 1 \\
            1 & -1
        \end{matrix}\right)
    \end{equation*}
    realizando la compleja conjugada de $H^T$ se obtiene lo siguiente:
    \begin{equation*}
        (H^T)^*= \frac{1}{\sqrt{2}} \left(\begin{matrix}
            1 & 1 \\
            1 & -1
        \end{matrix}\right)
    \end{equation*}
    como $H=(H^T)^*$, entonces H es hermitica.
    \item ¿Es unitaria?\\
    Para que una matriz sea unitaria se tiene que cumplir que $AA^*=A^*A=I$, calculando $H^*$ se tiene que:
    \begin{equation*}
        H^*= \frac{1}{\sqrt{2}} \left(\begin{matrix}
            1 & 1 \\
            1 & -1
        \end{matrix}\right)
    \end{equation*}
    entonces calculando $HH^*$:
    \begin{align*}
        HH^*&= \frac{1}{\sqrt{2}} \left(\begin{matrix}
            1 & 1 \\
            1 & -1
        \end{matrix}\right)\frac{1}{\sqrt{2}} \left(\begin{matrix}
            1 & 1 \\
            1 & -1
        \end{matrix}\right)  \\
        & =\frac{1}{2} \left(\begin{matrix}
            2 & 0 \\
            0 & 2
        \end{matrix}\right)\\
        & = \left(\begin{matrix}
            1 & 0 \\
            0 & 1
        \end{matrix}\right)
    \end{align*}
    calculando $H^*H$:
    \begin{align*}
        H^*H&= \frac{1}{\sqrt{2}} \left(\begin{matrix}
            1 & 1 \\
            1 & -1
        \end{matrix}\right)\frac{1}{\sqrt{2}} \left(\begin{matrix}
            1 & 1 \\
            1 & -1
        \end{matrix}\right)  \\
        & =\frac{1}{2} \left(\begin{matrix}
            2 & 0 \\
            0 & 2
        \end{matrix}\right)\\
        & = \left(\begin{matrix}
            1 & 0 \\
            0 & 1
        \end{matrix}\right)
    \end{align*}
    como $H^*H=HH^*=I$, entonces H es unitaria.
    \item Encuentre los eigenvalores y eigenvectores de $\hat{H}$.\\
    Haciendo uso del código () que fue escrito por el estudiante, se obtiene que los eigenvalores del operador H son:
    \begin{equation*}
        \lambda = \left(\begin{matrix}
            1 & 0\\
            0 &-1
        \end{matrix}\right) \qquad \lambda_1 = 1 \qquad \lambda_2=-1
    \end{equation*}
    y los eigenvectores son
    \begin{equation*}
        v_1=\left[\begin{matrix}
                1+\sqrt{2} \\ 1 
        \end{matrix}\right] \qquad v_2=\left[\begin{matrix}
            1-\sqrt{2} \\ 1 
    \end{matrix}\right]
    \end{equation*}
\end{enumerate}
\end{document}