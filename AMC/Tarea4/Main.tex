\documentclass[12pt,letterpaper]{report}
\usepackage{graphicx}
\usepackage{scrextend}
\usepackage{vmargin}
\usepackage{graphicx}
\usepackage{multirow}
\usepackage[utf8]{inputenc}
\usepackage[spanish]{babel}
\usepackage{multicol}
\usepackage{enumerate}
\usepackage{float}
\usepackage{amsmath, amsthm, amssymb, amsfonts}
\usepackage[usenames]{color}
\parindent=0mm
\pagestyle{empty}
\definecolor{miorange}{rgb}{0.91, 0.43, 0.0}
\begin{document}
\setmargins{2.5cm}      
{1.5cm}                     
{2cm}  
{24cm}                    
{10pt}                          
{1cm}                          
{0pt}                             
{2cm}
\begin{titlepage}
\begin{center}
\includegraphics[scale=0.40]{../../Logos/uanl.png} 
\hspace{2.5cm}
\includegraphics[scale=0.40]{../../Logos/fcfm.png}
\end{center}
\vspace{2cm}
\begin{center}
\textbf{
UNIVERSIDAD AUTÓNOMA DE NUEVO LEÓN\\
FACULTAD DE CIENCIAS
    FÍSICO MATEMÁTICAS}\\
\vspace*{2cm}
\begin{large}
\vspace{1cm}
\large{\textbf{Tópicos de Mécanica Cuántica}}\\
\textbf{Tarea 4}\\
Dr. Carlos Luna Criado\\
\end{large}
\vspace{3.5cm}
\begin{minipage}{0.6\linewidth}
\vspace{0.5cm}
\changefontsizes{14pt}
Nombre:\\                                                                                                                                                                                                                                                           
Giovanni Gamaliel López Padilla\\
\end{minipage}
\begin{minipage}{0.2\linewidth}
\changefontsizes{14pt} 
Matricula:\\                                                                                                                        
1837522
\end{minipage}
\end{center}
\vspace{4cm}
\begin{flushright}
\today
\end{flushright}
\end{titlepage}
Sean los vectores 
\begin{equation*}
\left|a \right\rangle = \left(\begin{matrix}
        x+iy \\ 3 \\ 2 
    \end{matrix} \right)
    \qquad \left|b \right\rangle =\left(\begin{matrix}
        1+2i \\ 0\\ 1+2i \\ 
    \end{matrix} \right) \qquad 
    \left|c \right\rangle = \left(\begin{matrix}
        0 \\ 1 \\ 3i
    \end{matrix} \right)
\end{equation*}
siendo $x,y \in \Re$. ¿Que valores puede tomar x e y para que $\left|a \right\rangle, \left|b \right\rangle,\left|c \right\rangle$ sean linealmente independientes?\\
Si un sistema de vectores son linealmente independientes se tiene que cumplir que el determinante de la matriz conformado por cada vector tiene que ser distinto de cero, por lo que 
se procedera a igualar este determinante para obtener que valores no pueden ser tomados por x e y.
\begin{align*}
    \left|\begin{matrix}
        x+iy & 1+2i & 0\\ 
        3 & 0 & 1 \\ 
        2 & 1+2i & 3i 
    \end{matrix} \right| & = 0 \\
    (-1-2i)x+(2-i)y+(20-5i)&=0+0i\\
    (-x+2y+20) + i( -2x-y-5)&=0+0i\\
\end{align*}
por lo que ahora resolveremos el sistema de ecuaciones
\begin{align*}
    -x+2y+20&=0\\
    -2x-y-5&=0
\end{align*}
dndo como resultado $x=2,y=-9$, por lo que los valores que pueden tomar x e y para que los estados $\left|a \right\rangle, \left|b \right\rangle,\left|c \right\rangle$
sean linealmente independientes es:
\begin{equation*}
    x\neq 2 \qquad y\neq -9
\end{equation*}
\end{document}