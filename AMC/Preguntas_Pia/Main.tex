\documentclass[12pt,letterpaper]{report}
\usepackage{graphicx}
\usepackage{scrextend}
\usepackage{vmargin}
\usepackage{graphicx}
\usepackage{multirow}
\usepackage[utf8]{inputenc}
\usepackage[spanish]{babel}
\usepackage{multicol}
\usepackage{enumerate}
\usepackage{hyperref}
\usepackage{float}
\usepackage{amsmath, amsthm, amssymb, amsfonts}
\usepackage[usenames]{color}
\definecolor{urlcolor}{rgb}{0,.145,.698}
    \definecolor{linkcolor}{rgb}{.71,0.21,0.01}
    \definecolor{citecolor}{rgb}{.12,.54,.11}
    \definecolor{def}{rgb}{0.00,0.27,0.87}
    \hypersetup{
      breaklinks=true,  % so long urls are correctly broken across lines
      colorlinks=true,
      urlcolor=urlcolor,
      linkcolor=linkcolor,
      citecolor=citecolor,
      }
\parindent=0mm
\pagestyle{empty}
\definecolor{miorange}{rgb}{0.91, 0.43, 0.0}
\begin{document}
\setmargins{2.5cm}      
{1.5cm}                     
{2cm}  
{24cm}                    
{10pt}                          
{1cm}                          
{0pt}                             
{2cm}
\begin{titlepage}
\begin{center}
\includegraphics[scale=0.40]{../../Logos/uanl.png} 
\hspace{2.5cm}
\includegraphics[scale=0.40]{../../Logos/fcfm.png}
\end{center}
\vspace{2cm}
\begin{center}
\textbf{
UNIVERSIDAD AUTÓNOMA DE NUEVO LEÓN\\
FACULTAD DE CIENCIAS
    FÍSICO MATEMÁTICAS}\\
\vspace*{2cm}
\begin{large}
\vspace{1cm}
\large{\textbf{Tópicos de Mécanica Cuántica}}\\
\textbf{Preguntas del proyecto final}\\
Dr. Carlos Luna Criado\\
\end{large}
\vspace{3.5cm}
\begin{minipage}{0.6\linewidth}
\changefontsizes{14pt}
Nombre:\\                                                                                                                                                                                                                                                           
Giovanni Gamaliel López Padilla
\end{minipage}
\begin{minipage}{0.2\linewidth}
\changefontsizes{14pt} 
Matricula:\\                                                                                                                        
1837522
\end{minipage}
\end{center}
\vspace{4cm}
\begin{flushright}
\today
\end{flushright}
\end{titlepage}
\begin{enumerate}
    \item ¿Átomos con número atómico muy grande son mucho más grandes que átomos con número atómico pequeño?. Razone su respuesta.\\
    Los átomos con un número atómico mayor es más "grande" que uno que tiene un número de atómico menor en ciertas circunstancias, si este número sobre pasa para que se agregue un nivel de energía, sí será más grande pero no necesariamente tener un número atómico mayor tendrá un mayor radio atómico.
    \item Mencione al menos dos puertas lógicas usadas en la computación cuántica, junto con su matriz de transformación.\\
    La puerta de pauli x \begin{equation*}
        \sigma_x=\left(\begin{matrix}
            0 & 1 \\
            1 & 0.
        \end{matrix}\right) 
    \end{equation*}
    La puerta de Hadamart \begin{equation*}
        H= \frac{1}{\sqrt{2}}
        \left(\begin{matrix}
        1 & 1 \\ 1 & -1.
        \end{matrix}\right)
    \end{equation*}
    \item Mencione la principal diferencia entre las puertas lógicas clásicas y las puertas lógicas cuánticas.\\
    Las puertas lógicas clásicas no necesariamente tienen una transformación inversa, en cambio las cuánticas sí tienen una ya que son transformaciones lineales.
    \item ¿Qué experimento descartó la teoría de Einstein sobre la incompletitud de la mecánica cuántica?\\
    El teorema de Bell o desigualdades de Bell
    \item Mencione y explique dos aplicaciones del entrelazamiento cuántico.
    \begin{enumerate}
        \item Criptografía cuántica\\
        Utiliza principios de la mecánica cuántica para garantizar la absoluta confidencialidad de la información transmitida.
        Una de las propiedades más importantes de la criptografía cuántica es que si un tercero intenta espiar durante la creación de la clave secreta, el proceso se altera advirtiéndose al intruso antes de que se transmita información privada. Esto es consecuencia del teorema de no clonado.
        \item Teleportación\\
        Es un proceso en el cuál se transmite información cuántica de una posición a otra suficientemente alejada mediante un canal clásico. Debido a que se produce un intercambio de información mediante un canal clásico, este intercambio no puede ir más rápido que la velocidad de la luz.
    \end{enumerate}
    \item Menciona y explica los 2 principios físicos con los que funciona el microscopio de efecto túnel.
    \begin{enumerate}
        \item Polartización\\
        Al tener un cuerpo conductor, estos pueden ser generar una corriente eléctrica producida por un cambio en el potencial, lo cual provocará una polarización.
        \item Niveles de fermi\\
        Al momento de realizar la excitación de los electrones de la última capa del material, estos pueden realizar el fenomeno del efecto tunel.
    \end{enumerate}
    \item Explique qué es un punto cuántico.\\
    Es una nanoestructura que confina el movimiento de los electrones de conducción en las tres direcciones espaciales.
    \item ¿Qué limita la eficiencia de una celda solar?.\\
    Los materiales con los cuales esta hecha junto a la estructura que estos tienen.
    \item ¿Por qué se usan puntos cuánticos en el diseño de celdas solares?.\\
    Por que esto tienen una mayor eficiencia energética a comparación de las celdas solares actuales.
\end{enumerate}
\end{document}