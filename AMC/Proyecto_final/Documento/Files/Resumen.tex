En este trabajo se planteará el algoritmo desarrollado por Peter Shor, este algoritmo fue creado para realizar la factorización de un número 
N en dos números primos a partir de principios cuánticos haciendo uso de estados cuánticos de qubits, compuertas lógicas cuánticas, transformadas de Fourier cuántica,
la periodicidad de una función modular y la estimación de la fase de un qubit. Se comparó los tiempos de procesamiento para el número 21 con el algoritmo de criba general de cuerpo
de números, el algoritmo de Shor ejecutado en una computadora clásica y el algoritmo de Shor ejecutado en una computadora cuántica, dando así que el algoritmo más eficiente
es el de Shor ejecutado en una computadora cuántica, esto debido a su estabilidad en el tiempo de calculo y que es el más rápido de los algoritmos propuestos.\\
\textbf{Palabras clave:} Algoritmo de Shor, Transformada de Fourier Cuántica, Python, Qiskit, complejidad algorítmica.