\begin{enumerate}
    \item \label{cod:shorquanrum} \href{https://github.com/giovannilopez9808/Notas_Agosto_2020/blob/master/AMC/Proyecto_final/Scripts/General/Shor_quantum.py}{Shor\_quantum.py}\\
    Este algoritmo realiza la factorización de un número N en dos números primos, este código es capaz de correr en una computadora clásica y en el propotito de IBM Q Experience
    \item \href{https://github.com/giovannilopez9808/Notas_Agosto_2020/blob/master/AMC/Proyecto_final/Scripts/General/Functions.py}{Functions.py \label{cod:functionsshor}}\\
    Esta serie de funciones son usadas en el programa \textit{Shor\_quantum.py}.
    \item \href{https://github.com/giovannilopez9808/Notas_Agosto_2020/blob/master/AMC/Proyecto_final/Scripts/General/r_period.py}{r\_period.py \label{cod:rperiod}}\\
    Este código realiza el calculo del periodo r dados a y N, da como resultado la figura  \ref{fig:condicionr}.
    \item \href{https://github.com/giovannilopez9808/Notas_Agosto_2020/blob/master/AMC/Proyecto_final/Scripts/General/bloch_graphics.py}{bloch\_graphics.py \label{cod:bloch}}\\
    Este código genera las figuras mostradas en \ref{fig:QFT_bloch} usando la representación de Qiskit.
    \item \href{https://github.com/giovannilopez9808/Notas_Agosto_2020/blob/master/AMC/Proyecto_final/Scripts/General/Time_clasic.py}{Time\_clasic.py \label{cod:time_clasic}}\\
    Este código calcula la factorización de un número dado de manera clásica 1000 veces y va tomando el tiempo que tarda en cada uno guardandolo en un archivo de texto.
    \item \href{https://github.com/giovannilopez9808/Notas_Agosto_2020/blob/master/AMC/Proyecto_final/Scripts/General/Time_graphics.py}{Time\_graphics.py \label{cod:time_graphics}}\\
    Este código crea la figura \ref{fig:time} a partir de los archivos de tiempo obtenidos con los códigos \ref{cod:shorquanrum} y \ref{cod:time_clasic}.
\end{enumerate}