% 3) Implementacion de metodos numericos 
% 4) Categorizacion de eficiencia de algoritmos
% 5) Algoritmos de eficiencia
% 6) Surgimiento de la computación cuantica por parte de IBM y Qiskit
En el área de la matemáticas, la factorización es una técnica que consiste en la descomposición en factores de una expresión algebraica
en forma de un producto. El teorema fundamental de la aritmética cubre la factorización de números enteros, este teorema pertenece a la teoría
de números. El teorema de factorización afirma lo siguiente:
\begin{center}
    \textit{Cada entero positivo tiene una única descomposición en números primos.}
\end{center}
El teorema fue demostrado por primera vez por Euclides, aunque la primer demostración completa apareció en las \textit{Disquisitiones Arithmeticae}
de Carl Friedrich Gauss.\\
\begin{center}
AÑADIR HISTORIA DE LOS METODOS MATEMATICOS
\end{center}
La mayor parte de los algoritmos de factorización elementales son de proposito general, es decir, permiten descomponer cualquier
número introducido, la diferencia entre algortimos es el tiempo que se toman para encontrar la factorización del número dado. El problema de 
factorizar enteros de tiempo polinómico no ha sido resulto en computación clásica. Esto puede ser de gran ayuda al avance en el ambito de la 
criptografía, ya que muchos sistemas criptográficos dependen de la imposibilidad de ser resueltos en un tiempo corto.\\
La complejidad de este problema se encuentra en el núcleo de varios sistemas criptográficos importantes. Un algoritmo veloz para la factorización
de enteros significaría que el algoritmo de clave pública RSA es inseguro. Si un número grande, de $b$ bits es el producto de dos primos
de aproximadamente el mismo tamaño, no existe algoritmo conocido capaz de factorizarlo en tiempo polinómico. Esto significa que ningún algoritmo
conocido puede factorizarlo en tiempo O$(b^K)$, para cualquier constante $k$. Aunque, existen algoritmos que son más rápidos que O$(a^b)$ para cualquier a 
mayor que 1. En otras palabras, los mejores algoritmos son súper-polinomiales, pero sub-exponenciales. En particular, el mejor tiempo
asintótico de ejecución lo contiene el algoritmo de \textit{criba general del cuerpo de números (CGCN)}, que para un número n es:
\begin{equation}
    O\left(exp\left(\left(\frac{64}{9}b\right)^\frac{1}{3} \left(log b\right)^\frac{2}{3} \right) \right).
    \label{eq:O(clasico)}
\end{equation}
Para una computadora ordinaria, la CGCN es el mejor algoritmo conocido para números grandes. 