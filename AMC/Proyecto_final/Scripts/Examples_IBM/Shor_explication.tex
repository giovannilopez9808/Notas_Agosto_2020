\documentclass[11pt]{article}

    \usepackage[breakable]{tcolorbox}
    \usepackage{parskip} % Stop auto-indenting (to mimic markdown behaviour)
    
    \usepackage{iftex}
    \ifPDFTeX
    	\usepackage[T1]{fontenc}
    	\usepackage{mathpazo}
    \else
    	\usepackage{fontspec}
    \fi

    % Basic figure setup, for now with no caption control since it's done
    % automatically by Pandoc (which extracts ![](path) syntax from Markdown).
    \usepackage{graphicx}
    % Maintain compatibility with old templates. Remove in nbconvert 6.0
    \let\Oldincludegraphics\includegraphics
    % Ensure that by default, figures have no caption (until we provide a
    % proper Figure object with a Caption API and a way to capture that
    % in the conversion process - todo).
    \usepackage{caption}
    \DeclareCaptionFormat{nocaption}{}
    \captionsetup{format=nocaption,aboveskip=0pt,belowskip=0pt}

    \usepackage[Export]{adjustbox} % Used to constrain images to a maximum size
    \adjustboxset{max size={0.9\linewidth}{0.9\paperheight}}
    \usepackage{float}
    \floatplacement{figure}{H} % forces figures to be placed at the correct location
    \usepackage{xcolor} % Allow colors to be defined
    \usepackage{enumerate} % Needed for markdown enumerations to work
    \usepackage{geometry} % Used to adjust the document margins
    \usepackage{amsmath} % Equations
    \usepackage{amssymb} % Equations
    \usepackage{textcomp} % defines textquotesingle
    % Hack from http://tex.stackexchange.com/a/47451/13684:
    \AtBeginDocument{%
        \def\PYZsq{\textquotesingle}% Upright quotes in Pygmentized code
    }
    \usepackage{upquote} % Upright quotes for verbatim code
    \usepackage{eurosym} % defines \euro
    \usepackage[mathletters]{ucs} % Extended unicode (utf-8) support
    \usepackage{fancyvrb} % verbatim replacement that allows latex
    \usepackage{grffile} % extends the file name processing of package graphics 
                         % to support a larger range
    \makeatletter % fix for grffile with XeLaTeX
    \def\Gread@@xetex#1{%
      \IfFileExists{"\Gin@base".bb}%
      {\Gread@eps{\Gin@base.bb}}%
      {\Gread@@xetex@aux#1}%
    }
    \makeatother

    % The hyperref package gives us a pdf with properly built
    % internal navigation ('pdf bookmarks' for the table of contents,
    % internal cross-reference links, web links for URLs, etc.)
    \usepackage{hyperref}
    % The default LaTeX title has an obnoxious amount of whitespace. By default,
    % titling removes some of it. It also provides customization options.
    \usepackage{titling}
    \usepackage{longtable} % longtable support required by pandoc >1.10
    \usepackage{booktabs}  % table support for pandoc > 1.12.2
    \usepackage[inline]{enumitem} % IRkernel/repr support (it uses the enumerate* environment)
    \usepackage[normalem]{ulem} % ulem is needed to support strikethroughs (\sout)
                                % normalem makes italics be italics, not underlines
    \usepackage{mathrsfs}
    

    
    % Colors for the hyperref package
    \definecolor{urlcolor}{rgb}{0,.145,.698}
    \definecolor{linkcolor}{rgb}{.71,0.21,0.01}
    \definecolor{citecolor}{rgb}{.12,.54,.11}

    % ANSI colors
    \definecolor{ansi-black}{HTML}{3E424D}
    \definecolor{ansi-black-intense}{HTML}{282C36}
    \definecolor{ansi-red}{HTML}{E75C58}
    \definecolor{ansi-red-intense}{HTML}{B22B31}
    \definecolor{ansi-green}{HTML}{00A250}
    \definecolor{ansi-green-intense}{HTML}{007427}
    \definecolor{ansi-yellow}{HTML}{DDB62B}
    \definecolor{ansi-yellow-intense}{HTML}{B27D12}
    \definecolor{ansi-blue}{HTML}{208FFB}
    \definecolor{ansi-blue-intense}{HTML}{0065CA}
    \definecolor{ansi-magenta}{HTML}{D160C4}
    \definecolor{ansi-magenta-intense}{HTML}{A03196}
    \definecolor{ansi-cyan}{HTML}{60C6C8}
    \definecolor{ansi-cyan-intense}{HTML}{258F8F}
    \definecolor{ansi-white}{HTML}{C5C1B4}
    \definecolor{ansi-white-intense}{HTML}{A1A6B2}
    \definecolor{ansi-default-inverse-fg}{HTML}{FFFFFF}
    \definecolor{ansi-default-inverse-bg}{HTML}{000000}

    % commands and environments needed by pandoc snippets
    % extracted from the output of `pandoc -s`
    \providecommand{\tightlist}{%
      \setlength{\itemsep}{0pt}\setlength{\parskip}{0pt}}
    \DefineVerbatimEnvironment{Highlighting}{Verbatim}{commandchars=\\\{\}}
    % Add ',fontsize=\small' for more characters per line
    \newenvironment{Shaded}{}{}
    \newcommand{\KeywordTok}[1]{\textcolor[rgb]{0.00,0.44,0.13}{\textbf{{#1}}}}
    \newcommand{\DataTypeTok}[1]{\textcolor[rgb]{0.56,0.13,0.00}{{#1}}}
    \newcommand{\DecValTok}[1]{\textcolor[rgb]{0.25,0.63,0.44}{{#1}}}
    \newcommand{\BaseNTok}[1]{\textcolor[rgb]{0.25,0.63,0.44}{{#1}}}
    \newcommand{\FloatTok}[1]{\textcolor[rgb]{0.25,0.63,0.44}{{#1}}}
    \newcommand{\CharTok}[1]{\textcolor[rgb]{0.25,0.44,0.63}{{#1}}}
    \newcommand{\StringTok}[1]{\textcolor[rgb]{0.25,0.44,0.63}{{#1}}}
    \newcommand{\CommentTok}[1]{\textcolor[rgb]{0.38,0.63,0.69}{\textit{{#1}}}}
    \newcommand{\OtherTok}[1]{\textcolor[rgb]{0.00,0.44,0.13}{{#1}}}
    \newcommand{\AlertTok}[1]{\textcolor[rgb]{1.00,0.00,0.00}{\textbf{{#1}}}}
    \newcommand{\FunctionTok}[1]{\textcolor[rgb]{0.02,0.16,0.49}{{#1}}}
    \newcommand{\RegionMarkerTok}[1]{{#1}}
    \newcommand{\ErrorTok}[1]{\textcolor[rgb]{1.00,0.00,0.00}{\textbf{{#1}}}}
    \newcommand{\NormalTok}[1]{{#1}}
    
    % Additional commands for more recent versions of Pandoc
    \newcommand{\ConstantTok}[1]{\textcolor[rgb]{0.53,0.00,0.00}{{#1}}}
    \newcommand{\SpecialCharTok}[1]{\textcolor[rgb]{0.25,0.44,0.63}{{#1}}}
    \newcommand{\VerbatimStringTok}[1]{\textcolor[rgb]{0.25,0.44,0.63}{{#1}}}
    \newcommand{\SpecialStringTok}[1]{\textcolor[rgb]{0.73,0.40,0.53}{{#1}}}
    \newcommand{\ImportTok}[1]{{#1}}
    \newcommand{\DocumentationTok}[1]{\textcolor[rgb]{0.73,0.13,0.13}{\textit{{#1}}}}
    \newcommand{\AnnotationTok}[1]{\textcolor[rgb]{0.38,0.63,0.69}{\textbf{\textit{{#1}}}}}
    \newcommand{\CommentVarTok}[1]{\textcolor[rgb]{0.38,0.63,0.69}{\textbf{\textit{{#1}}}}}
    \newcommand{\VariableTok}[1]{\textcolor[rgb]{0.10,0.09,0.49}{{#1}}}
    \newcommand{\ControlFlowTok}[1]{\textcolor[rgb]{0.00,0.44,0.13}{\textbf{{#1}}}}
    \newcommand{\OperatorTok}[1]{\textcolor[rgb]{0.40,0.40,0.40}{{#1}}}
    \newcommand{\BuiltInTok}[1]{{#1}}
    \newcommand{\ExtensionTok}[1]{{#1}}
    \newcommand{\PreprocessorTok}[1]{\textcolor[rgb]{0.74,0.48,0.00}{{#1}}}
    \newcommand{\AttributeTok}[1]{\textcolor[rgb]{0.49,0.56,0.16}{{#1}}}
    \newcommand{\InformationTok}[1]{\textcolor[rgb]{0.38,0.63,0.69}{\textbf{\textit{{#1}}}}}
    \newcommand{\WarningTok}[1]{\textcolor[rgb]{0.38,0.63,0.69}{\textbf{\textit{{#1}}}}}
    
    
    % Define a nice break command that doesn't care if a line doesn't already
    % exist.
    \def\br{\hspace*{\fill} \\* }
    % Math Jax compatibility definitions
    \def\gt{>}
    \def\lt{<}
    \let\Oldtex\TeX
    \let\Oldlatex\LaTeX
    \renewcommand{\TeX}{\textrm{\Oldtex}}
    \renewcommand{\LaTeX}{\textrm{\Oldlatex}}
    % Document parameters
    % Document title
    \title{Shor\_explication}
    
    
    
    
    
% Pygments definitions
\makeatletter
\def\PY@reset{\let\PY@it=\relax \let\PY@bf=\relax%
    \let\PY@ul=\relax \let\PY@tc=\relax%
    \let\PY@bc=\relax \let\PY@ff=\relax}
\def\PY@tok#1{\csname PY@tok@#1\endcsname}
\def\PY@toks#1+{\ifx\relax#1\empty\else%
    \PY@tok{#1}\expandafter\PY@toks\fi}
\def\PY@do#1{\PY@bc{\PY@tc{\PY@ul{%
    \PY@it{\PY@bf{\PY@ff{#1}}}}}}}
\def\PY#1#2{\PY@reset\PY@toks#1+\relax+\PY@do{#2}}

\expandafter\def\csname PY@tok@w\endcsname{\def\PY@tc##1{\textcolor[rgb]{0.73,0.73,0.73}{##1}}}
\expandafter\def\csname PY@tok@c\endcsname{\let\PY@it=\textit\def\PY@tc##1{\textcolor[rgb]{0.25,0.50,0.50}{##1}}}
\expandafter\def\csname PY@tok@cp\endcsname{\def\PY@tc##1{\textcolor[rgb]{0.74,0.48,0.00}{##1}}}
\expandafter\def\csname PY@tok@k\endcsname{\let\PY@bf=\textbf\def\PY@tc##1{\textcolor[rgb]{0.00,0.50,0.00}{##1}}}
\expandafter\def\csname PY@tok@kp\endcsname{\def\PY@tc##1{\textcolor[rgb]{0.00,0.50,0.00}{##1}}}
\expandafter\def\csname PY@tok@kt\endcsname{\def\PY@tc##1{\textcolor[rgb]{0.69,0.00,0.25}{##1}}}
\expandafter\def\csname PY@tok@o\endcsname{\def\PY@tc##1{\textcolor[rgb]{0.40,0.40,0.40}{##1}}}
\expandafter\def\csname PY@tok@ow\endcsname{\let\PY@bf=\textbf\def\PY@tc##1{\textcolor[rgb]{0.67,0.13,1.00}{##1}}}
\expandafter\def\csname PY@tok@nb\endcsname{\def\PY@tc##1{\textcolor[rgb]{0.00,0.50,0.00}{##1}}}
\expandafter\def\csname PY@tok@nf\endcsname{\def\PY@tc##1{\textcolor[rgb]{0.00,0.00,1.00}{##1}}}
\expandafter\def\csname PY@tok@nc\endcsname{\let\PY@bf=\textbf\def\PY@tc##1{\textcolor[rgb]{0.00,0.00,1.00}{##1}}}
\expandafter\def\csname PY@tok@nn\endcsname{\let\PY@bf=\textbf\def\PY@tc##1{\textcolor[rgb]{0.00,0.00,1.00}{##1}}}
\expandafter\def\csname PY@tok@ne\endcsname{\let\PY@bf=\textbf\def\PY@tc##1{\textcolor[rgb]{0.82,0.25,0.23}{##1}}}
\expandafter\def\csname PY@tok@nv\endcsname{\def\PY@tc##1{\textcolor[rgb]{0.10,0.09,0.49}{##1}}}
\expandafter\def\csname PY@tok@no\endcsname{\def\PY@tc##1{\textcolor[rgb]{0.53,0.00,0.00}{##1}}}
\expandafter\def\csname PY@tok@nl\endcsname{\def\PY@tc##1{\textcolor[rgb]{0.63,0.63,0.00}{##1}}}
\expandafter\def\csname PY@tok@ni\endcsname{\let\PY@bf=\textbf\def\PY@tc##1{\textcolor[rgb]{0.60,0.60,0.60}{##1}}}
\expandafter\def\csname PY@tok@na\endcsname{\def\PY@tc##1{\textcolor[rgb]{0.49,0.56,0.16}{##1}}}
\expandafter\def\csname PY@tok@nt\endcsname{\let\PY@bf=\textbf\def\PY@tc##1{\textcolor[rgb]{0.00,0.50,0.00}{##1}}}
\expandafter\def\csname PY@tok@nd\endcsname{\def\PY@tc##1{\textcolor[rgb]{0.67,0.13,1.00}{##1}}}
\expandafter\def\csname PY@tok@s\endcsname{\def\PY@tc##1{\textcolor[rgb]{0.73,0.13,0.13}{##1}}}
\expandafter\def\csname PY@tok@sd\endcsname{\let\PY@it=\textit\def\PY@tc##1{\textcolor[rgb]{0.73,0.13,0.13}{##1}}}
\expandafter\def\csname PY@tok@si\endcsname{\let\PY@bf=\textbf\def\PY@tc##1{\textcolor[rgb]{0.73,0.40,0.53}{##1}}}
\expandafter\def\csname PY@tok@se\endcsname{\let\PY@bf=\textbf\def\PY@tc##1{\textcolor[rgb]{0.73,0.40,0.13}{##1}}}
\expandafter\def\csname PY@tok@sr\endcsname{\def\PY@tc##1{\textcolor[rgb]{0.73,0.40,0.53}{##1}}}
\expandafter\def\csname PY@tok@ss\endcsname{\def\PY@tc##1{\textcolor[rgb]{0.10,0.09,0.49}{##1}}}
\expandafter\def\csname PY@tok@sx\endcsname{\def\PY@tc##1{\textcolor[rgb]{0.00,0.50,0.00}{##1}}}
\expandafter\def\csname PY@tok@m\endcsname{\def\PY@tc##1{\textcolor[rgb]{0.40,0.40,0.40}{##1}}}
\expandafter\def\csname PY@tok@gh\endcsname{\let\PY@bf=\textbf\def\PY@tc##1{\textcolor[rgb]{0.00,0.00,0.50}{##1}}}
\expandafter\def\csname PY@tok@gu\endcsname{\let\PY@bf=\textbf\def\PY@tc##1{\textcolor[rgb]{0.50,0.00,0.50}{##1}}}
\expandafter\def\csname PY@tok@gd\endcsname{\def\PY@tc##1{\textcolor[rgb]{0.63,0.00,0.00}{##1}}}
\expandafter\def\csname PY@tok@gi\endcsname{\def\PY@tc##1{\textcolor[rgb]{0.00,0.63,0.00}{##1}}}
\expandafter\def\csname PY@tok@gr\endcsname{\def\PY@tc##1{\textcolor[rgb]{1.00,0.00,0.00}{##1}}}
\expandafter\def\csname PY@tok@ge\endcsname{\let\PY@it=\textit}
\expandafter\def\csname PY@tok@gs\endcsname{\let\PY@bf=\textbf}
\expandafter\def\csname PY@tok@gp\endcsname{\let\PY@bf=\textbf\def\PY@tc##1{\textcolor[rgb]{0.00,0.00,0.50}{##1}}}
\expandafter\def\csname PY@tok@go\endcsname{\def\PY@tc##1{\textcolor[rgb]{0.53,0.53,0.53}{##1}}}
\expandafter\def\csname PY@tok@gt\endcsname{\def\PY@tc##1{\textcolor[rgb]{0.00,0.27,0.87}{##1}}}
\expandafter\def\csname PY@tok@err\endcsname{\def\PY@bc##1{\setlength{\fboxsep}{0pt}\fcolorbox[rgb]{1.00,0.00,0.00}{1,1,1}{\strut ##1}}}
\expandafter\def\csname PY@tok@kc\endcsname{\let\PY@bf=\textbf\def\PY@tc##1{\textcolor[rgb]{0.00,0.50,0.00}{##1}}}
\expandafter\def\csname PY@tok@kd\endcsname{\let\PY@bf=\textbf\def\PY@tc##1{\textcolor[rgb]{0.00,0.50,0.00}{##1}}}
\expandafter\def\csname PY@tok@kn\endcsname{\let\PY@bf=\textbf\def\PY@tc##1{\textcolor[rgb]{0.00,0.50,0.00}{##1}}}
\expandafter\def\csname PY@tok@kr\endcsname{\let\PY@bf=\textbf\def\PY@tc##1{\textcolor[rgb]{0.00,0.50,0.00}{##1}}}
\expandafter\def\csname PY@tok@bp\endcsname{\def\PY@tc##1{\textcolor[rgb]{0.00,0.50,0.00}{##1}}}
\expandafter\def\csname PY@tok@fm\endcsname{\def\PY@tc##1{\textcolor[rgb]{0.00,0.00,1.00}{##1}}}
\expandafter\def\csname PY@tok@vc\endcsname{\def\PY@tc##1{\textcolor[rgb]{0.10,0.09,0.49}{##1}}}
\expandafter\def\csname PY@tok@vg\endcsname{\def\PY@tc##1{\textcolor[rgb]{0.10,0.09,0.49}{##1}}}
\expandafter\def\csname PY@tok@vi\endcsname{\def\PY@tc##1{\textcolor[rgb]{0.10,0.09,0.49}{##1}}}
\expandafter\def\csname PY@tok@vm\endcsname{\def\PY@tc##1{\textcolor[rgb]{0.10,0.09,0.49}{##1}}}
\expandafter\def\csname PY@tok@sa\endcsname{\def\PY@tc##1{\textcolor[rgb]{0.73,0.13,0.13}{##1}}}
\expandafter\def\csname PY@tok@sb\endcsname{\def\PY@tc##1{\textcolor[rgb]{0.73,0.13,0.13}{##1}}}
\expandafter\def\csname PY@tok@sc\endcsname{\def\PY@tc##1{\textcolor[rgb]{0.73,0.13,0.13}{##1}}}
\expandafter\def\csname PY@tok@dl\endcsname{\def\PY@tc##1{\textcolor[rgb]{0.73,0.13,0.13}{##1}}}
\expandafter\def\csname PY@tok@s2\endcsname{\def\PY@tc##1{\textcolor[rgb]{0.73,0.13,0.13}{##1}}}
\expandafter\def\csname PY@tok@sh\endcsname{\def\PY@tc##1{\textcolor[rgb]{0.73,0.13,0.13}{##1}}}
\expandafter\def\csname PY@tok@s1\endcsname{\def\PY@tc##1{\textcolor[rgb]{0.73,0.13,0.13}{##1}}}
\expandafter\def\csname PY@tok@mb\endcsname{\def\PY@tc##1{\textcolor[rgb]{0.40,0.40,0.40}{##1}}}
\expandafter\def\csname PY@tok@mf\endcsname{\def\PY@tc##1{\textcolor[rgb]{0.40,0.40,0.40}{##1}}}
\expandafter\def\csname PY@tok@mh\endcsname{\def\PY@tc##1{\textcolor[rgb]{0.40,0.40,0.40}{##1}}}
\expandafter\def\csname PY@tok@mi\endcsname{\def\PY@tc##1{\textcolor[rgb]{0.40,0.40,0.40}{##1}}}
\expandafter\def\csname PY@tok@il\endcsname{\def\PY@tc##1{\textcolor[rgb]{0.40,0.40,0.40}{##1}}}
\expandafter\def\csname PY@tok@mo\endcsname{\def\PY@tc##1{\textcolor[rgb]{0.40,0.40,0.40}{##1}}}
\expandafter\def\csname PY@tok@ch\endcsname{\let\PY@it=\textit\def\PY@tc##1{\textcolor[rgb]{0.25,0.50,0.50}{##1}}}
\expandafter\def\csname PY@tok@cm\endcsname{\let\PY@it=\textit\def\PY@tc##1{\textcolor[rgb]{0.25,0.50,0.50}{##1}}}
\expandafter\def\csname PY@tok@cpf\endcsname{\let\PY@it=\textit\def\PY@tc##1{\textcolor[rgb]{0.25,0.50,0.50}{##1}}}
\expandafter\def\csname PY@tok@c1\endcsname{\let\PY@it=\textit\def\PY@tc##1{\textcolor[rgb]{0.25,0.50,0.50}{##1}}}
\expandafter\def\csname PY@tok@cs\endcsname{\let\PY@it=\textit\def\PY@tc##1{\textcolor[rgb]{0.25,0.50,0.50}{##1}}}

\def\PYZbs{\char`\\}
\def\PYZus{\char`\_}
\def\PYZob{\char`\{}
\def\PYZcb{\char`\}}
\def\PYZca{\char`\^}
\def\PYZam{\char`\&}
\def\PYZlt{\char`\<}
\def\PYZgt{\char`\>}
\def\PYZsh{\char`\#}
\def\PYZpc{\char`\%}
\def\PYZdl{\char`\$}
\def\PYZhy{\char`\-}
\def\PYZsq{\char`\'}
\def\PYZdq{\char`\"}
\def\PYZti{\char`\~}
% for compatibility with earlier versions
\def\PYZat{@}
\def\PYZlb{[}
\def\PYZrb{]}
\makeatother


    % For linebreaks inside Verbatim environment from package fancyvrb. 
    \makeatletter
        \newbox\Wrappedcontinuationbox 
        \newbox\Wrappedvisiblespacebox 
        \newcommand*\Wrappedvisiblespace {\textcolor{red}{\textvisiblespace}} 
        \newcommand*\Wrappedcontinuationsymbol {\textcolor{red}{\llap{\tiny$\m@th\hookrightarrow$}}} 
        \newcommand*\Wrappedcontinuationindent {3ex } 
        \newcommand*\Wrappedafterbreak {\kern\Wrappedcontinuationindent\copy\Wrappedcontinuationbox} 
        % Take advantage of the already applied Pygments mark-up to insert 
        % potential linebreaks for TeX processing. 
        %        {, <, #, %, $, ' and ": go to next line. 
        %        _, }, ^, &, >, - and ~: stay at end of broken line. 
        % Use of \textquotesingle for straight quote. 
        \newcommand*\Wrappedbreaksatspecials {% 
            \def\PYGZus{\discretionary{\char`\_}{\Wrappedafterbreak}{\char`\_}}% 
            \def\PYGZob{\discretionary{}{\Wrappedafterbreak\char`\{}{\char`\{}}% 
            \def\PYGZcb{\discretionary{\char`\}}{\Wrappedafterbreak}{\char`\}}}% 
            \def\PYGZca{\discretionary{\char`\^}{\Wrappedafterbreak}{\char`\^}}% 
            \def\PYGZam{\discretionary{\char`\&}{\Wrappedafterbreak}{\char`\&}}% 
            \def\PYGZlt{\discretionary{}{\Wrappedafterbreak\char`\<}{\char`\<}}% 
            \def\PYGZgt{\discretionary{\char`\>}{\Wrappedafterbreak}{\char`\>}}% 
            \def\PYGZsh{\discretionary{}{\Wrappedafterbreak\char`\#}{\char`\#}}% 
            \def\PYGZpc{\discretionary{}{\Wrappedafterbreak\char`\%}{\char`\%}}% 
            \def\PYGZdl{\discretionary{}{\Wrappedafterbreak\char`\$}{\char`\$}}% 
            \def\PYGZhy{\discretionary{\char`\-}{\Wrappedafterbreak}{\char`\-}}% 
            \def\PYGZsq{\discretionary{}{\Wrappedafterbreak\textquotesingle}{\textquotesingle}}% 
            \def\PYGZdq{\discretionary{}{\Wrappedafterbreak\char`\"}{\char`\"}}% 
            \def\PYGZti{\discretionary{\char`\~}{\Wrappedafterbreak}{\char`\~}}% 
        } 
        % Some characters . , ; ? ! / are not pygmentized. 
        % This macro makes them "active" and they will insert potential linebreaks 
        \newcommand*\Wrappedbreaksatpunct {% 
            \lccode`\~`\.\lowercase{\def~}{\discretionary{\hbox{\char`\.}}{\Wrappedafterbreak}{\hbox{\char`\.}}}% 
            \lccode`\~`\,\lowercase{\def~}{\discretionary{\hbox{\char`\,}}{\Wrappedafterbreak}{\hbox{\char`\,}}}% 
            \lccode`\~`\;\lowercase{\def~}{\discretionary{\hbox{\char`\;}}{\Wrappedafterbreak}{\hbox{\char`\;}}}% 
            \lccode`\~`\:\lowercase{\def~}{\discretionary{\hbox{\char`\:}}{\Wrappedafterbreak}{\hbox{\char`\:}}}% 
            \lccode`\~`\?\lowercase{\def~}{\discretionary{\hbox{\char`\?}}{\Wrappedafterbreak}{\hbox{\char`\?}}}% 
            \lccode`\~`\!\lowercase{\def~}{\discretionary{\hbox{\char`\!}}{\Wrappedafterbreak}{\hbox{\char`\!}}}% 
            \lccode`\~`\/\lowercase{\def~}{\discretionary{\hbox{\char`\/}}{\Wrappedafterbreak}{\hbox{\char`\/}}}% 
            \catcode`\.\active
            \catcode`\,\active 
            \catcode`\;\active
            \catcode`\:\active
            \catcode`\?\active
            \catcode`\!\active
            \catcode`\/\active 
            \lccode`\~`\~ 	
        }
    \makeatother

    \let\OriginalVerbatim=\Verbatim
    \makeatletter
    \renewcommand{\Verbatim}[1][1]{%
        %\parskip\z@skip
        \sbox\Wrappedcontinuationbox {\Wrappedcontinuationsymbol}%
        \sbox\Wrappedvisiblespacebox {\FV@SetupFont\Wrappedvisiblespace}%
        \def\FancyVerbFormatLine ##1{\hsize\linewidth
            \vtop{\raggedright\hyphenpenalty\z@\exhyphenpenalty\z@
                \doublehyphendemerits\z@\finalhyphendemerits\z@
                \strut ##1\strut}%
        }%
        % If the linebreak is at a space, the latter will be displayed as visible
        % space at end of first line, and a continuation symbol starts next line.
        % Stretch/shrink are however usually zero for typewriter font.
        \def\FV@Space {%
            \nobreak\hskip\z@ plus\fontdimen3\font minus\fontdimen4\font
            \discretionary{\copy\Wrappedvisiblespacebox}{\Wrappedafterbreak}
            {\kern\fontdimen2\font}%
        }%
        
        % Allow breaks at special characters using \PYG... macros.
        \Wrappedbreaksatspecials
        % Breaks at punctuation characters . , ; ? ! and / need catcode=\active 	
        \OriginalVerbatim[#1,codes*=\Wrappedbreaksatpunct]%
    }
    \makeatother

    % Exact colors from NB
    \definecolor{incolor}{HTML}{303F9F}
    \definecolor{outcolor}{HTML}{D84315}
    \definecolor{cellborder}{HTML}{CFCFCF}
    \definecolor{cellbackground}{HTML}{F7F7F7}
    
    % prompt
    \makeatletter
    \newcommand{\boxspacing}{\kern\kvtcb@left@rule\kern\kvtcb@boxsep}
    \makeatother
    \newcommand{\prompt}[4]{
        \ttfamily\llap{{\color{#2}[#3]:\hspace{3pt}#4}}\vspace{-\baselineskip}
    }
    

    
    % Prevent overflowing lines due to hard-to-break entities
    \sloppy 
    % Setup hyperref package
    \hypersetup{
      breaklinks=true,  % so long urls are correctly broken across lines
      colorlinks=true,
      urlcolor=urlcolor,
      linkcolor=linkcolor,
      citecolor=citecolor,
      }
    % Slightly bigger margins than the latex defaults
    
    \geometry{verbose,tmargin=1in,bmargin=1in,lmargin=1in,rmargin=1in}
    
    

\begin{document}
    
    \maketitle
    
    

    
    \section{Shor's Algorithm}\label{shors-algorithm}

    Shor's algorithm is famous for factoring integers in polynomial time.
Since the best-known classical algorithm requires superpolynomial time
to factor the product of two primes, the widely used cryptosystem, RSA,
relies on factoring being impossible for large enough integers.

In this chapter we will focus on the quantum part of Shor's algorithm,
which actually solves the problem of \emph{period finding}. Since a
factoring problem can be turned into a period finding problem in
polynomial time, an efficient period finding algorithm can be used to
factor integers efficiently too. For now its enough to show that if we
can compute the period of \(a^x\bmod N\) efficiently, then we can also
efficiently factor. Since period finding is a worthy problem in its own
right, we will first solve this, then discuss how this can be used to
factor in section 5.

    \begin{tcolorbox}[breakable, size=fbox, boxrule=1pt, pad at break*=1mm,colback=cellbackground, colframe=cellborder]
\prompt{In}{incolor}{2}{\boxspacing}
\begin{Verbatim}[commandchars=\\\{\}]
\PY{k+kn}{import} \PY{n+nn}{matplotlib}\PY{n+nn}{.}\PY{n+nn}{pyplot} \PY{k}{as} \PY{n+nn}{plt}
\PY{k+kn}{import} \PY{n+nn}{numpy} \PY{k}{as} \PY{n+nn}{np}
\PY{k+kn}{from} \PY{n+nn}{qiskit} \PY{k+kn}{import} \PY{n}{QuantumCircuit}\PY{p}{,} \PY{n}{Aer}\PY{p}{,} \PY{n}{execute}
\PY{k+kn}{from} \PY{n+nn}{qiskit}\PY{n+nn}{.}\PY{n+nn}{visualization} \PY{k+kn}{import} \PY{n}{plot\PYZus{}histogram}
\PY{k+kn}{from} \PY{n+nn}{math} \PY{k+kn}{import} \PY{n}{gcd}
\PY{k+kn}{from} \PY{n+nn}{numpy}\PY{n+nn}{.}\PY{n+nn}{random} \PY{k+kn}{import} \PY{n}{randint}
\PY{k+kn}{import} \PY{n+nn}{pandas} \PY{k}{as} \PY{n+nn}{pd}
\PY{k+kn}{from} \PY{n+nn}{fractions} \PY{k+kn}{import} \PY{n}{Fraction}
\PY{n+nb}{print}\PY{p}{(}\PY{l+s+s2}{\PYZdq{}}\PY{l+s+s2}{Imports Successful}\PY{l+s+s2}{\PYZdq{}}\PY{p}{)}
\end{Verbatim}
\end{tcolorbox}

    \begin{Verbatim}[commandchars=\\\{\}]
Imports Successful
    \end{Verbatim}

    \subsection{1. The Problem: Period
Finding}\label{the-problem-period-finding}

Let's look at the periodic function:

\[ f(x) = a^x \bmod{N}\]

 Reminder: Modulo \& Modular Arithmetic (Click here to expand)

The modulo operation (abbreviated to 'mod') simply means to find the
remainder when dividing one number by another. For example:

\[ 17 \bmod 5 = 2 \]

Since \(17 \div 5 = 3\) with remainder \(2\). (i.e.
\(17 = (3\times 5) + 2\)). In Python, the modulo operation is denoted
through the \% symbol.

This behaviour is used in modular arithmetic, where numbers 'wrap round'
after reaching a certain value (the modulus). Using modular arithmetic,
we could write:

\[ 17 = 2 \pmod 5\]

Note that here the \(\pmod 5\) applies to the entire equation (since it
is in parenthesis), unlike the equation above where it only applied to
the left-hand side of the equation.

where \(a\) and \(N\) are positive integers, \(a\) is less than \(N\),
and they have no common factors. The period, or order (\(r\)), is the
smallest (non-zero) integer such that:

\[a^r \bmod N = 1 \]

We can see an example of this function plotted on the graph below. Note
that the lines between points are to help see the periodicity and do not
represent the intermediate values between the x-markers.

    \begin{tcolorbox}[breakable, size=fbox, boxrule=1pt, pad at break*=1mm,colback=cellbackground, colframe=cellborder]
\prompt{In}{incolor}{3}{\boxspacing}
\begin{Verbatim}[commandchars=\\\{\}]
\PY{n}{N} \PY{o}{=} \PY{l+m+mi}{35}
\PY{n}{a} \PY{o}{=} \PY{l+m+mi}{3}

\PY{c+c1}{\PYZsh{} Calculate the plotting data}
\PY{n}{xvals} \PY{o}{=} \PY{n}{np}\PY{o}{.}\PY{n}{arange}\PY{p}{(}\PY{l+m+mi}{35}\PY{p}{)}
\PY{n}{yvals} \PY{o}{=} \PY{p}{[}\PY{n}{np}\PY{o}{.}\PY{n}{mod}\PY{p}{(}\PY{n}{a}\PY{o}{*}\PY{o}{*}\PY{n}{x}\PY{p}{,} \PY{n}{N}\PY{p}{)} \PY{k}{for} \PY{n}{x} \PY{o+ow}{in} \PY{n}{xvals}\PY{p}{]}

\PY{c+c1}{\PYZsh{} Use matplotlib to display it nicely}
\PY{n}{fig}\PY{p}{,} \PY{n}{ax} \PY{o}{=} \PY{n}{plt}\PY{o}{.}\PY{n}{subplots}\PY{p}{(}\PY{p}{)}
\PY{n}{ax}\PY{o}{.}\PY{n}{plot}\PY{p}{(}\PY{n}{xvals}\PY{p}{,} \PY{n}{yvals}\PY{p}{,} \PY{n}{linewidth}\PY{o}{=}\PY{l+m+mi}{1}\PY{p}{,} \PY{n}{linestyle}\PY{o}{=}\PY{l+s+s1}{\PYZsq{}}\PY{l+s+s1}{dotted}\PY{l+s+s1}{\PYZsq{}}\PY{p}{,} \PY{n}{marker}\PY{o}{=}\PY{l+s+s1}{\PYZsq{}}\PY{l+s+s1}{x}\PY{l+s+s1}{\PYZsq{}}\PY{p}{)}
\PY{n}{ax}\PY{o}{.}\PY{n}{set}\PY{p}{(}\PY{n}{xlabel}\PY{o}{=}\PY{l+s+s1}{\PYZsq{}}\PY{l+s+s1}{\PYZdl{}x\PYZdl{}}\PY{l+s+s1}{\PYZsq{}}\PY{p}{,} \PY{n}{ylabel}\PY{o}{=}\PY{l+s+s1}{\PYZsq{}}\PY{l+s+s1}{\PYZdl{}}\PY{l+s+si}{\PYZpc{}i}\PY{l+s+s1}{\PYZca{}x\PYZdl{} mod \PYZdl{}}\PY{l+s+si}{\PYZpc{}i}\PY{l+s+s1}{\PYZdl{}}\PY{l+s+s1}{\PYZsq{}} \PY{o}{\PYZpc{}} \PY{p}{(}\PY{n}{a}\PY{p}{,} \PY{n}{N}\PY{p}{)}\PY{p}{,}
       \PY{n}{title}\PY{o}{=}\PY{l+s+s2}{\PYZdq{}}\PY{l+s+s2}{Example of Periodic Function in Shor}\PY{l+s+s2}{\PYZsq{}}\PY{l+s+s2}{s Algorithm}\PY{l+s+s2}{\PYZdq{}}\PY{p}{)}
\PY{k}{try}\PY{p}{:} \PY{c+c1}{\PYZsh{} plot r on the graph}
    \PY{n}{r} \PY{o}{=} \PY{n}{yvals}\PY{p}{[}\PY{l+m+mi}{1}\PY{p}{:}\PY{p}{]}\PY{o}{.}\PY{n}{index}\PY{p}{(}\PY{l+m+mi}{1}\PY{p}{)} \PY{o}{+}\PY{l+m+mi}{1} 
    \PY{n}{plt}\PY{o}{.}\PY{n}{annotate}\PY{p}{(}\PY{n}{text}\PY{o}{=}\PY{l+s+s1}{\PYZsq{}}\PY{l+s+s1}{\PYZsq{}}\PY{p}{,} \PY{n}{xy}\PY{o}{=}\PY{p}{(}\PY{l+m+mi}{0}\PY{p}{,}\PY{l+m+mi}{1}\PY{p}{)}\PY{p}{,} \PY{n}{xytext}\PY{o}{=}\PY{p}{(}\PY{n}{r}\PY{p}{,}\PY{l+m+mi}{1}\PY{p}{)}\PY{p}{,} \PY{n}{arrowprops}\PY{o}{=}\PY{n+nb}{dict}\PY{p}{(}\PY{n}{arrowstyle}\PY{o}{=}\PY{l+s+s1}{\PYZsq{}}\PY{l+s+s1}{\PYZlt{}\PYZhy{}\PYZgt{}}\PY{l+s+s1}{\PYZsq{}}\PY{p}{)}\PY{p}{)}
    \PY{n}{plt}\PY{o}{.}\PY{n}{annotate}\PY{p}{(}\PY{n}{text}\PY{o}{=}\PY{l+s+s1}{\PYZsq{}}\PY{l+s+s1}{\PYZdl{}r=}\PY{l+s+si}{\PYZpc{}i}\PY{l+s+s1}{\PYZdl{}}\PY{l+s+s1}{\PYZsq{}} \PY{o}{\PYZpc{}} \PY{n}{r}\PY{p}{,} \PY{n}{xy}\PY{o}{=}\PY{p}{(}\PY{n}{r}\PY{o}{/}\PY{l+m+mi}{3}\PY{p}{,}\PY{l+m+mf}{1.5}\PY{p}{)}\PY{p}{)}
\PY{k}{except}\PY{p}{:}
    \PY{n+nb}{print}\PY{p}{(}\PY{l+s+s1}{\PYZsq{}}\PY{l+s+s1}{Could not find period, check a \PYZlt{} N and have no common factors.}\PY{l+s+s1}{\PYZsq{}}\PY{p}{)}
\PY{n}{plt}\PY{o}{.}\PY{n}{show}\PY{p}{(}\PY{p}{)}
\end{Verbatim}
\end{tcolorbox}

    \begin{center}
    \adjustimage{max size={0.9\linewidth}{0.9\paperheight}}{Shor_explication_files/Shor_explication_4_0.png}
    \end{center}
    { \hspace*{\fill} \\}
    
    \subsection{2. The Solution}\label{the-solution}

Shor's solution was to use
\href{./quantum-phase-estimation.html}{quantum phase estimation} on the
unitary operator:

\[ U|y\rangle \equiv |ay \bmod N \rangle \]

To see how this is helpful, let's work out what an eigenstate of U might
look like. If we started in the state \(|1\rangle\), we can see that
each successive application of U will multiply the state of our register
by \(a \pmod N\), and after \(r\) applications we will arrive at the
state \(|1\rangle\) again. For example with \(a = 3\) and \(N = 35\):

\[\begin{aligned}
U|1\rangle &= |3\rangle & \\
U^2|1\rangle &= |9\rangle \\
U^3|1\rangle &= |27\rangle \\
& \vdots \\
U^{(r-1)}|1\rangle &= |12\rangle \\
U^r|1\rangle &= |1\rangle 
\end{aligned}\]

    \begin{tcolorbox}[breakable, size=fbox, boxrule=1pt, pad at break*=1mm,colback=cellbackground, colframe=cellborder]
\prompt{In}{incolor}{4}{\boxspacing}
\begin{Verbatim}[commandchars=\\\{\}]
\PY{n}{ax}\PY{o}{.}\PY{n}{set}\PY{p}{(}\PY{n}{xlabel}\PY{o}{=}\PY{l+s+s1}{\PYZsq{}}\PY{l+s+s1}{Number of applications of U}\PY{l+s+s1}{\PYZsq{}}\PY{p}{,} \PY{n}{ylabel}\PY{o}{=}\PY{l+s+s1}{\PYZsq{}}\PY{l+s+s1}{End state of register}\PY{l+s+s1}{\PYZsq{}}\PY{p}{,}
       \PY{n}{title}\PY{o}{=}\PY{l+s+s2}{\PYZdq{}}\PY{l+s+s2}{Effect of Successive Applications of U}\PY{l+s+s2}{\PYZdq{}}\PY{p}{)}
\PY{n}{fig}
\end{Verbatim}
\end{tcolorbox}
 
            
\prompt{Out}{outcolor}{4}{}
    
    \begin{center}
    \adjustimage{max size={0.9\linewidth}{0.9\paperheight}}{Shor_explication_files/Shor_explication_6_0.png}
    \end{center}
    { \hspace*{\fill} \\}
    

    So a superposition of the states in this cycle (\(|u_0\rangle\)) would
be an eigenstate of \(U\):

\[|u_0\rangle = \tfrac{1}{\sqrt{r}}\sum_{k=0}^{r-1}{|a^k \bmod N\rangle} \]

 Click to Expand: Example with \(a = 3\) and \(N=35\)

\[\begin{aligned}
|u_0\rangle &= \tfrac{1}{\sqrt{12}}(|1\rangle + |3\rangle + |9\rangle \dots + |4\rangle + |12\rangle) \\[10pt]
U|u_0\rangle &= \tfrac{1}{\sqrt{12}}(U|1\rangle + U|3\rangle + U|9\rangle \dots + U|4\rangle + U|12\rangle) \\[10pt]
 &= \tfrac{1}{\sqrt{12}}(|3\rangle + |9\rangle + |27\rangle \dots + |12\rangle + |1\rangle) \\[10pt]
 &= |u_0\rangle
\end{aligned}\]

This eigenstate has an eigenvalue of 1, which isn't very interesting. A
more interesting eigenstate could be one in which the phase is different
for each of these computational basis states. Specifically, let's look
at the case in which the phase of the \(k\)th state is proportional to
\(k\):

\[\begin{aligned}
|u_1\rangle &= \tfrac{1}{\sqrt{r}}\sum_{k=0}^{r-1}{e^{-\tfrac{2\pi i k}{r}}|a^k \bmod N\rangle}\\[10pt]
U|u_1\rangle &= e^{\tfrac{2\pi i}{r}}|u_1\rangle 
\end{aligned}
\]

 Click to Expand: Example with \(a = 3\) and \(N=35\)

\[\begin{aligned}
|u_1\rangle &= \tfrac{1}{\sqrt{12}}(|1\rangle + e^{-\tfrac{2\pi i}{12}}|3\rangle + e^{-\tfrac{4\pi i}{12}}|9\rangle \dots + e^{-\tfrac{20\pi i}{12}}|4\rangle + e^{-\tfrac{22\pi i}{12}}|12\rangle) \\[10pt]
U|u_1\rangle &= \tfrac{1}{\sqrt{12}}(|3\rangle + e^{-\tfrac{2\pi i}{12}}|9\rangle + e^{-\tfrac{4\pi i}{12}}|27\rangle \dots + e^{-\tfrac{20\pi i}{12}}|12\rangle + e^{-\tfrac{22\pi i}{12}}|1\rangle) \\[10pt]
U|u_1\rangle &= e^{\tfrac{2\pi i}{12}}\cdot\tfrac{1}{\sqrt{12}}(e^{\tfrac{-2\pi i}{12}}|3\rangle + e^{-\tfrac{4\pi i}{12}}|9\rangle + e^{-\tfrac{6\pi i}{12}}|27\rangle \dots + e^{-\tfrac{22\pi i}{12}}|12\rangle + e^{-\tfrac{24\pi i}{12}}|1\rangle) \\[10pt]
U|u_1\rangle &= e^{\tfrac{2\pi i}{12}}|u_1\rangle
\end{aligned}\]

(We can see \(r = 12\) appears in the denominator of the phase.)

This is a particularly interesting eigenvalue as it contains \(r\). In
fact, \(r\) has to be included to make sure the phase differences
between the \(r\) computational basis states are equal. This is not the
only eigenstate with this behaviour; to generalise this further, we can
multiply an integer, \(s\), to this phase difference, which will show up
in our eigenvalue:

\[\begin{aligned}
|u_s\rangle &= \tfrac{1}{\sqrt{r}}\sum_{k=0}^{r-1}{e^{-\tfrac{2\pi i s k}{r}}|a^k \bmod N\rangle}\\[10pt]
U|u_s\rangle &= e^{\tfrac{2\pi i s}{r}}|u_s\rangle 
\end{aligned}
\]

 Click to Expand: Example with \(a = 3\) and \(N=35\)

\[\begin{aligned}
|u_s\rangle &= \tfrac{1}{\sqrt{12}}(|1\rangle + e^{-\tfrac{2\pi i s}{12}}|3\rangle + e^{-\tfrac{4\pi i s}{12}}|9\rangle \dots + e^{-\tfrac{20\pi i s}{12}}|4\rangle + e^{-\tfrac{22\pi i s}{12}}|12\rangle) \\[10pt]
U|u_s\rangle &= \tfrac{1}{\sqrt{12}}(|3\rangle + e^{-\tfrac{2\pi i s}{12}}|9\rangle + e^{-\tfrac{4\pi i s}{12}}|27\rangle \dots + e^{-\tfrac{20\pi i s}{12}}|12\rangle + e^{-\tfrac{22\pi i s}{12}}|1\rangle) \\[10pt]
U|u_s\rangle &= e^{\tfrac{2\pi i s}{12}}\cdot\tfrac{1}{\sqrt{12}}(e^{-\tfrac{2\pi i s}{12}}|3\rangle + e^{-\tfrac{4\pi i s}{12}}|9\rangle + e^{-\tfrac{6\pi i s}{12}}|27\rangle \dots + e^{-\tfrac{22\pi i s}{12}}|12\rangle + e^{-\tfrac{24\pi i s}{12}}|1\rangle) \\[10pt]
U|u_s\rangle &= e^{\tfrac{2\pi i s}{12}}|u_s\rangle
\end{aligned}\]

We now have a unique eigenstate for each integer value of \(s\) where
\[0 \leq s \leq r-1\]. Very conveniently, if we sum up all these
eigenstates, the different phases cancel out all computational basis
states except \(|1\rangle\):

\[ \tfrac{1}{\sqrt{r}}\sum_{s=0}^{r-1} |u_s\rangle = |1\rangle\]

 Click to Expand: Example with \(a = 7\) and \(N=15\)

For this, we will look at a smaller example where \(a = 7\) and
\(N=15\). In this case \(r=4\):

\[\begin{aligned}
\tfrac{1}{2}(\quad|u_0\rangle &= \tfrac{1}{2}(|1\rangle \hphantom{e^{-\tfrac{2\pi i}{12}}}+ |7\rangle \hphantom{e^{-\tfrac{12\pi i}{12}}} + |4\rangle \hphantom{e^{-\tfrac{12\pi i}{12}}} + |13\rangle)\dots \\[10pt]
+ |u_1\rangle &= \tfrac{1}{2}(|1\rangle + e^{-\tfrac{2\pi i}{4}}|7\rangle + e^{-\tfrac{\hphantom{1}4\pi i}{4}}|4\rangle + e^{-\tfrac{\hphantom{1}6\pi i}{4}}|13\rangle)\dots \\[10pt]
+ |u_2\rangle &= \tfrac{1}{2}(|1\rangle + e^{-\tfrac{4\pi i}{4}}|7\rangle + e^{-\tfrac{\hphantom{1}8\pi i}{4}}|4\rangle + e^{-\tfrac{12\pi i}{4}}|13\rangle)\dots \\[10pt]
+ |u_3\rangle &= \tfrac{1}{2}(|1\rangle + e^{-\tfrac{6\pi i}{4}}|7\rangle + e^{-\tfrac{12\pi i}{4}}|4\rangle + e^{-\tfrac{18\pi i}{4}}|13\rangle)\quad) = |1\rangle \\[10pt]
\end{aligned}\]

Since the computational basis state \(|1\rangle\) is a superposition of
these eigenstates, which means if we do QPE on \(U\) using the state
\(|1\rangle\), we will measure a phase:

\[\phi = \frac{s}{r}\]

Where \(s\) is a random integer between \(0\) and \(r-1\). We finally
use the
\href{https://en.wikipedia.org/wiki/Continued_fraction}{continued
fractions} algorithm on \(\phi\) to find \(r\). The circuit diagram
looks like this (note that this diagram uses Qiskit's qubit ordering
convention):

We will next demonstrate Shor's algorithm using Qiskit's simulators. For
this demonstration we will provide the circuits for \(U\) without
explanation, but in section 4 we will discuss how circuits for
\(U^{2^j}\) can be constructed efficiently.

    \subsection{3. Qiskit Implementation}\label{qiskit-implementation}

In this example we will solve the period finding problem for \(a=7\) and
\(N=15\). We provide the circuits for \(U\) where:

\[U|y\rangle = |ay\bmod 15\rangle \]

without explanation. To create \(U^x\), we will simply repeat the
circuit \(x\) times. In the next section we will discuss a general
method for creating these circuits efficiently. The function
\texttt{c\_amod15} returns the controlled-U gate for \texttt{a},
repeated \texttt{power} times.

    \begin{tcolorbox}[breakable, size=fbox, boxrule=1pt, pad at break*=1mm,colback=cellbackground, colframe=cellborder]
\prompt{In}{incolor}{5}{\boxspacing}
\begin{Verbatim}[commandchars=\\\{\}]
\PY{k}{def} \PY{n+nf}{c\PYZus{}amod15}\PY{p}{(}\PY{n}{a}\PY{p}{,} \PY{n}{power}\PY{p}{)}\PY{p}{:}
    \PY{l+s+sd}{\PYZdq{}\PYZdq{}\PYZdq{}Controlled multiplication by a mod 15\PYZdq{}\PYZdq{}\PYZdq{}}
    \PY{k}{if} \PY{n}{a} \PY{o+ow}{not} \PY{o+ow}{in} \PY{p}{[}\PY{l+m+mi}{2}\PY{p}{,}\PY{l+m+mi}{7}\PY{p}{,}\PY{l+m+mi}{8}\PY{p}{,}\PY{l+m+mi}{11}\PY{p}{,}\PY{l+m+mi}{13}\PY{p}{]}\PY{p}{:}
        \PY{k}{raise} \PY{n+ne}{ValueError}\PY{p}{(}\PY{l+s+s2}{\PYZdq{}}\PY{l+s+s2}{\PYZsq{}}\PY{l+s+s2}{a}\PY{l+s+s2}{\PYZsq{}}\PY{l+s+s2}{ must be 2,7,8,11 or 13}\PY{l+s+s2}{\PYZdq{}}\PY{p}{)}
    \PY{n}{U} \PY{o}{=} \PY{n}{QuantumCircuit}\PY{p}{(}\PY{l+m+mi}{4}\PY{p}{)}        
    \PY{k}{for} \PY{n}{iteration} \PY{o+ow}{in} \PY{n+nb}{range}\PY{p}{(}\PY{n}{power}\PY{p}{)}\PY{p}{:}
        \PY{k}{if} \PY{n}{a} \PY{o+ow}{in} \PY{p}{[}\PY{l+m+mi}{2}\PY{p}{,}\PY{l+m+mi}{13}\PY{p}{]}\PY{p}{:}
            \PY{n}{U}\PY{o}{.}\PY{n}{swap}\PY{p}{(}\PY{l+m+mi}{0}\PY{p}{,}\PY{l+m+mi}{1}\PY{p}{)}
            \PY{n}{U}\PY{o}{.}\PY{n}{swap}\PY{p}{(}\PY{l+m+mi}{1}\PY{p}{,}\PY{l+m+mi}{2}\PY{p}{)}
            \PY{n}{U}\PY{o}{.}\PY{n}{swap}\PY{p}{(}\PY{l+m+mi}{2}\PY{p}{,}\PY{l+m+mi}{3}\PY{p}{)}
        \PY{k}{if} \PY{n}{a} \PY{o+ow}{in} \PY{p}{[}\PY{l+m+mi}{7}\PY{p}{,}\PY{l+m+mi}{8}\PY{p}{]}\PY{p}{:}
            \PY{n}{U}\PY{o}{.}\PY{n}{swap}\PY{p}{(}\PY{l+m+mi}{2}\PY{p}{,}\PY{l+m+mi}{3}\PY{p}{)}
            \PY{n}{U}\PY{o}{.}\PY{n}{swap}\PY{p}{(}\PY{l+m+mi}{1}\PY{p}{,}\PY{l+m+mi}{2}\PY{p}{)}
            \PY{n}{U}\PY{o}{.}\PY{n}{swap}\PY{p}{(}\PY{l+m+mi}{0}\PY{p}{,}\PY{l+m+mi}{1}\PY{p}{)}
        \PY{k}{if} \PY{n}{a} \PY{o}{==} \PY{l+m+mi}{11}\PY{p}{:}
            \PY{n}{U}\PY{o}{.}\PY{n}{swap}\PY{p}{(}\PY{l+m+mi}{1}\PY{p}{,}\PY{l+m+mi}{3}\PY{p}{)}
            \PY{n}{U}\PY{o}{.}\PY{n}{swap}\PY{p}{(}\PY{l+m+mi}{0}\PY{p}{,}\PY{l+m+mi}{2}\PY{p}{)}
        \PY{k}{if} \PY{n}{a} \PY{o+ow}{in} \PY{p}{[}\PY{l+m+mi}{7}\PY{p}{,}\PY{l+m+mi}{11}\PY{p}{,}\PY{l+m+mi}{13}\PY{p}{]}\PY{p}{:}
            \PY{k}{for} \PY{n}{q} \PY{o+ow}{in} \PY{n+nb}{range}\PY{p}{(}\PY{l+m+mi}{4}\PY{p}{)}\PY{p}{:}
                \PY{n}{U}\PY{o}{.}\PY{n}{x}\PY{p}{(}\PY{n}{q}\PY{p}{)}
    \PY{n}{U} \PY{o}{=} \PY{n}{U}\PY{o}{.}\PY{n}{to\PYZus{}gate}\PY{p}{(}\PY{p}{)}
    \PY{n}{U}\PY{o}{.}\PY{n}{name} \PY{o}{=} \PY{l+s+s2}{\PYZdq{}}\PY{l+s+si}{\PYZpc{}i}\PY{l+s+s2}{\PYZca{}}\PY{l+s+si}{\PYZpc{}i}\PY{l+s+s2}{ mod 15}\PY{l+s+s2}{\PYZdq{}} \PY{o}{\PYZpc{}} \PY{p}{(}\PY{n}{a}\PY{p}{,} \PY{n}{power}\PY{p}{)}
    \PY{n}{c\PYZus{}U} \PY{o}{=} \PY{n}{U}\PY{o}{.}\PY{n}{control}\PY{p}{(}\PY{p}{)}
    \PY{k}{return} \PY{n}{c\PYZus{}U}
\end{Verbatim}
\end{tcolorbox}

    We will use 8 counting qubits:

    \begin{tcolorbox}[breakable, size=fbox, boxrule=1pt, pad at break*=1mm,colback=cellbackground, colframe=cellborder]
\prompt{In}{incolor}{6}{\boxspacing}
\begin{Verbatim}[commandchars=\\\{\}]
\PY{c+c1}{\PYZsh{} Specify variables}
\PY{n}{n\PYZus{}count} \PY{o}{=} \PY{l+m+mi}{8} \PY{c+c1}{\PYZsh{} number of counting qubits}
\PY{n}{a} \PY{o}{=} \PY{l+m+mi}{7}
\end{Verbatim}
\end{tcolorbox}

    We also provide the circuit for the inverse QFT (you can read more about
the QFT in the
\href{./quantum-fourier-transform.html\#generalqft}{quantum Fourier
transform chapter}):

    \begin{tcolorbox}[breakable, size=fbox, boxrule=1pt, pad at break*=1mm,colback=cellbackground, colframe=cellborder]
\prompt{In}{incolor}{7}{\boxspacing}
\begin{Verbatim}[commandchars=\\\{\}]
\PY{k}{def} \PY{n+nf}{qft\PYZus{}dagger}\PY{p}{(}\PY{n}{n}\PY{p}{)}\PY{p}{:}
    \PY{l+s+sd}{\PYZdq{}\PYZdq{}\PYZdq{}n\PYZhy{}qubit QFTdagger the first n qubits in circ\PYZdq{}\PYZdq{}\PYZdq{}}
    \PY{n}{qc} \PY{o}{=} \PY{n}{QuantumCircuit}\PY{p}{(}\PY{n}{n}\PY{p}{)}
    \PY{c+c1}{\PYZsh{} Don\PYZsq{}t forget the Swaps!}
    \PY{k}{for} \PY{n}{qubit} \PY{o+ow}{in} \PY{n+nb}{range}\PY{p}{(}\PY{n}{n}\PY{o}{/}\PY{o}{/}\PY{l+m+mi}{2}\PY{p}{)}\PY{p}{:}
        \PY{n}{qc}\PY{o}{.}\PY{n}{swap}\PY{p}{(}\PY{n}{qubit}\PY{p}{,} \PY{n}{n}\PY{o}{\PYZhy{}}\PY{n}{qubit}\PY{o}{\PYZhy{}}\PY{l+m+mi}{1}\PY{p}{)}
    \PY{k}{for} \PY{n}{j} \PY{o+ow}{in} \PY{n+nb}{range}\PY{p}{(}\PY{n}{n}\PY{p}{)}\PY{p}{:}
        \PY{k}{for} \PY{n}{m} \PY{o+ow}{in} \PY{n+nb}{range}\PY{p}{(}\PY{n}{j}\PY{p}{)}\PY{p}{:}
            \PY{n}{qc}\PY{o}{.}\PY{n}{cu1}\PY{p}{(}\PY{o}{\PYZhy{}}\PY{n}{np}\PY{o}{.}\PY{n}{pi}\PY{o}{/}\PY{n+nb}{float}\PY{p}{(}\PY{l+m+mi}{2}\PY{o}{*}\PY{o}{*}\PY{p}{(}\PY{n}{j}\PY{o}{\PYZhy{}}\PY{n}{m}\PY{p}{)}\PY{p}{)}\PY{p}{,} \PY{n}{m}\PY{p}{,} \PY{n}{j}\PY{p}{)}
        \PY{n}{qc}\PY{o}{.}\PY{n}{h}\PY{p}{(}\PY{n}{j}\PY{p}{)}
    \PY{n}{qc}\PY{o}{.}\PY{n}{name} \PY{o}{=} \PY{l+s+s2}{\PYZdq{}}\PY{l+s+s2}{QFT†}\PY{l+s+s2}{\PYZdq{}}
    \PY{k}{return} \PY{n}{qc}
\end{Verbatim}
\end{tcolorbox}

    With these building blocks we can easily construct the circuit for
Shor's algorithm:

    \begin{tcolorbox}[breakable, size=fbox, boxrule=1pt, pad at break*=1mm,colback=cellbackground, colframe=cellborder]
\prompt{In}{incolor}{8}{\boxspacing}
\begin{Verbatim}[commandchars=\\\{\}]
\PY{c+c1}{\PYZsh{} Create QuantumCircuit with n\PYZus{}count counting qubits}
\PY{c+c1}{\PYZsh{} plus 4 qubits for U to act on}
\PY{n}{qc} \PY{o}{=} \PY{n}{QuantumCircuit}\PY{p}{(}\PY{n}{n\PYZus{}count} \PY{o}{+} \PY{l+m+mi}{4}\PY{p}{,} \PY{n}{n\PYZus{}count}\PY{p}{)}

\PY{c+c1}{\PYZsh{} Initialise counting qubits}
\PY{c+c1}{\PYZsh{} in state |+\PYZgt{}}
\PY{k}{for} \PY{n}{q} \PY{o+ow}{in} \PY{n+nb}{range}\PY{p}{(}\PY{n}{n\PYZus{}count}\PY{p}{)}\PY{p}{:}
    \PY{n}{qc}\PY{o}{.}\PY{n}{h}\PY{p}{(}\PY{n}{q}\PY{p}{)}
    
\PY{c+c1}{\PYZsh{} And ancilla register in state |1\PYZgt{}}
\PY{n}{qc}\PY{o}{.}\PY{n}{x}\PY{p}{(}\PY{l+m+mi}{3}\PY{o}{+}\PY{n}{n\PYZus{}count}\PY{p}{)}

\PY{c+c1}{\PYZsh{} Do controlled\PYZhy{}U operations}
\PY{k}{for} \PY{n}{q} \PY{o+ow}{in} \PY{n+nb}{range}\PY{p}{(}\PY{n}{n\PYZus{}count}\PY{p}{)}\PY{p}{:}
    \PY{n}{qc}\PY{o}{.}\PY{n}{append}\PY{p}{(}\PY{n}{c\PYZus{}amod15}\PY{p}{(}\PY{n}{a}\PY{p}{,} \PY{l+m+mi}{2}\PY{o}{*}\PY{o}{*}\PY{n}{q}\PY{p}{)}\PY{p}{,} 
             \PY{p}{[}\PY{n}{q}\PY{p}{]} \PY{o}{+} \PY{p}{[}\PY{n}{i}\PY{o}{+}\PY{n}{n\PYZus{}count} \PY{k}{for} \PY{n}{i} \PY{o+ow}{in} \PY{n+nb}{range}\PY{p}{(}\PY{l+m+mi}{4}\PY{p}{)}\PY{p}{]}\PY{p}{)}

\PY{c+c1}{\PYZsh{} Do inverse\PYZhy{}QFT}
\PY{n}{qc}\PY{o}{.}\PY{n}{append}\PY{p}{(}\PY{n}{qft\PYZus{}dagger}\PY{p}{(}\PY{n}{n\PYZus{}count}\PY{p}{)}\PY{p}{,} \PY{n+nb}{range}\PY{p}{(}\PY{n}{n\PYZus{}count}\PY{p}{)}\PY{p}{)}

\PY{c+c1}{\PYZsh{} Measure circuit}
\PY{n}{qc}\PY{o}{.}\PY{n}{measure}\PY{p}{(}\PY{n+nb}{range}\PY{p}{(}\PY{n}{n\PYZus{}count}\PY{p}{)}\PY{p}{,} \PY{n+nb}{range}\PY{p}{(}\PY{n}{n\PYZus{}count}\PY{p}{)}\PY{p}{)}
\PY{n}{qc}\PY{o}{.}\PY{n}{draw}\PY{p}{(}\PY{l+s+s1}{\PYZsq{}}\PY{l+s+s1}{text}\PY{l+s+s1}{\PYZsq{}}\PY{p}{)}
\end{Verbatim}
\end{tcolorbox}

            \begin{tcolorbox}[breakable, size=fbox, boxrule=.5pt, pad at break*=1mm, opacityfill=0]
\prompt{Out}{outcolor}{8}{\boxspacing}
\begin{Verbatim}[commandchars=\\\{\}]
      ┌───┐                                                            »
 q\_0: ┤ H ├───────■────────────────────────────────────────────────────»
      ├───┤       │                                                    »
 q\_1: ┤ H ├───────┼──────────────■─────────────────────────────────────»
      ├───┤       │              │                                     »
 q\_2: ┤ H ├───────┼──────────────┼──────────────■──────────────────────»
      ├───┤       │              │              │                      »
 q\_3: ┤ H ├───────┼──────────────┼──────────────┼──────────────■───────»
      ├───┤       │              │              │              │       »
 q\_4: ┤ H ├───────┼──────────────┼──────────────┼──────────────┼───────»
      ├───┤       │              │              │              │       »
 q\_5: ┤ H ├───────┼──────────────┼──────────────┼──────────────┼───────»
      ├───┤       │              │              │              │       »
 q\_6: ┤ H ├───────┼──────────────┼──────────────┼──────────────┼───────»
      ├───┤       │              │              │              │       »
 q\_7: ┤ H ├───────┼──────────────┼──────────────┼──────────────┼───────»
      └───┘┌─────┴┼──────┐┌─────┴┼──────┐┌─────┴┼──────┐┌─────┴┼──────┐»
 q\_8: ─────┤0     │      ├┤0     │      ├┤0     │      ├┤0     │      ├»
           │             ││             ││             ││             │»
 q\_9: ─────┤1            ├┤1            ├┤1            ├┤1            ├»
           │  7\^{}1 mod 15 ││  7\^{}2 mod 15 ││  7\^{}4 mod 15 ││  7\^{}8 mod 15 │»
q\_10: ─────┤2            ├┤2            ├┤2            ├┤2            ├»
      ┌───┐│             ││             ││             ││             │»
q\_11: ┤ X ├┤3            ├┤3            ├┤3            ├┤3            ├»
      └───┘└─────────────┘└─────────────┘└─────────────┘└─────────────┘»
 c: 8/═════════════════════════════════════════════════════════════════»
                                                                       »
«                                                                       »
« q\_0: ─────────────────────────────────────────────────────────────────»
«                                                                       »
« q\_1: ─────────────────────────────────────────────────────────────────»
«                                                                       »
« q\_2: ─────────────────────────────────────────────────────────────────»
«                                                                       »
« q\_3: ─────────────────────────────────────────────────────────────────»
«                                                                       »
« q\_4: ───────■─────────────────────────────────────────────────────────»
«             │                                                         »
« q\_5: ───────┼───────────────■─────────────────────────────────────────»
«             │               │                                         »
« q\_6: ───────┼───────────────┼───────────────■─────────────────────────»
«             │               │               │                         »
« q\_7: ───────┼───────────────┼───────────────┼────────────────■────────»
«      ┌──────┴───────┐┌──────┴───────┐┌──────┴───────┐┌──────┴┼───────┐»
« q\_8: ┤0             ├┤0             ├┤0             ├┤0      │       ├»
«      │              ││              ││              ││               │»
« q\_9: ┤1             ├┤1             ├┤1             ├┤1              ├»
«      │  7\^{}16 mod 15 ││  7\^{}32 mod 15 ││  7\^{}64 mod 15 ││  7\^{}128 mod 15 │»
«q\_10: ┤2             ├┤2             ├┤2             ├┤2              ├»
«      │              ││              ││              ││               │»
«q\_11: ┤3             ├┤3             ├┤3             ├┤3              ├»
«      └──────────────┘└──────────────┘└──────────────┘└───────────────┘»
« c: 8/═════════════════════════════════════════════════════════════════»
«                                                                       »
«      ┌───────┐┌─┐
« q\_0: ┤0      ├┤M├─────────────────────
«      │       │└╥┘┌─┐
« q\_1: ┤1      ├─╫─┤M├──────────────────
«      │       │ ║ └╥┘┌─┐
« q\_2: ┤2      ├─╫──╫─┤M├───────────────
«      │       │ ║  ║ └╥┘┌─┐
« q\_3: ┤3      ├─╫──╫──╫─┤M├────────────
«      │  QFT† │ ║  ║  ║ └╥┘┌─┐
« q\_4: ┤4      ├─╫──╫──╫──╫─┤M├─────────
«      │       │ ║  ║  ║  ║ └╥┘┌─┐
« q\_5: ┤5      ├─╫──╫──╫──╫──╫─┤M├──────
«      │       │ ║  ║  ║  ║  ║ └╥┘┌─┐
« q\_6: ┤6      ├─╫──╫──╫──╫──╫──╫─┤M├───
«      │       │ ║  ║  ║  ║  ║  ║ └╥┘┌─┐
« q\_7: ┤7      ├─╫──╫──╫──╫──╫──╫──╫─┤M├
«      └───────┘ ║  ║  ║  ║  ║  ║  ║ └╥┘
« q\_8: ──────────╫──╫──╫──╫──╫──╫──╫──╫─
«                ║  ║  ║  ║  ║  ║  ║  ║
« q\_9: ──────────╫──╫──╫──╫──╫──╫──╫──╫─
«                ║  ║  ║  ║  ║  ║  ║  ║
«q\_10: ──────────╫──╫──╫──╫──╫──╫──╫──╫─
«                ║  ║  ║  ║  ║  ║  ║  ║
«q\_11: ──────────╫──╫──╫──╫──╫──╫──╫──╫─
«                ║  ║  ║  ║  ║  ║  ║  ║
« c: 8/══════════╩══╩══╩══╩══╩══╩══╩══╩═
«                0  1  2  3  4  5  6  7
\end{Verbatim}
\end{tcolorbox}
        
    Let's see what results we measure:

    \begin{tcolorbox}[breakable, size=fbox, boxrule=1pt, pad at break*=1mm,colback=cellbackground, colframe=cellborder]
\prompt{In}{incolor}{9}{\boxspacing}
\begin{Verbatim}[commandchars=\\\{\}]
\PY{n}{backend} \PY{o}{=} \PY{n}{Aer}\PY{o}{.}\PY{n}{get\PYZus{}backend}\PY{p}{(}\PY{l+s+s1}{\PYZsq{}}\PY{l+s+s1}{qasm\PYZus{}simulator}\PY{l+s+s1}{\PYZsq{}}\PY{p}{)}
\PY{n}{results} \PY{o}{=} \PY{n}{execute}\PY{p}{(}\PY{n}{qc}\PY{p}{,} \PY{n}{backend}\PY{p}{,} \PY{n}{shots}\PY{o}{=}\PY{l+m+mi}{2048}\PY{p}{)}\PY{o}{.}\PY{n}{result}\PY{p}{(}\PY{p}{)}
\PY{n}{counts} \PY{o}{=} \PY{n}{results}\PY{o}{.}\PY{n}{get\PYZus{}counts}\PY{p}{(}\PY{p}{)}
\PY{n}{plot\PYZus{}histogram}\PY{p}{(}\PY{n}{counts}\PY{p}{)}
\end{Verbatim}
\end{tcolorbox}
 
            
\prompt{Out}{outcolor}{9}{}
    
    % \begin{center}
    % \adjustimage{max size={0.9\linewidth}{0.9\paperheight}}{Shor_explication_files/Shor_explication_17_0.png}
    % \end{center}
    { \hspace*{\fill} \\}
    

    Since we have 3 qubits, these results correspond to measured phases of:

    \begin{tcolorbox}[breakable, size=fbox, boxrule=1pt, pad at break*=1mm,colback=cellbackground, colframe=cellborder]
\prompt{In}{incolor}{9}{\boxspacing}
\begin{Verbatim}[commandchars=\\\{\}]
\PY{n}{rows}\PY{p}{,} \PY{n}{measured\PYZus{}phases} \PY{o}{=} \PY{p}{[}\PY{p}{]}\PY{p}{,} \PY{p}{[}\PY{p}{]}
\PY{k}{for} \PY{n}{output} \PY{o+ow}{in} \PY{n}{counts}\PY{p}{:}
    \PY{n}{decimal} \PY{o}{=} \PY{n+nb}{int}\PY{p}{(}\PY{n}{output}\PY{p}{,} \PY{l+m+mi}{2}\PY{p}{)}  \PY{c+c1}{\PYZsh{} Convert (base 2) string to decimal}
    \PY{n}{phase} \PY{o}{=} \PY{n}{decimal}\PY{o}{/}\PY{p}{(}\PY{l+m+mi}{2}\PY{o}{*}\PY{o}{*}\PY{n}{n\PYZus{}count}\PY{p}{)} \PY{c+c1}{\PYZsh{} Find corresponding eigenvalue}
    \PY{n}{measured\PYZus{}phases}\PY{o}{.}\PY{n}{append}\PY{p}{(}\PY{n}{phase}\PY{p}{)}
    \PY{c+c1}{\PYZsh{} Add these values to the rows in our table:}
    \PY{n}{rows}\PY{o}{.}\PY{n}{append}\PY{p}{(}\PY{p}{[}\PY{l+s+s2}{\PYZdq{}}\PY{l+s+si}{\PYZpc{}s}\PY{l+s+s2}{(bin) = }\PY{l+s+si}{\PYZpc{}i}\PY{l+s+s2}{(dec)}\PY{l+s+s2}{\PYZdq{}} \PY{o}{\PYZpc{}} \PY{p}{(}\PY{n}{output}\PY{p}{,} \PY{n}{decimal}\PY{p}{)}\PY{p}{,} 
                 \PY{l+s+s2}{\PYZdq{}}\PY{l+s+si}{\PYZpc{}i}\PY{l+s+s2}{/}\PY{l+s+si}{\PYZpc{}i}\PY{l+s+s2}{ = }\PY{l+s+si}{\PYZpc{}.2f}\PY{l+s+s2}{\PYZdq{}} \PY{o}{\PYZpc{}} \PY{p}{(}\PY{n}{decimal}\PY{p}{,} \PY{l+m+mi}{2}\PY{o}{*}\PY{o}{*}\PY{n}{n\PYZus{}count}\PY{p}{,} \PY{n}{phase}\PY{p}{)}\PY{p}{]}\PY{p}{)}
\PY{c+c1}{\PYZsh{} Print the rows in a table}
\PY{n}{headers}\PY{o}{=}\PY{p}{[}\PY{l+s+s2}{\PYZdq{}}\PY{l+s+s2}{Register Output}\PY{l+s+s2}{\PYZdq{}}\PY{p}{,} \PY{l+s+s2}{\PYZdq{}}\PY{l+s+s2}{Phase}\PY{l+s+s2}{\PYZdq{}}\PY{p}{]}
\PY{n}{df} \PY{o}{=} \PY{n}{pd}\PY{o}{.}\PY{n}{DataFrame}\PY{p}{(}\PY{n}{rows}\PY{p}{,} \PY{n}{columns}\PY{o}{=}\PY{n}{headers}\PY{p}{)}
\PY{n+nb}{print}\PY{p}{(}\PY{n}{df}\PY{p}{)}
\end{Verbatim}
\end{tcolorbox}

    \begin{Verbatim}[commandchars=\\\{\}]
            Register Output           Phase
0    00000000(bin) = 0(dec)    0/256 = 0.00
1   01000000(bin) = 64(dec)   64/256 = 0.25
2  10000000(bin) = 128(dec)  128/256 = 0.50
3  11000000(bin) = 192(dec)  192/256 = 0.75
    \end{Verbatim}

    We can now use the continued fractions algorithm to attempt to find
\(s\) and \(r\). Python has this functionality built in: We can use the
\texttt{fractions} module to turn a float into a \texttt{Fraction}
object, for example:

    \begin{tcolorbox}[breakable, size=fbox, boxrule=1pt, pad at break*=1mm,colback=cellbackground, colframe=cellborder]
\prompt{In}{incolor}{10}{\boxspacing}
\begin{Verbatim}[commandchars=\\\{\}]
\PY{n}{Fraction}\PY{p}{(}\PY{l+m+mf}{0.666}\PY{p}{)}
\end{Verbatim}
\end{tcolorbox}

            \begin{tcolorbox}[breakable, size=fbox, boxrule=.5pt, pad at break*=1mm, opacityfill=0]
\prompt{Out}{outcolor}{10}{\boxspacing}
\begin{Verbatim}[commandchars=\\\{\}]
Fraction(5998794703657501, 9007199254740992)
\end{Verbatim}
\end{tcolorbox}
        
    \begin{tcolorbox}[breakable, size=fbox, boxrule=1pt, pad at break*=1mm,colback=cellbackground, colframe=cellborder]
\prompt{In}{incolor}{11}{\boxspacing}
\begin{Verbatim}[commandchars=\\\{\}]
\PY{l+m+mi}{5998794703657501}\PY{o}{/}\PY{l+m+mi}{9007199254740992}
\end{Verbatim}
\end{tcolorbox}

            \begin{tcolorbox}[breakable, size=fbox, boxrule=.5pt, pad at break*=1mm, opacityfill=0]
\prompt{Out}{outcolor}{11}{\boxspacing}
\begin{Verbatim}[commandchars=\\\{\}]
0.666
\end{Verbatim}
\end{tcolorbox}
        
    Because this gives fractions that return the result exactly (in this
case, \texttt{0.6660000...}), this can give gnarly results like the one
above. We can use the \texttt{.limit\_denominator()} method to get the
fraction that most closely resembles our float, with denominator below a
certain value:

    \begin{tcolorbox}[breakable, size=fbox, boxrule=1pt, pad at break*=1mm,colback=cellbackground, colframe=cellborder]
\prompt{In}{incolor}{12}{\boxspacing}
\begin{Verbatim}[commandchars=\\\{\}]
\PY{c+c1}{\PYZsh{} Get fraction that most closely resembles 0.666}
\PY{c+c1}{\PYZsh{} with denominator \PYZlt{} 15}
\PY{n}{Fraction}\PY{p}{(}\PY{l+m+mf}{0.666}\PY{p}{)}\PY{o}{.}\PY{n}{limit\PYZus{}denominator}\PY{p}{(}\PY{l+m+mi}{15}\PY{p}{)}
\end{Verbatim}
\end{tcolorbox}

            \begin{tcolorbox}[breakable, size=fbox, boxrule=.5pt, pad at break*=1mm, opacityfill=0]
\prompt{Out}{outcolor}{12}{\boxspacing}
\begin{Verbatim}[commandchars=\\\{\}]
Fraction(2, 3)
\end{Verbatim}
\end{tcolorbox}
        
    Much nicer! The order (r) must be less than N, so we will set the
maximum denominator to be \texttt{15}:

    \begin{tcolorbox}[breakable, size=fbox, boxrule=1pt, pad at break*=1mm,colback=cellbackground, colframe=cellborder]
\prompt{In}{incolor}{13}{\boxspacing}
\begin{Verbatim}[commandchars=\\\{\}]
\PY{n}{rows} \PY{o}{=} \PY{p}{[}\PY{p}{]}
\PY{k}{for} \PY{n}{phase} \PY{o+ow}{in} \PY{n}{measured\PYZus{}phases}\PY{p}{:}
    \PY{n}{frac} \PY{o}{=} \PY{n}{Fraction}\PY{p}{(}\PY{n}{phase}\PY{p}{)}\PY{o}{.}\PY{n}{limit\PYZus{}denominator}\PY{p}{(}\PY{l+m+mi}{15}\PY{p}{)}
    \PY{n}{rows}\PY{o}{.}\PY{n}{append}\PY{p}{(}\PY{p}{[}\PY{n}{phase}\PY{p}{,} \PY{l+s+s2}{\PYZdq{}}\PY{l+s+si}{\PYZpc{}i}\PY{l+s+s2}{/}\PY{l+s+si}{\PYZpc{}i}\PY{l+s+s2}{\PYZdq{}} \PY{o}{\PYZpc{}} \PY{p}{(}\PY{n}{frac}\PY{o}{.}\PY{n}{numerator}\PY{p}{,} \PY{n}{frac}\PY{o}{.}\PY{n}{denominator}\PY{p}{)}\PY{p}{,} \PY{n}{frac}\PY{o}{.}\PY{n}{denominator}\PY{p}{]}\PY{p}{)}
\PY{c+c1}{\PYZsh{} Print as a table}
\PY{n}{headers}\PY{o}{=}\PY{p}{[}\PY{l+s+s2}{\PYZdq{}}\PY{l+s+s2}{Phase}\PY{l+s+s2}{\PYZdq{}}\PY{p}{,} \PY{l+s+s2}{\PYZdq{}}\PY{l+s+s2}{Fraction}\PY{l+s+s2}{\PYZdq{}}\PY{p}{,} \PY{l+s+s2}{\PYZdq{}}\PY{l+s+s2}{Guess for r}\PY{l+s+s2}{\PYZdq{}}\PY{p}{]}
\PY{n}{df} \PY{o}{=} \PY{n}{pd}\PY{o}{.}\PY{n}{DataFrame}\PY{p}{(}\PY{n}{rows}\PY{p}{,} \PY{n}{columns}\PY{o}{=}\PY{n}{headers}\PY{p}{)}
\PY{n+nb}{print}\PY{p}{(}\PY{n}{df}\PY{p}{)}
\end{Verbatim}
\end{tcolorbox}

    \begin{Verbatim}[commandchars=\\\{\}]
   Phase Fraction  Guess for r
0   0.00      0/1            1
1   0.25      1/4            4
2   0.50      1/2            2
3   0.75      3/4            4
    \end{Verbatim}

    We can see that two of the measured eigenvalues provided us with the
correct result: \(r=4\), and we can see that Shor's algorithm has a
chance of failing. These bad results are because \(s = 0\), or because
\(s\) and \(r\) are not coprime and instead of \(r\) we are given a
factor of \(r\). The easiest solution to this is to simply repeat the
experiment until we get a satisfying result for \(r\).

\subsubsection{Quick Exercise}\label{quick-exercise}

\begin{itemize}
\tightlist
\item
  Modify the circuit above for values of \(a = 2, 8, 11\) and \(13\).
  What results do you get and why?
\end{itemize}

    \subsection{4. Modular Exponentiation}\label{modular-exponentiation}

You may have noticed that the method of creating the \(U^{2^j}\) gates
by repeating \(U\) grows exponentially with \(j\) and will not result in
a polynomial time algorithm. We want a way to create the operator:

\[ U^{2^j}|y\rangle = |a^{2^j}y \bmod N \rangle \]

that grows polynomially with \(j\). Fortunately, calculating:

\[ a^{2^j} \bmod N\]

efficiently is possible. Classical computers can use an algorithm known
as \emph{repeated squaring} to calculate an exponential. In our case,
since we are only dealing with exponentials of the form \(2^j\), the
repeated squaring algorithm becomes very simple:

    \begin{tcolorbox}[breakable, size=fbox, boxrule=1pt, pad at break*=1mm,colback=cellbackground, colframe=cellborder]
\prompt{In}{incolor}{19}{\boxspacing}
\begin{Verbatim}[commandchars=\\\{\}]
\PY{k}{def} \PY{n+nf}{a2jmodN}\PY{p}{(}\PY{n}{a}\PY{p}{,} \PY{n}{j}\PY{p}{,} \PY{n}{N}\PY{p}{)}\PY{p}{:}
    \PY{l+s+sd}{\PYZdq{}\PYZdq{}\PYZdq{}Compute a\PYZca{}\PYZob{}2\PYZca{}j\PYZcb{} (mod N) by repeated squaring\PYZdq{}\PYZdq{}\PYZdq{}}
    \PY{k}{for} \PY{n}{i} \PY{o+ow}{in} \PY{n+nb}{range}\PY{p}{(}\PY{n}{j}\PY{p}{)}\PY{p}{:}
        \PY{n}{a} \PY{o}{=} \PY{n}{np}\PY{o}{.}\PY{n}{mod}\PY{p}{(}\PY{n}{a}\PY{o}{*}\PY{o}{*}\PY{l+m+mi}{2}\PY{p}{,} \PY{n}{N}\PY{p}{)}
    \PY{k}{return} \PY{n}{a}
\end{Verbatim}
\end{tcolorbox}

    \begin{tcolorbox}[breakable, size=fbox, boxrule=1pt, pad at break*=1mm,colback=cellbackground, colframe=cellborder]
\prompt{In}{incolor}{24}{\boxspacing}
\begin{Verbatim}[commandchars=\\\{\}]
\PY{n}{a2jmodN}\PY{p}{(}\PY{l+m+mi}{7}\PY{p}{,} \PY{l+m+mi}{2049}\PY{p}{,} \PY{l+m+mi}{53}\PY{p}{)}
\end{Verbatim}
\end{tcolorbox}

            \begin{tcolorbox}[breakable, size=fbox, boxrule=.5pt, pad at break*=1mm, opacityfill=0]
\prompt{Out}{outcolor}{24}{\boxspacing}
\begin{Verbatim}[commandchars=\\\{\}]
47
\end{Verbatim}
\end{tcolorbox}
        
    If an efficient algorithm is possible in Python, then we can use the
same algorithm on a quantum computer. Unfortunately, despite scaling
polynomially with \(j\), modular exponentiation circuits are not
straightforward and are the bottleneck in Shor's algorithm. A
beginner-friendly implementation can be found in reference {[}1{]}.

\subsection{5. Factoring from Period
Finding}\label{factoring-from-period-finding}

Not all factoring problems are difficult; we can spot an even number
instantly and know that one of its factors is 2. In fact, there are
\href{https://nvlpubs.nist.gov/nistpubs/FIPS/NIST.FIPS.186-4.pdf\#\%5B\%7B\%22num\%22\%3A127\%2C\%22gen\%22\%3A0\%7D\%2C\%7B\%22name\%22\%3A\%22XYZ\%22\%7D\%2C70\%2C223\%2C0\%5D}{specific
criteria} for choosing numbers that are difficult to factor, but the
basic idea is to choose the product of two large prime numbers.

A general factoring algorithm will first check to see if there is a
shortcut to factoring the integer (is the number even? Is the number of
the form \(N = a^b\)?), before using Shor's period finding for the
worst-case scenario. Since we aim to focus on the quantum part of the
algorithm, we will jump straight to the case in which N is the product
of two primes.

\subsubsection{Example: Factoring 15}\label{example-factoring-15}

To see an example of factoring on a small number of qubits, we will
factor 15, which we all know is the product of the not-so-large prime
numbers 3 and 5.

    \begin{tcolorbox}[breakable, size=fbox, boxrule=1pt, pad at break*=1mm,colback=cellbackground, colframe=cellborder]
\prompt{In}{incolor}{48}{\boxspacing}
\begin{Verbatim}[commandchars=\\\{\}]
\PY{n}{N} \PY{o}{=} \PY{l+m+mi}{15}
\end{Verbatim}
\end{tcolorbox}

    The first step is to choose a random number, \(x\), between \(1\) and
\(N-1\):

    \begin{tcolorbox}[breakable, size=fbox, boxrule=1pt, pad at break*=1mm,colback=cellbackground, colframe=cellborder]
\prompt{In}{incolor}{49}{\boxspacing}
\begin{Verbatim}[commandchars=\\\{\}]
\PY{n}{np}\PY{o}{.}\PY{n}{random}\PY{o}{.}\PY{n}{seed}\PY{p}{(}\PY{l+m+mi}{1}\PY{p}{)} \PY{c+c1}{\PYZsh{} This is to make sure we get reproduceable results}
\PY{n}{a} \PY{o}{=} \PY{n}{randint}\PY{p}{(}\PY{l+m+mi}{2}\PY{p}{,} \PY{l+m+mi}{15}\PY{p}{)}
\PY{n+nb}{print}\PY{p}{(}\PY{n}{a}\PY{p}{)}
\end{Verbatim}
\end{tcolorbox}

    \begin{Verbatim}[commandchars=\\\{\}]
7
    \end{Verbatim}

    Next we quickly check it isn't already a non-trivial factor of \(N\):

    \begin{tcolorbox}[breakable, size=fbox, boxrule=1pt, pad at break*=1mm,colback=cellbackground, colframe=cellborder]
\prompt{In}{incolor}{50}{\boxspacing}
\begin{Verbatim}[commandchars=\\\{\}]
\PY{k+kn}{from} \PY{n+nn}{math} \PY{k+kn}{import} \PY{n}{gcd} \PY{c+c1}{\PYZsh{} greatest common divisor}
\PY{n}{gcd}\PY{p}{(}\PY{n}{a}\PY{p}{,} \PY{l+m+mi}{15}\PY{p}{)}
\end{Verbatim}
\end{tcolorbox}

            \begin{tcolorbox}[breakable, size=fbox, boxrule=.5pt, pad at break*=1mm, opacityfill=0]
\prompt{Out}{outcolor}{50}{\boxspacing}
\begin{Verbatim}[commandchars=\\\{\}]
1
\end{Verbatim}
\end{tcolorbox}
        
    Great. Next, we do Shor's order finding algorithm for \texttt{a\ =\ 7}
and \texttt{N\ =\ 15}. Remember that the phase we measure will be
\(s/r\) where:

\[ a^r \bmod N = 1 \]

and \(s\) is a random integer between 0 and \(r-1\).

    \begin{tcolorbox}[breakable, size=fbox, boxrule=1pt, pad at break*=1mm,colback=cellbackground, colframe=cellborder]
\prompt{In}{incolor}{51}{\boxspacing}
\begin{Verbatim}[commandchars=\\\{\}]
\PY{k}{def} \PY{n+nf}{qpe\PYZus{}amod15}\PY{p}{(}\PY{n}{a}\PY{p}{)}\PY{p}{:}
    \PY{n}{n\PYZus{}count} \PY{o}{=} \PY{l+m+mi}{3}
    \PY{n}{qc} \PY{o}{=} \PY{n}{QuantumCircuit}\PY{p}{(}\PY{l+m+mi}{4}\PY{o}{+}\PY{n}{n\PYZus{}count}\PY{p}{,} \PY{n}{n\PYZus{}count}\PY{p}{)}
    \PY{k}{for} \PY{n}{q} \PY{o+ow}{in} \PY{n+nb}{range}\PY{p}{(}\PY{n}{n\PYZus{}count}\PY{p}{)}\PY{p}{:}
        \PY{n}{qc}\PY{o}{.}\PY{n}{h}\PY{p}{(}\PY{n}{q}\PY{p}{)}     \PY{c+c1}{\PYZsh{} Initialise counting qubits in state |+\PYZgt{}}
    \PY{n}{qc}\PY{o}{.}\PY{n}{x}\PY{p}{(}\PY{l+m+mi}{3}\PY{o}{+}\PY{n}{n\PYZus{}count}\PY{p}{)} \PY{c+c1}{\PYZsh{} And ancilla register in state |1\PYZgt{}}
    \PY{k}{for} \PY{n}{q} \PY{o+ow}{in} \PY{n+nb}{range}\PY{p}{(}\PY{n}{n\PYZus{}count}\PY{p}{)}\PY{p}{:} \PY{c+c1}{\PYZsh{} Do controlled\PYZhy{}U operations}
        \PY{n}{qc}\PY{o}{.}\PY{n}{append}\PY{p}{(}\PY{n}{c\PYZus{}amod15}\PY{p}{(}\PY{n}{a}\PY{p}{,} \PY{l+m+mi}{2}\PY{o}{*}\PY{o}{*}\PY{n}{q}\PY{p}{)}\PY{p}{,} 
                 \PY{p}{[}\PY{n}{q}\PY{p}{]} \PY{o}{+} \PY{p}{[}\PY{n}{i}\PY{o}{+}\PY{n}{n\PYZus{}count} \PY{k}{for} \PY{n}{i} \PY{o+ow}{in} \PY{n+nb}{range}\PY{p}{(}\PY{l+m+mi}{4}\PY{p}{)}\PY{p}{]}\PY{p}{)}
    \PY{n}{qc}\PY{o}{.}\PY{n}{append}\PY{p}{(}\PY{n}{qft\PYZus{}dagger}\PY{p}{(}\PY{n}{n\PYZus{}count}\PY{p}{)}\PY{p}{,} \PY{n+nb}{range}\PY{p}{(}\PY{n}{n\PYZus{}count}\PY{p}{)}\PY{p}{)} \PY{c+c1}{\PYZsh{} Do inverse\PYZhy{}QFT}
    \PY{n}{qc}\PY{o}{.}\PY{n}{measure}\PY{p}{(}\PY{n+nb}{range}\PY{p}{(}\PY{n}{n\PYZus{}count}\PY{p}{)}\PY{p}{,} \PY{n+nb}{range}\PY{p}{(}\PY{n}{n\PYZus{}count}\PY{p}{)}\PY{p}{)}
    \PY{c+c1}{\PYZsh{} Simulate Results}
    \PY{n}{backend} \PY{o}{=} \PY{n}{Aer}\PY{o}{.}\PY{n}{get\PYZus{}backend}\PY{p}{(}\PY{l+s+s1}{\PYZsq{}}\PY{l+s+s1}{qasm\PYZus{}simulator}\PY{l+s+s1}{\PYZsq{}}\PY{p}{)}
    \PY{c+c1}{\PYZsh{} Setting memory=True below allows us to see a list of each sequential reading}
    \PY{n}{result} \PY{o}{=} \PY{n}{execute}\PY{p}{(}\PY{n}{qc}\PY{p}{,} \PY{n}{backend}\PY{p}{,} \PY{n}{shots}\PY{o}{=}\PY{l+m+mi}{1}\PY{p}{,} \PY{n}{memory}\PY{o}{=}\PY{k+kc}{True}\PY{p}{)}\PY{o}{.}\PY{n}{result}\PY{p}{(}\PY{p}{)}
    \PY{n}{readings} \PY{o}{=} \PY{n}{result}\PY{o}{.}\PY{n}{get\PYZus{}memory}\PY{p}{(}\PY{p}{)}
    \PY{n+nb}{print}\PY{p}{(}\PY{l+s+s2}{\PYZdq{}}\PY{l+s+s2}{Register Reading: }\PY{l+s+s2}{\PYZdq{}} \PY{o}{+} \PY{n}{readings}\PY{p}{[}\PY{l+m+mi}{0}\PY{p}{]}\PY{p}{)}
    \PY{n}{phase} \PY{o}{=} \PY{n+nb}{int}\PY{p}{(}\PY{n}{readings}\PY{p}{[}\PY{l+m+mi}{0}\PY{p}{]}\PY{p}{,}\PY{l+m+mi}{2}\PY{p}{)}\PY{o}{/}\PY{p}{(}\PY{l+m+mi}{2}\PY{o}{*}\PY{o}{*}\PY{n}{n\PYZus{}count}\PY{p}{)}
    \PY{n+nb}{print}\PY{p}{(}\PY{l+s+s2}{\PYZdq{}}\PY{l+s+s2}{Corresponding Phase: }\PY{l+s+si}{\PYZpc{}f}\PY{l+s+s2}{\PYZdq{}} \PY{o}{\PYZpc{}} \PY{n}{phase}\PY{p}{)}
    \PY{k}{return} \PY{n}{phase}
\end{Verbatim}
\end{tcolorbox}

    From this phase, we can easily find a guess for \(r\):

    \begin{tcolorbox}[breakable, size=fbox, boxrule=1pt, pad at break*=1mm,colback=cellbackground, colframe=cellborder]
\prompt{In}{incolor}{52}{\boxspacing}
\begin{Verbatim}[commandchars=\\\{\}]
\PY{n}{np}\PY{o}{.}\PY{n}{random}\PY{o}{.}\PY{n}{seed}\PY{p}{(}\PY{l+m+mi}{3}\PY{p}{)} \PY{c+c1}{\PYZsh{} This is to make sure we get reproduceable results}
\PY{n}{phase} \PY{o}{=} \PY{n}{qpe\PYZus{}amod15}\PY{p}{(}\PY{n}{a}\PY{p}{)} \PY{c+c1}{\PYZsh{} Phase = s/r}
\PY{n}{Fraction}\PY{p}{(}\PY{n}{phase}\PY{p}{)}\PY{o}{.}\PY{n}{limit\PYZus{}denominator}\PY{p}{(}\PY{l+m+mi}{15}\PY{p}{)} \PY{c+c1}{\PYZsh{} Denominator should (hopefully!) tell us r}
\end{Verbatim}
\end{tcolorbox}

    \begin{Verbatim}[commandchars=\\\{\}]
Register Reading: 010
Corresponding Phase: 0.250000
    \end{Verbatim}

            \begin{tcolorbox}[breakable, size=fbox, boxrule=.5pt, pad at break*=1mm, opacityfill=0]
\prompt{Out}{outcolor}{52}{\boxspacing}
\begin{Verbatim}[commandchars=\\\{\}]
Fraction(1, 4)
\end{Verbatim}
\end{tcolorbox}
        
    \begin{tcolorbox}[breakable, size=fbox, boxrule=1pt, pad at break*=1mm,colback=cellbackground, colframe=cellborder]
\prompt{In}{incolor}{53}{\boxspacing}
\begin{Verbatim}[commandchars=\\\{\}]
\PY{n}{frac} \PY{o}{=} \PY{n}{Fraction}\PY{p}{(}\PY{n}{phase}\PY{p}{)}\PY{o}{.}\PY{n}{limit\PYZus{}denominator}\PY{p}{(}\PY{l+m+mi}{15}\PY{p}{)}
\PY{n}{s}\PY{p}{,} \PY{n}{r} \PY{o}{=} \PY{n}{frac}\PY{o}{.}\PY{n}{numerator}\PY{p}{,} \PY{n}{frac}\PY{o}{.}\PY{n}{denominator}
\PY{n+nb}{print}\PY{p}{(}\PY{n}{r}\PY{p}{)}
\end{Verbatim}
\end{tcolorbox}

    \begin{Verbatim}[commandchars=\\\{\}]
4
    \end{Verbatim}

    Now we have \(r\), we might be able to use this to find a factor of
\(N\). Since:

\[a^r \bmod N = 1 \]

then:

\[(a^r - 1) \bmod N = 0 \]

which mean \(N\) must divide \(a^r-1\). And if \(r\) is also even, then
we can write:

\[a^r -1 = (a^{r/2}-1)(a^{r/2}+1)\]

(if \(r\) is not even, we cannot go further and must try again with a
different value for \(a\)). There is then a high probability that the
greatest common divisor of either \(a^{r/2}-1\), or \(a^{r/2}+1\) is a
factor of \(N\) {[}2{]}:

    \begin{tcolorbox}[breakable, size=fbox, boxrule=1pt, pad at break*=1mm,colback=cellbackground, colframe=cellborder]
\prompt{In}{incolor}{55}{\boxspacing}
\begin{Verbatim}[commandchars=\\\{\}]
\PY{n}{guesses} \PY{o}{=} \PY{p}{[}\PY{n}{gcd}\PY{p}{(}\PY{n}{a}\PY{o}{*}\PY{o}{*}\PY{p}{(}\PY{n}{r}\PY{o}{/}\PY{o}{/}\PY{l+m+mi}{2}\PY{p}{)}\PY{o}{\PYZhy{}}\PY{l+m+mi}{1}\PY{p}{,} \PY{n}{N}\PY{p}{)}\PY{p}{,} \PY{n}{gcd}\PY{p}{(}\PY{n}{a}\PY{o}{*}\PY{o}{*}\PY{p}{(}\PY{n}{r}\PY{o}{/}\PY{o}{/}\PY{l+m+mi}{2}\PY{p}{)}\PY{o}{+}\PY{l+m+mi}{1}\PY{p}{,} \PY{n}{N}\PY{p}{)}\PY{p}{]}
\PY{n+nb}{print}\PY{p}{(}\PY{n}{guesses}\PY{p}{)}
\end{Verbatim}
\end{tcolorbox}

    \begin{Verbatim}[commandchars=\\\{\}]
[3, 5]
    \end{Verbatim}

    The cell below repeats the algorithm until at least one factor of 15 is
found. You should try re-running the cell a few times to see how it
behaves.

    \begin{tcolorbox}[breakable, size=fbox, boxrule=1pt, pad at break*=1mm,colback=cellbackground, colframe=cellborder]
\prompt{In}{incolor}{23}{\boxspacing}
\begin{Verbatim}[commandchars=\\\{\}]
\PY{n}{a} \PY{o}{=} \PY{l+m+mi}{7}
\PY{n}{factor\PYZus{}found} \PY{o}{=} \PY{k+kc}{False}
\PY{n}{attempt} \PY{o}{=} \PY{l+m+mi}{0}
\PY{k}{while} \PY{o+ow}{not} \PY{n}{factor\PYZus{}found}\PY{p}{:}
    \PY{n}{attempt} \PY{o}{+}\PY{o}{=} \PY{l+m+mi}{1}
    \PY{n+nb}{print}\PY{p}{(}\PY{l+s+s2}{\PYZdq{}}\PY{l+s+se}{\PYZbs{}n}\PY{l+s+s2}{Attempt }\PY{l+s+si}{\PYZpc{}i}\PY{l+s+s2}{:}\PY{l+s+s2}{\PYZdq{}} \PY{o}{\PYZpc{}} \PY{n}{attempt}\PY{p}{)}
    \PY{n}{phase} \PY{o}{=} \PY{n}{qpe\PYZus{}amod15}\PY{p}{(}\PY{n}{a}\PY{p}{)} \PY{c+c1}{\PYZsh{} Phase = s/r}
    \PY{n}{frac} \PY{o}{=} \PY{n}{Fraction}\PY{p}{(}\PY{n}{phase}\PY{p}{)}\PY{o}{.}\PY{n}{limit\PYZus{}denominator}\PY{p}{(}\PY{l+m+mi}{15}\PY{p}{)} \PY{c+c1}{\PYZsh{} Denominator should (hopefully!) tell us r}
    \PY{n}{r} \PY{o}{=} \PY{n}{frac}\PY{o}{.}\PY{n}{denominator}
    \PY{n+nb}{print}\PY{p}{(}\PY{l+s+s2}{\PYZdq{}}\PY{l+s+s2}{Result: r = }\PY{l+s+si}{\PYZpc{}i}\PY{l+s+s2}{\PYZdq{}} \PY{o}{\PYZpc{}} \PY{n}{r}\PY{p}{)}
    \PY{k}{if} \PY{n}{phase} \PY{o}{!=} \PY{l+m+mi}{0}\PY{p}{:}
        \PY{c+c1}{\PYZsh{} Guesses for factors are gcd(x\PYZca{}\PYZob{}r/2\PYZcb{} ±1 , 15)}
        \PY{n}{guesses} \PY{o}{=} \PY{p}{[}\PY{n}{gcd}\PY{p}{(}\PY{n}{a}\PY{o}{*}\PY{o}{*}\PY{p}{(}\PY{n}{r}\PY{o}{/}\PY{o}{/}\PY{l+m+mi}{2}\PY{p}{)}\PY{o}{\PYZhy{}}\PY{l+m+mi}{1}\PY{p}{,} \PY{l+m+mi}{15}\PY{p}{)}\PY{p}{,} \PY{n}{gcd}\PY{p}{(}\PY{n}{a}\PY{o}{*}\PY{o}{*}\PY{p}{(}\PY{n}{r}\PY{o}{/}\PY{o}{/}\PY{l+m+mi}{2}\PY{p}{)}\PY{o}{+}\PY{l+m+mi}{1}\PY{p}{,} \PY{l+m+mi}{15}\PY{p}{)}\PY{p}{]}
        \PY{n+nb}{print}\PY{p}{(}\PY{l+s+s2}{\PYZdq{}}\PY{l+s+s2}{Guessed Factors: }\PY{l+s+si}{\PYZpc{}i}\PY{l+s+s2}{ and }\PY{l+s+si}{\PYZpc{}i}\PY{l+s+s2}{\PYZdq{}} \PY{o}{\PYZpc{}} \PY{p}{(}\PY{n}{guesses}\PY{p}{[}\PY{l+m+mi}{0}\PY{p}{]}\PY{p}{,} \PY{n}{guesses}\PY{p}{[}\PY{l+m+mi}{1}\PY{p}{]}\PY{p}{)}\PY{p}{)}
        \PY{k}{for} \PY{n}{guess} \PY{o+ow}{in} \PY{n}{guesses}\PY{p}{:}
            \PY{k}{if} \PY{n}{guess} \PY{o}{!=} \PY{l+m+mi}{1} \PY{o+ow}{and} \PY{p}{(}\PY{l+m+mi}{15} \PY{o}{\PYZpc{}} \PY{n}{guess}\PY{p}{)} \PY{o}{==} \PY{l+m+mi}{0}\PY{p}{:} \PY{c+c1}{\PYZsh{} Check to see if guess is a factor}
                \PY{n+nb}{print}\PY{p}{(}\PY{l+s+s2}{\PYZdq{}}\PY{l+s+s2}{*** Non\PYZhy{}trivial factor found: }\PY{l+s+si}{\PYZpc{}i}\PY{l+s+s2}{ ***}\PY{l+s+s2}{\PYZdq{}} \PY{o}{\PYZpc{}} \PY{n}{guess}\PY{p}{)}
                \PY{n}{factor\PYZus{}found} \PY{o}{=} \PY{k+kc}{True}
\end{Verbatim}
\end{tcolorbox}

    \begin{Verbatim}[commandchars=\\\{\}]

Attempt 1:
Register Reading: 000
Corresponding Phase: 0.000000
Result: r = 1

Attempt 2:
Register Reading: 100
Corresponding Phase: 0.500000
Result: r = 2
Guessed Factors: 3 and 1
*** Non-trivial factor found: 3 ***
    \end{Verbatim}

    \subsection{6. References}\label{references}

\begin{enumerate}
\def\labelenumi{\arabic{enumi}.}
\item
  Stephane Beauregard, \emph{Circuit for Shor's algorithm using 2n+3
  qubits,}
  \href{https://arxiv.org/abs/quant-ph/0205095}{arXiv:quant-ph/0205095}
\item
  M. Nielsen and I. Chuang, \emph{Quantum Computation and Quantum
  Information,} Cambridge Series on Information and the Natural Sciences
  (Cambridge University Press, Cambridge, 2000). (Page 633)
\end{enumerate}

    \begin{tcolorbox}[breakable, size=fbox, boxrule=1pt, pad at break*=1mm,colback=cellbackground, colframe=cellborder]
\prompt{In}{incolor}{24}{\boxspacing}
\begin{Verbatim}[commandchars=\\\{\}]
\PY{k+kn}{import} \PY{n+nn}{qiskit}
\PY{n}{qiskit}\PY{o}{.}\PY{n}{\PYZus{}\PYZus{}qiskit\PYZus{}version\PYZus{}\PYZus{}}
\end{Verbatim}
\end{tcolorbox}

            \begin{tcolorbox}[breakable, size=fbox, boxrule=.5pt, pad at break*=1mm, opacityfill=0]
\prompt{Out}{outcolor}{24}{\boxspacing}
\begin{Verbatim}[commandchars=\\\{\}]
\{'qiskit-terra': '0.14.2',
 'qiskit-aer': '0.5.2',
 'qiskit-ignis': '0.3.3',
 'qiskit-ibmq-provider': '0.7.2',
 'qiskit-aqua': '0.7.3',
 'qiskit': '0.19.6'\}
\end{Verbatim}
\end{tcolorbox}
        

    % Add a bibliography block to the postdoc
    
    
    
\end{document}
