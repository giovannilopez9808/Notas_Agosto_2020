\section*{Ejericio 30}
Mostrar que
\begin{equation*}
    {T^i}_j = -\psi^* \alpha^i \hat{P}_j c \psi
\end{equation*}
Se tiene que el lagrangiano es:
\begin{equation*}
    \mathcal{L} = \psi^* \left[i\hbar c \gamma^\mu \partial_\mu + m_0 c^2\right] \psi
\end{equation*}
y el tensor energía momento es
\begin{equation*}
    {T^\mu}_\nu = \frac{\partial \mathcal{L}}{\partial (\partial_\mu \psi)} \partial_\nu \psi + 
    \frac{\partial \mathcal{L}}{\partial (\partial_\mu \psi^*)} \partial_\nu \psi^* -\delta_{\nu}^\mu \mathcal{L}
\end{equation*}
calculando $\frac{\partial \mathcal{L}}{\partial (\partial_\mu \psi)} \partial_\nu \psi $:
\begin{equation*}
    \frac{\partial \mathcal{L}}{\partial (\partial_\mu \psi)} \partial_\nu \psi  = \psi^* ci \hbar \gamma^\mu \psi
\end{equation*}
como la lagrangiana no depende de $\partial_\mu \psi^*$, entonces 
\begin{equation*}
    \frac{\partial \mathcal{L}}{\partial (\partial_\mu \psi^*)} \partial_\nu \psi^*  =0
\end{equation*}
por lo tanto, el tensor energía momento es
\begin{equation*}
    {T^\mu}_\nu = \psi^* ci \hbar \gamma^\mu \delta_\nu\psi - \delta^\mu_\nu \left[\psi^* \left[i\hbar c \gamma^\mu \partial_\mu + m_0 c^2\right] \psi\right]
\end{equation*}
Tomando el caso de $\mu=i , \nu=j$, se tiene que:
\begin{equation*}
    {T^i}_j= \psi^* ci \hbar \alpha^i \partial_j\psi
\end{equation*}
Como el operador del momento lineal es $\hat{P}_j = -i\hbar \partial_j$, entonces
\begin{align*}
    {T^i}_j &= -\psi^* c \alpha^i (-i\hbar \partial_j) \psi \\
    &=\psi^* c\alpha^i \hat{P}_j \psi
\end{align*}
por lo tanto:
\begin{equation*}
    {T^i}_j = -\psi^* \alpha^i \hat{P}_j c \psi
\end{equation*}