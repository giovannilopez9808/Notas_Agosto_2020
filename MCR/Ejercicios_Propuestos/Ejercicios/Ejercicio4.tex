\section*{Ejercicio 4}
Muestre que:
    \begin{equation*}
        \partial_\alpha A^\alpha = \partial^\alpha A_\alpha 
    \end{equation*}
    Se tiene que:\\
    \begin{minipage}{0.5\linewidth}
        \begin{equation*}
            x_\alpha = g_{\alpha \beta} x^\beta 
        \end{equation*}
    \end{minipage}
    \begin{minipage}{0.5\linewidth}
        \begin{equation*}
            x^\alpha = g^{\alpha \beta}x_\beta
        \end{equation*}
    \end{minipage}
    por lo tanto: 
    \begin{align*}
        g^{\alpha \beta}\partial_\alpha &= \partial^\alpha \\
        g_{\alpha \beta}\partial^\alpha &= \partial_\alpha 
    \end{align*}
    calculando $\partial^\alpha A_\alpha$
    \begin{align*}
        \partial^\alpha A_\alpha &= \left(\frac{\partial A_0}{\partial x_0} \right)-\left(\frac{\partial A_1}{\partial x_1} \right)-\left(\frac{\partial A_2}{\partial x_2} \right)-\left(\frac{\partial A_3}{\partial x_3}\right)\\
        & = \frac{\partial A_0}{\partial x_0} - \nabla A
    \end{align*}
    por lo que se encuentra que:
    \begin{align*}
        A_0&=A^0 \\ A_1&=-A^1 \\ A_2&=-A^2 \\ A_3&=-A^3 \\
    \end{align*}
    \begin{align*}
        \partial^\alpha A_\alpha &= (g^{\alpha \beta}\partial_\beta)(g_{\alpha \gamma}A^\gamma)\\
        \delta^\beta_\gamma &= \partial_\beta A^\gamma
    \end{align*}