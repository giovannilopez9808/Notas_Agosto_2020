\documentclass[12pt,letterpaper]{article}
\usepackage{graphicx}
\usepackage{scrextend}
\usepackage{vmargin}
\usepackage{graphicx}
\usepackage{multirow}
\usepackage[utf8]{inputenc}
\usepackage[spanish]{babel}
\usepackage{multicol}
\usepackage{enumerate}
\usepackage{float}
\usepackage{amsmath, amsthm, amssymb, amsfonts}
\usepackage[usenames]{color}
\parindent=0mm
\pagestyle{empty}
\definecolor{miorange}{rgb}{0.91, 0.43, 0.0}
\begin{document}
\setmargins{2.5cm}      
{1.5cm}                     
{2cm}  
{24cm}                    
{10pt}                          
{1cm}                          
{0pt}                             
{2cm}
\begin{titlepage}
\begin{center}
\includegraphics[scale=0.40]{../../Logos/uanl.png} 
\hspace{2.5cm}
\includegraphics[scale=0.40]{../../Logos/fcfm.png}
\end{center}
\vspace{2cm}
\begin{center}
\textbf{
UNIVERSIDAD AUTÓNOMA DE NUEVO LEÓN\\
FACULTAD DE CIENCIAS
    FÍSICO MATEMÁTICAS}\\
\vspace*{2cm}
\begin{large}
\vspace{1cm}
\large{\textbf{Mécaninca Cuántica Relativista}}\\
\textbf{Problemas propuestos}\\
Francisco Baez\\
\end{large}
\end{center}
\end{titlepage}
\tableofcontents
\section*{Ejercicio 1}
Mostrar que:
\begin{equation*}
    U_\perp=\frac{u_i}{\gamma_v \left[1+\frac{\upsilon \cdot v}{c^2} \right]}
\end{equation*}
De las transformaciones:
\begin{align*}
    r_\parallel &= \gamma_v [{r_\parallel}'+v{t}']\\
    r_\perp &= {r_\perp}'\\
    t&= \gamma_v \left[ {t}'+\frac{\upsilon \cdot v}{c^2}\right]
\end{align*}
tomando los diferenciales:
\begin{align*}
    dr_\parallel &= \gamma_v \left[{dr}'_\parallel + vdt\right]\\
    dr_\perp &= {dr}'_perp\\
    dt&=\gamma_v \left[dt+ \frac{vdr}{c^2} \right]
\end{align*}
entonces:
\begin{align*}
    \frac{dr_\perp}{dt}&= \frac{{dr}'_\perp}{\gamma_v {dt}'\left[1+\frac{v}{c^2} \frac{{dr}'}{{dt}'} \right]}\\
    u_\perp&= \frac{{dr}'_\perp}{\gamma_v {dt}' \left[1+\frac{v}{c^2}\frac{{dr}'}{{dt}'} \right]}\\
    & = \frac{{u}'_\perp}{\gamma_v \left[1+\frac{v\cdot {u}'}{c^2} \right]}
\end{align*}
por lo tanto:
\begin{equation}
    u_\perp = \frac{{u}'_\perp}{\gamma_v \left[1+\frac{v\cdot {u}'}{c^2} \right]}
\end{equation}
\section*{Ejercicio 2}
Mostrar que
    \begin{align*}
        (U_d)_x &= \frac{-c\beta \sin({\theta}')}{\gamma_v (1-\beta^2 \cos({\theta}'))}\\
        (U_d)_z &= \frac{c\beta (1-\cos({\theta}'))}{1-\beta^2 \cos({\theta}')}
    \end{align*}
    Se sabe que por convención:
    \begin{align*}
        (U_d)_\perp = (U_d)_x &= \frac{({U}'_d)_\perp}{\gamma_v \left(1+\frac{v \cdot {u}'}{c^2} \right)}\\
        (U_d)_\parallel = (U_c)_z &= \frac{({U}'_d)_\parallel+v}{1+\frac{v \cdot {u}'_d}{c^2}}\\
    \end{align*}
    pero, del diagrama
    \begin{align*}
        ({U}'_d)_\perp &= {U}'_d \sin({\theta}')\\
        ({U}'_d)_\parallel&={U}'_d \cos({\theta}')
    \end{align*}
    por lo tanto:
    \begin{align*}
        (U_d)_x &= \frac{{U}'_d \sin({\theta}')}{\gamma_v \left(1+\frac{|v||u_d|\cos(\theta)}{c^2} \right)}\\
        (U_d)_z &= \frac{{U}'_d \cos({\theta}'_d)+v}{\gamma_v \left(1+\frac{|v||u_d|\cos(\theta)}{c^2} \right)}
    \end{align*}
    pero ${U}'_d =-v$
    \begin{align*}
        (U_d)_x &= \frac{-v \sin({\theta}')}{\gamma_v \left(1- \frac{v^2}{c^2}\cos({\theta}') \right)}\\
        & = \frac{-c\beta \sin({\theta}')}{\gamma_v (1-\beta^2 \cos({\theta}'))}\\
        (U_d)_z &= \frac{c\beta (1-\cos({\theta}'))}{1-\beta^2 \cos({\theta}')}
    \end{align*}
\section*{Ejercicio 3}
Mostrar que
    \begin{equation}
        u_c^2 = u_a^2 - \frac{\eta}{\gamma_a}
    \end{equation}
    \begin{align*}
        (U_c)_x &= \frac{c\beta \sin({\theta}')}{\gamma_v \left[1+\beta^2 \cos({\theta}') \right]}\\
        & \approx \frac{c \beta  \theta}{\gamma_v \left[1+\beta^2 \left(1- \frac{\theta^2}{2} \right) \right]}\\
        (U_c)_z & \approx \frac{c \beta (1+\left(1-\frac{\theta^2}{2} \right))}{1+\beta^2\left(1-\frac{\theta^2}{2}\right)}
    \end{align*}
    realizando el calculo para ángulos pequeños, tomando en cuenta que $\cos(\theta)=1-\theta^2/2$ y $\sin(\theta)=\theta$
    \begin{align*}
        (U_c)_z^2 &= \frac{c^2 \beta^2 \left(4-2\theta^2+\frac{\theta^4}{4}\right)}{(1+\beta^2)\left(1- \frac{\beta^2 \theta^2}{2(1+\beta^2)}\right)^2}\\
        & = \frac{c^2 \beta^2 (4-2\theta^2)}{(1+\beta^2)^2 \left(1- \frac{\beta^2 \theta^2 }{2(1+\beta^2)}\right)^2}\\
        & \approx \frac{c^2 \beta^2 (4-2\theta^2)}{(1+\beta^2)^2} \left(1+ \frac{\beta^2}{1+\beta^2}\theta^2 \right)\\
        & \approx \frac{4c^2 \beta^2}{(1+\beta^2)^2} - \frac{2c^2 \beta^2 \theta^2}{(1+\beta^2)^2} + \frac{4c^2 \beta^4 \theta^2}{(1+\beta^2)^3}\\
        & \approx u_a^2 - \frac{2c^2 \beta^2 \theta^2}{(1+\beta^2)^2} + \frac{4c^2 \beta^4 \theta^2}{(1+\beta^2)^3} 
    \end{align*}
    \begin{align*}
        (U_c)_x^2 &= \frac{c^2 \beta^2 \theta^2}{\gamma_v^2 (1+\beta^2)^2 \left(1-\frac{\beta^2 \theta^2}{2(1+\beta^2)} \right)^2}\\
        &\approx \frac{c^2 \beta^2 \theta^2 }{\gamma^2_v (1+\beta^2)} \left(1+\frac{\beta}{1+\beta^2}\theta^2 \right)\\
        &\approx \frac{c^2 \beta^2 \theta^2}{\gamma^2 (1+\beta^2)^2} + \frac{c^2 \beta^4 \theta^4}{\gamma^2 (1+\beta^2)^3}\\
        &\approx \frac{c^2 \beta^2 \theta^2}{\gamma^2 (1+\beta^2)^2}
    \end{align*}
    se tiene que:\\
    \begin{minipage}{0.5\linewidth}
        \begin{equation*}
            u_a = \frac{2\beta c}{1+\beta^2}
        \end{equation*}
    \end{minipage}
    \begin{minipage}{0.5\linewidth}
        \begin{equation*}
            \gamma_a = \frac{1+\beta^2}{1-\beta^2}
        \end{equation*}
    \end{minipage}
    por lo tanto:
    \begin{align*}
        u_c^2&= (u_c)_x^2+(u_c)_z^2 \\
        & = u_a^2 - \frac{2c^2 \beta^2 \theta^2}{(1+\beta^2)^2} + \frac{4c^2 \beta^4 \theta^2}{(1+\beta^2)^3}+\frac{c^2 \beta^2 \theta^2}{\gamma^2 (1+\beta^2)^2} \\
        & =u_a^2 + \frac{c^2 \beta^2 \theta^2}{(1+\beta^2)^2}\left(1-\beta^2-2 + \frac{4\beta^2}{1+\beta^2} \right)\\
        & = u_a^2 + \frac{c^2 \beta^2 \theta^2 }{(1+\beta^2)^2 }\left(\frac{1-2\beta^2-\beta^4}{1+\beta^2} \right) \\
        &= u_a^2 - \frac{c^2 \beta^2 \theta^2 }{(1+\beta^2)^2 }\left(\frac{(1-\beta^2)^2}{1+\beta^2} \right) \\
        & = u_a^2 -\frac{c^2 \beta^2 \theta^2}{1-\beta^2} \left(\frac{(1-\beta^2)^3}{(1+\beta^2)^3}\right)\\
        & = u_a^2 - \eta \frac{1}{\gamma^3_a}
    \end{align*}
\section*{Ejercicio 4}
Muestre que:
    \begin{equation*}
        \partial_\alpha A^\alpha = \partial^\alpha A_\alpha 
    \end{equation*}
    Se tiene que:\\
    \begin{minipage}{0.5\linewidth}
        \begin{equation*}
            x_\alpha = g_{\alpha \beta} x^\beta 
        \end{equation*}
    \end{minipage}
    \begin{minipage}{0.5\linewidth}
        \begin{equation*}
            x^\alpha = g^{\alpha \beta}x_\beta
        \end{equation*}
    \end{minipage}
    por lo tanto: 
    \begin{align*}
        g^{\alpha \beta}\partial_\alpha &= \partial^\alpha \\
        g_{\alpha \beta}\partial^\alpha &= \partial_\alpha 
    \end{align*}
    calculando $\partial^\alpha A_\alpha$
    \begin{align*}
        \partial^\alpha A_\alpha &= \left(\frac{\partial A_0}{\partial x_0} \right)-\left(\frac{\partial A_1}{\partial x_1} \right)-\left(\frac{\partial A_2}{\partial x_2} \right)-\left(\frac{\partial A_3}{\partial x_3}\right)\\
        & = \frac{\partial A_0}{\partial x_0} - \nabla A
    \end{align*}
    por lo que se encuentra que:
    \begin{align*}
        A_0&=A^0 \\ A_1&=-A^1 \\ A_2&=-A^2 \\ A_3&=-A^3 \\
    \end{align*}
    \begin{align*}
        \partial^\alpha A_\alpha &= (g^{\alpha \beta}\partial_\beta)(g_{\alpha \gamma}A^\gamma)\\
        \delta^\beta_\gamma &= \partial_\beta A^\gamma
    \end{align*}
\section*{Ejercicio 5}
Por verificar que:
    \begin{equation*}
        \partial^\alpha = \left(\frac{\partial}{\partial x_0}, -\nabla  \right)
    \end{equation*}
    Sea $A^\alpha$ un tensor covariante, entonces:
    \begin{align*}
        \partial^\alpha A_\alpha &= \left(\frac{\partial A_0}{\partial x_0} \right)-\left(\frac{\partial A_1}{\partial x_1} \right)-\left(\frac{\partial A_2}{\partial x_2} \right)-\left(\frac{\partial A_3}{\partial x_3} \right)\\
        &= \left(\frac{\partial}{\partial x_0},-\frac{\partial}{\partial x_1},-\frac{\partial}{\partial x_2},-\frac{\partial}{\partial x_3} \right) \cdot (A_0,A_1,A_2,A_3)\\
        &= \left(\frac{\partial}{\partial x_0},-\nabla \right) \cdot A_\alpha
    \end{align*}
    por lo tanto:
    \begin{equation*}
        \partial^\alpha = \left(\frac{\partial}{\partial x_0}, -\nabla  \right)
    \end{equation*}
\section*{Ejercicio 6}
Probar que las matrices $S_1^2,S_2^2,S_3^2$ son diagonales con -1 y que las matrices $K_1^2,K_2^2,K_3^2$ son diagonales con 1:
    Se tiene la matriz $S_1$ igual a:
    \begin{equation*}
        S_1 =\left( \begin{matrix}
            0 & 0 & 0 & 0 \\
            0 & 0 & 0 & 0 \\
            0 & 0 & 0 & -1 \\
            0 & 0 & 1 & 0 \\
        \end{matrix}\right)
    \end{equation*}
    entonces, calculando $S_1^2$, se tiene que:
    \begin{align*}
        S_1^2 &=\left( \begin{matrix}
            0 & 0 & 0 & 0 \\
            0 & 0 & 0 & 0 \\
            0 & 0 & 0 & -1 \\
            0 & 0 & 1 & 0 \\
        \end{matrix}\right)^2 \\
        & =\left( \begin{matrix}
            0 & 0 & 0 & 0 \\
            0 & 0 & 0 & 0 \\
            0 & 0 & 0 & -1 \\
            0 & 0 & 1 & 0 \\
        \end{matrix}\right)\left( \begin{matrix}
            0 & 0 & 0 & 0 \\
            0 & 0 & 0 & 0 \\
            0 & 0 & 0 & -1 \\
            0 & 0 & 1 & 0 \\
        \end{matrix}\right)\\
        & =\left( \begin{matrix}
            0 & 0 & 0 & 0 \\
            0 & 0 & 0 & 0 \\
            0 & 0 & -1 & 0 \\
            0 & 0 & 0 & -1 \\
        \end{matrix}\right)
    \end{align*}
    Se tiene la matriz $S_2$ igual a:
    \begin{equation*}
        S_2 =\left( \begin{matrix}
            0 & 0 & 0 & 0 \\
            0 & 0 & 0 & 0 \\
            0 & 0 & 0 & -1 \\
            0 & 0 & 1 & 0 \\
        \end{matrix}\right)
    \end{equation*}
    entonces, calculando $S_2^2$, se tiene que:
    \begin{align*}
        S_2^2 &=\left( \begin{matrix}
            0 & 0 & 0 & 0 \\
            0 & 0 & 0 & 1 \\
            0 & 0 & 0 & 0 \\
            0 & -1 & 0 & 0 \\
        \end{matrix}\right)^2 \\
        & =\left( \begin{matrix}
            0 & 0 & 0 & 0 \\
            0 & 0 & 0 & 1 \\
            0 & 0 & 0 & 0 \\
            0 & -1 & 0 & 0 \\
        \end{matrix}\right)\left( \begin{matrix}
            0 & 0 & 0 & 0 \\
            0 & 0 & 0 & 1 \\
            0 & 0 & 0 & 0 \\
            0 & -1 & 0 & 0 \\
        \end{matrix}\right)\\
        & =\left( \begin{matrix}
            0 & 0 & 0 & 0 \\
            0 & -1 & 0 & 0 \\
            0 & 0 & 0 & 0 \\
            0 & 0 & 0 & -1 \\
        \end{matrix}\right)
    \end{align*}
    Se tiene la matriz $S_3$ igual a:
    \begin{equation*}
        S_3 =\left( \begin{matrix}
            0 & 0 & 0 & 0 \\
            0 & 0 & -1 & 0 \\
            0 & 1 & 0 & -0 \\
            0 & 0 & 0 & 0 \\
        \end{matrix}\right)
    \end{equation*}
    entonces, calculando $S_3^2$, se tiene que:
    \begin{align*}
        S_3^2 &=\left( \begin{matrix}
            0 & 0 & 0 & 0 \\
            0 & 0 & -1 & 0 \\
            0 & 1 & 0 & 0 \\
            0 & 0 & 0 & 0 \\
        \end{matrix}\right)^2 \\
        & =\left( \begin{matrix}
            0 & 0 & 0 & 0 \\
            0 & 0 & -1 & 0 \\
            0 & 1 & 0 & 0 \\
            0 & 0 & 0 & 0 \\
        \end{matrix}\right)\left( \begin{matrix}
            0 & 0 & 0 & 0 \\
            0 & 0 & -1 & 0 \\
            0 & 1 & 0 & 0 \\
            0 & 0 & 0 & 0 \\
        \end{matrix}\right)\\
        & =\left( \begin{matrix}
            0 & 0 & 0 & 0 \\
            0 & -1 & 0 & 0 \\
            0 & 0 & -1 & 0 \\
            0 & 0 & 0 & 0 \\
        \end{matrix}\right)
    \end{align*}
    por lo tanto las matrices $S_\mu^2$ son diagonales con -1
    Se tiene la matriz $K_1$ igual a:
    \begin{equation*}
        S_1 =\left( \begin{matrix}
            0 & 1 & 0 & 0 \\
            1 & 0 & 0 & 0 \\
            0 & 0 & 0 & 0 \\
            0 & 0 & 0 & 0 \\
        \end{matrix}\right)
    \end{equation*}
    entonces, calculando $K_1^2$, se tiene que:
    \begin{align*}
        K_1^2 &=\left( \begin{matrix}
            0 & 1 & 0 & 0 \\
            1 & 0 & 0 & 0 \\
            0 & 0 & 0 & 0 \\
            0 & 0 & 0 & 0 \\
        \end{matrix}\right)^2 \\
        & =\left( \begin{matrix}
            0 & 1 & 0 & 0 \\
            1 & 0 & 0 & 0 \\
            0 & 0 & 0 & 0 \\
            0 & 0 & 0 & 0 \\
        \end{matrix}\right)\left( \begin{matrix}
            0 & 1 & 0 & 0 \\
            1 & 0 & 0 & 0 \\
            0 & 0 & 0 & 0 \\
            0 & 0 & 0 & 0 \\
        \end{matrix}\right)\\
        & =\left( \begin{matrix}
            1 & 0 & 0 & 0 \\
            0 & 1 & 0 & 0 \\
            0 & 0 & 0 & 0 \\
            0 & 0 & 0 & 0 \\
        \end{matrix}\right)
    \end{align*}
    Se tiene la matriz $K_2$ igual a:
    \begin{equation*}
        K_2 =\left( \begin{matrix}
            0 & 0 & 1 & 0 \\
            0 & 0 & 0 & 0 \\
            1 & 0 & 0 & 0 \\
            0 & 0 & 0 & 0 \\
        \end{matrix}\right)
    \end{equation*}
    entonces, calculando $k_2^2$, se tiene que:
    \begin{align*}
        K_2^2 &=\left( \begin{matrix}
            0 & 0 & 1 & 0 \\
            0 & 0 & 0 & 0 \\
            1 & 0 & 0 & 0 \\
            0 & 0 & 0 & 0 \\
        \end{matrix}\right)^2 \\
        & =\left( \begin{matrix}
            0 & 0 & 1 & 0 \\
            0 & 0 & 0 & 0 \\
            1 & 0 & 0 & 0 \\
            0 & 0 & 0 & 0 \\
        \end{matrix}\right)\left( \begin{matrix}
            0 & 0 & 1 & 0 \\
            0 & 0 & 0 & 0 \\
            1 & 0 & 0 & 0 \\
            0 & 0 & 0 & 0 \\
        \end{matrix}\right)\\
        & =\left( \begin{matrix}
            1 & 0 & 0 & 0 \\
            0 & 0 & 0 & 0 \\
            0 & 0 & 1 & 0 \\
            0 & 0 & 0 & 0 \\
        \end{matrix}\right)
    \end{align*}
    Se tiene la matriz $K_3$ igual a:
    \begin{equation*}
        K_3 =\left( \begin{matrix}
            0 & 0 & 0 & 1 \\
            0 & 0 & 0 & 0 \\
            0 & 0 & 0 & 0 \\
            1 & 0 & 0 & 0 \\
        \end{matrix}\right)
    \end{equation*}
    entonces, calculando $K_3^2$, se tiene que:
    \begin{align*}
        K_3^2 &=\left( \begin{matrix}
            0 & 0 & 0 & 1 \\
            0 & 0 & 0 & 0 \\
            0 & 0 & 0 & 0 \\
            1 & 0 & 0 & 0 \\
        \end{matrix}\right)^2 \\
        & =\left( \begin{matrix}
            0 & 0 & 0 & 1 \\
            0 & 0 & 0 & 0 \\
            0 & 0 & 0 & 0 \\
            1 & 0 & 0 & 0 \\
        \end{matrix}\right)\left( \begin{matrix}
            0 & 0 & 0 & 1 \\
            0 & 0 & 0 & 0 \\
            0 & 0 & 0 & 0 \\
            1 & 0 & 0 & 0 \\
        \end{matrix}\right)\\
        & =\left( \begin{matrix}
            1 & 0 & 0 & 0 \\
            0 & 0 & 0 & 0 \\
            0 & 0 & 0 & 0 \\
            0 & 0 & 0 & 1 \\
        \end{matrix}\right)
    \end{align*}
    por lo tanto las matrices $K_\mu^2$ son diagonales con 1
\section*{Ejercicio 7}
Mostrar que $F_{\alpha\gamma}=g_{\alpha\gamma}F^{\gamma \delta}g_{\delta \beta }$\\
    Se sabe que:\\
    \begin{minipage}{0.5\linewidth}
    \begin{equation*}
        F^{\gamma \delta} = \left( \begin{matrix}
            0 & -Ex & -Ey   & -Ez \\
            E_x &  0  & -B_z & B_y \\
            E_y & B_z & 0 &-B_x \\
            E_z & -B_y & B_x & 0  
        \end{matrix}\right)
    \end{equation*}
\end{minipage}
\begin{minipage}{0.5\linewidth}
\begin{equation*}
    g_{\alpha \gamma} = g_{\delta \beta} =\left(\begin{matrix}
        1 & 0 & 0 & 0\\
        0 & -1 & 0 & 0\\
        0 & 0 & -1 & 0\\
        0 & 0 & 0 & -1\\
    \end{matrix}\right)
\end{equation*}
\end{minipage}
realizando la multiplicacion $F^{\gamma \delta}g_{\delta \beta }$
\begin{align*}
    F^{\gamma \delta}g_{\delta \beta }&= \left( \begin{matrix}
        0 & -Ex & -Ey   & -Ez \\
        E_x &  0  & -B_z & B_y \\
        E_y & B_z & 0 &-B_x \\
        E_z & -B_y & B_x & 0  
    \end{matrix}\right)\left(\begin{matrix}
        1 & 0 & 0 & 0\\
        0 & -1 & 0 & 0\\
        0 & 0 & -1 & 0\\
        0 & 0 & 0 & -1\\
    \end{matrix}\right)\\
    & =\left( \begin{matrix}
        0 & Ex & Ey   & Ez \\
        E_x &  0  & B_z & -B_y \\
        E_y & -B_z & 0 &B_x \\
        E_z & B_y & -B_x & 0  
    \end{matrix}\right)
\end{align*}
por lo tanto:
\begin{equation*}
    F^{\gamma}_\beta = \left( \begin{matrix}
        0 & Ex & Ey   & Ez \\
        E_x &  0  & B_z & -B_y \\
        E_y & -B_z & 0 &B_x \\
        E_z & B_y & -B_x & 0  
    \end{matrix}\right)
\end{equation*}
realizando la multiplicacion $g_{\alpha\gamma}F^{\gamma}_\beta$ se obtiene que:
\begin{align*}
    g_{\alpha\gamma}F^{\gamma}_\beta &= \left(\begin{matrix}
        1 & 0 & 0 & 0\\
        0 & -1 & 0 & 0\\
        0 & 0 & -1 & 0\\
        0 & 0 & 0 & -1\\
    \end{matrix}\right)\left( \begin{matrix}
        0 & Ex & Ey   & Ez \\
        E_x &  0  & B_z & -B_y \\
        E_y & -B_z & 0 &B_x \\
        E_z & B_y & -B_x & 0  
    \end{matrix}\right) \\
    & = \left( \begin{matrix}
        0 & Ex & Ey   & Ez \\
        -E_x &  0  & -B_z & B_y \\
        -E_y & B_z & 0 &-B_x \\
        -E_z & -B_y & B_x & 0  
    \end{matrix}\right)
\end{align*}
por lo tanto:
\begin{equation*}
    F_{\alpha \beta} = \left( \begin{matrix}
        0 & Ex & Ey   & Ez \\
        -E_x &  0  & -B_z & B_y \\
        -E_y & B_z & 0 &-B_x \\
        -E_z & -B_y & B_x & 0  
    \end{matrix}\right)
\end{equation*}
\section*{Ejercicio 8}
Compruebe la forma de L, que cumple $L^Tg=-gL$ donde L tiene diagonal de ceros y g es la representación matricial de $g_{\mu \nu \rho}$
    Se tiene que:
    \begin{equation*}
        g=diag(1,-1,-1,-1)
    \end{equation*}
    y que
    \begin{equation*}
        g^T=g=g^{-1}
    \end{equation*}
    por lo tanto:
    \begin{align*}
        c^T&= (gL)^T \\
        &=L^T g \\
        &=-gL\\
        &=-c
    \end{align*}
    por lo tanto:
    \begin{equation*}
        c_{ij}=-c_{ji}
    \end{equation*}
    si $i=j$, entonces $c_{ii}=0$, por lo tanto:
    \begin{equation*}
        gL=C=\left(\begin{matrix}
            0 & C_{12} & C_{13} & C_{14} \\
            -C_{12} & 0 & C_{23} & C_{24} \\
            -C_{13} & -C_{23} & 0 & C_{34} \\
            -C_{14} & -C_{24} & -C_{34} & 0 \\
        \end{matrix}\right)
    \end{equation*}
    realizando la operación $gc=ggL$, se tiene que:
    \begin{align*}
        gc&=g(gL)\\
        &=(gg)L\\
        &=L
    \end{align*}
    por lo tanto $gc=L$
\section*{Ejercicio 9}
Formule la matriz de rotación respecto a $\hat{z}$ por medio de la transformación de Lorentz partiendo de la invarianza de $S^2$ se obtiene la 
forma (dependiente de 6 parámetros) de L, tal que $A=e^L$ es la transformación de Lorentz.\\
Se tiene la base, $S_\mu , K_\mu$, donde $L=-\vec{\omega}\cdot \vec{s}- \vec{\xi}\cdot \vec{k}$, para rotar con respecto $\hat{z}$, se tiene que cumplir que:
$\vec{\xi}=0 , \vec{\omega}=\omega_z \hat{z}$, entonces:
\begin{equation*}
    L=-\vec{\omega}\cdot \vec{s}=-\omega_z s_3
\end{equation*}
por lo tanto:
\begin{align*}
    A&=e^L \\
    &=e^{-\omega s_3}\\
    &=\sum_{i=0}^\infty \frac{(-1)^i(\omega s_3)^i}{i!} 
\end{align*}
donde se cumple que: 
\begin{align*}
    s^3_3&=-s_3\\
s^4_3&=-s_3^2\\
s_3^5&=s_3
\end{align*}
entonces:
\begin{align*}
    A&=1-\omega s_3 + \frac{\omega^2}{2!}s_3^2 + \frac{\omega^3}{3!}s_3 - \frac{\omega^4}{4!}s_3^2\\
    &=(1+s_3^2)-s_3^2\left[1-\frac{\omega^2}{2!}+\frac{\omega^4}{4!}-\dots \right] - s_3 \left[\omega - \frac{\omega^3}{3!}+\frac{\omega^5}{5!}-\dots \right]\\
    &= \left(\begin{matrix}
        1 & 0 & 0 & 0 \\
        0 & 0 & 0 & 0 \\
        0 & 0 & 0 & 0 \\
        0 & 0 & 0 & 1 \\
    \end{matrix}\right) - \left(\begin{matrix}
        0 & 0 & 0 & 0 \\
        0 &-1 & 0 & 0 \\
        0 & 0 & -1 & 0 \\
        0 & 0 & 0 & 0 \\
    \end{matrix}\right) cos(\omega) - \left(\begin{matrix}
        0 & 0 & 0 & 0 \\
        0 & 0 & -1 & 0 \\
        0 & 1 & 0 & 0 \\
        0 & 0 & 0 & 0 \\
    \end{matrix}\right) sen(\omega)
\end{align*}
entonces:
\begin{equation*}
    A=\left(\begin{matrix}
        1 & 0 & 0 & 0 \\
        0 & cos(\omega) & sen(\omega) & 0 \\
        0 & -sen(\omega) & cos(\omega) & 0 \\
        0 & 0 & 0 & 1 \\
    \end{matrix}\right) 
\end{equation*}
\section*{Ejercicio 10}
Determina la energía threshold para las siguientes reacciones, asumiendo que el proton blanco está
en reposo. Consulta en la página de Partice Data Group las masas de las partıículas.
\begin{itemize}
    \item $p+p \rightarrow p+p+\pi^0$
    \item $p+p \rightarrow p+p+\pi^++\pi^-$
    \item $\pi^-+p \rightarrow p+\bar{p}+n$
    \item $\pi^-+p \rightarrow K^0+\sum^0$
\end{itemize}
\section*{Ejercicio 11}
Una partícula A en reposo, decae en 2 partículas B y C $(A\rightarrow B+C)$. Mostrar que la energía
de la partícula que emergió es 
\begin{equation*}
    E_B= \frac{m_A^2+m_B^2-m_C^2}{2m_A}c^2
\end{equation*}
\section*{Ejercicio 12}
En una dispersión de 2 cuerpos $A+B \rightarrow C+D$, es conveniente introducir las variables de Mandelstam 
\begin{align*}
    s&=(p_A+p_B)^2/c^2 \\
    t&=(p_A-p_C)^2/c^2 \\
    u&= (p_A-p_D)^2/c^2
\end{align*}
\begin{enumerate}
    \item Mostrar que $s+t+u=m_A^2+m_B^2+m_c^2+m_D^2$\\
    Realizando la suma de $s+t+u$, se tiene que:
    \begin{align*}
        s+t+u&= \frac{(p_A+p_B)^2+(p_A-p_C)^2+(p_A-p_D)^2}{c^2}\\
        &=\frac{p_A^2+2p_Ap_B+p_B^2+p_A^2-2p_Ap_C+p_C^2+p_A^2-2p_Ap_D+p_D^2}{c^2}\\
        &=\frac{3p_A^2+2p_A(p_B-p_C-p_D)+p_B^2+p_C^2+p_D^2}{c^2}\\
        &=\frac{3p_A^2-2p_A^2+p_B^2+p_C^2+p_D^2}{c^2}\\
        &=\frac{p_A^2+p_B^2+p_C^2+p_D^2}{c^2}\\
        &=m_A^2+m_B^2+m_C^2+m_D^2
    \end{align*}
    \item Mostrar que la energía de centro de masa de A es $E_A^{CM}=(s+m_A^2-m_B^2)c^2/2\sqrt{s}$
    \item Mostrar que la energía de A en el sistema de laboratorio (B en reposo) es $E_A^{LAB}=(s-m_A^2-m_B^2)c^2/2m_B$
\end{enumerate}
\section*{Ejercicio 13}
Mostrar que:
\begin{align*}
    {\rho}'_+&=\frac{e|E_p|}{m_0c^2} \psi_+^* \psi_+\\
    {\rho}'_-&=\frac{e|E_p|}{m_0c^2} \psi_-^* \psi_-\\
\end{align*}
considerando que:
\begin{align*}
    \psi_+ &= A_+ exp\left[\frac{i}{\hbar} \left(\vec{p}\cdot \vec{x} - |E_pt| \right)\right]\\
    \psi_- &= A_- exp\left[\frac{i}{\hbar} \left(\vec{p}\cdot \vec{x} + |E_pt| \right)\right]\\
\end{align*}
 Sea 
 \begin{equation*}
     {\rho}' =\frac{i\hbar e}{2m_0 c^2} \left[\psi^* \frac{\partial }{\partial t} \psi -\psi \frac{\partial }{\partial t} \psi^*\right]
 \end{equation*}
 Usando a $\psi_+$ se calcularan las derivadas partiales 
 \begin{align*}
    \frac{\partial }{\partial t} \psi_+= -\frac{|E_p|i}{\hbar}\psi_+ &\qquad \frac{\partial }{\partial t} \psi_+^*= \frac{|E_p|i}{\hbar}\psi_+^* \\
    \psi_+^* \frac{\partial }{\partial t} \psi_+= -\frac{|E_p|i}{\hbar}\psi_+ \psi_+^*&\qquad \psi_+\frac{\partial }{\partial t} \psi_+^*= \frac{|E_p|i}{\hbar}\psi_+^*\psi_+ \\
 \end{align*}
entonces:
\begin{align*}
    \psi^* \frac{\partial }{\partial t} \psi -\psi \frac{\partial }{\partial t} \psi^*&=-\frac{2|E_p|i}{\hbar} \psi_+^*\psi_+\\
    \frac{i\hbar e}{2m_0 c^2} \left[\psi^* \frac{\partial }{\partial t} \psi -\psi \frac{\partial }{\partial t} \psi^*\right]&=\frac{e|E_p|}{m_0c^2} \psi_+^* \psi_+
\end{align*}
por lo tanto:
\begin{equation*}
    {\rho}'_+=\frac{e|E_p|}{m_0c^2} \psi_+^* \psi_+
\end{equation*}
Usando a $\psi_-$ se calcularan las derivadas partiales 
\begin{align*}
   \frac{\partial }{\partial t} \psi_-= \frac{|E_p|i}{\hbar}\psi_- &\qquad \frac{\partial }{\partial t} \psi_-^*= -\frac{|E_p|i}{\hbar}\psi_-^* \\
   \psi_-^* \frac{\partial }{\partial t} \psi_-= \frac{|E_p|i}{\hbar}\psi_- \psi_-^*&\qquad \psi_-\frac{\partial }{\partial t} \psi_-^*= -\frac{|E_p|i}{\hbar}\psi_-^*\psi_- \\
\end{align*}
entonces:
\begin{align*}
   \psi^* \frac{\partial }{\partial t} \psi -\psi \frac{\partial }{\partial t} \psi^*&=\frac{2|E_p|i}{\hbar} \psi_-^*\psi_-\\
   \frac{i\hbar e}{2m_0 c^2} \left[\psi^* \frac{\partial }{\partial t} \psi -\psi \frac{\partial }{\partial t} \psi^*\right]&=-\frac{e|E_p|}{m_0c^2} \psi_-^* \psi_-
\end{align*}
por lo tanto:
\begin{equation*}
   {\rho}'_-=-\frac{e|E_p|}{m_0c^2} \psi_-^* \psi_-
\end{equation*}
\section*{Ejercicio 14}
Usar la ecuación de Euler-Lagrande para $\psi^*$ y obtener la ecuación de Klein Gordon para $\psi$.\\
Sea
\begin{equation*}
    \frac{\mathcal{L}}{\partial \psi_0} - \frac{\partial}{\partial x_\beta }\left[\frac{\partial \mathcal{L}}{\partial \left(\frac{\partial \psi_\sigma}{\partial x_\mu} \right)} \right]=0
\end{equation*}
y la densidad lagrangiana
\begin{equation*}
    \mathcal{L}\left(\psi,\psi^*, \frac{\partial \psi}{\partial x^\beta},\frac{\partial \psi^*}{\partial x^\beta} \right)= \frac{\hbar^2}{2m_0} \left[g^{\beta \nu}\frac{\partial \psi^*}{\partial x^\mu}\frac{\partial \psi}{\partial x^\nu}-\frac{m_0^2c^2}{\hbar^2}\psi^*\psi \right]
\end{equation*}
Para el campo $\psi_\sigma=\psi^*$, calculando $\frac{\mathcal{L}}{\partial \psi^*}$
\begin{equation*}
    \frac{\partial \mathcal{L}}{\partial \psi^*}=\frac{\hbar^2}{2m_0}\left[- \frac{m_0^2c^2}{\hbar}\psi \right]
\end{equation*}
calculando $\frac{\partial \mathcal{L}}{\partial \left(\frac{\partial \psi^*}{\partial x_\mu}\right)}$
\begin{align*}
    \frac{\partial \mathcal{L}}{\partial \left(\frac{\partial \psi^*}{\partial x_\mu}\right)}&= \frac{\hbar^2}{2m_0}\frac{\partial}{\partial \left(\frac{\partial \psi^*}{\partial x_\mu}\right)}\left(g^{\mu \nu}\frac{\partial \psi}{\partial x^\nu} \frac{\partial \psi^*}{\partial x^\mu}\right)\\
    &=\frac{\hbar^2}{2m_0}\left(g^{\mu \nu} \frac{\partial \psi}{\partial x^\nu} \delta_\beta^\mu\right) \\
    &=\frac{\hbar^2}{2m_0}\left(g_{\mu \nu} \frac{\partial \psi}{\partial x_\nu} \delta_\beta^\mu\right) \\
    &=\frac{\hbar^2}{2m_0}\left(g_{\beta \nu} \frac{\partial \psi}{\partial x_\nu}\right)
\end{align*}
calculando $\frac{\partial}{\partial x_\beta}\left(\frac{\partial \mathcal{L}}{\partial \left(\frac{\partial \psi^*}{\partial x_\mu}\right)}\right)$:
\begin{align*}
    \frac{\partial}{\partial x_\beta}\left(\frac{\partial \mathcal{L}}{\partial \left(\frac{\partial \psi^*}{\partial x_\mu}\right)}\right)&= \frac{\partial}{\partial x_\beta} \left(\frac{\hbar^2}{2m_0} \left(g_{\beta \nu} \frac{\partial \psi}{\partial x_\nu}\right) \right)\\
    &=\frac{\hbar^2}{2m_0}\left(\partial^\beta \left(g_{\beta \nu} \frac{\partial \psi}{\partial x_\nu} \right)\right)\\
    &=\frac{\hbar^2}{2m_0}\partial^\beta \left(g_{\beta \nu} \partial^\mu \psi \right)\\
    &=\frac{\hbar^2}{2m_0} \partial_\mu \partial^\mu \psi
\end{align*}
por lo tanto:
\begin{equation*}
    \left[\frac{m_0^2c^2}{\hbar^2}+\partial_\mu \partial^\mu \right]\psi=0
\end{equation*}
\section*{Ejercicio 15}
Mostrar que el tensor de energía momento para la densidad lagrangiana es 
\begin{equation*}
    T_\mu^\nu = \frac{\hbar^2}{2m_0} \left[g^{\sigma \nu} \frac{\partial \psi^*}{\partial x^\sigma}
    \frac{\partial \psi}{\partial x^\mu}+g^{\sigma \nu} \frac{\partial \psi}{\partial x^\sigma}
    \frac{\partial \psi^*}{\partial x^\mu}-\left(g^{\sigma \rho} \frac{\partial \psi^*}{\partial x^\sigma}
    \frac{\partial \psi}{\partial x^\rho}-\frac{m_0^2 c^2}{\hbar}\psi^*\psi\right)g_\mu^\nu\right]
\end{equation*}
Sea el ternsor energía momento definido por:
\begin{equation*}
    T_\mu^\nu = \sum\limits_{\sigma} \frac{\partial \psi_\sigma}{\partial x^\mu} \frac{\partial \mathcal{L}}{\partial \left[\partial \psi_\sigma / \partial x^\nu\right]}- \mathcal{L} g_\mu^\nu 
\end{equation*}
Utilizando la siguiente densidad lagrangiana
\begin{equation*}
    \mathcal{L}\left(\psi,\psi^*, \frac{\partial \psi}{\partial x^\beta},\frac{\partial \psi^*}{\partial x^\alpha} \right)= \frac{\hbar^2}{2m_0} \left[g^{\beta \nu}\frac{\partial \psi^*}{\partial x^\alpha}\frac{\partial \psi}{\partial x^\beta}-\frac{m_0^2c^2}{\hbar^2}\psi^*\psi \right]
\end{equation*}
Calculando $\partial \mathcal{L}/ \partial (\partial \psi / \partial x^\mu )$
\begin{align*}
    \frac{\partial \mathcal{L}}{\partial (\partial \psi / \partial x^\mu)} &= \frac{\hbar^2}{2m_0}\left(g^{\sigma \beta} \frac{\partial \psi^*}{\partial x^{\sigma}} \delta_\nu^\beta\right)\\
    & =\frac{\hbar^2}{2m_0}\left( g^{\sigma \nu } \frac{\partial \psi^*}{\partial x^{\sigma}}\right)\\
\end{align*}
Calculando $\partial \mathcal{L}/ \partial (\partial \psi^* / \partial x^\mu )$
\begin{align*}
    \frac{\partial \mathcal{L}}{\partial (\partial \psi^* / \partial x^\mu)} &= \frac{\hbar^2}{2m_0}\left(g^{\sigma \beta} \frac{\partial \psi}{\partial x^{\sigma}} \delta_\nu^\beta\right)\\
    & =\frac{\hbar^2}{2m_0}\left( g^{\sigma \nu } \frac{\partial \psi}{\partial x^{\sigma}}\right)\\
\end{align*}
    \begin{equation*}
        T_\mu^\nu = \frac{\hbar^2}{2m_0} \left[g^{\sigma \nu} \frac{\partial \psi^*}{\partial x^\sigma}
        \frac{\partial \psi}{\partial x^\mu}+g^{\sigma \nu} \frac{\partial \psi}{\partial x^\sigma}
        \frac{\partial \psi^*}{\partial x^\mu}-\left(g^{\sigma \rho} \frac{\partial \psi^*}{\partial x^\sigma}
        \frac{\partial \psi}{\partial x^\rho}-\frac{m_0^2 c^2}{\hbar}\psi^*\psi\right)g_\mu^\nu\right]
    \end{equation*}
\section*{Ejercicio 16}
Mostrar que:
\begin{equation*}
    T_0^0 = \frac{\hbar^2}{2m_0} \left[\frac{1}{c^2} \frac{\partial \psi^*}{\partial t}\frac{\partial \psi}{\partial t}+ \left(\nabla \psi^*\right)\cdot \left(\nabla \psi \right)+ \frac{m_0^2c^2}{\hbar^2} \psi^* \psi \right]  
\end{equation*}
Sea el tensor energía momento:
\begin{equation*}
    T_\mu^\nu = \frac{\hbar^2}{2m_0} \left[g^{\sigma \nu} \frac{\partial \psi^*}{\partial x^\sigma}
    \frac{\partial \psi}{\partial x^\mu}+g^{\sigma \nu} \frac{\partial \psi}{\partial x^\sigma}
    \frac{\partial \psi^*}{\partial x^\mu}-\left(g^{\sigma \rho} \frac{\partial \psi^*}{\partial x^\sigma}
    \frac{\partial \psi}{\partial x^\rho}-\frac{m_0^2 c^2}{\hbar}\psi^*\psi\right)g_\mu^\nu\right]
\end{equation*}
Tomando el caso de $\mu=0,\nu=0$ y el tensor metrico $g_{\mu\nu}$ tal que
\begin{equation*}
    g^{\mu\nu}= \left[\begin{matrix}
        \frac{1}{c^2} & 0 & 0 & 0 \\
        0 & -1 & 0 & 0 \\
        0 &  0& -1 & 0 \\
        0 & 0 & 0 & -1 \\
    \end{matrix}\right]
\end{equation*}
entonces:% \frac{\partial}{\partial}
\begin{align*}
    T_0^0   & =\frac{\hbar^2}{2m_0}\left[\frac{1}{c^2} \frac{\partial \psi^*}{\partial t}\frac{\partial \psi}{\partial t}+\frac{1}{c^2} \frac{\partial \psi^*}{\partial t}\frac{\partial \psi}{\partial t}-\frac{1}{c^2} \frac{\partial \psi^*}{\partial t}\frac{\partial \psi}{\partial t}+\frac{\partial \psi^*}{\partial x^1} \frac{\partial \psi}{\partial x^1}+\frac{\partial \psi^*}{\partial x^2} \frac{\partial \psi}{\partial x^2}+\frac{\partial \psi^*}{\partial x^3} \frac{\partial \psi}{\partial x^3}+\frac{m_0^2 c^2}{\hbar^2}\psi^* \psi \right]\\
            & =\frac{\hbar^2}{2m_0}\left[ \frac{1}{c^2} \frac{\partial \psi^*}{\partial t}\frac{\partial \psi}{\partial t} ,\left(\frac{\partial \psi^*}{\partial x^1},\frac{\partial \psi^*}{\partial x^2},\frac{\partial \psi^*}{\partial x^3}\right) \cdot \left(\frac{\partial \psi}{\partial x^1},\frac{\partial \psi}{\partial x^2},\frac{\partial \psi}{\partial x^3}\right), \frac{m_0^2 c^2}{\hbar^2} \psi^* \psi\right]\\
            & =\frac{\hbar^2}{2m_0}\left[\frac{1}{c^2} \frac{\partial \psi^*}{\partial t}\frac{\partial \psi}{\partial t} +\left(\nabla \psi^*\right) \cdot \left(\nabla \psi\right)+ \frac{m_0^2 c^2}{\hbar^2} \psi^* \psi\right]\\
\end{align*}
por lo tanto:
\begin{equation*}
    T_0^0 = \frac{\hbar^2}{2m_0} \left[\frac{1}{c^2} \frac{\partial \psi^*}{\partial t}\frac{\partial \psi}{\partial t}+ \left(\nabla \psi^*\right)\cdot \left(\nabla \psi \right)+ \frac{m_0^2c^2}{\hbar^2} \psi^* \psi \right]  
\end{equation*}
\section*{Ejercicio 17}
Mostras que
\begin{itemize}
    \item $H(+)=E_{pn}$
    \item $H(-)=E_{pn}$
\end{itemize}
\section*{Ejercicio 18}
Obtener la constante de la función de onda para $E=-E_p$
\section*{Ejercicio 19}
Mostrar que:
\begin{equation*}
    {\vec{J}}' = -\frac{i\hbar e}{2m_0} \left[\psi^* \nabla \psi - \psi \nabla \psi^*  \right] - \frac{e^2}{m_0c} \vec{A} \psi \psi^*
\end{equation*}
\section*{Ejercicio 20}
Sea 
\begin{equation*}
    \mathcal{L } = -\frac{1}{4}F_{\mu\nu}F^{\mu\nu} \qquad F_{\mu \nu}= \partial_\mu A_\nu - \partial_\nu A_\mu
\end{equation*}
\begin{itemize}
    \item Determinar las ecuaciones que satisface el campo $A_\nu$\\
    Se tiene que la ecuación de Euler-Lagrande para campos es la siguiente:
\begin{equation}
  \nabla_\nu \left( \frac{\partial \mathcal{L}}{\partial (\nabla_\nu A_\mu)}\right) = \frac{\partial \mathcal{L}}{\partial A_\mu}
  \label{eq:euler}
\end{equation}
Como la ecuación \ref{eq:l} no depende del campo $A_\mu$, entonces:
\begin{equation*}
  \frac{\partial \mathcal{L}}{\partial A_\mu} = 0
\end{equation*}
por lo tanto, la ecuación \ref{eq:euler} se escribe de la siguiente manera:
\begin{equation}
  \nabla_\nu \left( \frac{\partial \mathcal{L}}{\partial (\nabla_\nu A_\mu)}\right) = 0
  \label{eq:euler2}
\end{equation}
Calculando $ \frac{\partial \mathcal{L}}{\partial (\nabla_\nu A_\mu)}$ se tiene que :
  \begin{align*}
    \frac{\partial \mathcal{L}}{\partial (\nabla_\nu A_\mu)} &=   \frac{\partial }{\partial (\nabla_\nu A_\mu)} \left(\frac{1}{4}F_{\mu \nu} F^{\mu \nu} \right)\\
    &= \frac{\partial }{\partial (\nabla_\nu A_\mu)}(\left(\partial_\mu A_\nu - \partial_\nu A_\mu\right) F^{\mu \nu})\\
    & = F^{\mu \nu }
  \end{align*}
  entonces
  \begin{align*}
      \nabla_\nu \left( \frac{\partial \mathcal{L}}{\partial (\nabla_\nu A_\mu)}\right)=
      \nabla_\nu F^{\mu \nu} 
  \end{align*}
  por lo tanto el campo $A_\nu$ debe cumplir la siguiente ecuación:
  \begin{equation*}
    \nabla_\nu F^{\mu \nu} =0
  \end{equation*} 
    \item Determinar el tensor $T^\mu_\nu T^{\mu\nu}$\\
    Se tiene que 
    \begin{equation*}
      {T^\mu}_\nu =\frac{\partial \mathcal{L}}{\partial(\partial_\mu A_\nu)} - g^{\mu}_\nu \mathcal{L} \qquad
      T^{\mu\nu} =\frac{\partial \mathcal{L}}{\partial(\partial_\mu A_\nu)} - g^{\mu\nu} \mathcal{L}
    \end{equation*}
    Calculando ${T^{\mu}}_\nu(T^{\mu \nu})$
    \begin{align*}
      {T^{\mu}}_\nu(T^{\mu \nu}) =& \left(\frac{\partial \mathcal{L}}{\partial(\partial_\mu A_\nu)} - g^{\mu}_\nu \mathcal{L}\right)\left(
        \frac{\partial \mathcal{L}}{\partial(\partial_\mu A_\nu)} - g^{\mu\nu} \right)\\
        =&({F^\mu}_\nu \partial_\nu A_\mu )(F^{\mu \nu} \partial_\nu A_\mu) - ({F^\mu}_\nu \partial_\nu A_\mu )(g^{\mu \nu} \mathcal{L})
        \\ &-({g^{\mu}}_\nu\mathcal{L})(F^{\mu \nu} \partial_\nu A_\mu)  + ({g^\mu}_\nu \mathcal{L})(g^{\mu \nu}\mathcal{L}) \\
         =& ({F^\mu}_\nu \partial_\nu A_\mu )(F^{\mu \nu} \partial_\nu A_\mu) 
    \end{align*}
    por lo tanto
    \begin{equation*}
      {T^{\mu}}_\nu(T^{\mu \nu})=({F^\mu}_\nu \partial_\nu A_\mu )(F^{\mu \nu} \partial_\nu A_\mu) 
    \end{equation*}
\end{itemize}
\section*{Ejercicio 21}
\begin{itemize}
    \item Usando la ecuación de Euler-Lagrange para campos, determinar que $\psi$ satisface 
\begin{equation*}
    \left(p^\mu-\frac{e}{c}A^\mu\right) \left(p_\mu - \frac{e}{c} A_\mu \right) \psi = m_0^2c^2 \psi
\end{equation*}
    \item Mostrar que la ecuación para $A_\mu$
    \begin{equation*}
        \partial^\mu F_{\mu \nu} = J_\nu = \frac{ie\hbar}{2m_0} \left[\begin{matrix}
            \psi^* \left[\partial_\nu +\frac{ie}{\hbar c} A_\nu \right]\psi \\
            -\psi \left[\partial_\nu - \frac{ie}{\hbar c}A_\nu\right]\psi^*
        \end{matrix}
        \right]
    \end{equation*}
\end{itemize}

\section*{Ejercicio 22}
Mostrar que:
\begin{equation*}
    \mathcal{L}=-\frac{1}{4} F_{\mu\nu}F^{\mu\nu}
\end{equation*}
es invariante ante la transformacion 
\begin{equation*}
    {A}'_\mu = A_\mu +\partial_\mu \xi(x)
\end{equation*}
Se tiene que:
\begin{equation*}
    {F}'_{\mu\nu}= \partial_\mu {A}'_\nu -\partial_\nu {A}'_\mu
\end{equation*}
entonces:
\begin{align*}
    {F}'_{\mu\nu}&= \partial_\mu {A}'_\nu -\partial_\nu {A}'_\mu\\
    &=\partial_\mu\left[A_\nu +\partial_\nu \xi(x)\right] - \partial_\nu  \left[A_\mu+\partial_\mu \xi(x)\right]\\
    &=\partial_\mu A_\nu +\partial_\mu \partial_\nu \xi(x) - \partial_\nu A_\mu -\partial_\nu \partial_\mu \xi(x)\\
    &=\partial_\mu A_\nu +\partial_\mu \partial_\nu \xi(x) - \partial_\nu A_\mu -\partial_\mu \partial_\nu \xi(x)\\
    &=\partial_\mu A_\nu - \partial_\nu A_\mu \\
    &= F_{\mu \nu}
\end{align*}
\begin{align*}
    {F}'^{\mu\nu}&= \partial^\mu {A}'^\nu -\partial^\nu {A}'^\mu\\
    &=\partial^\mu\left[A^\nu +\partial^\nu \xi(x)\right] - \partial^\nu  \left[A^\mu+\partial^\mu \xi(x)\right]\\
    &=\partial^\mu A^\nu +\partial^\mu \partial^\nu \xi(x) - \partial^\nu A^\mu -\partial^\nu \partial^\mu \xi(x)\\
    &=\partial^\mu A^\nu +\partial^\mu \partial^\nu \xi(x) - \partial^\nu A^\mu -\partial^\mu \partial^\nu \xi(x)\\
    &=\partial^\mu A^\nu - \partial^\nu A^\mu \\
    &= F^{\mu \nu}
\end{align*}
por lo tanto:
\begin{align*}
    -\frac{1}{4}{F}'_{\mu\nu}{F}'^{\mu\nu}&=-\frac{1}{4}F_{\mu \nu}F^{\mu \nu}\\
    {\mathcal{L}}' &= \mathcal{L}
\end{align*}
por lo tanto $\mathcal{L}$ es invariante ante la transformacion.
\section*{Ejercicio 23}
Mostrar que 
\begin{equation*}
    \left[i\hbar \frac{\partial}{\partial t} -eA_0\right]^2 \psi \approx \left[-i e \hbar \frac{\partial }{\partial t} A_0 \varphi+2i\hbar m_0c^2 \frac{\partial}{\partial t}\varphi-2eA_0m_0c^2\varphi +m_0^2c^4 \varphi \right]e^{\frac{-im_0c^2t}{\hbar}}
\end{equation*}
se han omitido 
\begin{equation*}
    (i\hbar A_0 \frac{\partial}{\partial t}\varphi) << A_0 m_0c^2 |\varphi| \qquad |A_0\varphi_e| << m_0c^2 |\varphi|
\end{equation*}
tomando en cuenta que:
\begin{equation*}
    \psi=\varphi e^{\frac{-i}{\hbar}m_0c^2t}
\end{equation*}
Calculando $\left[i\hbar \frac{\partial}{\partial t} -eA_0\right]^2$
\begin{align*}
    \left[i\hbar \frac{\partial}{\partial t} -eA_0\right]^2\psi &=\left[-\hbar^2\frac{\partial^2}{\partial t^2}-2i\hbar e A_0 \frac{\partial}{\partial t}+e^2A_0\right]\psi
\end{align*}
calculando $\frac{\partial \psi }{\partial t}$
\begin{align*}
    \frac{\partial \psi }{\partial t} &= \left[\frac{-i}{\hbar}m_0c^2 \varphi+ \frac{\partial \varphi}{\partial t}\right] e^{\frac{-i}{\hbar}m_0c^2t}
\end{align*}
calculando $\frac{\partial^2 \psi }{\partial t^2}$
\begin{align*}
    \frac{\partial^2 \psi }{\partial t^2} &= \left[\frac{\partial^2 \varphi}{ \partial t^2}-\frac{2i}{\hbar}m_0c^2\frac{\partial \varphi}{\partial t}-\frac{1}{\hbar^2}m_0^2c^4 \varphi\right] e^{\frac{-i}{\hbar}m_0c^2t}
\end{align*}
por lo tanto:
\begin{align*}
    \left[i\hbar \frac{\partial}{\partial t} -eA_0\right]^2\psi &=\left[-\hbar^2\frac{\partial^2}{\partial t^2}-2i\hbar e A_0 \frac{\partial}{\partial t}+e^2A_0\right]\psi\\
    &=\left[-\hbar^2\frac{\partial^2 \varphi}{ \partial t^2}+2i\hbar m_0c^2\frac{\partial \varphi}{\partial t}+m_0^2c^4 \varphi -2m_0c^2eA_0\varphi - 2i \hbar e A_0 \frac{\partial \varphi}{\partial t}\right] e^{\frac{-i}{\hbar}m_0c^2t}\\
    & \approx \left[2i\hbar m_0c^2\frac{\partial \varphi}{\partial t}+m_0^2c^4 \varphi -2m_0c^2eA_0\varphi - 2i \hbar e A_0 \frac{\partial \varphi}{\partial t}\right] e^{\frac{-i}{\hbar}m_0c^2t}
\end{align*}
\section*{Ejercicio 24}
Mostrar que al desarrollar 
\begin{equation*}
    \left[i\hbar \vec{\nabla} + \frac{e}{c}\vec{A}\right]^2 \varphi 
\end{equation*}
de la ecuación:
\begin{equation*}
    i\hbar \frac{\partial}{\partial t} \varphi = \left[\frac{1}{2m_0} \left[i\hbar \vec{\nabla} + \frac{e}{c}\vec{A}\right]^2
    + \frac{i\hbar e}{2m_0 c^2}\frac{\partial }{\partial t}A_0 +e A_0 \right] \varphi
\end{equation*}
se escribe como:
\begin{equation*}
    i\hbar \frac{\partial}{\partial t} \varphi = \left[\frac{{\vec{P}^2}}{2m}-\frac{e}{m_0c}\vec{A}\cdot \vec{P}+eA_0 + \frac{i\hbar e}{2m_0}\left[\vec{\nabla \cdot \vec{A}}\right]+ \frac{i\hbar e}{2m_0c^2}\frac{\partial }{\partial t}A_0\right] \varphi
\end{equation*}
Desarrollando el termino $\left[i\hbar \vec{\nabla} + \frac{e}{c}\vec{A}\right]^2 \varphi $, se tiene que:
\begin{align*}
    \frac{1}{2m_0}\left[i\hbar \vec{\nabla} + \frac{e}{c}\vec{A}\right]^2 \varphi =& \frac{1}{2m_0}\left[-\hbar^2 \vec{\nabla}^2 + \frac{i\hbar e}{c} \vec{\nabla} \cdot \vec{A} + \frac{ei\hbar}{c}\vec{A}\cdot \vec{\nabla} + \frac{e^2}{c^2}(\vec{A}\cdot\vec{A})\right]\\
    =& \frac{\vec{P}}{2m_0}+ \frac{i\hbar e}{2m_0c} \vec{\nabla} \cdot \vec{A} + \frac{e^2}{2m_0c^2}\vec{A}\cdot \vec{A} + \frac{ei\hbar}{2m_0c}\vec{A}\cdot \vec{\nabla} \\
    =& \frac{\vec{P}}{2m_0}+ \frac{i\hbar e}{2m_0c} \vec{\nabla} \cdot \vec{A} + \frac{e^2}{2m_0c^2}\vec{A}\cdot \vec{A} - \frac{e}{2m_0c}\vec{A}\cdot \vec{P} 
\end{align*}
entonces:
\begin{align*}
    \frac{1}{2m_0} \left[i\hbar \vec{\nabla} + \frac{e}{c}\vec{A}\right]^2
    + \frac{i\hbar e}{2m_0 c^2}\frac{\partial }{\partial t}A_0 +e A_0 =&
    \frac{\vec{P}}{2m_0}+ \frac{i\hbar e}{2m_0c} \vec{\nabla} \cdot \vec{A} + \frac{e^2}{2m_0c^2}\vec{A}\cdot \vec{A} - \frac{e}{2m_0c}\vec{A}\cdot \vec{P} +\frac{i\hbar e}{2m_0c^2}\frac{\partial}{\partial t}A_0+eA_0\\
    =&\frac{\vec{P}}{2m_0}- \frac{e}{2m_0c}\vec{A}\cdot \vec{P} +\frac{i\hbar e}{2m_0c}\left(\frac{\partial}{\partial t}A_0+\vec{\nabla} \cdot \vec{A} \right)+eA_0\\
    =&\frac{\vec{P}}{2m_0}- \frac{e}{2m_0c}\vec{A}\cdot \vec{P}+eA_0\\
\end{align*}
por lo tanto:
\begin{equation*}
    i\hbar \frac{\partial}{\partial t} \varphi=\left[\frac{\vec{P}}{2m_0}- \frac{e}{2m_0c}\vec{A}\cdot \vec{P}+eA_0\right]\varphi
\end{equation*}
\end{document}