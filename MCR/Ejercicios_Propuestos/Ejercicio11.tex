\section*{Ejercicio 11}
Una partícula A en reposo, decae en 2 partículas B y C $(A\rightarrow B+C)$. Mostrar que la energía
de la partícula que emergió es 
\begin{equation*}
    E_B= \frac{m_A^2+m_B^2-m_C^2}{2m_A}c^2
\end{equation*}
De la conservación de la energía se tiene que:
\begin{equation*}
    m_a=m_b\gamma_b+m_c\gamma_c
\end{equation*}
y de la conservación del momento lineal:
\begin{equation*}
    0=p_b+p_c
\end{equation*}
por lo tanto $p_b=-p_c$, entonces:
\begin{align*}
    0=&p_b+p_c\\
    0=&m_b\gamma_bv_b -m_c\gamma_cv_c\\
    0=&m_b\sqrt{\gamma^2_b-1}-m_c\sqrt{\gamma^2_c-1}\\
    m_b^2\gamma_b^2-m_c^2\gamma_c^2=&m_b^2-m_c^2
\end{align*}
nombando a $\chi_b\equiv m_b\gamma_b$ y a $\chi_c\equiv m_c\gamma_c$, entonces se tiene el siguiente sistema de ecuaciones
\begin{equation*}
    \chi_b^2-\chi_c^2=m_b^2-m_c^2 \qquad \chi_b + \chi_c = m_a
\end{equation*}
resolviendo ese sistema se obtine que $\chi_b$ es igual a 
\begin{equation*}
    \chi_b = \frac{m_a^2+m_b^2-m_c^2}{2m_a}
\end{equation*}
pero $E_b=c^2\chi_b$, por lo tanto:
\begin{equation*}
    E_b = \frac{m_a^2+m_b^2-m_c^2}{2m_a}c^2
\end{equation*}