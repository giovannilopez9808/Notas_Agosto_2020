%Para slide em "Wide-Screen" usar:
\documentclass[aspectratio=169]{beamer} 

%Para slide "quadrado" usar:
%\documentclass{beamer} 




\usepackage{tikz}

%Elaborado por Mateus Moro Lumertz

\author{mateuslumertz}
\definecolor{cor1}{RGB}{0,100,166}
\definecolor{cor2}{RGB}{100,195,213}
\definecolor{cor5}{RGB}{150,203,226}
\definecolor{cor3}{RGB}{30,130,186}
\definecolor{cor4}{RGB}{40,185,218}
\definecolor{preto}{RGB}{0,0,0}
\definecolor{branco}{RGB}{255,255,255}
% Configuração das Cores
\setbeamercolor{paleta1}{fg=cor1,bg=white}
\setbeamercolor{paleta2}{fg=cor1,bg=white}
\setbeamercolor{estrutura}{fg=cor1,bg=white}
\setbeamercolor{titulo_rodape}{fg=black,bg=white}
\setbeamercolor{data_rodape}{fg=gray,bg=white}
\setbeamercolor{frametitle}{fg=cor1,bg=branco}


% Modelo do rodapé
\defbeamertemplate*{footline}{mytheme}{%
  \leavevmode%
  \hbox{\begin{beamercolorbox}[wd=.5\paperwidth,ht=2.5ex,dp=1.125ex,leftskip=.3cm,rightskip=.3cm]{titulo_rodape}%
    \makebox[2em][l]{{\usebeamerfont{titulo_rodape}\textcolor{cor1}{\insertframenumber}}}%
    {\usebeamercolor{titulo_rodape}\insertshorttitle}
  \end{beamercolorbox}%
  \begin{beamercolorbox}[wd=.2\paperwidth,ht=2.5ex,dp=1.125ex,leftskip=.3cm,rightskip=.3cm]{data_rodape}%
    \usebeamerfont{data_rodape}\insertshortdate%
  \end{beamercolorbox}%
  \begin{beamercolorbox}[wd=.3\paperwidth,ht=2.5ex,dp=1.125ex,leftskip=.3cm,rightskip=.3cm,right]{titulo_rodape}%
    \includegraphics[width=.2\paperwidth,height=7ex,keepaspectratio]{../../Logos/fcfm.png}\hspace*{2em}%
  \end{beamercolorbox}}%
  \vskip0pt%
  
    \begin{tikzpicture}[remember picture,overlay]
  
    \filldraw[cor2]
    ([xshift=-12cm,yshift=7.5cm]current page.south east)--
    ([xshift=-16cm,yshift=7.5cm]current page.south east)--
    ([xshift=-16cm,yshift=7.62cm]current page.south east)--
    ([xshift=-12cm,yshift=7.62cm]current page.south east)-- cycle
    ;
    \filldraw[cor3]
    ([xshift=-8cm,yshift=7.5cm]current page.south east)--
    ([xshift=-12cm,yshift=7.5cm]current page.south east)--
    ([xshift=-12cm,yshift=7.62cm]current page.south east)--
    ([xshift=-8cm,yshift=7.62cm]current page.south east)-- cycle
    ;
    \filldraw[cor4]
    ([xshift=-4cm,yshift=7.5cm]current page.south east)--
    ([xshift=-8cm,yshift=7.5cm]current page.south east)--
    ([xshift=-8cm,yshift=7.62cm]current page.south east)--
    ([xshift=-4cm,yshift=7.62cm]current page.south east)-- cycle
    ;
    \filldraw[cor5]
    ([xshift=-0cm,yshift=7.5cm]current page.south east)--
    ([xshift=-4cm,yshift=7.5cm]current page.south east)--
    ([xshift=-4cm,yshift=7.62cm]current page.south east)--
    ([xshift=-0cm,yshift=7.62cm]current page.south east)-- cycle
    ;

  \end{tikzpicture}
  
}

% Slide de Título
\defbeamertemplate*{title page}{mytheme}[1][]
{%
  \begin{tikzpicture}[remember picture,overlay]
    \filldraw[cor1]
    (current page.north west) --
    ([yshift=-12cm]current page.north west) --
    ([xshift=-4cm,yshift=-12cm]current page.north east) {[rounded corners=15pt]--
    ([xshift=-4cm,yshift=3cm]current page.south east)} --
    ([yshift=3cm]current page.south west) --
    (current page.south west) --
    (current page.south east) --
    (current page.north east) -- cycle
    ;
  \filldraw[branco]
    (current page.north west) --
    ([yshift=-2.15cm]current page.north west) --
    ([xshift=-3cm,yshift=-2.15cm]current page.north east) {[rounded corners=15pt]--
    ([xshift=-3cm,yshift=3cm]current page.south east)} --
    ([yshift=3cm]current page.south west) --
    (current page.south west) --
    (current page.south east) --
    (current page.north east) -- cycle
    ;
    \filldraw[cor2]
    ([xshift=-0.25cm,yshift=3cm]current page.south east)--
    ([xshift=-2.75cm,yshift=3cm]current page.south east)--
    ([xshift=-2.75cm,yshift=3.85cm]current page.south east)--
    ([xshift=-0.25cm,yshift=3.85cm]current page.south east)-- cycle
    ;
    \filldraw[cor3]
    ([xshift=-0.25cm,yshift=4cm]current page.south east)--
    ([xshift=-2.75cm,yshift=4cm]current page.south east)--
    ([xshift=-2.75cm,yshift=4.85cm]current page.south east)--
    ([xshift=-0.25cm,yshift=4.85cm]current page.south east)-- cycle
    ;
    \filldraw[cor4]
    ([xshift=-0.25cm,yshift=5cm]current page.south east)--
    ([xshift=-2.75cm,yshift=5cm]current page.south east)--
    ([xshift=-2.75cm,yshift=5.85cm]current page.south east)--
    ([xshift=-0.25cm,yshift=5.85cm]current page.south east)-- cycle
    ;
    \filldraw[cor5]
    ([xshift=-0.25cm,yshift=6cm]current page.south east)--
    ([xshift=-2.75cm,yshift=6cm]current page.south east)--
    ([xshift=-2.75cm,yshift=6.85cm]current page.south east)--
    ([xshift=-0.25cm,yshift=6.85cm]current page.south east)-- cycle
    ;
  \node[text=branco,anchor=south west,font=\sffamily\LARGE,text width=.68\paperwidth] 
  at ([xshift=10pt,yshift=-0.5cm]current page.west)
  (title)
  {\raggedright\inserttitle};  
  
  \node[text=cor1,anchor=south west,font=\sffamily\small,text width=.75\paperwidth] 
  at ([xshift=10pt,yshift=3.6cm]current page.west)
  (title)
  {\raggedright};  
  
  
  % \node[anchor=west]
  %at ([xshift=10.1cm,yshift=8.5cm]current page.south west)
  %{\includegraphics[height=1.5cm]{imagens/logo.png}};
  
  \node[anchor=east]
  at ([xshift=-0.15cm,yshift=-1cm]current page.north east)
  {\includegraphics[width=2.5cm]{../../Logos/fcfm.png}};
  
  \node[text=preto,font=\large\sffamily,anchor=south west]
  at ([xshift=30pt,yshift=0.5cm]current page.south west)
  (date)
  {\insertdate};
  \node[text=preto,font=\large\sffamily,anchor=south west]
  at ([yshift=5pt]date.north west)
  (author)
  {\insertauthor};
  \end{tikzpicture}%
}

% remove navigation symbols
\setbeamertemplate{navigation symbols}{}



% definition of the itemize templates
\setbeamertemplate{itemize item}[circle]
\setbeamercolor{itemize item}{fg=cor3,bg=white}
\setbeamercolor{itemize subitem}{fg=cor4,bg=white}
\setbeamercolor{itemize subsubitem}{fg=cor2,bg=white}
\title[]{Tarea N$^{\circ}$2}
\author{Ivan Pla, Axel Rangel, Paulina Vazquez, Giovanni López}
\date{\today}
\begin{document}
\begin{frame}[plain]
\maketitle
\end{frame}
\begin{frame}{Problema 1}
  Sea \begin{equation}
    \mathcal{L}=  \frac{1}{4}F_{\mu \nu} F^{\mu \nu} 
    \label{eq:l}
  \end{equation}
  donde
  \begin{equation}
    F_{\mu \nu}=\left(\partial_\mu A_\nu - \partial_\nu A_\mu\right)
    \label{eq:ten}
  \end{equation}
  \begin{itemize}
    \item Determinar las ecuaciones que satisface el campo $A_\nu$.
  \end{itemize}
\end{frame}
\begin{frame}{Procedimiento}
  Se tiene que la ecuación de Euler-Lagrande para campos es la siguiente:
\begin{equation}
  \nabla_\nu \left( \frac{\partial \mathcal{L}}{\partial (\nabla_\nu A_\mu)}\right) = \frac{\partial \mathcal{L}}{\partial A_\mu}
  \label{eq:euler}
\end{equation}
Como la ecuación \ref{eq:l} no depende del campo $A_\mu$, entonces:
\begin{equation*}
  \frac{\partial \mathcal{L}}{\partial A_\mu} = 0
\end{equation*}
por lo tanto, la ecuación \ref{eq:euler} se escribe de la siguiente manera:
\begin{equation}
  \nabla_\nu \left( \frac{\partial \mathcal{L}}{\partial (\nabla_\nu A_\mu)}\right) = 0
  \label{eq:euler2}
\end{equation}
\end{frame}
\begin{frame}{Procedimiento}
  Calculando $ \frac{\partial \mathcal{L}}{\partial (\nabla_\nu A_\mu)}$ se tiene que :
  \begin{align*}
    \frac{\partial \mathcal{L}}{\partial (\nabla_\nu A_\mu)} &=   \frac{\partial }{\partial (\nabla_\nu A_\mu)} \left(\frac{1}{4}F_{\mu \nu} F^{\mu \nu} \right)\\
    &= \frac{\partial }{\partial (\nabla_\nu A_\mu)}(\left(\partial_\mu A_\nu - \partial_\nu A_\mu\right) F^{\mu \nu})\\
    & = F^{\mu \nu }
  \end{align*}
\end{frame}
\begin{frame}{Resultado}
entonces
\begin{align*}
    \nabla_\nu \left( \frac{\partial \mathcal{L}}{\partial (\nabla_\nu A_\mu)}\right)=
    \nabla_\nu F^{\mu \nu} 
\end{align*}
por lo tanto el campo $A_\nu$ debe cumplir la siguiente ecuación:
\begin{equation*}
  \nabla_\nu F^{\mu \nu} =0
\end{equation*}
\end{frame}
\begin{frame}{Problema 2}
  \begin{itemize}
    \item Determinar el tensor ${T^{\mu}}_\nu(T^{\mu \nu})$
  \end{itemize}
\end{frame}
\begin{frame}{Procedimiento}
  Se tiene que 
  \begin{equation*}
    {T^\mu}_\nu =\frac{\partial \mathcal{L}}{\partial(\partial_\mu A_\nu)} - g^{\mu}_\nu \mathcal{L} \qquad
    T^{\mu\nu} =\frac{\partial \mathcal{L}}{\partial(\partial_\mu A_\nu)} - g^{\mu\nu} \mathcal{L}
  \end{equation*}
\end{frame}
\begin{frame}{Procedimiento}
  Calculando ${T^{\mu}}_\nu(T^{\mu \nu})$
  \begin{align*}
    {T^{\mu}}_\nu(T^{\mu \nu}) =& \left(\frac{\partial \mathcal{L}}{\partial(\partial_\mu A_\nu)} - g^{\mu}_\nu \mathcal{L}\right)\left(
      \frac{\partial \mathcal{L}}{\partial(\partial_\mu A_\nu)} - g^{\mu\nu} \right)\\
      =&({F^\mu}_\nu \partial_\nu A_\mu )(F^{\mu \nu} \partial_\nu A_\mu) - ({F^\mu}_\nu \partial_\nu A_\mu )(g^{\mu \nu} \mathcal{L})
      \\ &-({g^{\mu}}_\nu\mathcal{L})(F^{\mu \nu} \partial_\nu A_\mu)  + ({g^\mu}_\nu \mathcal{L})(g^{\mu \nu}\mathcal{L}) \\
       =& ({F^\mu}_\nu \partial_\nu A_\mu )(F^{\mu \nu} \partial_\nu A_\mu) 
  \end{align*}
\end{frame}
\begin{frame}{Resultado}
  por lo tanto
  \begin{equation*}
    {T^{\mu}}_\nu(T^{\mu \nu})=({F^\mu}_\nu \partial_\nu A_\mu )(F^{\mu \nu} \partial_\nu A_\mu) 
  \end{equation*}
\end{frame}
\end{document}