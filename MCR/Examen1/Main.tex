\documentclass[12pt,letterpaper]{report}
\usepackage{graphicx}
\usepackage{scrextend}
\usepackage{vmargin}
\usepackage{graphicx}
\usepackage{multirow}
\usepackage[utf8]{inputenc}
\usepackage[spanish]{babel}
\usepackage{multicol}
\usepackage{enumerate}
\usepackage{float}
\usepackage{amsmath, amsthm, amssymb, amsfonts}
\usepackage[usenames]{color}
\parindent=0mm
\pagestyle{empty}
\definecolor{miorange}{rgb}{0.91, 0.43, 0.0}
\begin{document}
\setmargins{2.5cm}      
{1.5cm}                     
{2cm}  
{24cm}                    
{10pt}                          
{1cm}                          
{0pt}                             
{2cm}
\begin{titlepage}
\begin{center}
\includegraphics[scale=0.40]{../../Logos/uanl.png} 
\hspace{2.5cm}
\includegraphics[scale=0.40]{../../Logos/fcfm.png}
\end{center}
\vspace{2cm}
\begin{center}
\textbf{
UNIVERSIDAD AUTÓNOMA DE NUEVO LEÓN\\
FACULTAD DE CIENCIAS
    FÍSICO MATEMÁTICAS}\\
\vspace*{2cm}
\begin{large}
\vspace{1cm}
\large{\textbf{Mécaninca Cuántica Relativista}}\\
\textbf{Problemas propuestos}\\
Francisco Baez\\
\end{large}
\end{center}
\end{titlepage}
\begin{enumerate}
    \item Mostrar que:
    \begin{equation*}
        U_\perp=\frac{u_i}{\gamma_v \left[1+\frac{\upsilon \cdot v}{c^2} \right]}
    \end{equation*}
    De las transformaciones:
    \begin{align*}
        r_\parallel &= \gamma_v [{r_\parallel}'+v{t}']\\
        r_\perp &= {r_\perp}'\\
        t&= \gamma_v \left[ {t}'+\frac{\upsilon \cdot v}{c^2}\right]
    \end{align*}
    tomando los diferenciales:
    \begin{align*}
        dr_\parallel &= \gamma_v \left[{dr}'_\parallel + vdt\right]\\
        dr_\perp &= {dr}'_perp\\
        dt&=\gamma_v \left[dt+ \frac{vdr}{c^2} \right]
    \end{align*}
    entonces:
    \begin{align*}
        \frac{dr_\perp}{dt}&= \frac{{dr}'_\perp}{\gamma_v {dt}'\left[1+\frac{v}{c^2} \frac{{dr}'}{{dt}'} \right]}\\
        u_\perp&= \frac{{dr}'_\perp}{\gamma_v {dt}' \left[1+\frac{v}{c^2}\frac{{dr}'}{{dt}'} \right]}\\
        & = \frac{{u}'_\perp}{\gamma_v \left[1+\frac{v\cdot {u}'}{c^2} \right]}
    \end{align*}
    por lo tanto:
    \begin{equation}
        u_\perp = \frac{{u}'_\perp}{\gamma_v \left[1+\frac{v\cdot {u}'}{c^2} \right]}
    \end{equation}
    \item Mostrar que
    \begin{align*}
        (U_d)_x &= \frac{-c\beta \sin({\theta}')}{\gamma_v (1-\beta^2 \cos({\theta}'))}\\
        (U_d)_z &= \frac{c\beta (1-\cos({\theta}'))}{1-\beta^2 \cos({\theta}')}
    \end{align*}
    Se sabe que por convención:
    \begin{align*}
        (U_d)_\perp = (U_d)_x &= \frac{({U}'_d)_\perp}{\gamma_v \left(1+\frac{v \cdot {u}'}{c^2} \right)}\\
        (U_d)_\parallel = (U_c)_z &= \frac{({U}'_d)_\parallel+v}{1+\frac{v \cdot {u}'_d}{c^2}}\\
    \end{align*}
    pero, del diagrama
    \begin{align*}
        ({U}'_d)_\perp &= {U}'_d \sin({\theta}')\\
        ({U}'_d)_\parallel&={U}'_d \cos({\theta}')
    \end{align*}
    por lo tanto:
    \begin{align*}
        (U_d)_x &= \frac{{U}'_d \sin({\theta}')}{\gamma_v \left(1+\frac{|v||u_d|\cos(\theta)}{c^2} \right)}\\
        (U_d)_z &= \frac{{U}'_d \cos({\theta}'_d)+v}{\gamma_v \left(1+\frac{|v||u_d|\cos(\theta)}{c^2} \right)}
    \end{align*}
    pero ${U}'_d =-v$
    \begin{align*}
        (U_d)_x &= \frac{-v \sin({\theta}')}{\gamma_v \left(1- \frac{v^2}{c^2}\cos({\theta}') \right)}\\
        & = \frac{-c\beta \sin({\theta}')}{\gamma_v (1-\beta^2 \cos({\theta}'))}\\
        (U_d)_z &= \frac{c\beta (1-\cos({\theta}'))}{1-\beta^2 \cos({\theta}')}
    \end{align*}
    \item Mostrar que
    \begin{equation}
        u_c^2 = u_a^2 - \frac{\eta}{\gamma_a}
    \end{equation}
    \begin{align*}
        (U_c)_x &= \frac{c\beta \sin({\theta}')}{\gamma_v \left[1+\beta^2 \cos({\theta}') \right]}\\
        & \approx \frac{c \beta  \theta}{\gamma_v \left[1+\beta^2 \left(1- \frac{\theta^2}{2} \right) \right]}\\
        (U_c)_z & \approx \frac{c \beta (1+\left(1-\frac{\theta^2}{2} \right))}{1+\beta^2\left(1-\frac{\theta^2}{2}\right)}
    \end{align*}
    realizando el calculo para ángulos pequeños, tomando en cuenta que $\cos(\theta)=1-\theta^2/2$ y $\sin(\theta)=\theta$
    \begin{align*}
        (U_c)_z^2 &= \frac{c^2 \beta^2 \left(4-2\theta^2+\frac{\theta^4}{4}\right)}{(1+\beta^2)\left(1- \frac{\beta^2 \theta^2}{2(1+\beta^2)}\right)^2}\\
        & = \frac{c^2 \beta^2 (4-2\theta^2)}{(1+\beta^2)^2 \left(1- \frac{\beta^2 \theta^2 }{2(1+\beta^2)}\right)^2}\\
        & \approx \frac{c^2 \beta^2 (4-2\theta^2)}{(1+\beta^2)^2} \left(1+ \frac{\beta^2}{1+\beta^2}\theta^2 \right)\\
        & \approx \frac{4c^2 \beta^2}{(1+\beta^2)^2} - \frac{2c^2 \beta^2 \theta^2}{(1+\beta^2)^2} + \frac{4c^2 \beta^4 \theta^2}{(1+\beta^2)^3}\\
        & \approx u_a^2 - \frac{2c^2 \beta^2 \theta^2}{(1+\beta^2)^2} + \frac{4c^2 \beta^4 \theta^2}{(1+\beta^2)^3} 
    \end{align*}
    \begin{align*}
        (U_c)_x^2 &= \frac{c^2 \beta^2 \theta^2}{\gamma_v^2 (1+\beta^2)^2 \left(1-\frac{\beta^2 \theta^2}{2(1+\beta^2)} \right)^2}\\
        &\approx \frac{c^2 \beta^2 \theta^2 }{\gamma^2_v (1+\beta^2)} \left(1+\frac{\beta}{1+\beta^2}\theta^2 \right)\\
        &\approx \frac{c^2 \beta^2 \theta^2}{\gamma^2 (1+\beta^2)^2} + \frac{c^2 \beta^4 \theta^4}{\gamma^2 (1+\beta^2)^3}\\
        &\approx \frac{c^2 \beta^2 \theta^2}{\gamma^2 (1+\beta^2)^2}
    \end{align*}
    se tiene que:\\
    \begin{minipage}{0.5\linewidth}
        \begin{equation*}
            u_a = \frac{2\beta c}{1+\beta^2}
        \end{equation*}
    \end{minipage}
    \begin{minipage}{0.5\linewidth}
        \begin{equation*}
            \gamma_a = \frac{1+\beta^2}{1-\beta^2}
        \end{equation*}
    \end{minipage}
    por lo tanto:
    \begin{align*}
        u_c^2&= (u_c)_x^2+(u_c)_z^2 \\
        & = u_a^2 - \frac{2c^2 \beta^2 \theta^2}{(1+\beta^2)^2} + \frac{4c^2 \beta^4 \theta^2}{(1+\beta^2)^3}+\frac{c^2 \beta^2 \theta^2}{\gamma^2 (1+\beta^2)^2} \\
        & =u_a^2 + \frac{c^2 \beta^2 \theta^2}{(1+\beta^2)^2}\left(1-\beta^2-2 + \frac{4\beta^2}{1+\beta^2} \right)\\
        & = u_a^2 + \frac{c^2 \beta^2 \theta^2 }{(1+\beta^2)^2 }\left(\frac{1-2\beta^2-\beta^4}{1+\beta^2} \right) \\
        &= u_a^2 - \frac{c^2 \beta^2 \theta^2 }{(1+\beta^2)^2 }\left(\frac{(1-\beta^2)^2}{1+\beta^2} \right) \\
        & = u_a^2 -\frac{c^2 \beta^2 \theta^2}{1-\beta^2} \left(\frac{(1-\beta^2)^3}{(1+\beta^2)^3}\right)\\
        & = u_a^2 - \eta \frac{1}{\gamma^3_a}
    \end{align*}
    \item Muestre que:
    \begin{equation*}
        \partial_\alpha A^\alpha = \partial^\alpha A_\alpha 
    \end{equation*}
    Se tiene que:\\
    \begin{minipage}{0.5\linewidth}
        \begin{equation*}
            x_\alpha = g_{\alpha \beta} x^\beta 
        \end{equation*}
    \end{minipage}
    \begin{minipage}{0.5\linewidth}
        \begin{equation*}
            x^\alpha = g^{\alpha \beta}x_\beta
        \end{equation*}
    \end{minipage}
    por lo tanto: 
    \begin{align*}
        g^{\alpha \beta}\partial_\alpha &= \partial^\alpha \\
        g_{\alpha \beta}\partial^\alpha &= \partial_\alpha 
    \end{align*}
    calculando $\partial^\alpha A_\alpha$
    \begin{align*}
        \partial^\alpha A_\alpha &= \left(\frac{\partial A_0}{\partial x_0} \right)-\left(\frac{\partial A_1}{\partial x_1} \right)-\left(\frac{\partial A_2}{\partial x_2} \right)-\left(\frac{\partial A_3}{\partial x_3}\right)\\
        & = \frac{\partial A_0}{\partial x_0} - \nabla A
    \end{align*}
    por lo que se encuentra que:
    \begin{align*}
        A_0&=A^0 \\ A_1&=-A^1 \\ A_2&=-A^2 \\ A_3&=-A^3 \\
    \end{align*}
    \begin{align*}
        \partial^\alpha A_\alpha &= (g^{\alpha \beta}\partial_\beta)(g_{\alpha \gamma}A^\gamma)\\
        \delta^\beta_\gamma &= \partial_\beta A^\gamma
    \end{align*}
    \item Por verificar que:
    \begin{equation*}
        \partial^\alpha = \left(\frac{\partial}{\partial x_0}, -\nabla  \right)
    \end{equation*}
    Sea $A^\alpha$ un tensor covariante, entonces:
    \begin{align*}
        \partial^\alpha A_\alpha &= \left(\frac{\partial A_0}{\partial x_0} \right)-\left(\frac{\partial A_1}{\partial x_1} \right)-\left(\frac{\partial A_2}{\partial x_2} \right)-\left(\frac{\partial A_3}{\partial x_3} \right)\\
        &= \left(\frac{\partial}{\partial x_0},-\frac{\partial}{\partial x_1},-\frac{\partial}{\partial x_2},-\frac{\partial}{\partial x_3} \right) \cdot (A_0,A_1,A_2,A_3)\\
        &= \left(\frac{\partial}{\partial x_0},-\nabla \right) \cdot A_\alpha
    \end{align*}
    por lo tanto:
    \begin{equation*}
        \partial^\alpha = \left(\frac{\partial}{\partial x_0}, -\nabla  \right)
    \end{equation*}
    \item Probar que las matrices $S_1^2,S_2^2,S_3^2$ son diagonales con -1 y que las matrices $K_1^2,K_2^2,K_3^2$ son diagonales con 1:
    Se tiene la matriz $S_1$ igual a:
    \begin{equation*}
        S_1 =\left( \begin{matrix}
            0 & 0 & 0 & 0 \\
            0 & 0 & 0 & 0 \\
            0 & 0 & 0 & -1 \\
            0 & 0 & 1 & 0 \\
        \end{matrix}\right)
    \end{equation*}
    entonces, calculando $S_1^2$, se tiene que:
    \begin{align*}
        S_1^2 &=\left( \begin{matrix}
            0 & 0 & 0 & 0 \\
            0 & 0 & 0 & 0 \\
            0 & 0 & 0 & -1 \\
            0 & 0 & 1 & 0 \\
        \end{matrix}\right)^2 \\
        & =\left( \begin{matrix}
            0 & 0 & 0 & 0 \\
            0 & 0 & 0 & 0 \\
            0 & 0 & 0 & -1 \\
            0 & 0 & 1 & 0 \\
        \end{matrix}\right)\left( \begin{matrix}
            0 & 0 & 0 & 0 \\
            0 & 0 & 0 & 0 \\
            0 & 0 & 0 & -1 \\
            0 & 0 & 1 & 0 \\
        \end{matrix}\right)\\
        & =\left( \begin{matrix}
            0 & 0 & 0 & 0 \\
            0 & 0 & 0 & 0 \\
            0 & 0 & -1 & 0 \\
            0 & 0 & 0 & -1 \\
        \end{matrix}\right)
    \end{align*}
    Se tiene la matriz $S_2$ igual a:
    \begin{equation*}
        S_2 =\left( \begin{matrix}
            0 & 0 & 0 & 0 \\
            0 & 0 & 0 & 0 \\
            0 & 0 & 0 & -1 \\
            0 & 0 & 1 & 0 \\
        \end{matrix}\right)
    \end{equation*}
    entonces, calculando $S_2^2$, se tiene que:
    \begin{align*}
        S_2^2 &=\left( \begin{matrix}
            0 & 0 & 0 & 0 \\
            0 & 0 & 0 & 1 \\
            0 & 0 & 0 & 0 \\
            0 & -1 & 0 & 0 \\
        \end{matrix}\right)^2 \\
        & =\left( \begin{matrix}
            0 & 0 & 0 & 0 \\
            0 & 0 & 0 & 1 \\
            0 & 0 & 0 & 0 \\
            0 & -1 & 0 & 0 \\
        \end{matrix}\right)\left( \begin{matrix}
            0 & 0 & 0 & 0 \\
            0 & 0 & 0 & 1 \\
            0 & 0 & 0 & 0 \\
            0 & -1 & 0 & 0 \\
        \end{matrix}\right)\\
        & =\left( \begin{matrix}
            0 & 0 & 0 & 0 \\
            0 & -1 & 0 & 0 \\
            0 & 0 & 0 & 0 \\
            0 & 0 & 0 & -1 \\
        \end{matrix}\right)
    \end{align*}
    Se tiene la matriz $S_3$ igual a:
    \begin{equation*}
        S_3 =\left( \begin{matrix}
            0 & 0 & 0 & 0 \\
            0 & 0 & -1 & 0 \\
            0 & 1 & 0 & -0 \\
            0 & 0 & 0 & 0 \\
        \end{matrix}\right)
    \end{equation*}
    entonces, calculando $S_3^2$, se tiene que:
    \begin{align*}
        S_3^2 &=\left( \begin{matrix}
            0 & 0 & 0 & 0 \\
            0 & 0 & -1 & 0 \\
            0 & 1 & 0 & 0 \\
            0 & 0 & 0 & 0 \\
        \end{matrix}\right)^2 \\
        & =\left( \begin{matrix}
            0 & 0 & 0 & 0 \\
            0 & 0 & -1 & 0 \\
            0 & 1 & 0 & 0 \\
            0 & 0 & 0 & 0 \\
        \end{matrix}\right)\left( \begin{matrix}
            0 & 0 & 0 & 0 \\
            0 & 0 & -1 & 0 \\
            0 & 1 & 0 & 0 \\
            0 & 0 & 0 & 0 \\
        \end{matrix}\right)\\
        & =\left( \begin{matrix}
            0 & 0 & 0 & 0 \\
            0 & -1 & 0 & 0 \\
            0 & 0 & -1 & 0 \\
            0 & 0 & 0 & 0 \\
        \end{matrix}\right)
    \end{align*}
    por lo tanto las matrices $S_\mu^2$ son diagonales con -1
    Se tiene la matriz $K_1$ igual a:
    \begin{equation*}
        S_1 =\left( \begin{matrix}
            0 & 1 & 0 & 0 \\
            1 & 0 & 0 & 0 \\
            0 & 0 & 0 & 0 \\
            0 & 0 & 0 & 0 \\
        \end{matrix}\right)
    \end{equation*}
    entonces, calculando $K_1^2$, se tiene que:
    \begin{align*}
        K_1^2 &=\left( \begin{matrix}
            0 & 1 & 0 & 0 \\
            1 & 0 & 0 & 0 \\
            0 & 0 & 0 & 0 \\
            0 & 0 & 0 & 0 \\
        \end{matrix}\right)^2 \\
        & =\left( \begin{matrix}
            0 & 1 & 0 & 0 \\
            1 & 0 & 0 & 0 \\
            0 & 0 & 0 & 0 \\
            0 & 0 & 0 & 0 \\
        \end{matrix}\right)\left( \begin{matrix}
            0 & 1 & 0 & 0 \\
            1 & 0 & 0 & 0 \\
            0 & 0 & 0 & 0 \\
            0 & 0 & 0 & 0 \\
        \end{matrix}\right)\\
        & =\left( \begin{matrix}
            1 & 0 & 0 & 0 \\
            0 & 1 & 0 & 0 \\
            0 & 0 & 0 & 0 \\
            0 & 0 & 0 & 0 \\
        \end{matrix}\right)
    \end{align*}
    Se tiene la matriz $K_2$ igual a:
    \begin{equation*}
        K_2 =\left( \begin{matrix}
            0 & 0 & 1 & 0 \\
            0 & 0 & 0 & 0 \\
            1 & 0 & 0 & 0 \\
            0 & 0 & 0 & 0 \\
        \end{matrix}\right)
    \end{equation*}
    entonces, calculando $k_2^2$, se tiene que:
    \begin{align*}
        K_2^2 &=\left( \begin{matrix}
            0 & 0 & 1 & 0 \\
            0 & 0 & 0 & 0 \\
            1 & 0 & 0 & 0 \\
            0 & 0 & 0 & 0 \\
        \end{matrix}\right)^2 \\
        & =\left( \begin{matrix}
            0 & 0 & 1 & 0 \\
            0 & 0 & 0 & 0 \\
            1 & 0 & 0 & 0 \\
            0 & 0 & 0 & 0 \\
        \end{matrix}\right)\left( \begin{matrix}
            0 & 0 & 1 & 0 \\
            0 & 0 & 0 & 0 \\
            1 & 0 & 0 & 0 \\
            0 & 0 & 0 & 0 \\
        \end{matrix}\right)\\
        & =\left( \begin{matrix}
            1 & 0 & 0 & 0 \\
            0 & 0 & 0 & 0 \\
            0 & 0 & 1 & 0 \\
            0 & 0 & 0 & 0 \\
        \end{matrix}\right)
    \end{align*}
    Se tiene la matriz $K_3$ igual a:
    \begin{equation*}
        K_3 =\left( \begin{matrix}
            0 & 0 & 0 & 1 \\
            0 & 0 & 0 & 0 \\
            0 & 0 & 0 & 0 \\
            1 & 0 & 0 & 0 \\
        \end{matrix}\right)
    \end{equation*}
    entonces, calculando $K_3^2$, se tiene que:
    \begin{align*}
        K_3^2 &=\left( \begin{matrix}
            0 & 0 & 0 & 1 \\
            0 & 0 & 0 & 0 \\
            0 & 0 & 0 & 0 \\
            1 & 0 & 0 & 0 \\
        \end{matrix}\right)^2 \\
        & =\left( \begin{matrix}
            0 & 0 & 0 & 1 \\
            0 & 0 & 0 & 0 \\
            0 & 0 & 0 & 0 \\
            1 & 0 & 0 & 0 \\
        \end{matrix}\right)\left( \begin{matrix}
            0 & 0 & 0 & 1 \\
            0 & 0 & 0 & 0 \\
            0 & 0 & 0 & 0 \\
            1 & 0 & 0 & 0 \\
        \end{matrix}\right)\\
        & =\left( \begin{matrix}
            1 & 0 & 0 & 0 \\
            0 & 0 & 0 & 0 \\
            0 & 0 & 0 & 0 \\
            0 & 0 & 0 & 1 \\
        \end{matrix}\right)
    \end{align*}
    por lo tanto las matrices $K_\mu^2$ son diagonales con 1
    \item Compruebe la forma de L, que cumple $L^Tg=-gL$ donde L tiene diagonal de ceros y g es la representación matricial de $g_{\mu \nu \rho}$
    Se tiene que:
    \begin{equation*}
        g=diag(1,-1,-1,-1)
    \end{equation*}
    y que
    \begin{equation*}
        g^T=g=g^{-1}
    \end{equation*}
    por lo tanto:
    \begin{align*}
        c^T&= (gL)^T \\
        &=L^T g \\
        &=-gL\\
        &=-c
    \end{align*}
    por lo tanto:
    \begin{equation*}
        c_{ij}=-c_{ji}
    \end{equation*}
    si $i=j$, entonces $c_{ii}=0$, por lo tanto:
    \begin{equation*}
        gL=C=\left(\begin{matrix}
            0 & C_{12} & C_{13} & C_{14} \\
            -C_{12} & 0 & C_{23} & C_{24} \\
            -C_{13} & -C_{23} & 0 & C_{34} \\
            -C_{14} & -C_{24} & -C_{34} & 0 \\
        \end{matrix}\right)
    \end{equation*}
    realizando la operación $gc=ggL$, se tiene que:
    \begin{align*}
        gc&=g(gL)\\
        &=(gg)L\\
        &=L
    \end{align*}
    por lo tanto $gc=L$
    \item Mostrar que $F_{\alpha\gamma}=g_{\alpha\gamma}F^{\gamma \delta}g_{\delta \beta }$\\
    Se sabe que:\\
    \begin{minipage}{0.5\linewidth}
    \begin{equation*}
        F^{\gamma \delta} = \left( \begin{matrix}
            0 & -Ex & -Ey   & -Ez \\
            E_x &  0  & -B_z & B_y \\
            E_y & B_z & 0 &-B_x \\
            E_z & -B_y & B_x & 0  
        \end{matrix}\right)
    \end{equation*}
\end{minipage}
\begin{minipage}{0.5\linewidth}
\begin{equation*}
    g_{\alpha \gamma} = g_{\delta \beta} =\left(\begin{matrix}
        1 & 0 & 0 & 0\\
        0 & -1 & 0 & 0\\
        0 & 0 & -1 & 0\\
        0 & 0 & 0 & -1\\
    \end{matrix}\right)
\end{equation*}
\end{minipage}
realizando la multiplicacion $F^{\gamma \delta}g_{\delta \beta }$
\begin{align*}
    F^{\gamma \delta}g_{\delta \beta }&= \left( \begin{matrix}
        0 & -Ex & -Ey   & -Ez \\
        E_x &  0  & -B_z & B_y \\
        E_y & B_z & 0 &-B_x \\
        E_z & -B_y & B_x & 0  
    \end{matrix}\right)\left(\begin{matrix}
        1 & 0 & 0 & 0\\
        0 & -1 & 0 & 0\\
        0 & 0 & -1 & 0\\
        0 & 0 & 0 & -1\\
    \end{matrix}\right)\\
    & =\left( \begin{matrix}
        0 & Ex & Ey   & Ez \\
        E_x &  0  & B_z & -B_y \\
        E_y & -B_z & 0 &B_x \\
        E_z & B_y & -B_x & 0  
    \end{matrix}\right)
\end{align*}
por lo tanto:
\begin{equation*}
    F^{\gamma}_\beta = \left( \begin{matrix}
        0 & Ex & Ey   & Ez \\
        E_x &  0  & B_z & -B_y \\
        E_y & -B_z & 0 &B_x \\
        E_z & B_y & -B_x & 0  
    \end{matrix}\right)
\end{equation*}
realizando la multiplicacion $g_{\alpha\gamma}F^{\gamma}_\beta$ se obtiene que:
\begin{align*}
    g_{\alpha\gamma}F^{\gamma}_\beta &= \left(\begin{matrix}
        1 & 0 & 0 & 0\\
        0 & -1 & 0 & 0\\
        0 & 0 & -1 & 0\\
        0 & 0 & 0 & -1\\
    \end{matrix}\right)\left( \begin{matrix}
        0 & Ex & Ey   & Ez \\
        E_x &  0  & B_z & -B_y \\
        E_y & -B_z & 0 &B_x \\
        E_z & B_y & -B_x & 0  
    \end{matrix}\right) \\
    & = \left( \begin{matrix}
        0 & Ex & Ey   & Ez \\
        -E_x &  0  & -B_z & B_y \\
        -E_y & B_z & 0 &-B_x \\
        -E_z & -B_y & B_x & 0  
    \end{matrix}\right)
\end{align*}
por lo tanto:
\begin{equation*}
    F_{\alpha \beta} = \left( \begin{matrix}
        0 & Ex & Ey   & Ez \\
        -E_x &  0  & -B_z & B_y \\
        -E_y & B_z & 0 &-B_x \\
        -E_z & -B_y & B_x & 0  
    \end{matrix}\right)
\end{equation*}
\item Formule la matriz de rotación respecto a $\hat{z}$ por medio de la transformación de Lorentz partiendo de la invarianza de $S^2$ se obtiene la 
forma (dependiente de 6 parámetros) de L, tal que $A=e^L$ es la transformación de Lorentz.\\
Se tiene la base, $S_\mu , K_\mu$, donde $L=-\vec{\omega}\cdot \vec{s}- \vec{\xi}\cdot \vec{k}$, para rotar con respecto $\hat{z}$, se tiene que cumplir que:
$\vec{\xi}=0 , \vec{\omega}=\omega_z \hat{z}$, entonces:
\begin{equation*}
    L=-\vec{\omega}\cdot \vec{s}=-\omega_z s_3
\end{equation*}
por lo tanto:
\begin{align*}
    A&=e^L \\
    &=e^{-\omega s_3}\\
    &=\sum_{i=0}^\infty \frac{(-1)^i(\omega s_3)^i}{i!} 
\end{align*}
donde se cumple que: 
\begin{align*}
    s^3_3&=-s_3\\
s^4_3&=-s_3^2\\
s_3^5&=s_3
\end{align*}
entonces:
\begin{align*}
    A&=1-\omega s_3 + \frac{\omega^2}{2!}s_3^2 + \frac{\omega^3}{3!}s_3 - \frac{\omega^4}{4!}s_3^2\\
    &=(1+s_3^2)-s_3^2\left[1-\frac{\omega^2}{2!}+\frac{\omega^4}{4!}-\dots \right] - s_3 \left[\omega - \frac{\omega^3}{3!}+\frac{\omega^5}{5!}-\dots \right]\\
    &= \left(\begin{matrix}
        1 & 0 & 0 & 0 \\
        0 & 0 & 0 & 0 \\
        0 & 0 & 0 & 0 \\
        0 & 0 & 0 & 1 \\
    \end{matrix}\right) - \left(\begin{matrix}
        0 & 0 & 0 & 0 \\
        0 &-1 & 0 & 0 \\
        0 & 0 & -1 & 0 \\
        0 & 0 & 0 & 0 \\
    \end{matrix}\right) cos(\omega) - \left(\begin{matrix}
        0 & 0 & 0 & 0 \\
        0 & 0 & -1 & 0 \\
        0 & 1 & 0 & 0 \\
        0 & 0 & 0 & 0 \\
    \end{matrix}\right) sen(\omega)
\end{align*}
entonces:
\begin{equation*}
    A=\left(\begin{matrix}
        1 & 0 & 0 & 0 \\
        0 & cos(\omega) & sen(\omega) & 0 \\
        0 & -sen(\omega) & cos(\omega) & 0 \\
        0 & 0 & 0 & 1 \\
    \end{matrix}\right) 
\end{equation*}
\end{enumerate}
\end{document}