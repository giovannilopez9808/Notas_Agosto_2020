\documentclass[12pt,letterpaper]{report}
\usepackage{graphicx}
\usepackage{scrextend}
\usepackage{vmargin}
\usepackage{graphicx}
\usepackage{multirow}
\usepackage[utf8]{inputenc}
\usepackage[spanish]{babel}
\usepackage{multicol}
\usepackage{enumerate}
\usepackage{float}
\usepackage{amsmath, amsthm, amssymb, amsfonts}
\usepackage[usenames]{color}
\parindent=0mm
\pagestyle{empty}
\definecolor{miorange}{rgb}{0.91, 0.43, 0.0}
\begin{document}
\setmargins{2.5cm}      
{1.5cm}                     
{2cm}  
{24cm}                    
{10pt}                          
{1cm}                          
{0pt}                             
{2cm}
\begin{titlepage}
\begin{center}
\includegraphics[scale=0.40]{../../Logos/uanl.png} 
\hspace{2.5cm}
\includegraphics[scale=0.40]{../../Logos/fcfm.png}
\end{center}
\vspace{2cm}
\begin{center}
\textbf{
UNIVERSIDAD AUTÓNOMA DE NUEVO LEÓN\\
FACULTAD DE CIENCIAS
    FÍSICO MATEMÁTICAS}\\
\vspace*{2cm}
\begin{large}
\vspace{1cm}
\large{\textbf{Tópicos de Mécanica Cuántica}}\\
\textbf{Tarea 1: Operador P$_{z}$ sobre el Hamiltoniano}\\
Enrique Valbuena Ordonez\\
\end{large}
\end{center}
\end{titlepage}
\section*{Transformaciones de Lorentz}
\vspace*{-1cm}
\begin{flushright}
    \today
\end{flushright}
\subsection*{Transformaciones de Galileo}
\begin{equation}
    \vec{x}=\vec{x}'+\vec{R}t 
\end{equation}
\begin{equation}
    \vec{v}=\vec{v}'+\vec{v}_f 
\end{equation}
\begin{equation}
    \vec{a}=\vec{a}'
\end{equation}
Ejemplo: Sea un avión con $|\vec{v}|=C=|\vec{v}_f|$\\
Por lo que:
\begin{align*}
    v&=v'+v_R\\
     &=c+c\\
     &=2c
\end{align*}
Sean $\theta$ y $\theta'$ S,R,I\\
En $\theta$, un evento ocurre en $x,y,z,t$\\
En $\theta'$, un evento ocurre en $x',y',z',t'$\\
\begin{equation*}
    x,y,z,t \leftrightarrow x',y',z',t'
\end{equation*}
¿Cuál es la relación entre los parámetros?\\
\begin{eqnarray*}
x'=\gamma (x-vt)\\
y'=y\\
z'=z\\
t'=\gamma(t-\frac{v}{c^2}x)\\
\gamma=\frac{1}{\sqrt{1-\frac{v^2}{c^2}}}
\end{eqnarray*}
Las transformaciones inversas
\begin{eqnarray*}
    x=\gamma(x'+vt')\\
    y=y'
    z=z'
    t=\gamma(t'+\frac{v}{c^2}x')
\end{eqnarray*}
Las transformaciones de Lorentz (deducción)\\
En t=0, los orígenes coinciden, se emite un pulso luminoso, en el orígen de $\theta \Rightarrow \theta$ y $\theta'$ observaran un cascarón esférico de radiación exponiendose hacia afuera, con la misma rapidez c.\\
En el sistema $\theta$, el fuente de onda, alcanza un punto P, de coordenadas $(x,y,z)$ y se satisface 
\begin{equation*}
    c^2t^2-(x^2+y^2+z^2)=0
\end{equation*}
En el sistema $\theta'$, se satisface
\begin{equation*}
    c^{2}t^{'2}-(x^{'2}+y^{'2}+z^{'2})=0
\end{equation*}
El espacio-tiempo es homogene e isotrópico, la conexión, transformación, entre $(x,y,z,t)$ y $(x',y',z',t')$ es lineal\\
\begin{minipage}{0.5\linewidth}
\begin{eqnarray*}
    x'=x\\
    y'y
\end{eqnarray*}
\end{minipage}
\begin{minipage}{0.5\linewidth}
\begin{eqnarray*}
    z=z'(z,t',v,c)\\
    t=t'(z',t',v,c)
\end{eqnarray*}
\end{minipage}
Se propone que:
\begin{eqnarray}
    \label{sistema}
    z'=Az+Bt\\
    t'=Gz+Dt
    \label{sistemas}
\end{eqnarray}
Determinar A,B,D,G.\\
El sistema $\theta$ se encuentra en reposo, z=0
\begin{eqnarray}
    z'=A(0)+Bt \rightarrow z'=Bt \\
    t'=G(0)+Dt \rightarrow t'=Dt
\end{eqnarray}
derivamos con respecto $t'$
\begin{align*}
    \frac{dz'}{dt'} &=\frac{d}{dt'}[Bt]\\
                    &=B\frac{dt}{dt'}\\
                    & =\frac{B}{D}\\
                v'  &=\frac{B}{D}
\end{align*}
\begin{equation}
    -v=\frac{B}{D}
    \label{-v}
\end{equation}
$\theta'$ esta en reposo, con respecto a si mismo $z'=0$\\
Derivamos $Az+Bt=0$ con respecto a t
\begin{equation}
    A\frac{dz}{dt}+B=0 \rightarrow A\gamma +B=0
    \label{derivada}
\end{equation}
Usamos \ref{-v} en \ref{derivada}:
\begin{equation}
    A=D
    \label{A=D}
\end{equation}
por ende:
\begin{equation}
    B=-vA
    \label{B=-vA}
\end{equation}
usando \ref{A=D} y \ref{B=-vA} en \ref{sistemas} y \ref{sistema}
\begin{eqnarray}
    \label{z}
    z'=A[z-vt]\\
    t'=A=\left[t+\frac{c}{A}z \right]
    \label{tt}
\end{eqnarray}
En $t=0=t'$ y $z=z'$, se emite el pulso y se cumplen las metricas, por ende
\begin{equation}
    (ct)^2-z^2=(ct')^2-z^{'2}
    \label{ct}
\end{equation}
Usando las transformaciones \ref{z}, \ref{tt} y \ref{ct} se obtiene entonces\\
\begin{minipage}{0.5\linewidth}
\begin{eqnarray*}
A=\frac{1}{\sqrt{1-\frac{v^2}{c^2}}}\\
t'=\frac{t-z\frac{v}{c^2}}{\sqrt{1-\frac{v^2}{c^2}}}
\end{eqnarray*}
\end{minipage}
\begin{minipage}{0.5\linewidth}
\begin{eqnarray*}
G=\frac{-\frac{v}{c^2}}{\sqrt{1-\frac{v^2}{c^2}}}\\
z'=\frac{z-vt}{\sqrt{1-\frac{v^2}{c^2}}}
\end{eqnarray*}
\end{minipage}
Donde $\beta^2=\frac{v^2}{c^2}$\\
\begin{minipage}{0.5\linewidth}
\begin{eqnarray*}
x_0=ct\\
x_1=z\\
x_2=x\\
x_3=y
\end{eqnarray*}
\end{minipage}
\begin{minipage}{0.5\linewidth}
\begin{eqnarray*}
x_0'=\gamma[x_0-\beta x_1]\\
x_1'=\gamma[x_1-\beta x_0]\\
x_2'=x_2\\
x_3'=x_3
\end{eqnarray*}
\end{minipage}
Se define que
\begin{equation}
    \beta=Tanh(\varsigma)
    \label{beta}
\end{equation}
ya que $\beta=\frac{v}{c}$
\begin{equation}
    \gamma=[1-Tanh^2\varsigma]^{-\frac{1}{2}}
    \label{gamma}
\end{equation}
entonces:
\begin{eqnarray*}
    x_0'=x_0Cosh \varsigma - x_1 Senh \varsigma \\
    x_1'=-x_0Senh \varsigma +x_1Cosh\varsigma
\end{eqnarray*}
$\varsigma \rightarrow$ Boost parameter 
\pagebreak
\end{document}