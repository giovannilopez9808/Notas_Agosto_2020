\documentclass[12pt,letterpaper]{report}
\usepackage{graphicx}
\usepackage{scrextend}
\usepackage{vmargin}
\usepackage{graphicx}
\usepackage{multirow}
\usepackage[utf8]{inputenc}
\usepackage[spanish]{babel}
\usepackage{multicol}
\usepackage{enumerate}
\usepackage{float}
\usepackage{amsmath, amsthm, amssymb, amsfonts}
\usepackage[usenames]{color}
\parindent=0mm
\pagestyle{empty}
\definecolor{miorange}{rgb}{0.91, 0.43, 0.0}
\begin{document}
\setmargins{2.5cm}      
{1.5cm}                     
{2cm}  
{24cm}                    
{10pt}                          
{1cm}                          
{0pt}                             
{2cm}
\begin{titlepage}
\begin{center}
\includegraphics[scale=0.40]{../../Logos/uanl.png} 
\hspace{2.5cm}
\includegraphics[scale=0.40]{../../Logos/fcfm.png}
\end{center}
\vspace{2cm}
\begin{center}
\textbf{
UNIVERSIDAD AUTÓNOMA DE NUEVO LEÓN\\
FACULTAD DE CIENCIAS
    FÍSICO MATEMÁTICAS}\\
\vspace*{2cm}
\begin{large}
\vspace{1cm}
\large{\textbf{Mécaninca Cuántica Relativista}}\\
\textbf{Examen parcial 1}\\
Francisco Baez\\
\end{large}
\end{center}
\vspace{3.5cm}
\begin{center}
\begin{minipage}{0.6\linewidth}
\vspace{0.5cm}
\changefontsizes{14pt}
Nombre:\\
Giovanni Gamaliel López Padilla\\
\end{minipage}
\begin{minipage}{0.2\linewidth}
\changefontsizes{14pt}
Matricula:\\
1837522
\end{minipage}
\end{center}
\vspace{4cm}
\begin{flushright}
\today
\end{flushright}
\end{titlepage}
\begin{enumerate}
    \item Sean las transformaciones de Lorentz
    \begin{align*}
        {A}'_0&=\gamma (A_0-\vec{\beta}\cdot \vec{A})\\
        {A}'_\parallel&=\gamma (A_\parallel - \beta A_0)\\
        {A}'_\perp &= A_\perp
        \end{align*}
    Mostrar que el producto escalar tensorial es un invariante, es decir, \\${A}'_0{B}'_0-\vec{A}'\cdot\vec{B}'={A}_0{B}_0-\vec{A}\cdot\vec{B}$
    \item Sea un sistema ${k}'$ que se mueve en el eje $x_i$ con una velocidad de $\nu = c\beta$ con respecto a K. Mostrar que la velocidad
    perpendicular a la dirección de movimiento ${k}'$ vistas en el sistema K es:
    \begin{equation*}
        U_\perp=\frac{u_i}{\gamma_v \left[1+\frac{\upsilon \cdot v}{c^2} \right]}
    \end{equation*}
    De las transformaciones:
    \begin{align*}
        r_\parallel &= \gamma_v [{r_\parallel}'+v{t}']\\
        r_\perp &= {r_\perp}'\\
        t&= \gamma_v \left[ {t}'+\frac{\upsilon \cdot v}{c^2}\right]
    \end{align*}
    tomando los diferenciales:
    \begin{align*}
        dr_\parallel &= \gamma_v \left[{dr}'_\parallel + vdt\right]\\
        dr_\perp &= {dr}'_perp\\
        dt&=\gamma_v \left[dt+ \frac{vdr}{c^2} \right]
    \end{align*}
    entonces:
    \begin{align*}
        \frac{dr_\perp}{dt}&= \frac{{dr}'_\perp}{\gamma_v {dt}'\left[1+\frac{v}{c^2} \frac{{dr}'}{{dt}'} \right]}\\
        u_\perp&= \frac{{dr}'_\perp}{\gamma_v {dt}' \left[1+\frac{v}{c^2}\frac{{dr}'}{{dt}'} \right]}\\
        & = \frac{{u}'_\perp}{\gamma_v \left[1+\frac{v\cdot {u}'}{c^2} \right]}
    \end{align*}
    por lo tanto:
    \begin{equation*}
        u_\perp = \frac{{u}'_\perp}{\gamma_v \left[1+\frac{v\cdot {u}'}{c^2} \right]}
    \end{equation*}
    \item Sea el tensor de primer rango contravariante $A^\mu=(A^0,\vec{A})$. Expresar las componentes del tensor de primer rango contravariante
    $A_\mu$ en términos de las componentes de $A^\mu$.\\
    La transformación de un tensor contracovariante a covariante es la siguiente:
    \begin{equation*}
        A_\mu = g_{\gamma \mu} A^\gamma 
    \end{equation*}
    donde $g=diag(1,-1,-1,-1)$, entonces:
    \begin{align*}
        \left(\begin{matrix}
            A_0 \\ A_1 \\ A_2 \\ A_3 
        \end{matrix}\right)&= 
        \left(\begin{matrix}
        1 & 0 & 0 & 0 \\
        0 & -1 & 0 & 0 \\
        0 & 0 & -1 & 0 \\
        0 & 0 & 0 & -1 \\
        \end{matrix}\right)\left(
           \begin{matrix}          
        A^0 \\ A^1 \\ A^2 \\ A^3 \end{matrix}
        \right)\\
        & =\left( \begin{matrix}
            A^0 \\ -A^1 \\ -A^2 \\ -A^3
        \end{matrix}\right)
    \end{align*}
    por lo tanto:
    \begin{align*}
        A_0 &= A^0 \\
        A_1 &= -A^1 \\
        A_2 &= -A^2 \\
        A_3 &= -A^3 \\
    \end{align*}
    \item Sea el tensor métrico en una representación matricial en la notación tradicional diag(1,-1,-1,-1) y sea L la matriz
    sin traza tal que se cumple que $(Lg)^T=-gL$, donde T significa la transpuesta. Mostrar que la forma matricial de L es:
    \begin{equation*}
        L=\left(\begin{matrix}
            0 & L_{01} & L_{02} & L_{03} \\
            L_{01} & 0 & L_{12} & L_{13} \\
            L_{02} & -L_{12} & 0 & L_{23} \\
            L_{03} & -L_{13} & -L_{23} & 0 \\
        \end{matrix}\right)
    \end{equation*}
    tomando en cuenta que:
    \begin{equation*}
        g=diag(1,-1,-1,-1)
    \end{equation*}
    y que
    \begin{equation*}
        g^T=g=g^{-1}
    \end{equation*}
    por lo tanto:
    \begin{align*}
        (Lg)^T&=-gL\\
        gL^T&=-gL\\
        g(gL^T)&=g(-gL)\\
        (gg)L^T&=-(gg)L\\
        L^T=-L
    \end{align*}
    por lo tanto:
    \begin{equation*}
        L_{ji}=-L_{ij}
    \end{equation*}
    si $i=j$, entonces $L_{ii}=0$
\item Sea la matriz de transformación de Lorentz más general dada por $A=e^{\vec{\omega}\cdot\vec{s}-\vec{\xi}\cdot  \vec{k}}$, donde 
$\vec{\omega},\vec{\xi}$ contienen seis parámetros de la transformación general de Lorentz y $\vec{S},\vec{K}$ son los generadores de las transformaciones
de Lorentz. Determina la matriz de transformación de Lorentz para $\vec{\xi}=0$ y $\vec{\omega}=\omega \hat{z}$.\\
Se tiene la base, $S_\mu , K_\mu$, donde $L=-\vec{\omega}\cdot \vec{s}- \vec{\xi}\cdot \vec{k}$, para rotar con respecto $\hat{z}$, se tiene que cumplir que:
$\vec{\xi}=0 , \vec{\omega}=\omega_z \hat{z}$, entonces:
\begin{equation*}
    L=-\vec{\omega}\cdot \vec{s}=-\omega_z s_3
\end{equation*}
por lo tanto:
\begin{align*}
    A&=e^L \\
    &=e^{-\omega s_3}\\
    &=\sum_{i=0}^\infty \frac{(-1)^i(\omega s_3)^i}{i!} 
\end{align*}
donde se cumple que: 
\begin{align*}
    s^3_3&=-s_3\\
s^4_3&=-s_3^2\\
s_3^5&=s_3
\end{align*}
entonces:
\begin{align*}
    A&=1-\omega s_3 + \frac{\omega^2}{2!}s_3^2 + \frac{\omega^3}{3!}s_3 - \frac{\omega^4}{4!}s_3^2\\
    &=(1+s_3^2)-s_3^2\left[1-\frac{\omega^2}{2!}+\frac{\omega^4}{4!}-\dots \right] - s_3 \left[\omega - \frac{\omega^3}{3!}+\frac{\omega^5}{5!}-\dots \right]\\
    &= \left(\begin{matrix}
        1 & 0 & 0 & 0 \\
        0 & 0 & 0 & 0 \\
        0 & 0 & 0 & 0 \\
        0 & 0 & 0 & 1 \\
    \end{matrix}\right) - \left(\begin{matrix}
        0 & 0 & 0 & 0 \\
        0 &-1 & 0 & 0 \\
        0 & 0 & -1 & 0 \\
        0 & 0 & 0 & 0 \\
    \end{matrix}\right) cos(\omega) - \left(\begin{matrix}
        0 & 0 & 0 & 0 \\
        0 & 0 & -1 & 0 \\
        0 & 1 & 0 & 0 \\
        0 & 0 & 0 & 0 \\
    \end{matrix}\right) sen(\omega)
\end{align*}
entonces:
\begin{equation*}
    A=\left(\begin{matrix}
        1 & 0 & 0 & 0 \\
        0 & cos(\omega) & sen(\omega) & 0 \\
        0 & -sen(\omega) & cos(\omega) & 0 \\
        0 & 0 & 0 & 1 \\
    \end{matrix}\right) 
\end{equation*}
\end{enumerate}
\end{document}