\section{Resumen}
En este trabajo se presenta la simulación de una cadena de moleculas de carbono que se encuentran bajo los efectos del potencial
de Lennard-Jones y el potencial no lineal de atracción finita (FENE). Se realizo la misma simulación un total de nueve veces, en cada una
con una configuración inicial diferente en la cual se monitorizo la temperatura, energía potencial y la posición de cada moleucula a lo largo de 
cada paso de la simulación. Al analizar el comportamieno de cada ente se observo que la tendencia de la energía potencial y la temperatura son muy semejantes 
llegando a tener una desviación estandar de $6.65x10^{-4}$ en la energía potencial y $5.40x10^{-4}$ en la temperatura, es por ello que apesar que tengan
una gran diferencia en el movimiento de cada partícula, al analizar aspectos macroscopicos no una diferencia notable entre simulaciones.
\textbf{Palabras clave:} FENE, Lennard-Jones, bidimensional, temperatura, energía potencial,cadenas de moleculas.