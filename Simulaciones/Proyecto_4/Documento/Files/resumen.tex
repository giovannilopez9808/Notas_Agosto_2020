\preprint{APS/123-QED}
\begin{abstract}
    En este trabajo se presenta la simulación de una red cuadrangular de átomos de carbono para diferentes densidades bajo la interacción del potencial de 
    Lennard-Jones y un termosatato de Andersen. Se realizó un monitoreo a la energía cinetica, energía potencial, energía total, la temperatura a 
    y la velocidad de cada particula lo largo del sistema para observar su comportamiento. Esta ultima se logro apreciar que describe una distribución normal
    ,ya que fue afectada por el termosatato de Andersen. Al termino de cada simulación se obtuvo la distribución radial observando que los átomos
    tienden a centrarse a un radio de 1 presentado fluctuaciones en distancias largas.\\
\textbf{Palabras clave:} Termostato de Andersen, Lennard-Jones, bidimensional, temperatura, distribución normal
\end{abstract}