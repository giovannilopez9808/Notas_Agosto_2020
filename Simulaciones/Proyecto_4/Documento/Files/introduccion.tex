\section{Introducción}
La simualación de sistemas moleculares es de gran ayuda para poder analizar estrucuturas y obtener información como tensiones, fuerzas
que resiste el material o los efectos en el sistema cuando esta bajo ciertos ambientes o potenciales de interacción. En algunos casos
se busca que el sistema se encuentre a una tempereratura constante para tener un equilibrio termodinámico o mantener el sistema con esa energía interna,
para lograr esto se implementan termostatos en el sistema. La simualación de estos termostatos de manera numérica puede ser implementada de
diversas formas, una de ellas es el termostato de Andersen.