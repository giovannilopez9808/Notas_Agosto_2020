\section{Código}
\begin{itemize}
    \item \href{https://github.com/giovannilopez9808/Notas_Agosto_2020/blob/master/Simulaciones/Proyecto_4/Scripts/MD-n2.f}{MD-n2.f}\\
    Este código realiza la simulación de cada sistema tomando en cuenta al potencial de Lennard-Jones y al potencial FENE.
    \item \href{https://github.com/giovannilopez9808/Notas_Agosto_2020/blob/master/Simulaciones/Proyecto_4/Scripts/run.py}{run.py}\\
    Este código es en el encargado de ejecutar al modelo varias veces automaticamente.
    \item \href{https://github.com/giovannilopez9808/Notas_Agosto_2020/blob/master/Simulaciones/Proyecto_4/Scripts/Potencial_Graphics.py}{Potencial\_Graphics.py}\\
    Este código realiza el calculo de los potenciales y fuerzas de Lennard-Jones y FENE para generar las figuras \ref{fig:pot-len-jones} y \ref{fig:force-len-jones}.
    \item \href{https://github.com/giovannilopez9808/Notas_Agosto_2020/blob/master/Simulaciones/Proyecto_4_2/Scripts/Dim_Graphics.py}{Dim\_Graphics.py}\\
    Este código realiza cuatro capturas de la dinámica de una simulación para generar la figura \ref{fig:dim}.
    \item \href{https://github.com/giovannilopez9808/Notas_Agosto_2020/blob/master/Simulaciones/Proyecto_4/Scripts/temp_graphics.py}{temp\_graphics.py}\\
    Este código realiza el calculo de la temperatura para generar la figura \ref{fig:temp}.
    \item \href{https://github.com/giovannilopez9808/Notas_Agosto_2020/blob/master/Simulaciones/Proyecto_4_2/Scripts/Energy_Graphics.py}{Energy\_Graphics.py}\\
    Este código realiza la captura de la energía potencial de cada simulación para generar la figura \ref{fig:energy}.
    \item \href{https://github.com/giovannilopez9808/Notas_Agosto_2020/blob/master/Simulaciones/Proyecto_4_2/Scripts/dim_gif.py}{Dim\_gif.py}\\
    Este código realiza una animación de la dinámica de cada simulación.
    \item \href{https://github.com/giovannilopez9808/Notas_Agosto_2020/blob/master/Simulaciones/Proyecto_4_2/Scripts/dis_mean.py}{dis\_mean.py}\\
    Este código realiza el calculo promedio de la distancia entre el primer y último átomo, creando así la figura \ref{eq:dis}.
\end{itemize}