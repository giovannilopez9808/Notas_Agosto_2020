\section{Resumen}
En este trabajo se presenta la simulación de una cadena de átomos de carbono que se encuentran bajo los efectos del potencial
de Lennard-Jones y el potencial no lineal de atracción finita (FENE). Se realizo la misma simulación un total de nueve veces, en cada una
con una configuración inicial diferente en la cual se monitorizo la temperatura, energía potencial y la posición de cada moleucula a lo largo de 
cada paso de la simulación. Al analizar el comportamieno de cada ente se observo que la tendencia de la energía potencial y la temperatura es ser oscilantes
en el tiempo, se observa también que conforme se aumentan el número de átomos en la simulación, las graficas se definen más, llegando a tener la apariencia de una onda la cual es 
amortiguada, esto nos puede dar indicios a que la cadena de átomos llegara a estar en equilibrio después de un tiempo. Observando los promedios de la distancia entre el primer y último átomo de la simulación
se observa que esta sigue una tendencia ascendente con una semejanza a una función logaritmica.\\
\textbf{Palabras clave:} FENE, Lennard-Jones, bidimensional, temperatura, energía potencial,cadenas de átomos.