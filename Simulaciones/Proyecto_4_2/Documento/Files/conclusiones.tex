\section{Conclusiones}
Las simulaciones de cadenas de átomos afectadas por los potenciales de Lennard-Jones y el potencial FENE
presentan situaciones caóticas, esto debido a que aun que sean mínimas las diferencias entre posiciones y velocidades iniciales de 
cada molecua esta lleva a presentar una dinámica muy diferente, en cambio la temperatura y la energía potencial 
presentan una evolución oscilante independientemente de cual sean las condiciones iniciales, se observó que conforme aumenta la cantidad de 
átomos en la simulación, esta oscilación en las energías es más visible y conforme se aumentan los pasos el valor máximo de la oscilación va disminuyendo
haciendo notar que esta en algún momento llegara a un estado de equilibrio.\\
Visualizando las diferentes animaciones de cada simulación se aprecia como esta presenta un movimiento ondulatirio pareciendose casi como una cuerda atada de los extremos
creando efectos de ondas constructivas y destructivas.