\section{Conclusiones}
    La implemententación del termostato de Andersen es eficiente para obtener un sistema en equilibro termodinamico, pero la distribución radial
    se ve afectada, ya que no presenta valores definidos en los cuales tenga curvas suaves.
    En cambio, si lo que se prefiere es que las velocidades de las particulas formen una distribución uniforme el termostato de Andersen podrá
    realizarlo.\\
    La estabilidad en las energías no es algo de esperarse de este contexto, ya que el cambio en las velocidades puede ser drastico y con ello formar
    pendientes altas en esta función.