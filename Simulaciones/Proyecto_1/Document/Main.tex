\documentclass[12pt,letterpaper]{article}
\usepackage{graphicx}
\usepackage{scrextend}
\usepackage{vmargin}
\usepackage{graphicx}
\usepackage{multirow}
\usepackage[utf8]{inputenc}
\usepackage[spanish]{babel}
\usepackage{multicol}
\usepackage{enumerate}
\usepackage{float}
\usepackage{amsmath, amsthm, amssymb, amsfonts}
\usepackage[usenames]{color}
\usepackage[breaklinks=true,hidelinks]{hyperref}
\spanishdecimal{.}
\parindent=0mm
\pagestyle{empty}
\definecolor{miorange}{rgb}{0.91, 0.43, 0.0}
\begin{document}
\setmargins{2.5cm}      
{1.5cm}                     
{2cm}  
{24cm}                    
{10pt}                          
{1cm}                          
{0pt}                             
{2cm}
\begin{titlepage}
\begin{center}
\includegraphics[scale=0.40]{../../../Logos/uanl.png} 
\hspace{2.5cm}
\includegraphics[scale=0.40]{../../../Logos/fcfm.png}
\end{center}
\vspace{2cm}
\begin{center}
\textbf{
UNIVERSIDAD AUTÓNOMA DE NUEVO LEÓN\\
FACULTAD DE CIENCIAS
FÍSICO MATEMÁTICAS}\\
\vspace*{2cm}
\begin{large}
\vspace{1cm}
\large{\textbf{Simuladores Moleculares}}\\
\textbf{--}\\
Omar Gonzalez Amezcua\\
\end{large}
\vspace{3.5cm}
\begin{minipage}{0.6\linewidth}
\vspace{0.5cm}
\changefontsizes{14pt}
Nombre:\\
Giovanni Gamaliel López Padilla\\
\end{minipage}
\begin{minipage}{0.2\linewidth}
\changefontsizes{14pt}
Matricula:\\
1837522
\end{minipage}
\end{center}
\vspace{4cm}
\begin{flushright}
\today
\end{flushright}
\end{titlepage}
\begin{multicols}{2}
\section*{Resumen}
\subsection*{Palabras clave}
\section*{Introducción}
\section*{Objetivo general}
\section*{Objetivo específico}
\section*{Marco teórico}
\section*{Resultados}
\section*{Conclusiones}
El planteamiento que habían realizado Rayleigh y Jeans sobre que la energía estaba distribuida de manera uniforme era erronea y que el cambio que realizo Plank al proponer la distribución de Boltzmann esto debido a los limites de la energía media fue lo que ayudo a que el problema de la radiación de cuerpo negro sea solucionado y corroborado por mediciones experimentales.
\section*{Código}
\begin{itemize}
\item \href{https://github.com/giovannilopez9808/Notas_Agosto_2020/blob/master/AMC/Reto1/black_body.py}{Github - black\_body.py}\\
Este código realiza el fit de los datos de COBE guardados en el archivo  \textit{data.txt} y crea la figura \ref{fit}
\item \href{https://github.com/giovannilopez9808/Notas_Agosto_2020/blob/master/AMC/Reto1/wien_law.py}{Github - wien\_law.py}\\
Este código realiza el calculo del espectro electromagnético de un cuerpo negro para diferentes temperaturas, localiza sus picos y en base a ello opera una regresión lineal para obtener la constante de Wien.
\end{itemize}
\bibliographystyle{plain}
\bibliography{Main}
\nocite{*}
\end{multicols}
\end{document}