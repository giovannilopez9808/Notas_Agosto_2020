\documentclass[12pt,letterpaper]{report}
\usepackage{graphicx}
\usepackage{scrextend}
\usepackage{vmargin}
\usepackage{graphicx}
\usepackage{multirow}
\usepackage[utf8]{inputenc}
\usepackage[spanish]{babel}
\usepackage{multicol}
\usepackage{enumerate}
\usepackage{float}
\usepackage{amsmath, amsthm, amssymb, amsfonts}
\usepackage[usenames]{color}
\parindent=0mm
\spanishdecimal{.}
\pagestyle{empty}
\definecolor{miorange}{rgb}{0.91, 0.43, 0.0}
\begin{document}
\setmargins{2.5cm}      
{1.5cm}                     
{2cm}  
{24cm}                    
{10pt}                          
{1cm}                          
{0pt}                             
{2cm}
\begin{titlepage}
\begin{center}
\includegraphics[scale=0.40]{../../Logos/uanl.png} 
\hspace{2.5cm}
\includegraphics[scale=0.40]{../../Logos/fcfm.png}
\end{center}
\vspace{2cm}
\begin{center}
\textbf{
UNIVERSIDAD AUTÓNOMA DE NUEVO LEÓN\\
FACULTAD DE CIENCIAS
    FÍSICO MATEMÁTICAS}\\
\vspace*{2cm}
\begin{large}
\vspace{1cm}
\large{\textbf{Relatividad General}}\\
\textbf{Dilatación del tiempo}\\
Carlos Luna Criado\\
\end{large}
\vspace{3.5cm}
\begin{minipage}{0.6\linewidth}
\vspace{0.5cm}
\changefontsizes{14pt}
Nombre:\\
Giovanni Gamaliel López Padilla\\
\end{minipage}
\begin{minipage}{0.2\linewidth}
\changefontsizes{14pt}
Matricula:\\
1837522
\end{minipage}
\end{center}
\vspace{4cm}
\begin{flushright}
\today
\end{flushright}
\end{titlepage}
Un cohete sale de la Tierra a una velocidad de $\frac{3}{5}c$. Cuando un reloj en el cohete indica que ha pasado 1 hora desde haber abandonado la Tierra, el cohete envía una señal luminosa a la Tierra.
\begin{enumerate}[I)]
    \item De acuerdo con los relojes de tierra, ¿cuándo se envió la señal?\\
    De acuerdo con la ecuación:
    \begin{equation}
        \label{t'}
        {t}'=t\sqrt{1-\frac{v^2}{c^2}}
    \end{equation}
    Y sustituyendo la velocidad del cohete en la ecuación \ref{t'} se tiene lo siguiente:
    \begin{align*}
        {t}'&= t\sqrt{1-\frac{v^2}{c^2}}\\
            &= t\sqrt{1-\frac{3^2c^2}{5^2c^2}}\\
            &= t\sqrt{1-\frac{9}{25}}\\
            &= t\sqrt{\frac{16}{25}}\\
            &=\frac{4}{5}t
    \end{align*}
    por ende:
    \begin{equation}
        \label{t'resul}
        {t}'=\frac{4}{5}t
    \end{equation}
    por lo que el tiempo en tierra fue de:
    \begin{align*}
        {t}'&=\frac{4}{5}t\\
        t   &=\frac{5}{4}{t}'\\
            &=\frac{5}{4} hr\\
            & = 1.25 hr
    \end{align*}
    \begin{equation}
        \label{ttierra}
        t=1.25 hr
    \end{equation}
    \item Según los relojes de tierra, ¿cuánto tiempo después de que saliera el cohete, regresó la señal a la Tierra?\\
    Considerando que el cohete salio con una velocidad constante de $\frac{3}{5}c$, se puede calcular la distancia que ha recorrido en una hora:
    \begin{align*}
        d   &=vt\\
            &=\left(\frac{3}{5}c\right)\left(60^2 seg\right)\\
            &=6.4755x10^{11} km 
    \end{align*}
    con esta distancia calcularemos el tiempo en el la señal la reccore, tomando en cuenta que su velocidad es $c$.
    \begin{align*}
        t   &= \frac{d}{v}\\
            &=\frac{60^2 c(3)}{5c}\\
            &=2160 seg\\
            & = 0.6 hr
    \end{align*}
    por lo que el tiempo que le tomo fue 
    \begin{equation}
        \label{tluz}
        t= 0.6 hr
    \end{equation}
    entonces, el tiempo que se observo en la tierra es la suma de la expresión \ref{ttierra} y \ref{tluz}, que es:
    \begin{align*}
        t_{total}   &=t_{luz}+t_{tierra}\\
                    &=0.6+1.25\\
                    &=1.85 hr
    \end{align*}
    por lo que llegamos al resultado de $t_{total}=1.85 hr$
    \item Según el observador del cohete, ¿cuánto tiempo después de la partida del cohete llegó la señal a la Tierra?\\
    Calculando la distancia recorrida por el cohete, es la siguiente:\\
    \begin{align*}
        {d}'&=d\sqrt{1-\frac{v^2}{c^2}}\\
            &=\left(6.4755x10^{11} km  \right)\sqrt{1-\frac{(3c)^2}{(5c)^2}}\\
            &=\left(6.4755x10^{11} km  \right)\sqrt{1-\frac{9}{25}}\\
            &=\left(6.4755x10^{11} km  \right)\sqrt{\frac{16}{25}}\\
            &=\left(6.4755x10^{11} km  \right)\left(\frac{4}{5}\right)\\
            &=5.1804x10^{11} km
    \end{align*}
    por lo que calculando el tiempo que recorrio la distancia ${d}'$ el haz, se tiene que:
    \begin{align*}
        {t}'&=\frac{{d}'}{v}\\
            &=\frac{{d}'}{c}\\
            &=1727seg\\
            &=0.48hr
    \end{align*}
    por lo que el tiempo en el cual llego la señal después de la partida del cohete es de :
    \begin{align*}
        {t}'_{total}  &=t+{t}'\\
                    &=1hr+0.48hr\\
                    &=1.48 hr
    \end{align*}
\end{enumerate}
\end{document}