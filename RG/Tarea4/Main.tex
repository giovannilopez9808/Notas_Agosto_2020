\documentclass[12pt,letterpaper]{report}
\usepackage{graphicx}
\usepackage{scrextend}
\usepackage{vmargin}
\usepackage{graphicx}
\usepackage{multirow}
\usepackage[utf8]{inputenc}
\usepackage[spanish]{babel}
\usepackage{multicol}
\usepackage{enumerate}
\usepackage{float}
\usepackage{amsmath, amsthm, amssymb, amsfonts}
\usepackage[usenames]{color}
\parindent=0mm
\spanishdecimal{.}
\pagestyle{empty}
\definecolor{miorange}{rgb}{0.91, 0.43, 0.0}
\begin{document}
\setmargins{2.5cm}      
{1.5cm}                     
{2cm}  
{24cm}                    
{10pt}                          
{1cm}                          
{0pt}                             
{2cm}
\begin{titlepage}
\begin{center}
\includegraphics[scale=0.40]{../../Logos/uanl.png} 
\hspace{2.5cm}
\includegraphics[scale=0.40]{../../Logos/fcfm.png}
\end{center}
\vspace{2cm}
\begin{center}
\textbf{
UNIVERSIDAD AUTÓNOMA DE NUEVO LEÓN\\
FACULTAD DE CIENCIAS
    FÍSICO MATEMÁTICAS}\\
\vspace*{2cm}
\begin{large}
\vspace{1cm}
\large{\textbf{Relatividad General}}\\
\textbf{Símbolos de Christoffel}\\
Carlos Luna Criado\\
\end{large}
\vspace{3.5cm}
\begin{minipage}{0.6\linewidth}
\vspace{0.5cm}
\changefontsizes{14pt}
Nombre:\\
Giovanni Gamaliel López Padilla\\
\end{minipage}
\begin{minipage}{0.2\linewidth}
\changefontsizes{14pt}
Matricula:\\
1837522
\end{minipage}
\end{center}
\vspace{4cm}
\begin{flushright}
\today
\end{flushright}
\end{titlepage}
Calcule los símbolos de Christoffel para las coordenadas esfericas $(r,\theta,\phi)$, partiendo del elemento diferencial de desplazamiento:
\begin{equation*}
    dS^2=dr^2+r^2d\theta^2 +r^2\sin^2(\theta)d\phi^2
\end{equation*}
A partir del elemento diferencial de desplazamiento se tiene que el tensor metrico es:
\begin{equation*}
    g_{ij} = \left(\begin{matrix}
        1 & 0 & 0 \\
        0 & r^2 & 0 \\ 
        0 & 0 & r^2\sin^2(\theta)
    \end{matrix}\right), \qquad
    g^{ij} = \left(\begin{matrix}
        1 & 0 & 0 \\
        0 & \frac{1}{r^2} & 0 \\ 
        0 & 0 & \frac{1}{r^2\sin^2(\theta)}
    \end{matrix}\right)
\end{equation*}
Se tiene que los símbolos de Christoffel estan definidos por:
\begin{equation*}
    \Gamma^{i}_{kl} = \frac{1}{2}g^{im} \left(\frac{\partial g_{mk}}{\partial x^l}+\frac{\partial g_{ml}}{\partial x^k}-\frac{\partial g_{kl}}{\partial x^m} \right)
\end{equation*}
como el tensor metrico $g$ es diferente de cero en su diagonal, entonces, los símbolos de Christoffel puede ser distinto de cero para $i=m$, y para $i\neq m$ iguales a cero, calculando para $i=1$, se tiene que:
\begin{equation*}
    \Gamma^{i}_{kl}= \frac{1}{2} g^{ii}\left(\frac{\partial g_{ik}}{\partial x^l}+\frac{\partial g_{il}}{\partial x^k}-\frac{\partial g_{kl}}{\partial x^i} \right)
\end{equation*}
calculando sus permutaciones, se tiene que:\\
\begin{minipage}{0.3\linewidth}
    \begin{align*}
        \Gamma^{1}_{11} &= \frac{1}{2} \left(\frac{\partial 1}{\partial r}+ \frac{\partial 1}{\partial r} - \frac{\partial 1}{\partial r}\right)\\
        &=0\\
        \Gamma^{1}_{12} &= \frac{1}{2} \left(\frac{\partial 1}{\partial \theta}+\frac{\partial 0}{\partial r}-\frac{\partial 0}{\partial r} \right)\\
        &=0\\
        \Gamma^{1}_{13} &= \frac{1}{2} \left(\frac{\partial 1}{\partial \phi}+\frac{\partial 0}{\partial r}-\frac{\partial 0}{\partial r} \right)\\
        &=0\\
    \end{align*}
\end{minipage}
\hspace{0.5cm}
\begin{minipage}{0.3\linewidth}
    \begin{align*}
        \Gamma^{1}_{21} &= \frac{1}{2} \left(\frac{\partial 0}{\partial r}+ \frac{\partial 1}{\partial \theta} - \frac{\partial 1}{\partial r}\right)\\
&=0\\
\Gamma^{1}_{22} &= \frac{1}{2} \left(\frac{\partial 0}{\partial \theta}+\frac{\partial 0}{\partial \theta}-\frac{\partial r^2}{\partial r} \right)\\
&=-r\\
\Gamma^{1}_{23} &= \frac{1}{2} \left(\frac{\partial 0}{\partial \phi}+\frac{\partial 0}{\partial \theta}-\frac{\partial 0}{\partial r} \right)\\
&=0\\
    \end{align*}
\end{minipage}
\hspace{0.5cm}
\begin{minipage}{0.3\linewidth}
    \begin{align*}
        \Gamma^{1}_{31} &= \frac{1}{2} \left(\frac{\partial 0}{\partial r}+ \frac{\partial 1}{\partial \phi} - \frac{\partial 1}{\partial r}\right)\\
        &=0\\
        \Gamma^{1}_{32} &= \frac{1}{2} \left(\frac{\partial 0}{\partial \theta}+\frac{\partial 0}{\partial \phi}-\frac{\partial 0}{\partial r} \right)\\
        &=0\\
        \Gamma^{1}_{33} &= \frac{1}{2} \left(\frac{\partial 0}{\partial \phi}+\frac{\partial 0}{\partial \phi}-\frac{\partial r^2\sin^2(\theta)}{\partial r} \right)\\
        &=-2r\sin^2(\theta)\\  
    \end{align*}
\end{minipage}
calculando para $i=2$, se tiene que:\\
\begin{minipage}{0.3\linewidth}
    \begin{align*}
        \Gamma^{2}_{11} &= \frac{1}{2r^2} \left(\frac{\partial 0}{\partial r}+ \frac{\partial 0}{\partial r} - \frac{\partial 1}{\partial \theta}\right)\\
        &=0\\
        \Gamma^{2}_{12} &= \frac{1}{2r^2} \left(\frac{\partial 0}{\partial \theta}+\frac{\partial r^2}{\partial r}-\frac{\partial 0}{\partial \theta} \right)\\
        &=\frac{1}{r}\\
        \Gamma^{2}_{13} &= \frac{1}{2r^2} \left(\frac{\partial 0}{\partial \phi}+\frac{\partial 0}{\partial r}-\frac{\partial 0}{\partial \theta} \right)\\
        &=0\\
    \end{align*}
\end{minipage}
\hspace{0.25cm}
\begin{minipage}{0.3\linewidth}
    \begin{align*}
        \Gamma^{2}_{21} &= \frac{1}{2r^2} \left(\frac{\partial r^2}{\partial r}+ \frac{\partial 0}{\partial \theta} - \frac{\partial 1}{\partial \theta}\right)\\
&=\frac{1}{r}\\
\Gamma^{2}_{22} &= \frac{1}{2r^2} \left(\frac{\partial r^2}{\partial \theta}+\frac{\partial r^2}{\partial \theta}-\frac{\partial r^2}{\partial \theta} \right)\\
&=0\\
\Gamma^{2}_{23} &= \frac{1}{2r^2} \left(\frac{\partial r^2}{\partial \phi}+\frac{\partial 0}{\partial \theta}-\frac{\partial 0}{\partial \theta} \right)\\
&=0\\
    \end{align*}
\end{minipage}
\hspace{0.25cm}
\begin{minipage}{0.3\linewidth}
    \begin{align*}
        \Gamma^{2}_{31} &= \frac{1}{2r^2} \left(\frac{\partial 0}{\partial r}+ \frac{\partial 0}{\partial \phi} - \frac{\partial 1}{\partial \theta}\right)\\
        &=0\\
        \Gamma^{2}_{32} &= \frac{1}{2r^2} \left(\frac{\partial 0}{\partial \theta}+\frac{\partial r^2}{\partial \phi}-\frac{\partial 0}{\partial \theta} \right)\\
        &=0\\
        \Gamma^{2}_{33} &= \frac{1}{2r^2} \left(\frac{\partial 0}{\partial \phi}+\frac{\partial 0}{\partial \phi}-\frac{\partial r^2\sin^2(\theta)}{\partial \theta} \right)\\
        &=-2\sin(\theta)\cos(\theta)\\  
    \end{align*}
\end{minipage}
calculando para $i=3$, se tiene que:\\
\begin{minipage}{0.3\linewidth}
    \begin{align*}
        \Gamma^{3}_{11} &= \frac{1}{2r^2\sin^2(\theta)} \left(\frac{\partial 0}{\partial r}+ \frac{\partial 0}{\partial r} - \frac{\partial 1}{\partial \phi}\right)\\
        &=0\\
        \Gamma^{3}_{12} &= \frac{1}{2r^2\sin^2(\theta)} \left(\frac{\partial 0}{\partial \theta}+\frac{\partial 0}{\partial r}-\frac{\partial 0}{\partial \phi} \right)\\
        &=0\\
        \Gamma^{3}_{13} &= \frac{1}{2r^2\sin^2(\theta)} \left(\frac{\partial 0}{\partial \phi}+\frac{\partial r^2\sin^2(\theta)}{\partial r}-\frac{\partial 0}{\partial \phi} \right)\\
        &=\frac{1}{r}\\
    \end{align*}
\end{minipage}
\hspace{0.25cm}
\begin{minipage}{0.3\linewidth}
    \begin{align*}
        \Gamma^{3}_{21} &= \frac{1}{2r^2\sin^2(\theta)} \left(\frac{\partial 0}{\partial r}+ \frac{\partial 0}{\partial \theta} - \frac{\partial 1}{\partial \phi}\right)\\
&=0\\
\Gamma^{3}_{22} &= \frac{1}{2r^2\sin^2(\theta)} \left(\frac{\partial 0}{\partial \theta}+\frac{\partial 0}{\partial \theta}-\frac{\partial r^2}{\partial \phi} \right)\\
&=0\\
\Gamma^{3}_{23} &= \frac{1}{2r^2\sin^2(\theta)} \left(\frac{\partial 0}{\partial \phi}+\frac{\partial r^2\sin^2(\theta)}{\partial \theta}-\frac{\partial 0}{\partial \phi} \right)\\
&=\cot(\theta)\\
    \end{align*}
\end{minipage}\\
\begin{minipage}{0.3\linewidth}
    \begin{align*}
        \Gamma^{3}_{31} &= \frac{1}{2r^2\sin^2(\theta)} \left(\frac{\partial r^2\sin^2(\theta)}{\partial r}+ \frac{\partial 0}{\partial \phi} - \frac{\partial 1}{\partial \phi}\right)\\
        &=\frac{1}{r}\\
        \Gamma^{3}_{32} &= \frac{1}{2r^2\sin^2(\theta)} \left(\frac{\partial r^2\sin^2(\theta)}{\partial \theta}+\frac{\partial 0}{\partial \phi}-\frac{\partial 0}{\partial \phi} \right)\\
        &=\cot(\theta)\\
        \Gamma^{3}_{33} &= \frac{1}{2r^2\sin^2(\theta)} \left(\frac{\partial r^2\sin^2(\theta)}{\partial \phi}+\frac{\partial r^2\sin^2(\theta)}{\partial \phi}-\frac{\partial r^2\sin^2(\theta)}{\partial \phi} \right)\\
        &=0\\  
    \end{align*}
\end{minipage}\\
dando asi como resultado que los símbolos de Christoffel diferentes a cero son:
\begin{minipage}{0.3\linewidth}
    \begin{align*}
        \Gamma^{1}_{22} &= -r \\
        \Gamma^1_{33} &= -2r\sin^2(\theta)
    \end{align*}
\end{minipage}
\hspace{0.5cm}
\begin{minipage}{0.3\linewidth}
    \begin{align*}
        \Gamma^2_{12}&= \frac{1}{r}\\
        \Gamma^2_{32}&= \frac{1}{r}\\
        \Gamma^2_{33}&=-2\sin(\theta)\cos(\theta)
    \end{align*}
\end{minipage}
\hspace{0.5cm}
\begin{minipage}{0.3\linewidth}
    \begin{align*}
        \Gamma^3_{13} &= \frac{1}{r}\\
        \Gamma^3_{31}&= \frac{1}{r}\\
         \Gamma^3_{23} &=\cot(\theta)\\
         \Gamma^3_{32} &=\cot(\theta)
    \end{align*}
\end{minipage}
\end{document}