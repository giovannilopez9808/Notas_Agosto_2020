\documentclass[11pt]{article}

    \usepackage[breakable]{tcolorbox}
    \usepackage{parskip} % Stop auto-indenting (to mimic markdown behaviour)
    
    \usepackage{iftex}
    \ifPDFTeX
    	\usepackage[T1]{fontenc}
    	\usepackage{mathpazo}
    \else
    	\usepackage{fontspec}
    \fi

    % Basic figure setup, for now with no caption control since it's done
    % automatically by Pandoc (which extracts ![](path) syntax from Markdown).
    \usepackage{graphicx}
    % Maintain compatibility with old templates. Remove in nbconvert 6.0
    \let\Oldincludegraphics\includegraphics
    % Ensure that by default, figures have no caption (until we provide a
    % proper Figure object with a Caption API and a way to capture that
    % in the conversion process - todo).
    \usepackage{caption}
    \DeclareCaptionFormat{nocaption}{}
    \captionsetup{format=nocaption,aboveskip=0pt,belowskip=0pt}

    \usepackage[Export]{adjustbox} % Used to constrain images to a maximum size
    \adjustboxset{max size={0.9\linewidth}{0.9\paperheight}}
    \usepackage{float}
    \floatplacement{figure}{H} % forces figures to be placed at the correct location
    \usepackage{xcolor} % Allow colors to be defined
    \usepackage{enumerate} % Needed for markdown enumerations to work
    \usepackage{geometry} % Used to adjust the document margins
    \usepackage{amsmath} % Equations
    \usepackage{amssymb} % Equations
    \usepackage{textcomp} % defines textquotesingle
    % Hack from http://tex.stackexchange.com/a/47451/13684:
    \AtBeginDocument{%
        \def\PYZsq{\textquotesingle}% Upright quotes in Pygmentized code
    }
    \usepackage{upquote} % Upright quotes for verbatim code
    \usepackage{eurosym} % defines \euro
    \usepackage[mathletters]{ucs} % Extended unicode (utf-8) support
    \usepackage{fancyvrb} % verbatim replacement that allows latex

    % The hyperref package gives us a pdf with properly built
    % internal navigation ('pdf bookmarks' for the table of contents,
    % internal cross-reference links, web links for URLs, etc.)
    \usepackage{hyperref}
    % The default LaTeX title has an obnoxious amount of whitespace. By default,
    % titling removes some of it. It also provides customization options.
    \usepackage{titling}
    \usepackage{longtable} % longtable support required by pandoc >1.10
    \usepackage{booktabs}  % table support for pandoc > 1.12.2
    \usepackage[inline]{enumitem} % IRkernel/repr support (it uses the enumerate* environment)
    \usepackage[normalem]{ulem} % ulem is needed to support strikethroughs (\sout)
                                % normalem makes italics be italics, not underlines
    \usepackage{mathrsfs}
    

    
    % Colors for the hyperref package
    \definecolor{urlcolor}{rgb}{0,.145,.698}
    \definecolor{linkcolor}{rgb}{.71,0.21,0.01}
    \definecolor{citecolor}{rgb}{.12,.54,.11}

    % ANSI colors
    \definecolor{ansi-black}{HTML}{3E424D}
    \definecolor{ansi-black-intense}{HTML}{282C36}
    \definecolor{ansi-red}{HTML}{E75C58}
    \definecolor{ansi-red-intense}{HTML}{B22B31}
    \definecolor{ansi-green}{HTML}{00A250}
    \definecolor{ansi-green-intense}{HTML}{007427}
    \definecolor{ansi-yellow}{HTML}{DDB62B}
    \definecolor{ansi-yellow-intense}{HTML}{B27D12}
    \definecolor{ansi-blue}{HTML}{208FFB}
    \definecolor{ansi-blue-intense}{HTML}{0065CA}
    \definecolor{ansi-magenta}{HTML}{D160C4}
    \definecolor{ansi-magenta-intense}{HTML}{A03196}
    \definecolor{ansi-cyan}{HTML}{60C6C8}
    \definecolor{ansi-cyan-intense}{HTML}{258F8F}
    \definecolor{ansi-white}{HTML}{C5C1B4}
    \definecolor{ansi-white-intense}{HTML}{A1A6B2}
    \definecolor{ansi-default-inverse-fg}{HTML}{FFFFFF}
    \definecolor{ansi-default-inverse-bg}{HTML}{000000}

    % commands and environments needed by pandoc snippets
    % extracted from the output of `pandoc -s`
    \providecommand{\tightlist}{%
      \setlength{\itemsep}{0pt}\setlength{\parskip}{0pt}}
    \DefineVerbatimEnvironment{Highlighting}{Verbatim}{commandchars=\\\{\}}
    % Add ',fontsize=\small' for more characters per line
    \newenvironment{Shaded}{}{}
    \newcommand{\KeywordTok}[1]{\textcolor[rgb]{0.00,0.44,0.13}{\textbf{{#1}}}}
    \newcommand{\DataTypeTok}[1]{\textcolor[rgb]{0.56,0.13,0.00}{{#1}}}
    \newcommand{\DecValTok}[1]{\textcolor[rgb]{0.25,0.63,0.44}{{#1}}}
    \newcommand{\BaseNTok}[1]{\textcolor[rgb]{0.25,0.63,0.44}{{#1}}}
    \newcommand{\FloatTok}[1]{\textcolor[rgb]{0.25,0.63,0.44}{{#1}}}
    \newcommand{\CharTok}[1]{\textcolor[rgb]{0.25,0.44,0.63}{{#1}}}
    \newcommand{\StringTok}[1]{\textcolor[rgb]{0.25,0.44,0.63}{{#1}}}
    \newcommand{\CommentTok}[1]{\textcolor[rgb]{0.38,0.63,0.69}{\textit{{#1}}}}
    \newcommand{\OtherTok}[1]{\textcolor[rgb]{0.00,0.44,0.13}{{#1}}}
    \newcommand{\AlertTok}[1]{\textcolor[rgb]{1.00,0.00,0.00}{\textbf{{#1}}}}
    \newcommand{\FunctionTok}[1]{\textcolor[rgb]{0.02,0.16,0.49}{{#1}}}
    \newcommand{\RegionMarkerTok}[1]{{#1}}
    \newcommand{\ErrorTok}[1]{\textcolor[rgb]{1.00,0.00,0.00}{\textbf{{#1}}}}
    \newcommand{\NormalTok}[1]{{#1}}
    
    % Additional commands for more recent versions of Pandoc
    \newcommand{\ConstantTok}[1]{\textcolor[rgb]{0.53,0.00,0.00}{{#1}}}
    \newcommand{\SpecialCharTok}[1]{\textcolor[rgb]{0.25,0.44,0.63}{{#1}}}
    \newcommand{\VerbatimStringTok}[1]{\textcolor[rgb]{0.25,0.44,0.63}{{#1}}}
    \newcommand{\SpecialStringTok}[1]{\textcolor[rgb]{0.73,0.40,0.53}{{#1}}}
    \newcommand{\ImportTok}[1]{{#1}}
    \newcommand{\DocumentationTok}[1]{\textcolor[rgb]{0.73,0.13,0.13}{\textit{{#1}}}}
    \newcommand{\AnnotationTok}[1]{\textcolor[rgb]{0.38,0.63,0.69}{\textbf{\textit{{#1}}}}}
    \newcommand{\CommentVarTok}[1]{\textcolor[rgb]{0.38,0.63,0.69}{\textbf{\textit{{#1}}}}}
    \newcommand{\VariableTok}[1]{\textcolor[rgb]{0.10,0.09,0.49}{{#1}}}
    \newcommand{\ControlFlowTok}[1]{\textcolor[rgb]{0.00,0.44,0.13}{\textbf{{#1}}}}
    \newcommand{\OperatorTok}[1]{\textcolor[rgb]{0.40,0.40,0.40}{{#1}}}
    \newcommand{\BuiltInTok}[1]{{#1}}
    \newcommand{\ExtensionTok}[1]{{#1}}
    \newcommand{\PreprocessorTok}[1]{\textcolor[rgb]{0.74,0.48,0.00}{{#1}}}
    \newcommand{\AttributeTok}[1]{\textcolor[rgb]{0.49,0.56,0.16}{{#1}}}
    \newcommand{\InformationTok}[1]{\textcolor[rgb]{0.38,0.63,0.69}{\textbf{\textit{{#1}}}}}
    \newcommand{\WarningTok}[1]{\textcolor[rgb]{0.38,0.63,0.69}{\textbf{\textit{{#1}}}}}
    
    
    % Define a nice break command that doesn't care if a line doesn't already
    % exist.
    \def\br{\hspace*{\fill} \\* }
    % Math Jax compatibility definitions
    \def\gt{>}
    \def\lt{<}
    \let\Oldtex\TeX
    \let\Oldlatex\LaTeX
    \renewcommand{\TeX}{\textrm{\Oldtex}}
    \renewcommand{\LaTeX}{\textrm{\Oldlatex}}
    % Document parameters
    % Document title
    \title{Sistema\_Binario-Evolucion\_Temporal\_y\_observaciones\_HyT}
    
    
    
    
    
% Pygments definitions
\makeatletter
\def\PY@reset{\let\PY@it=\relax \let\PY@bf=\relax%
    \let\PY@ul=\relax \let\PY@tc=\relax%
    \let\PY@bc=\relax \let\PY@ff=\relax}
\def\PY@tok#1{\csname PY@tok@#1\endcsname}
\def\PY@toks#1+{\ifx\relax#1\empty\else%
    \PY@tok{#1}\expandafter\PY@toks\fi}
\def\PY@do#1{\PY@bc{\PY@tc{\PY@ul{%
    \PY@it{\PY@bf{\PY@ff{#1}}}}}}}
\def\PY#1#2{\PY@reset\PY@toks#1+\relax+\PY@do{#2}}

\expandafter\def\csname PY@tok@w\endcsname{\def\PY@tc##1{\textcolor[rgb]{0.73,0.73,0.73}{##1}}}
\expandafter\def\csname PY@tok@c\endcsname{\let\PY@it=\textit\def\PY@tc##1{\textcolor[rgb]{0.25,0.50,0.50}{##1}}}
\expandafter\def\csname PY@tok@cp\endcsname{\def\PY@tc##1{\textcolor[rgb]{0.74,0.48,0.00}{##1}}}
\expandafter\def\csname PY@tok@k\endcsname{\let\PY@bf=\textbf\def\PY@tc##1{\textcolor[rgb]{0.00,0.50,0.00}{##1}}}
\expandafter\def\csname PY@tok@kp\endcsname{\def\PY@tc##1{\textcolor[rgb]{0.00,0.50,0.00}{##1}}}
\expandafter\def\csname PY@tok@kt\endcsname{\def\PY@tc##1{\textcolor[rgb]{0.69,0.00,0.25}{##1}}}
\expandafter\def\csname PY@tok@o\endcsname{\def\PY@tc##1{\textcolor[rgb]{0.40,0.40,0.40}{##1}}}
\expandafter\def\csname PY@tok@ow\endcsname{\let\PY@bf=\textbf\def\PY@tc##1{\textcolor[rgb]{0.67,0.13,1.00}{##1}}}
\expandafter\def\csname PY@tok@nb\endcsname{\def\PY@tc##1{\textcolor[rgb]{0.00,0.50,0.00}{##1}}}
\expandafter\def\csname PY@tok@nf\endcsname{\def\PY@tc##1{\textcolor[rgb]{0.00,0.00,1.00}{##1}}}
\expandafter\def\csname PY@tok@nc\endcsname{\let\PY@bf=\textbf\def\PY@tc##1{\textcolor[rgb]{0.00,0.00,1.00}{##1}}}
\expandafter\def\csname PY@tok@nn\endcsname{\let\PY@bf=\textbf\def\PY@tc##1{\textcolor[rgb]{0.00,0.00,1.00}{##1}}}
\expandafter\def\csname PY@tok@ne\endcsname{\let\PY@bf=\textbf\def\PY@tc##1{\textcolor[rgb]{0.82,0.25,0.23}{##1}}}
\expandafter\def\csname PY@tok@nv\endcsname{\def\PY@tc##1{\textcolor[rgb]{0.10,0.09,0.49}{##1}}}
\expandafter\def\csname PY@tok@no\endcsname{\def\PY@tc##1{\textcolor[rgb]{0.53,0.00,0.00}{##1}}}
\expandafter\def\csname PY@tok@nl\endcsname{\def\PY@tc##1{\textcolor[rgb]{0.63,0.63,0.00}{##1}}}
\expandafter\def\csname PY@tok@ni\endcsname{\let\PY@bf=\textbf\def\PY@tc##1{\textcolor[rgb]{0.60,0.60,0.60}{##1}}}
\expandafter\def\csname PY@tok@na\endcsname{\def\PY@tc##1{\textcolor[rgb]{0.49,0.56,0.16}{##1}}}
\expandafter\def\csname PY@tok@nt\endcsname{\let\PY@bf=\textbf\def\PY@tc##1{\textcolor[rgb]{0.00,0.50,0.00}{##1}}}
\expandafter\def\csname PY@tok@nd\endcsname{\def\PY@tc##1{\textcolor[rgb]{0.67,0.13,1.00}{##1}}}
\expandafter\def\csname PY@tok@s\endcsname{\def\PY@tc##1{\textcolor[rgb]{0.73,0.13,0.13}{##1}}}
\expandafter\def\csname PY@tok@sd\endcsname{\let\PY@it=\textit\def\PY@tc##1{\textcolor[rgb]{0.73,0.13,0.13}{##1}}}
\expandafter\def\csname PY@tok@si\endcsname{\let\PY@bf=\textbf\def\PY@tc##1{\textcolor[rgb]{0.73,0.40,0.53}{##1}}}
\expandafter\def\csname PY@tok@se\endcsname{\let\PY@bf=\textbf\def\PY@tc##1{\textcolor[rgb]{0.73,0.40,0.13}{##1}}}
\expandafter\def\csname PY@tok@sr\endcsname{\def\PY@tc##1{\textcolor[rgb]{0.73,0.40,0.53}{##1}}}
\expandafter\def\csname PY@tok@ss\endcsname{\def\PY@tc##1{\textcolor[rgb]{0.10,0.09,0.49}{##1}}}
\expandafter\def\csname PY@tok@sx\endcsname{\def\PY@tc##1{\textcolor[rgb]{0.00,0.50,0.00}{##1}}}
\expandafter\def\csname PY@tok@m\endcsname{\def\PY@tc##1{\textcolor[rgb]{0.40,0.40,0.40}{##1}}}
\expandafter\def\csname PY@tok@gh\endcsname{\let\PY@bf=\textbf\def\PY@tc##1{\textcolor[rgb]{0.00,0.00,0.50}{##1}}}
\expandafter\def\csname PY@tok@gu\endcsname{\let\PY@bf=\textbf\def\PY@tc##1{\textcolor[rgb]{0.50,0.00,0.50}{##1}}}
\expandafter\def\csname PY@tok@gd\endcsname{\def\PY@tc##1{\textcolor[rgb]{0.63,0.00,0.00}{##1}}}
\expandafter\def\csname PY@tok@gi\endcsname{\def\PY@tc##1{\textcolor[rgb]{0.00,0.63,0.00}{##1}}}
\expandafter\def\csname PY@tok@gr\endcsname{\def\PY@tc##1{\textcolor[rgb]{1.00,0.00,0.00}{##1}}}
\expandafter\def\csname PY@tok@ge\endcsname{\let\PY@it=\textit}
\expandafter\def\csname PY@tok@gs\endcsname{\let\PY@bf=\textbf}
\expandafter\def\csname PY@tok@gp\endcsname{\let\PY@bf=\textbf\def\PY@tc##1{\textcolor[rgb]{0.00,0.00,0.50}{##1}}}
\expandafter\def\csname PY@tok@go\endcsname{\def\PY@tc##1{\textcolor[rgb]{0.53,0.53,0.53}{##1}}}
\expandafter\def\csname PY@tok@gt\endcsname{\def\PY@tc##1{\textcolor[rgb]{0.00,0.27,0.87}{##1}}}
\expandafter\def\csname PY@tok@err\endcsname{\def\PY@bc##1{\setlength{\fboxsep}{0pt}\fcolorbox[rgb]{1.00,0.00,0.00}{1,1,1}{\strut ##1}}}
\expandafter\def\csname PY@tok@kc\endcsname{\let\PY@bf=\textbf\def\PY@tc##1{\textcolor[rgb]{0.00,0.50,0.00}{##1}}}
\expandafter\def\csname PY@tok@kd\endcsname{\let\PY@bf=\textbf\def\PY@tc##1{\textcolor[rgb]{0.00,0.50,0.00}{##1}}}
\expandafter\def\csname PY@tok@kn\endcsname{\let\PY@bf=\textbf\def\PY@tc##1{\textcolor[rgb]{0.00,0.50,0.00}{##1}}}
\expandafter\def\csname PY@tok@kr\endcsname{\let\PY@bf=\textbf\def\PY@tc##1{\textcolor[rgb]{0.00,0.50,0.00}{##1}}}
\expandafter\def\csname PY@tok@bp\endcsname{\def\PY@tc##1{\textcolor[rgb]{0.00,0.50,0.00}{##1}}}
\expandafter\def\csname PY@tok@fm\endcsname{\def\PY@tc##1{\textcolor[rgb]{0.00,0.00,1.00}{##1}}}
\expandafter\def\csname PY@tok@vc\endcsname{\def\PY@tc##1{\textcolor[rgb]{0.10,0.09,0.49}{##1}}}
\expandafter\def\csname PY@tok@vg\endcsname{\def\PY@tc##1{\textcolor[rgb]{0.10,0.09,0.49}{##1}}}
\expandafter\def\csname PY@tok@vi\endcsname{\def\PY@tc##1{\textcolor[rgb]{0.10,0.09,0.49}{##1}}}
\expandafter\def\csname PY@tok@vm\endcsname{\def\PY@tc##1{\textcolor[rgb]{0.10,0.09,0.49}{##1}}}
\expandafter\def\csname PY@tok@sa\endcsname{\def\PY@tc##1{\textcolor[rgb]{0.73,0.13,0.13}{##1}}}
\expandafter\def\csname PY@tok@sb\endcsname{\def\PY@tc##1{\textcolor[rgb]{0.73,0.13,0.13}{##1}}}
\expandafter\def\csname PY@tok@sc\endcsname{\def\PY@tc##1{\textcolor[rgb]{0.73,0.13,0.13}{##1}}}
\expandafter\def\csname PY@tok@dl\endcsname{\def\PY@tc##1{\textcolor[rgb]{0.73,0.13,0.13}{##1}}}
\expandafter\def\csname PY@tok@s2\endcsname{\def\PY@tc##1{\textcolor[rgb]{0.73,0.13,0.13}{##1}}}
\expandafter\def\csname PY@tok@sh\endcsname{\def\PY@tc##1{\textcolor[rgb]{0.73,0.13,0.13}{##1}}}
\expandafter\def\csname PY@tok@s1\endcsname{\def\PY@tc##1{\textcolor[rgb]{0.73,0.13,0.13}{##1}}}
\expandafter\def\csname PY@tok@mb\endcsname{\def\PY@tc##1{\textcolor[rgb]{0.40,0.40,0.40}{##1}}}
\expandafter\def\csname PY@tok@mf\endcsname{\def\PY@tc##1{\textcolor[rgb]{0.40,0.40,0.40}{##1}}}
\expandafter\def\csname PY@tok@mh\endcsname{\def\PY@tc##1{\textcolor[rgb]{0.40,0.40,0.40}{##1}}}
\expandafter\def\csname PY@tok@mi\endcsname{\def\PY@tc##1{\textcolor[rgb]{0.40,0.40,0.40}{##1}}}
\expandafter\def\csname PY@tok@il\endcsname{\def\PY@tc##1{\textcolor[rgb]{0.40,0.40,0.40}{##1}}}
\expandafter\def\csname PY@tok@mo\endcsname{\def\PY@tc##1{\textcolor[rgb]{0.40,0.40,0.40}{##1}}}
\expandafter\def\csname PY@tok@ch\endcsname{\let\PY@it=\textit\def\PY@tc##1{\textcolor[rgb]{0.25,0.50,0.50}{##1}}}
\expandafter\def\csname PY@tok@cm\endcsname{\let\PY@it=\textit\def\PY@tc##1{\textcolor[rgb]{0.25,0.50,0.50}{##1}}}
\expandafter\def\csname PY@tok@cpf\endcsname{\let\PY@it=\textit\def\PY@tc##1{\textcolor[rgb]{0.25,0.50,0.50}{##1}}}
\expandafter\def\csname PY@tok@c1\endcsname{\let\PY@it=\textit\def\PY@tc##1{\textcolor[rgb]{0.25,0.50,0.50}{##1}}}
\expandafter\def\csname PY@tok@cs\endcsname{\let\PY@it=\textit\def\PY@tc##1{\textcolor[rgb]{0.25,0.50,0.50}{##1}}}

\def\PYZbs{\char`\\}
\def\PYZus{\char`\_}
\def\PYZob{\char`\{}
\def\PYZcb{\char`\}}
\def\PYZca{\char`\^}
\def\PYZam{\char`\&}
\def\PYZlt{\char`\<}
\def\PYZgt{\char`\>}
\def\PYZsh{\char`\#}
\def\PYZpc{\char`\%}
\def\PYZdl{\char`\$}
\def\PYZhy{\char`\-}
\def\PYZsq{\char`\'}
\def\PYZdq{\char`\"}
\def\PYZti{\char`\~}
% for compatibility with earlier versions
\def\PYZat{@}
\def\PYZlb{[}
\def\PYZrb{]}
\makeatother


    % For linebreaks inside Verbatim environment from package fancyvrb. 
    \makeatletter
        \newbox\Wrappedcontinuationbox 
        \newbox\Wrappedvisiblespacebox 
        \newcommand*\Wrappedvisiblespace {\textcolor{red}{\textvisiblespace}} 
        \newcommand*\Wrappedcontinuationsymbol {\textcolor{red}{\llap{\tiny$\m@th\hookrightarrow$}}} 
        \newcommand*\Wrappedcontinuationindent {3ex } 
        \newcommand*\Wrappedafterbreak {\kern\Wrappedcontinuationindent\copy\Wrappedcontinuationbox} 
        % Take advantage of the already applied Pygments mark-up to insert 
        % potential linebreaks for TeX processing. 
        %        {, <, #, %, $, ' and ": go to next line. 
        %        _, }, ^, &, >, - and ~: stay at end of broken line. 
        % Use of \textquotesingle for straight quote. 
        \newcommand*\Wrappedbreaksatspecials {% 
            \def\PYGZus{\discretionary{\char`\_}{\Wrappedafterbreak}{\char`\_}}% 
            \def\PYGZob{\discretionary{}{\Wrappedafterbreak\char`\{}{\char`\{}}% 
            \def\PYGZcb{\discretionary{\char`\}}{\Wrappedafterbreak}{\char`\}}}% 
            \def\PYGZca{\discretionary{\char`\^}{\Wrappedafterbreak}{\char`\^}}% 
            \def\PYGZam{\discretionary{\char`\&}{\Wrappedafterbreak}{\char`\&}}% 
            \def\PYGZlt{\discretionary{}{\Wrappedafterbreak\char`\<}{\char`\<}}% 
            \def\PYGZgt{\discretionary{\char`\>}{\Wrappedafterbreak}{\char`\>}}% 
            \def\PYGZsh{\discretionary{}{\Wrappedafterbreak\char`\#}{\char`\#}}% 
            \def\PYGZpc{\discretionary{}{\Wrappedafterbreak\char`\%}{\char`\%}}% 
            \def\PYGZdl{\discretionary{}{\Wrappedafterbreak\char`\$}{\char`\$}}% 
            \def\PYGZhy{\discretionary{\char`\-}{\Wrappedafterbreak}{\char`\-}}% 
            \def\PYGZsq{\discretionary{}{\Wrappedafterbreak\textquotesingle}{\textquotesingle}}% 
            \def\PYGZdq{\discretionary{}{\Wrappedafterbreak\char`\"}{\char`\"}}% 
            \def\PYGZti{\discretionary{\char`\~}{\Wrappedafterbreak}{\char`\~}}% 
        } 
        % Some characters . , ; ? ! / are not pygmentized. 
        % This macro makes them "active" and they will insert potential linebreaks 
        \newcommand*\Wrappedbreaksatpunct {% 
            \lccode`\~`\.\lowercase{\def~}{\discretionary{\hbox{\char`\.}}{\Wrappedafterbreak}{\hbox{\char`\.}}}% 
            \lccode`\~`\,\lowercase{\def~}{\discretionary{\hbox{\char`\,}}{\Wrappedafterbreak}{\hbox{\char`\,}}}% 
            \lccode`\~`\;\lowercase{\def~}{\discretionary{\hbox{\char`\;}}{\Wrappedafterbreak}{\hbox{\char`\;}}}% 
            \lccode`\~`\:\lowercase{\def~}{\discretionary{\hbox{\char`\:}}{\Wrappedafterbreak}{\hbox{\char`\:}}}% 
            \lccode`\~`\?\lowercase{\def~}{\discretionary{\hbox{\char`\?}}{\Wrappedafterbreak}{\hbox{\char`\?}}}% 
            \lccode`\~`\!\lowercase{\def~}{\discretionary{\hbox{\char`\!}}{\Wrappedafterbreak}{\hbox{\char`\!}}}% 
            \lccode`\~`\/\lowercase{\def~}{\discretionary{\hbox{\char`\/}}{\Wrappedafterbreak}{\hbox{\char`\/}}}% 
            \catcode`\.\active
            \catcode`\,\active 
            \catcode`\;\active
            \catcode`\:\active
            \catcode`\?\active
            \catcode`\!\active
            \catcode`\/\active 
            \lccode`\~`\~ 	
        }
    \makeatother

    \let\OriginalVerbatim=\Verbatim
    \makeatletter
    \renewcommand{\Verbatim}[1][1]{%
        %\parskip\z@skip
        \sbox\Wrappedcontinuationbox {\Wrappedcontinuationsymbol}%
        \sbox\Wrappedvisiblespacebox {\FV@SetupFont\Wrappedvisiblespace}%
        \def\FancyVerbFormatLine ##1{\hsize\linewidth
            \vtop{\raggedright\hyphenpenalty\z@\exhyphenpenalty\z@
                \doublehyphendemerits\z@\finalhyphendemerits\z@
                \strut ##1\strut}%
        }%
        % If the linebreak is at a space, the latter will be displayed as visible
        % space at end of first line, and a continuation symbol starts next line.
        % Stretch/shrink are however usually zero for typewriter font.
        \def\FV@Space {%
            \nobreak\hskip\z@ plus\fontdimen3\font minus\fontdimen4\font
            \discretionary{\copy\Wrappedvisiblespacebox}{\Wrappedafterbreak}
            {\kern\fontdimen2\font}%
        }%
        
        % Allow breaks at special characters using \PYG... macros.
        \Wrappedbreaksatspecials
        % Breaks at punctuation characters . , ; ? ! and / need catcode=\active 	
        \OriginalVerbatim[#1,codes*=\Wrappedbreaksatpunct]%
    }
    \makeatother

    % Exact colors from NB
    \definecolor{incolor}{HTML}{303F9F}
    \definecolor{outcolor}{HTML}{D84315}
    \definecolor{cellborder}{HTML}{CFCFCF}
    \definecolor{cellbackground}{HTML}{F7F7F7}
    
    % prompt
    \makeatletter
    \newcommand{\boxspacing}{\kern\kvtcb@left@rule\kern\kvtcb@boxsep}
    \makeatother
    \newcommand{\prompt}[4]{
        \ttfamily\llap{{\color{#2}[#3]:\hspace{3pt}#4}}\vspace{-\baselineskip}
    }
    

    
    % Prevent overflowing lines due to hard-to-break entities
    \sloppy 
    % Setup hyperref package
    \hypersetup{
      breaklinks=true,  % so long urls are correctly broken across lines
      colorlinks=true,
      urlcolor=urlcolor,
      linkcolor=linkcolor,
      citecolor=citecolor,
      }
    % Slightly bigger margins than the latex defaults
    
    \geometry{verbose,tmargin=1in,bmargin=1in,lmargin=1in,rmargin=1in}
    
    

\begin{document}
    
    \maketitle
    
    

    
    \hypertarget{modelo-de-evoluciuxf3n-de-un-pulsar-binario}{%
\section{Modelo de evolución de un Pulsar
Binario}\label{modelo-de-evoluciuxf3n-de-un-pulsar-binario}}

    \hypertarget{cuxe1lculo-simbuxf3lico-de-a-en-funciuxf3n-de-e}{%
\section{\texorpdfstring{Cálculo Simbólico de \(a\) en función de
\(e\)}{Cálculo Simbólico de a en función de e}}\label{cuxe1lculo-simbuxf3lico-de-a-en-funciuxf3n-de-e}}

    Dividiendo las ecuaciones para \(\dot{a}\) y \(\dot{e}\) podemos
eliminar el tiempo de estas expresiones y encontrar una ecuación que
relaciona directamente \(a\) con \(e\):
\[\frac{da}{de}=\frac{12}{19}a\frac{1+(73/24)e^2+(37/96)e^4}{e(1-e^2)[1+(121/304)e^2]}\]

    Integrando esta relación respecto a \(e\), encontramos: \[
\int_{a_{0}}^{a}\frac{d\bar{a}}{\bar{a}}=\int_{e_{0}}^{e}\frac{12}{19}\frac{1+(73/24)\bar{e}^2+(37/96)\bar{e}^4}{\bar{e}(1-\bar{e}^2)[1+(121/304)\bar{e}^2]}\,d\bar{e}
\] Por lo tanto, \[
\ln \left(\frac{a}{a_{0}}\right)=\int_{e_{0}}^{e}\frac{12}{19}\frac{1+(73/24)\bar{e}^2+(37/96)\bar{e}^4}{\bar{e}(1-\bar{e}^2)[1+(121/304)\bar{e}^2]}\,d\bar{e}
\]

    Despejando \(a\), obtenemos la expresión \[
\bar{a}= a_{0}\exp\left[\int_{e_{0}}^{e}\frac{12}{19}\frac{1+(73/24)\bar{e}^2+(37/96)\bar{e}^4}{\bar{e}(1-\bar{e}^2)[1+(121/304)\bar{e}^2]}\,d\bar{e}\right].
\]

    Primero calcularemos la integral en la expresión anterior, usando
\texttt{sympy}

    \begin{tcolorbox}[breakable, size=fbox, boxrule=1pt, pad at break*=1mm,colback=cellbackground, colframe=cellborder]
\prompt{In}{incolor}{1}{\boxspacing}
\begin{Verbatim}[commandchars=\\\{\}]
\PY{k+kn}{from} \PY{n+nn}{sympy} \PY{k}{import} \PY{o}{*}
\PY{n}{init\PYZus{}printing}\PY{p}{(}\PY{n}{use\PYZus{}unicode}\PY{o}{=}\PY{k+kc}{True}\PY{p}{)}
\end{Verbatim}
\end{tcolorbox}

    \begin{tcolorbox}[breakable, size=fbox, boxrule=1pt, pad at break*=1mm,colback=cellbackground, colframe=cellborder]
\prompt{In}{incolor}{2}{\boxspacing}
\begin{Verbatim}[commandchars=\\\{\}]
\PY{n}{a0} \PY{o}{=} \PY{n}{Symbol}\PY{p}{(}\PY{l+s+s1}{\PYZsq{}}\PY{l+s+s1}{a\PYZus{}0}\PY{l+s+s1}{\PYZsq{}}\PY{p}{)}
\PY{n}{e0} \PY{o}{=} \PY{n}{Symbol}\PY{p}{(}\PY{l+s+s1}{\PYZsq{}}\PY{l+s+s1}{e\PYZus{}0}\PY{l+s+s1}{\PYZsq{}}\PY{p}{)}
\PY{n}{e} \PY{o}{=} \PY{n}{Symbol}\PY{p}{(}\PY{l+s+s1}{\PYZsq{}}\PY{l+s+s1}{e}\PY{l+s+s1}{\PYZsq{}}\PY{p}{)}
\PY{n}{a} \PY{o}{=} \PY{n}{Symbol}\PY{p}{(}\PY{l+s+s1}{\PYZsq{}}\PY{l+s+s1}{a}\PY{l+s+s1}{\PYZsq{}}\PY{p}{)}
\end{Verbatim}
\end{tcolorbox}

    \begin{tcolorbox}[breakable, size=fbox, boxrule=1pt, pad at break*=1mm,colback=cellbackground, colframe=cellborder]
\prompt{In}{incolor}{3}{\boxspacing}
\begin{Verbatim}[commandchars=\\\{\}]
\PY{n}{integrando} \PY{o}{=} \PY{n}{Rational}\PY{p}{(}\PY{l+m+mi}{12}\PY{p}{,}\PY{l+m+mi}{19}\PY{p}{)}\PY{o}{*}\PY{p}{(}\PY{p}{(}\PY{l+m+mi}{1}\PY{o}{+}\PY{n}{Rational}\PY{p}{(}\PY{l+m+mi}{73}\PY{p}{,}\PY{l+m+mi}{24}\PY{p}{)}\PY{o}{*}\PY{n}{e}\PY{o}{*}\PY{o}{*}\PY{l+m+mi}{2}\PY{o}{+}\PY{n}{Rational}\PY{p}{(}\PY{l+m+mi}{37}\PY{p}{,}\PY{l+m+mi}{96}\PY{p}{)}\PY{o}{*}\PY{n}{e}\PY{o}{*}\PY{o}{*}\PY{l+m+mi}{4}\PY{p}{)}
                                 \PY{o}{/}\PY{p}{(}\PY{n}{e}\PY{o}{*}\PY{p}{(}\PY{l+m+mi}{1}\PY{o}{\PYZhy{}}\PY{n}{e}\PY{o}{*}\PY{o}{*}\PY{l+m+mi}{2}\PY{p}{)}\PY{o}{*}\PY{p}{(}\PY{l+m+mi}{1}\PY{o}{+}\PY{n}{Rational}\PY{p}{(}\PY{l+m+mi}{121}\PY{p}{,}\PY{l+m+mi}{304}\PY{p}{)}\PY{o}{*}\PY{n}{e}\PY{o}{*}\PY{o}{*}\PY{l+m+mi}{2}\PY{p}{)}\PY{p}{)}\PY{p}{)}
\end{Verbatim}
\end{tcolorbox}

    \begin{tcolorbox}[breakable, size=fbox, boxrule=1pt, pad at break*=1mm,colback=cellbackground, colframe=cellborder]
\prompt{In}{incolor}{4}{\boxspacing}
\begin{Verbatim}[commandchars=\\\{\}]
\PY{n}{integrando}
\end{Verbatim}
\end{tcolorbox}
 
            
\prompt{Out}{outcolor}{4}{}
    
    $$\frac{\frac{37 e^{4}}{8} + \frac{73 e^{2}}{2} + 12}{19 e \left(- e^{2} + 1\right) \left(\frac{121 e^{2}}{304} + 1\right)}$$

    

    \begin{tcolorbox}[breakable, size=fbox, boxrule=1pt, pad at break*=1mm,colback=cellbackground, colframe=cellborder]
\prompt{In}{incolor}{5}{\boxspacing}
\begin{Verbatim}[commandchars=\\\{\}]
\PY{n}{Integral} \PY{o}{=} \PY{n}{integrate}\PY{p}{(}\PY{n}{integrando}\PY{p}{,}\PY{p}{(}\PY{n}{e}\PY{p}{,}\PY{n}{e0}\PY{p}{,}\PY{n}{e}\PY{p}{)}\PY{p}{)}
\end{Verbatim}
\end{tcolorbox}

    \begin{tcolorbox}[breakable, size=fbox, boxrule=1pt, pad at break*=1mm,colback=cellbackground, colframe=cellborder]
\prompt{In}{incolor}{6}{\boxspacing}
\begin{Verbatim}[commandchars=\\\{\}]
\PY{n}{Integral}
\end{Verbatim}
\end{tcolorbox}
 
            
\prompt{Out}{outcolor}{6}{}
    
    $$\frac{12}{19} \log{\left (e \right )} - \frac{12}{19} \log{\left (e_{0} \right )} - \log{\left (e^{2} - 1 \right )} + \frac{870}{2299} \log{\left (e^{2} + \frac{304}{121} \right )} + \log{\left (e_{0}^{2} - 1 \right )} - \frac{870}{2299} \log{\left (e_{0}^{2} + \frac{304}{121} \right )}$$

    

    y ahora exponenciamos:

    \begin{tcolorbox}[breakable, size=fbox, boxrule=1pt, pad at break*=1mm,colback=cellbackground, colframe=cellborder]
\prompt{In}{incolor}{7}{\boxspacing}
\begin{Verbatim}[commandchars=\\\{\}]
\PY{n}{a} \PY{o}{=} \PY{n}{a0}\PY{o}{*}\PY{n}{exp}\PY{p}{(}\PY{n}{Integral}\PY{p}{)}
\end{Verbatim}
\end{tcolorbox}

    \begin{tcolorbox}[breakable, size=fbox, boxrule=1pt, pad at break*=1mm,colback=cellbackground, colframe=cellborder]
\prompt{In}{incolor}{8}{\boxspacing}
\begin{Verbatim}[commandchars=\\\{\}]
\PY{n}{a}
\end{Verbatim}
\end{tcolorbox}
 
            
\prompt{Out}{outcolor}{8}{}
    
    $$\frac{a_{0} e^{\frac{12}{19}} \left(e^{2} + \frac{304}{121}\right)^{\frac{870}{2299}} \left(e_{0}^{2} - 1\right)}{e_{0}^{\frac{12}{19}} \left(e^{2} - 1\right) \left(e_{0}^{2} + \frac{304}{121}\right)^{\frac{870}{2299}}}$$

    

    Si definimos \[
g(e):= \frac{e^{12/19}}{1-e^2}\left(1+\frac{121}{304} \right)^{870/2299},
\] entonces la solución para \(a(e)\) puede escribirse como \[
a(e)=a_{0}\frac{g(e)}{g(e_{0})}.
\]

    Alternativamente, podemos intentar usar la función \texttt{dsolve} de
\texttt{sympy} para resolver directamente la EDO determinada por la
expresión de \(da/dt\) descrita arriba:

    \begin{tcolorbox}[breakable, size=fbox, boxrule=1pt, pad at break*=1mm,colback=cellbackground, colframe=cellborder]
\prompt{In}{incolor}{9}{\boxspacing}
\begin{Verbatim}[commandchars=\\\{\}]
\PY{n}{af} \PY{o}{=} \PY{n}{Function}\PY{p}{(}\PY{l+s+s1}{\PYZsq{}}\PY{l+s+s1}{a}\PY{l+s+s1}{\PYZsq{}}\PY{p}{)}
\PY{n}{solucion} \PY{o}{=} \PY{n}{dsolve}\PY{p}{(}\PY{n}{Derivative}\PY{p}{(}\PY{n}{af}\PY{p}{(}\PY{n}{e}\PY{p}{)}\PY{p}{,}\PY{n}{e}\PY{p}{)}\PY{o}{\PYZhy{}}\PY{n}{af}\PY{p}{(}\PY{n}{e}\PY{p}{)}\PY{o}{*}\PY{n}{integrando}\PY{p}{,}\PY{n}{af}\PY{p}{(}\PY{n}{e}\PY{p}{)}\PY{p}{)}
\PY{n}{solucion}
\end{Verbatim}
\end{tcolorbox}
 
            
\prompt{Out}{outcolor}{9}{}
    
    $$a{\left (e \right )} = \frac{C_{1} e^{\frac{12}{19}} \left(121 e^{2} + 304\right)^{\frac{870}{2299}}}{11 e^{2} - 11}$$

    

    La constante \(C_1\) es determinada por la condición inicial
\(a(e_0)=a_0\) que, luego de ser reemplazada de vuelta en la solución,
entrega la misma expresión encontrada por el otro método.

    \hypertarget{soluciuxf3n-numuxe9rica}{%
\section{Solución numérica}\label{soluciuxf3n-numuxe9rica}}

    \begin{tcolorbox}[breakable, size=fbox, boxrule=1pt, pad at break*=1mm,colback=cellbackground, colframe=cellborder]
\prompt{In}{incolor}{10}{\boxspacing}
\begin{Verbatim}[commandchars=\\\{\}]
\PY{o}{\PYZpc{}}\PY{k}{matplotlib} inline
\end{Verbatim}
\end{tcolorbox}

    \begin{tcolorbox}[breakable, size=fbox, boxrule=1pt, pad at break*=1mm,colback=cellbackground, colframe=cellborder]
\prompt{In}{incolor}{11}{\boxspacing}
\begin{Verbatim}[commandchars=\\\{\}]
\PY{k+kn}{import} \PY{n+nn}{numpy} \PY{k}{as} \PY{n+nn}{np}
\PY{k+kn}{from} \PY{n+nn}{scipy}\PY{n+nn}{.}\PY{n+nn}{integrate} \PY{k}{import} \PY{n}{odeint}
\PY{k+kn}{import} \PY{n+nn}{matplotlib}\PY{n+nn}{.}\PY{n+nn}{pyplot} \PY{k}{as} \PY{n+nn}{plt}
\PY{k+kn}{from} \PY{n+nn}{\PYZus{}\PYZus{}future\PYZus{}\PYZus{}} \PY{k}{import} \PY{n}{division}
\end{Verbatim}
\end{tcolorbox}

    Usaremos la siguiente adimencionalización de las variables: \[
\tilde{a}:= \frac{a}{R_{*}}, \qquad \tilde{t}:= \frac{c t}{R_{*}},
\] con \[
R_{*}^3:=\frac{4 G^3 \mu M^2}{c^6}.
\]

    Las ecuaciones \emph{adimensionalizadas} que describen el decaimiento (y
circularización de la órbita) son:

\begin{align}
\frac{d\tilde{a}}{d\tilde{t}} &= -\frac{16}{5}\frac{1}{\tilde{a}^3}\frac{1}{\left(1-e^2\right)^{7/2}}\left(1+\frac{73}{24}e^2+\frac{37}{96}e^4\right) ,\\
\frac{de}{d\tilde{t}} &= -\frac{76}{15}\frac{1}{\tilde{a}^4}\frac{e}{\left(1-e^2\right)^{5/2}}\left(1+\frac{121}{304}e^2\right) .
\end{align}

Como el sistema es de primer orden, basta definir el vector
(bidimensional) solución \(x\) por medio de \(x[0]:=\tilde{a}\),
\(x[1]=e\).

Por lo tanto, la función \texttt{dotx}, que define la derivada temporal
de \(x\) (en nuestro caso, derivada con respecto al tiempo
adimensionalizado \(\tilde{t}\), es dada por

    \begin{tcolorbox}[breakable, size=fbox, boxrule=1pt, pad at break*=1mm,colback=cellbackground, colframe=cellborder]
\prompt{In}{incolor}{12}{\boxspacing}
\begin{Verbatim}[commandchars=\\\{\}]
\PY{k}{def} \PY{n+nf}{dotx}\PY{p}{(}\PY{n}{x}\PY{p}{,}\PY{n}{t}\PY{p}{)}\PY{p}{:}
    \PY{n}{a} \PY{o}{=} \PY{n}{x}\PY{p}{[}\PY{l+m+mi}{0}\PY{p}{]}
    \PY{n}{e} \PY{o}{=} \PY{n}{x}\PY{p}{[}\PY{l+m+mi}{1}\PY{p}{]}
    \PY{k}{return} \PY{p}{[}\PY{o}{\PYZhy{}}\PY{p}{(}\PY{l+m+mi}{16}\PY{o}{/}\PY{p}{(}\PY{l+m+mi}{5}\PY{o}{*}\PY{n}{a}\PY{o}{*}\PY{o}{*}\PY{l+m+mi}{3}\PY{p}{)}\PY{p}{)}\PY{o}{*}\PY{p}{(}\PY{l+m+mi}{1}\PY{o}{+}\PY{p}{(}\PY{l+m+mi}{73}\PY{o}{/}\PY{l+m+mi}{24}\PY{p}{)}\PY{o}{*}\PY{n}{e}\PY{o}{*}\PY{o}{*}\PY{l+m+mi}{2}\PY{o}{+}\PY{p}{(}\PY{l+m+mi}{37}\PY{o}{/}\PY{l+m+mi}{96}\PY{p}{)}\PY{o}{*}\PY{n}{e}\PY{o}{*}\PY{o}{*}\PY{l+m+mi}{4}\PY{p}{)}\PY{o}{/}\PY{p}{(}\PY{p}{(}\PY{l+m+mi}{1}\PY{o}{\PYZhy{}}\PY{n}{e}\PY{o}{*}\PY{o}{*}\PY{l+m+mi}{2}\PY{p}{)}\PY{o}{*}\PY{o}{*}\PY{p}{(}\PY{l+m+mi}{7}\PY{o}{/}\PY{l+m+mi}{2}\PY{p}{)}\PY{p}{)}\PY{p}{,}
            \PY{o}{\PYZhy{}}\PY{p}{(}\PY{l+m+mi}{76}\PY{o}{/}\PY{p}{(}\PY{l+m+mi}{15}\PY{o}{*}\PY{n}{a}\PY{o}{*}\PY{o}{*}\PY{l+m+mi}{4}\PY{p}{)}\PY{p}{)}\PY{o}{*}\PY{n}{e}\PY{o}{*}\PY{p}{(}\PY{l+m+mi}{1}\PY{o}{+}\PY{p}{(}\PY{l+m+mi}{121}\PY{o}{/}\PY{l+m+mi}{304}\PY{p}{)}\PY{o}{*}\PY{n}{e}\PY{o}{*}\PY{o}{*}\PY{l+m+mi}{2}\PY{p}{)}\PY{o}{/}\PY{p}{(}\PY{p}{(}\PY{l+m+mi}{1}\PY{o}{\PYZhy{}}\PY{n}{e}\PY{o}{*}\PY{o}{*}\PY{l+m+mi}{2}\PY{p}{)}\PY{o}{*}\PY{o}{*}\PY{p}{(}\PY{l+m+mi}{5}\PY{o}{/}\PY{l+m+mi}{2}\PY{p}{)}\PY{p}{)}\PY{p}{]}
\end{Verbatim}
\end{tcolorbox}

    Usamos los datos del Pulsar de Hulse y Taulor, de acuerdo a lo por
\href{http://dx.doi.org/10.1088/0004-637X/722/2/1030}{Weisberg, Nice y
Taylor (2010)} (http://arxiv.org/abs/1011.0718v1):

    \begin{tcolorbox}[breakable, size=fbox, boxrule=1pt, pad at break*=1mm,colback=cellbackground, colframe=cellborder]
\prompt{In}{incolor}{13}{\boxspacing}
\begin{Verbatim}[commandchars=\\\{\}]
\PY{n}{T0\PYZus{}d} \PY{o}{=} \PY{l+m+mf}{0.322997448911} \PY{c+c1}{\PYZsh{} periodo inicial, en días}
\PY{n}{e0} \PY{o}{=} \PY{l+m+mf}{0.6171334} \PY{c+c1}{\PYZsh{} excentricidad inicial}
\PY{n}{M\PYZus{}c} \PY{o}{=} \PY{l+m+mf}{1.3886} \PY{c+c1}{\PYZsh{} masa de la compañera, en masas solares}
\PY{n}{M\PYZus{}p} \PY{o}{=} \PY{l+m+mf}{1.4398} \PY{c+c1}{\PYZsh{} masa del pulsar, en masas solares}
\PY{n}{c} \PY{o}{=} \PY{l+m+mi}{299792458} \PY{c+c1}{\PYZsh{} rapidez de la luz, en metros por segundo}
\PY{n}{MGcm3} \PY{o}{=} \PY{l+m+mf}{4.925490947E\PYZhy{}6} \PY{c+c1}{\PYZsh{} MG/c\PYZca{}3, en segundos}
\end{Verbatim}
\end{tcolorbox}

    Calculamos algunos otros parámetros que nos serán útiles:

    \begin{tcolorbox}[breakable, size=fbox, boxrule=1pt, pad at break*=1mm,colback=cellbackground, colframe=cellborder]
\prompt{In}{incolor}{14}{\boxspacing}
\begin{Verbatim}[commandchars=\\\{\}]
\PY{n}{m\PYZus{}sol} \PY{o}{=} \PY{n}{MGcm3}\PY{o}{*}\PY{n}{c} \PY{c+c1}{\PYZsh{} parametro de masa del Sol m=GM/c\PYZca{}2, en metros}
\PY{n}{M} \PY{o}{=} \PY{n}{M\PYZus{}c}\PY{o}{+}\PY{n}{M\PYZus{}p} \PY{c+c1}{\PYZsh{} masa total, en masas solares}
\PY{n}{mu} \PY{o}{=} \PY{p}{(}\PY{n}{M\PYZus{}c}\PY{o}{*}\PY{n}{M\PYZus{}p}\PY{p}{)}\PY{o}{/}\PY{n}{M} \PY{c+c1}{\PYZsh{} masa reducida, en masas solares}
\PY{n}{R\PYZus{}ast} \PY{o}{=} \PY{n}{m\PYZus{}sol}\PY{o}{*}\PY{p}{(}\PY{l+m+mi}{4}\PY{o}{*}\PY{n}{mu}\PY{o}{*}\PY{n}{M}\PY{o}{*}\PY{o}{*}\PY{l+m+mi}{2}\PY{p}{)}\PY{o}{*}\PY{o}{*}\PY{p}{(}\PY{l+m+mi}{1}\PY{o}{/}\PY{l+m+mi}{3}\PY{p}{)} \PY{c+c1}{\PYZsh{} R\PYZus{}\PYZbs{}ast en metros}
\PY{n}{T0\PYZus{}s} \PY{o}{=} \PY{n}{T0\PYZus{}d}\PY{o}{*}\PY{l+m+mi}{86400} \PY{c+c1}{\PYZsh{} periodo inicial, en segundos}

\PY{n+nb}{print}\PY{p}{(}\PY{l+s+s1}{\PYZsq{}}\PY{l+s+s1}{R\PYZus{}ast = }\PY{l+s+s1}{\PYZsq{}}\PY{o}{+}\PY{n+nb}{str}\PY{p}{(}\PY{n}{R\PYZus{}ast}\PY{p}{)}\PY{o}{+}\PY{l+s+s1}{\PYZsq{}}\PY{l+s+s1}{ [m]}\PY{l+s+s1}{\PYZsq{}}\PY{p}{)}
\PY{n+nb}{print}\PY{p}{(}\PY{l+s+s1}{\PYZsq{}}\PY{l+s+s1}{T0 = }\PY{l+s+s1}{\PYZsq{}}\PY{o}{+}\PY{n+nb}{str}\PY{p}{(}\PY{n}{T0\PYZus{}s}\PY{p}{)} \PY{o}{+} \PY{l+s+s1}{\PYZsq{}}\PY{l+s+s1}{ [s]}\PY{l+s+s1}{\PYZsq{}}\PY{p}{)}
\end{Verbatim}
\end{tcolorbox}

    \begin{Verbatim}[commandchars=\\\{\}]
R\_ast = 4176.03001482 [m]
T0 = 27906.9795859 [s]
    \end{Verbatim}

    Definimos también un par de funciones que relacionan el periodo orbital
\(T\) (en segundos) con el semieje mayor \(a\) (de la coordenada
relativa, en metros), y viceversa:

    \begin{tcolorbox}[breakable, size=fbox, boxrule=1pt, pad at break*=1mm,colback=cellbackground, colframe=cellborder]
\prompt{In}{incolor}{15}{\boxspacing}
\begin{Verbatim}[commandchars=\\\{\}]
\PY{k}{def} \PY{n+nf}{a}\PY{p}{(}\PY{n}{T\PYZus{}s}\PY{p}{)}\PY{p}{:}
    \PY{k}{return} \PY{p}{(}\PY{n}{m\PYZus{}sol}\PY{o}{*}\PY{n}{M}\PY{o}{*}\PY{p}{(}\PY{n}{c}\PY{o}{*}\PY{n}{T\PYZus{}s}\PY{o}{/}\PY{p}{(}\PY{l+m+mi}{2}\PY{o}{*}\PY{n}{np}\PY{o}{.}\PY{n}{pi}\PY{p}{)}\PY{p}{)}\PY{o}{*}\PY{o}{*}\PY{l+m+mi}{2}\PY{p}{)}\PY{o}{*}\PY{o}{*}\PY{p}{(}\PY{l+m+mi}{1}\PY{o}{/}\PY{l+m+mi}{3}\PY{p}{)}

\PY{k}{def} \PY{n+nf}{T}\PY{p}{(}\PY{n}{a\PYZus{}m}\PY{p}{)}\PY{p}{:}
    \PY{k}{return} \PY{p}{(}\PY{l+m+mi}{2}\PY{o}{*}\PY{n}{np}\PY{o}{.}\PY{n}{pi}\PY{o}{/}\PY{n}{c}\PY{p}{)}\PY{o}{*}\PY{p}{(}\PY{n}{a\PYZus{}m}\PY{o}{*}\PY{o}{*}\PY{l+m+mi}{3}\PY{o}{/}\PY{p}{(}\PY{n}{M}\PY{o}{*}\PY{n}{m\PYZus{}sol}\PY{p}{)}\PY{p}{)}\PY{o}{*}\PY{o}{*}\PY{p}{(}\PY{l+m+mi}{1}\PY{o}{/}\PY{l+m+mi}{2}\PY{p}{)}
\end{Verbatim}
\end{tcolorbox}

    \begin{tcolorbox}[breakable, size=fbox, boxrule=1pt, pad at break*=1mm,colback=cellbackground, colframe=cellborder]
\prompt{In}{incolor}{16}{\boxspacing}
\begin{Verbatim}[commandchars=\\\{\}]
\PY{n}{a0\PYZus{}m} \PY{o}{=} \PY{n}{a}\PY{p}{(}\PY{n}{T0\PYZus{}s}\PY{p}{)} \PY{c+c1}{\PYZsh{} a inicial, en metros}
\PY{n}{at0} \PY{o}{=} \PY{n}{a0\PYZus{}m}\PY{o}{/}\PY{n}{R\PYZus{}ast} \PY{c+c1}{\PYZsh{} a tilde inicial}
\PY{n+nb}{print}\PY{p}{(}\PY{l+s+s1}{\PYZsq{}}\PY{l+s+s1}{a0 = }\PY{l+s+s1}{\PYZsq{}}\PY{o}{+}\PY{n+nb}{str}\PY{p}{(}\PY{n}{a0\PYZus{}m}\PY{p}{)}\PY{o}{+}\PY{l+s+s1}{\PYZsq{}}\PY{l+s+s1}{ m}\PY{l+s+s1}{\PYZsq{}}\PY{p}{)}
\PY{n+nb}{print}\PY{p}{(}\PY{l+s+s1}{\PYZsq{}}\PY{l+s+s1}{at0 = }\PY{l+s+s1}{\PYZsq{}}\PY{o}{+}\PY{n+nb}{str}\PY{p}{(}\PY{n}{at0}\PY{p}{)}\PY{p}{)}
\end{Verbatim}
\end{tcolorbox}

    \begin{Verbatim}[commandchars=\\\{\}]
a0 = 1949123981.98 m
at0 = 466740.893878
    \end{Verbatim}

    Dado que aquí resolveremos el sistema de ecuaciones dos veces (con
distintas condiciones iniciales), definiremos una función que nos
entrega todas las soluciones:

    \begin{tcolorbox}[breakable, size=fbox, boxrule=1pt, pad at break*=1mm,colback=cellbackground, colframe=cellborder]
\prompt{In}{incolor}{17}{\boxspacing}
\begin{Verbatim}[commandchars=\\\{\}]
\PY{k}{def} \PY{n+nf}{solucion}\PY{p}{(}\PY{n}{x0}\PY{p}{,}\PY{n}{tt\PYZus{}int}\PY{p}{)}\PY{p}{:}
    \PY{n+nb}{print} \PY{l+s+s1}{\PYZsq{}}\PY{l+s+s1}{Se resuelve con at0 = }\PY{l+s+s1}{\PYZpc{}}\PY{l+s+s1}{2.f y e0 = }\PY{l+s+s1}{\PYZpc{}}\PY{l+s+s1}{2.f}\PY{l+s+s1}{\PYZsq{}}\PY{o}{\PYZpc{}}\PY{p}{(}\PY{n}{x0}\PY{p}{[}\PY{l+m+mi}{0}\PY{p}{]}\PY{p}{,}\PY{n}{x0}\PY{p}{[}\PY{l+m+mi}{1}\PY{p}{]}\PY{p}{)}
    \PY{n}{sol} \PY{o}{=} \PY{n}{odeint}\PY{p}{(}\PY{n}{dotx}\PY{p}{,}\PY{n}{x0}\PY{p}{,}\PY{n}{tt\PYZus{}int}\PY{p}{)}
    \PY{n}{at\PYZus{}todos} \PY{o}{=} \PY{n}{sol}\PY{p}{[}\PY{p}{:}\PY{p}{,}\PY{l+m+mi}{0}\PY{p}{]}
    \PY{c+c1}{\PYZsh{} verifica si at llega a 2. En caso positivo corta el arreglo de soluciones}
    \PY{n}{restriccion} \PY{o}{=} \PY{n}{np}\PY{o}{.}\PY{n}{where}\PY{p}{(}\PY{n}{at\PYZus{}todos}\PY{o}{\PYZlt{}}\PY{l+m+mi}{2}\PY{p}{)}\PY{p}{[}\PY{l+m+mi}{0}\PY{p}{]}
    \PY{k}{if} \PY{n+nb}{len}\PY{p}{(}\PY{n}{restriccion}\PY{p}{)} \PY{o+ow}{is} \PY{o+ow}{not} \PY{l+m+mi}{0}\PY{p}{:}
        \PY{n}{pos\PYZus{}ttmax} \PY{o}{=} \PY{n}{restriccion}\PY{p}{[}\PY{l+m+mi}{0}\PY{p}{]} \PY{c+c1}{\PYZsh{} determina el tiempo en el que at=2}
        \PY{n+nb}{print}\PY{p}{(}\PY{l+s+s1}{\PYZsq{}}\PY{l+s+s1}{Acortando intervalo a tt\PYZus{}max = }\PY{l+s+s1}{\PYZsq{}}\PY{o}{+}\PY{n+nb}{str}\PY{p}{(}\PY{n}{tt\PYZus{}int}\PY{p}{[}\PY{n}{pos\PYZus{}ttmax}\PY{p}{]}\PY{p}{)}\PY{p}{)}
    \PY{k}{else}\PY{p}{:} 
        \PY{n}{pos\PYZus{}ttmax} \PY{o}{=} \PY{n+nb}{len}\PY{p}{(}\PY{n}{tt\PYZus{}int}\PY{p}{)}
    \PY{n}{tt} \PY{o}{=} \PY{n}{tt\PYZus{}int}\PY{p}{[}\PY{p}{:}\PY{n}{pos\PYZus{}ttmax}\PY{p}{]}
    \PY{n}{t\PYZus{}a} \PY{o}{=} \PY{n}{tt}\PY{o}{*}\PY{n}{R\PYZus{}ast}\PY{o}{/}\PY{n}{c}\PY{o}{/}\PY{l+m+mi}{31557600} \PY{c+c1}{\PYZsh{} el tiempo, en años}
    \PY{n}{at} \PY{o}{=} \PY{n}{sol}\PY{p}{[}\PY{p}{:}\PY{n}{pos\PYZus{}ttmax}\PY{p}{,}\PY{l+m+mi}{0}\PY{p}{]}
    \PY{n}{e} \PY{o}{=} \PY{n}{sol}\PY{p}{[}\PY{p}{:}\PY{n}{pos\PYZus{}ttmax}\PY{p}{,}\PY{l+m+mi}{1}\PY{p}{]}
    \PY{n}{a\PYZus{}m} \PY{o}{=} \PY{n}{at}\PY{o}{*}\PY{n}{R\PYZus{}ast} \PY{c+c1}{\PYZsh{} solución de a, en metros}
    \PY{n}{T\PYZus{}s} \PY{o}{=} \PY{n}{T}\PY{p}{(}\PY{n}{a\PYZus{}m}\PY{p}{)} \PY{c+c1}{\PYZsh{} solución de T, en segundos}
    \PY{k}{return} \PY{n}{tt}\PY{p}{,}\PY{n}{t\PYZus{}a}\PY{p}{,}\PY{n}{at}\PY{p}{,}\PY{n}{e}\PY{p}{,}\PY{n}{a\PYZus{}m}\PY{p}{,}\PY{n}{T\PYZus{}s}
\end{Verbatim}
\end{tcolorbox}

    \hypertarget{primera-integracion-hasta-el-colapso-final}{%
\section{Primera integracion: hasta el colapso
final!}\label{primera-integracion-hasta-el-colapso-final}}

    \begin{tcolorbox}[breakable, size=fbox, boxrule=1pt, pad at break*=1mm,colback=cellbackground, colframe=cellborder]
\prompt{In}{incolor}{18}{\boxspacing}
\begin{Verbatim}[commandchars=\\\{\}]
\PY{n}{tt\PYZus{}int\PYZus{}max} \PY{o}{=} \PY{l+m+mi}{10}\PY{o}{*}\PY{o}{*}\PY{l+m+mi}{22} \PY{c+c1}{\PYZsh{} tiempo adimensional máximo de integración. Con este valor se llega hasta a=2}
\PY{n}{tt\PYZus{}int} \PY{o}{=} \PY{n}{np}\PY{o}{.}\PY{n}{linspace}\PY{p}{(}\PY{l+m+mi}{0}\PY{p}{,}\PY{n}{tt\PYZus{}int\PYZus{}max}\PY{p}{,}\PY{l+m+mi}{100000}\PY{p}{)} \PY{c+c1}{\PYZsh{} tiempos en los que se integrará el sistema}
\PY{n+nb}{print}\PY{p}{(}\PY{l+s+s1}{\PYZsq{}}\PY{l+s+s1}{tt\PYZus{}int\PYZus{}max = }\PY{l+s+s1}{\PYZsq{}}\PY{o}{+}\PY{n+nb}{str}\PY{p}{(}\PY{n}{tt\PYZus{}int\PYZus{}max}\PY{p}{)}\PY{p}{)}
\PY{n}{x0} \PY{o}{=} \PY{p}{[}\PY{n}{at0}\PY{p}{,}\PY{n}{e0}\PY{p}{]} \PY{c+c1}{\PYZsh{} valores iniciales}
\PY{n}{tt}\PY{p}{,}\PY{n}{t\PYZus{}a}\PY{p}{,}\PY{n}{at}\PY{p}{,}\PY{n}{e}\PY{p}{,}\PY{n}{a\PYZus{}m}\PY{p}{,}\PY{n}{T\PYZus{}s} \PY{o}{=} \PY{n}{solucion}\PY{p}{(}\PY{n}{x0}\PY{p}{,}\PY{n}{tt\PYZus{}int}\PY{p}{)} \PY{c+c1}{\PYZsh{} calcula y asigna valores de la solución}
\end{Verbatim}
\end{tcolorbox}

    \begin{Verbatim}[commandchars=\\\{\}]
tt\_int\_max = 10000000000000000000000
Se resuelve con at0 = 466741 y e0 =  1
Acortando intervalo a tt\_max = 6.81206812068e+20
    \end{Verbatim}

    \begin{Verbatim}[commandchars=\\\{\}]
/usr/local/lib/python2.7/dist-packages/scipy/integrate/odepack.py:218:
ODEintWarning: Excess work done on this call (perhaps wrong Dfun type). Run with
full\_output = 1 to get quantitative information.
  warnings.warn(warning\_msg, ODEintWarning)
    \end{Verbatim}
Graficamos primero la solución para las cantidades adimensionalizadas:
    \begin{tcolorbox}[breakable, size=fbox, boxrule=1pt, pad at break*=1mm,colback=cellbackground, colframe=cellborder]
\prompt{In}{incolor}{19}{\boxspacing}
\begin{Verbatim}[commandchars=\\\{\}]
\PY{n}{fig}\PY{p}{,}\PY{n}{eje} \PY{o}{=} \PY{n}{plt}\PY{o}{.}\PY{n}{subplots}\PY{p}{(}\PY{l+m+mi}{1}\PY{p}{,}\PY{l+m+mi}{1}\PY{p}{,}\PY{n}{figsize}\PY{o}{=}\PY{p}{(}\PY{l+m+mi}{5}\PY{p}{,}\PY{l+m+mi}{5}\PY{p}{)}\PY{p}{)}
\PY{n}{eje}\PY{o}{.}\PY{n}{plot}\PY{p}{(}\PY{n}{tt}\PY{p}{,}\PY{n}{at}\PY{p}{)}
\PY{n}{eje}\PY{o}{.}\PY{n}{set\PYZus{}xlabel}\PY{p}{(}\PY{l+s+sa}{r}\PY{l+s+s1}{\PYZsq{}}\PY{l+s+s1}{\PYZdl{}}\PY{l+s+s1}{\PYZbs{}}\PY{l+s+s1}{tilde}\PY{l+s+si}{\PYZob{}t\PYZcb{}}\PY{l+s+s1}{\PYZdl{}}\PY{l+s+s1}{\PYZsq{}}\PY{p}{,}\PY{n}{fontsize}\PY{o}{=}\PY{l+m+mi}{15}\PY{p}{)}
\PY{n}{eje}\PY{o}{.}\PY{n}{set\PYZus{}ylabel}\PY{p}{(}\PY{l+s+sa}{r}\PY{l+s+s1}{\PYZsq{}}\PY{l+s+s1}{\PYZdl{}}\PY{l+s+s1}{\PYZbs{}}\PY{l+s+s1}{tilde}\PY{l+s+si}{\PYZob{}a\PYZcb{}}\PY{l+s+s1}{\PYZdl{}}\PY{l+s+s1}{\PYZsq{}}\PY{p}{,}\PY{n}{fontsize}\PY{o}{=}\PY{l+m+mi}{15}\PY{p}{)}
\PY{n}{plt}\PY{o}{.}\PY{n}{grid}\PY{p}{(}\PY{p}{)}
\end{Verbatim}
\end{tcolorbox}

    \begin{center}
    \adjustimage{max size={0.9\linewidth}{0.9\paperheight}}{Sistema_Binario-Evolucion_Temporal_y_observaciones_HyT_files/Sistema_Binario-Evolucion_Temporal_y_observaciones_HyT_37_0.png}
    \end{center}
    { \hspace*{\fill} \\}
    
    Como vemos, dadas las condiciones iniciales, se requiere un tiempo
adimensionalizado del orden de \(10^{21}\) para que el sistema colapse
(suponiendo que el modelo es válido incluso a pequeñas distancias, cosa
que en realidad no es cierta).

A continuación, graficamos las cantidades físicas (con dimensiones):

    \begin{tcolorbox}[breakable, size=fbox, boxrule=1pt, pad at break*=1mm,colback=cellbackground, colframe=cellborder]
\prompt{In}{incolor}{20}{\boxspacing}
\begin{Verbatim}[commandchars=\\\{\}]
\PY{n}{fig}\PY{p}{,}\PY{n}{eje} \PY{o}{=} \PY{n}{plt}\PY{o}{.}\PY{n}{subplots}\PY{p}{(}\PY{l+m+mi}{1}\PY{p}{,}\PY{l+m+mi}{1}\PY{p}{,}\PY{n}{figsize}\PY{o}{=}\PY{p}{(}\PY{l+m+mi}{5}\PY{p}{,}\PY{l+m+mi}{5}\PY{p}{)}\PY{p}{)}
\PY{n}{eje}\PY{o}{.}\PY{n}{plot}\PY{p}{(}\PY{n}{t\PYZus{}a}\PY{p}{,}\PY{n}{a\PYZus{}m}\PY{p}{)}
\PY{n}{eje}\PY{o}{.}\PY{n}{set\PYZus{}title}\PY{p}{(}\PY{l+s+sa}{u}\PY{l+s+s1}{\PYZsq{}}\PY{l+s+s1}{Evolución del semieje mayor}\PY{l+s+s1}{\PYZsq{}}\PY{p}{,}\PY{n}{fontsize}\PY{o}{=}\PY{l+m+mi}{15}\PY{p}{)}
\PY{n}{eje}\PY{o}{.}\PY{n}{set\PYZus{}xlabel}\PY{p}{(}\PY{l+s+sa}{r}\PY{l+s+s1}{\PYZsq{}}\PY{l+s+s1}{\PYZdl{}t}\PY{l+s+s1}{\PYZbs{}}\PY{l+s+s1}{ (a}\PY{l+s+s1}{\PYZbs{}}\PY{l+s+s1}{\PYZti{}nos)\PYZdl{}}\PY{l+s+s1}{\PYZsq{}}\PY{p}{,}\PY{n}{fontsize}\PY{o}{=}\PY{l+m+mi}{15}\PY{p}{)}
\PY{n}{eje}\PY{o}{.}\PY{n}{set\PYZus{}ylabel}\PY{p}{(}\PY{l+s+sa}{r}\PY{l+s+s1}{\PYZsq{}}\PY{l+s+s1}{\PYZdl{}a (m)\PYZdl{}}\PY{l+s+s1}{\PYZsq{}}\PY{p}{,}\PY{n}{fontsize}\PY{o}{=}\PY{l+m+mi}{15}\PY{p}{)}
\PY{n}{plt}\PY{o}{.}\PY{n}{grid}\PY{p}{(}\PY{p}{)}
\end{Verbatim}
\end{tcolorbox}

    \begin{center}
    \adjustimage{max size={0.9\linewidth}{0.9\paperheight}}{Sistema_Binario-Evolucion_Temporal_y_observaciones_HyT_files/Sistema_Binario-Evolucion_Temporal_y_observaciones_HyT_39_0.png}
    \end{center}
    { \hspace*{\fill} \\}
    
    Vemos entonces que el tiempo de colapso es del orden de \(10^8\) años.

Podemos también graficar cómo evoluciona la excentricidad:

    \begin{tcolorbox}[breakable, size=fbox, boxrule=1pt, pad at break*=1mm,colback=cellbackground, colframe=cellborder]
\prompt{In}{incolor}{21}{\boxspacing}
\begin{Verbatim}[commandchars=\\\{\}]
\PY{n}{fig}\PY{p}{,}\PY{n}{eje} \PY{o}{=} \PY{n}{plt}\PY{o}{.}\PY{n}{subplots}\PY{p}{(}\PY{l+m+mi}{1}\PY{p}{,}\PY{l+m+mi}{1}\PY{p}{,}\PY{n}{figsize}\PY{o}{=}\PY{p}{(}\PY{l+m+mi}{5}\PY{p}{,}\PY{l+m+mi}{5}\PY{p}{)}\PY{p}{)}
\PY{n}{eje}\PY{o}{.}\PY{n}{plot}\PY{p}{(}\PY{n}{t\PYZus{}a}\PY{p}{,}\PY{n}{e}\PY{p}{)}
\PY{n}{eje}\PY{o}{.}\PY{n}{set\PYZus{}title}\PY{p}{(}\PY{l+s+sa}{u}\PY{l+s+s1}{\PYZsq{}}\PY{l+s+s1}{Evolución de la excentricidad}\PY{l+s+s1}{\PYZsq{}}\PY{p}{,}\PY{n}{fontsize}\PY{o}{=}\PY{l+m+mi}{15}\PY{p}{)}
\PY{n}{eje}\PY{o}{.}\PY{n}{set\PYZus{}xlabel}\PY{p}{(}\PY{l+s+sa}{r}\PY{l+s+s1}{\PYZsq{}}\PY{l+s+s1}{\PYZdl{}t}\PY{l+s+s1}{\PYZbs{}}\PY{l+s+s1}{ (a}\PY{l+s+s1}{\PYZbs{}}\PY{l+s+s1}{\PYZti{}nos)\PYZdl{}}\PY{l+s+s1}{\PYZsq{}}\PY{p}{,}\PY{n}{fontsize}\PY{o}{=}\PY{l+m+mi}{15}\PY{p}{)}
\PY{n}{eje}\PY{o}{.}\PY{n}{set\PYZus{}ylabel}\PY{p}{(}\PY{l+s+sa}{r}\PY{l+s+s1}{\PYZsq{}}\PY{l+s+s1}{\PYZdl{}e\PYZdl{}}\PY{l+s+s1}{\PYZsq{}}\PY{p}{,}\PY{n}{fontsize}\PY{o}{=}\PY{l+m+mi}{15}\PY{p}{)}
\PY{n}{plt}\PY{o}{.}\PY{n}{grid}\PY{p}{(}\PY{p}{)}
\end{Verbatim}
\end{tcolorbox}

    \begin{center}
    \adjustimage{max size={0.9\linewidth}{0.9\paperheight}}{Sistema_Binario-Evolucion_Temporal_y_observaciones_HyT_files/Sistema_Binario-Evolucion_Temporal_y_observaciones_HyT_41_0.png}
    \end{center}
    { \hspace*{\fill} \\}
    
    Finalmente, graficamos la evolución del periodo orbital del sistema:

    \begin{tcolorbox}[breakable, size=fbox, boxrule=1pt, pad at break*=1mm,colback=cellbackground, colframe=cellborder]
\prompt{In}{incolor}{22}{\boxspacing}
\begin{Verbatim}[commandchars=\\\{\}]
\PY{n}{T\PYZus{}h} \PY{o}{=} \PY{n}{T\PYZus{}s}\PY{o}{/}\PY{l+m+mf}{3600.} \PY{c+c1}{\PYZsh{} periodo orbital, en horas}
\PY{n}{plt}\PY{o}{.}\PY{n}{figure}\PY{p}{(}\PY{n}{figsize}\PY{o}{=}\PY{p}{(}\PY{l+m+mi}{5}\PY{p}{,}\PY{l+m+mi}{5}\PY{p}{)}\PY{p}{)}
\PY{n}{plt}\PY{o}{.}\PY{n}{plot}\PY{p}{(}\PY{n}{t\PYZus{}a}\PY{p}{,}\PY{n}{T\PYZus{}h}\PY{p}{)}
\PY{n}{plt}\PY{o}{.}\PY{n}{title}\PY{p}{(}\PY{l+s+sa}{u}\PY{l+s+s1}{\PYZsq{}}\PY{l+s+s1}{Evolución del Periodo orbital}\PY{l+s+s1}{\PYZsq{}}\PY{p}{,}\PY{n}{fontsize}\PY{o}{=}\PY{l+m+mi}{15}\PY{p}{)}
\PY{n}{plt}\PY{o}{.}\PY{n}{xlabel}\PY{p}{(}\PY{l+s+sa}{u}\PY{l+s+s1}{\PYZsq{}}\PY{l+s+s1}{\PYZdl{}t\PYZdl{} (años)}\PY{l+s+s1}{\PYZsq{}}\PY{p}{,}\PY{n}{fontsize}\PY{o}{=}\PY{l+m+mi}{15}\PY{p}{)}
\PY{n}{plt}\PY{o}{.}\PY{n}{ylabel}\PY{p}{(}\PY{l+s+sa}{r}\PY{l+s+s1}{\PYZsq{}}\PY{l+s+s1}{\PYZdl{}T\PYZdl{} (horas)}\PY{l+s+s1}{\PYZsq{}}\PY{p}{,}\PY{n}{fontsize}\PY{o}{=}\PY{l+m+mi}{15}\PY{p}{)}
\PY{n}{plt}\PY{o}{.}\PY{n}{grid}\PY{p}{(}\PY{p}{)}
\end{Verbatim}
\end{tcolorbox}

    \begin{center}
    \adjustimage{max size={0.9\linewidth}{0.9\paperheight}}{Sistema_Binario-Evolucion_Temporal_y_observaciones_HyT_files/Sistema_Binario-Evolucion_Temporal_y_observaciones_HyT_43_0.png}
    \end{center}
    { \hspace*{\fill} \\}
    
    \hypertarget{graficando-la-dependencia-de-a-con-e}{%
\section{\texorpdfstring{Graficando la dependencia de \(a\) con
\(e\)}{Graficando la dependencia de a con e}}\label{graficando-la-dependencia-de-a-con-e}}

    Primero definimos la función \(g(e)\) y la graficamos

    \begin{tcolorbox}[breakable, size=fbox, boxrule=1pt, pad at break*=1mm,colback=cellbackground, colframe=cellborder]
\prompt{In}{incolor}{23}{\boxspacing}
\begin{Verbatim}[commandchars=\\\{\}]
\PY{k}{def} \PY{n+nf}{g}\PY{p}{(}\PY{n}{e}\PY{p}{)}\PY{p}{:}
    \PY{k}{return} \PY{n}{e}\PY{o}{*}\PY{o}{*}\PY{p}{(}\PY{l+m+mi}{12}\PY{o}{/}\PY{l+m+mi}{19}\PY{p}{)}\PY{o}{*}\PY{p}{(}\PY{l+m+mi}{1}\PY{o}{+}\PY{l+m+mi}{121}\PY{o}{*}\PY{n}{e}\PY{o}{*}\PY{o}{*}\PY{l+m+mi}{2}\PY{o}{/}\PY{l+m+mi}{304}\PY{p}{)}\PY{o}{*}\PY{o}{*}\PY{p}{(}\PY{l+m+mi}{870}\PY{o}{/}\PY{l+m+mi}{2299}\PY{p}{)}\PY{o}{/}\PY{p}{(}\PY{l+m+mi}{1}\PY{o}{\PYZhy{}}\PY{n}{e}\PY{o}{*}\PY{o}{*}\PY{l+m+mi}{2}\PY{p}{)}
\end{Verbatim}
\end{tcolorbox}

    \begin{tcolorbox}[breakable, size=fbox, boxrule=1pt, pad at break*=1mm,colback=cellbackground, colframe=cellborder]
\prompt{In}{incolor}{24}{\boxspacing}
\begin{Verbatim}[commandchars=\\\{\}]
\PY{n}{fig}\PY{p}{,}\PY{n}{eje}\PY{o}{=} \PY{n}{plt}\PY{o}{.}\PY{n}{subplots}\PY{p}{(}\PY{l+m+mi}{1}\PY{p}{,}\PY{l+m+mi}{1}\PY{p}{,}\PY{n}{figsize}\PY{o}{=}\PY{p}{(}\PY{l+m+mi}{5}\PY{p}{,}\PY{l+m+mi}{5}\PY{p}{)}\PY{p}{)}
\PY{n}{ee} \PY{o}{=} \PY{n}{np}\PY{o}{.}\PY{n}{linspace}\PY{p}{(}\PY{l+m+mi}{0}\PY{p}{,}\PY{l+m+mi}{1}\PY{p}{,}\PY{l+m+mi}{100}\PY{p}{)}
\PY{n}{eje}\PY{o}{.}\PY{n}{plot}\PY{p}{(}\PY{n}{ee}\PY{p}{,}\PY{n}{g}\PY{p}{(}\PY{n}{ee}\PY{p}{)}\PY{p}{)}
\PY{n}{eje}\PY{o}{.}\PY{n}{set\PYZus{}yscale}\PY{p}{(}\PY{l+s+s1}{\PYZsq{}}\PY{l+s+s1}{log}\PY{l+s+s1}{\PYZsq{}}\PY{p}{)}
\PY{n}{eje}\PY{o}{.}\PY{n}{set\PYZus{}title}\PY{p}{(}\PY{l+s+sa}{r}\PY{l+s+s1}{\PYZsq{}}\PY{l+s+s1}{\PYZdl{}g\PYZdl{} versus \PYZdl{}e\PYZdl{}}\PY{l+s+s1}{\PYZsq{}}\PY{p}{,}\PY{n}{fontsize}\PY{o}{=}\PY{l+m+mi}{15}\PY{p}{)}
\PY{n}{eje}\PY{o}{.}\PY{n}{set\PYZus{}xlabel}\PY{p}{(}\PY{l+s+sa}{r}\PY{l+s+s1}{\PYZsq{}}\PY{l+s+s1}{\PYZdl{}e\PYZdl{}}\PY{l+s+s1}{\PYZsq{}}\PY{p}{,}\PY{n}{fontsize}\PY{o}{=}\PY{l+m+mi}{15}\PY{p}{)}
\PY{n}{eje}\PY{o}{.}\PY{n}{set\PYZus{}ylabel}\PY{p}{(}\PY{l+s+sa}{r}\PY{l+s+s1}{\PYZsq{}}\PY{l+s+s1}{\PYZdl{}g\PYZdl{}}\PY{l+s+s1}{\PYZsq{}}\PY{p}{,}\PY{n}{fontsize}\PY{o}{=}\PY{l+m+mi}{15}\PY{p}{)}
\PY{c+c1}{\PYZsh{}plt.legend(loc=\PYZsq{}best\PYZsq{})}
\PY{n}{plt}\PY{o}{.}\PY{n}{grid}\PY{p}{(}\PY{p}{)}
\end{Verbatim}
\end{tcolorbox}

    \begin{Verbatim}[commandchars=\\\{\}]
/usr/local/lib/python2.7/dist-packages/ipykernel/\_\_main\_\_.py:2: RuntimeWarning:
divide by zero encountered in true\_divide
  from ipykernel import kernelapp as app
    \end{Verbatim}

    \begin{center}
    \adjustimage{max size={0.9\linewidth}{0.9\paperheight}}{Sistema_Binario-Evolucion_Temporal_y_observaciones_HyT_files/Sistema_Binario-Evolucion_Temporal_y_observaciones_HyT_47_1.png}
    \end{center}
    { \hspace*{\fill} \\}
    
    Ahora graficamos \(a\) en términos de \(e\), tanto para la solución
analítica como numérica

    \begin{tcolorbox}[breakable, size=fbox, boxrule=1pt, pad at break*=1mm,colback=cellbackground, colframe=cellborder]
\prompt{In}{incolor}{25}{\boxspacing}
\begin{Verbatim}[commandchars=\\\{\}]
\PY{n}{fig}\PY{p}{,}\PY{n}{eje}\PY{o}{=} \PY{n}{plt}\PY{o}{.}\PY{n}{subplots}\PY{p}{(}\PY{l+m+mi}{1}\PY{p}{,}\PY{l+m+mi}{1}\PY{p}{,}\PY{n}{figsize}\PY{o}{=}\PY{p}{(}\PY{l+m+mi}{5}\PY{p}{,}\PY{l+m+mi}{5}\PY{p}{)}\PY{p}{)}
\PY{n}{eje}\PY{o}{.}\PY{n}{plot}\PY{p}{(}\PY{n}{e}\PY{p}{,}\PY{n}{at}\PY{o}{*}\PY{n}{R\PYZus{}ast}\PY{p}{,} \PY{n}{label}\PY{o}{=}\PY{l+s+sa}{u}\PY{l+s+s1}{\PYZsq{}}\PY{l+s+s1}{sol. numérica}\PY{l+s+s1}{\PYZsq{}}\PY{p}{)}
\PY{n}{ee} \PY{o}{=} \PY{n}{np}\PY{o}{.}\PY{n}{linspace}\PY{p}{(}\PY{n+nb}{min}\PY{p}{(}\PY{n}{e}\PY{p}{)}\PY{p}{,}\PY{n}{e0}\PY{p}{,}\PY{l+m+mi}{10}\PY{p}{)}
\PY{n}{a\PYZus{}an} \PY{o}{=} \PY{n}{a0\PYZus{}m}\PY{o}{*}\PY{n}{g}\PY{p}{(}\PY{n}{ee}\PY{p}{)}\PY{o}{/}\PY{n}{g}\PY{p}{(}\PY{n}{e0}\PY{p}{)}
\PY{n}{eje}\PY{o}{.}\PY{n}{plot}\PY{p}{(}\PY{n}{ee}\PY{p}{,}\PY{n}{a\PYZus{}an}\PY{p}{,}\PY{l+s+s1}{\PYZsq{}}\PY{l+s+s1}{o}\PY{l+s+s1}{\PYZsq{}}\PY{p}{,}\PY{n}{label}\PY{o}{=}\PY{l+s+sa}{u}\PY{l+s+s1}{\PYZsq{}}\PY{l+s+s1}{sol. analítica}\PY{l+s+s1}{\PYZsq{}}\PY{p}{)}
\PY{n}{eje}\PY{o}{.}\PY{n}{set\PYZus{}title}\PY{p}{(}\PY{l+s+sa}{u}\PY{l+s+s1}{\PYZsq{}}\PY{l+s+s1}{Semieje mayor v/s excentricidad}\PY{l+s+s1}{\PYZsq{}}\PY{p}{,}\PY{n}{fontsize}\PY{o}{=}\PY{l+m+mi}{14}\PY{p}{)}
\PY{n}{eje}\PY{o}{.}\PY{n}{set\PYZus{}xlabel}\PY{p}{(}\PY{l+s+sa}{r}\PY{l+s+s1}{\PYZsq{}}\PY{l+s+s1}{\PYZdl{}e\PYZdl{}}\PY{l+s+s1}{\PYZsq{}}\PY{p}{,}\PY{n}{fontsize}\PY{o}{=}\PY{l+m+mi}{15}\PY{p}{)}
\PY{n}{eje}\PY{o}{.}\PY{n}{set\PYZus{}ylabel}\PY{p}{(}\PY{l+s+sa}{r}\PY{l+s+s1}{\PYZsq{}}\PY{l+s+s1}{\PYZdl{}a\PYZdl{}}\PY{l+s+s1}{\PYZsq{}}\PY{p}{,}\PY{n}{fontsize}\PY{o}{=}\PY{l+m+mi}{15}\PY{p}{)}
\PY{n}{plt}\PY{o}{.}\PY{n}{legend}\PY{p}{(}\PY{n}{loc}\PY{o}{=}\PY{l+s+s1}{\PYZsq{}}\PY{l+s+s1}{best}\PY{l+s+s1}{\PYZsq{}}\PY{p}{)}
\PY{n}{plt}\PY{o}{.}\PY{n}{grid}\PY{p}{(}\PY{p}{)}
\end{Verbatim}
\end{tcolorbox}

    \begin{center}
    \adjustimage{max size={0.9\linewidth}{0.9\paperheight}}{Sistema_Binario-Evolucion_Temporal_y_observaciones_HyT_files/Sistema_Binario-Evolucion_Temporal_y_observaciones_HyT_49_0.png}
    \end{center}
    { \hspace*{\fill} \\}
    
    Como vemos, en el tiempo de observación del pulsar binario,
aproximadamente 30 años, el decaimiento tanto de \(a\) como \(e\) será
en la práctica a una tasa constante (línea recta en el gráfico en
función del tiempo).

Resolvemos nuevamente el sistema, pero sólo en el intervalo de tiempo de
30 años:

    \begin{tcolorbox}[breakable, size=fbox, boxrule=1pt, pad at break*=1mm,colback=cellbackground, colframe=cellborder]
\prompt{In}{incolor}{26}{\boxspacing}
\begin{Verbatim}[commandchars=\\\{\}]
\PY{n}{t\PYZus{}max\PYZus{}a} \PY{o}{=} \PY{l+m+mi}{30} \PY{c+c1}{\PYZsh{} tiempo de integración, en años}
\PY{n}{tt\PYZus{}int\PYZus{}max} \PY{o}{=} \PY{l+m+mi}{31557600}\PY{o}{*}\PY{n}{c}\PY{o}{*}\PY{n}{t\PYZus{}max\PYZus{}a}\PY{o}{/}\PY{n}{R\PYZus{}ast} \PY{c+c1}{\PYZsh{}tiempo adimensional máximo de integración}

\PY{n}{tt\PYZus{}int} \PY{o}{=} \PY{n}{np}\PY{o}{.}\PY{n}{linspace}\PY{p}{(}\PY{l+m+mi}{0}\PY{p}{,}\PY{n}{tt\PYZus{}int\PYZus{}max}\PY{p}{,}\PY{l+m+mi}{100000}\PY{p}{)}
\PY{n+nb}{print}\PY{p}{(}\PY{l+s+s1}{\PYZsq{}}\PY{l+s+s1}{tt\PYZus{}int\PYZus{}max = }\PY{l+s+s1}{\PYZsq{}}\PY{o}{+}\PY{n+nb}{str}\PY{p}{(}\PY{n}{tt\PYZus{}int\PYZus{}max}\PY{p}{)}\PY{p}{)}

\PY{n}{tt}\PY{p}{,}\PY{n}{t\PYZus{}a}\PY{p}{,}\PY{n}{at}\PY{p}{,}\PY{n}{e}\PY{p}{,}\PY{n}{a\PYZus{}m}\PY{p}{,}\PY{n}{T\PYZus{}s} \PY{o}{=} \PY{n}{solucion}\PY{p}{(}\PY{n}{x0}\PY{p}{,}\PY{n}{tt\PYZus{}int}\PY{p}{)}
\end{Verbatim}
\end{tcolorbox}

    \begin{Verbatim}[commandchars=\\\{\}]
tt\_int\_max = 6.79645292707e+13
Se resuelve con at0 = 466741 y e0 =  1
    \end{Verbatim}

    \begin{tcolorbox}[breakable, size=fbox, boxrule=1pt, pad at break*=1mm,colback=cellbackground, colframe=cellborder]
\prompt{In}{incolor}{27}{\boxspacing}
\begin{Verbatim}[commandchars=\\\{\}]
\PY{n}{fig}\PY{p}{,}\PY{n}{eje} \PY{o}{=} \PY{n}{plt}\PY{o}{.}\PY{n}{subplots}\PY{p}{(}\PY{l+m+mi}{1}\PY{p}{,}\PY{l+m+mi}{1}\PY{p}{,}\PY{n}{figsize}\PY{o}{=}\PY{p}{(}\PY{l+m+mi}{5}\PY{p}{,}\PY{l+m+mi}{5}\PY{p}{)}\PY{p}{)}
\PY{n}{eje}\PY{o}{.}\PY{n}{plot}\PY{p}{(}\PY{n}{t\PYZus{}a}\PY{p}{,}\PY{n}{at}\PY{o}{/}\PY{n}{at0}\PY{o}{\PYZhy{}}\PY{l+m+mi}{1}\PY{p}{)}
\PY{n}{eje}\PY{o}{.}\PY{n}{set\PYZus{}title}\PY{p}{(}\PY{l+s+sa}{u}\PY{l+s+s1}{\PYZsq{}}\PY{l+s+s1}{Evolución del semieje mayor}\PY{l+s+s1}{\PYZsq{}}\PY{p}{,}\PY{n}{fontsize}\PY{o}{=}\PY{l+m+mi}{15}\PY{p}{)}
\PY{n}{eje}\PY{o}{.}\PY{n}{set\PYZus{}xlabel}\PY{p}{(}\PY{l+s+sa}{r}\PY{l+s+s1}{\PYZsq{}}\PY{l+s+s1}{\PYZdl{}t}\PY{l+s+s1}{\PYZbs{}}\PY{l+s+s1}{ (a}\PY{l+s+s1}{\PYZbs{}}\PY{l+s+s1}{\PYZti{}nos)\PYZdl{}}\PY{l+s+s1}{\PYZsq{}}\PY{p}{,}\PY{n}{fontsize}\PY{o}{=}\PY{l+m+mi}{15}\PY{p}{)}
\PY{n}{eje}\PY{o}{.}\PY{n}{set\PYZus{}ylabel}\PY{p}{(}\PY{l+s+sa}{r}\PY{l+s+s1}{\PYZsq{}}\PY{l+s+s1}{\PYZdl{}(a\PYZhy{}a\PYZus{}0)/a\PYZus{}0\PYZdl{}}\PY{l+s+s1}{\PYZsq{}}\PY{p}{,}\PY{n}{fontsize}\PY{o}{=}\PY{l+m+mi}{15}\PY{p}{)}
\PY{n}{plt}\PY{o}{.}\PY{n}{grid}\PY{p}{(}\PY{p}{)}
\end{Verbatim}
\end{tcolorbox}

    \begin{center}
    \adjustimage{max size={0.9\linewidth}{0.9\paperheight}}{Sistema_Binario-Evolucion_Temporal_y_observaciones_HyT_files/Sistema_Binario-Evolucion_Temporal_y_observaciones_HyT_52_0.png}
    \end{center}
    { \hspace*{\fill} \\}
    
    \begin{tcolorbox}[breakable, size=fbox, boxrule=1pt, pad at break*=1mm,colback=cellbackground, colframe=cellborder]
\prompt{In}{incolor}{28}{\boxspacing}
\begin{Verbatim}[commandchars=\\\{\}]
\PY{n}{fig}\PY{p}{,}\PY{n}{eje} \PY{o}{=} \PY{n}{plt}\PY{o}{.}\PY{n}{subplots}\PY{p}{(}\PY{l+m+mi}{1}\PY{p}{,}\PY{l+m+mi}{1}\PY{p}{,}\PY{n}{figsize}\PY{o}{=}\PY{p}{(}\PY{l+m+mi}{5}\PY{p}{,}\PY{l+m+mi}{5}\PY{p}{)}\PY{p}{)}
\PY{n}{eje}\PY{o}{.}\PY{n}{plot}\PY{p}{(}\PY{n}{t\PYZus{}a}\PY{p}{,}\PY{n}{e}\PY{o}{\PYZhy{}}\PY{n}{e0}\PY{p}{)}
\PY{n}{eje}\PY{o}{.}\PY{n}{set\PYZus{}title}\PY{p}{(}\PY{l+s+sa}{u}\PY{l+s+s1}{\PYZsq{}}\PY{l+s+s1}{Evolución de la excentricidad}\PY{l+s+s1}{\PYZsq{}}\PY{p}{,}\PY{n}{fontsize}\PY{o}{=}\PY{l+m+mi}{15}\PY{p}{)}
\PY{n}{eje}\PY{o}{.}\PY{n}{set\PYZus{}xlabel}\PY{p}{(}\PY{l+s+sa}{r}\PY{l+s+s1}{\PYZsq{}}\PY{l+s+s1}{\PYZdl{}t}\PY{l+s+s1}{\PYZbs{}}\PY{l+s+s1}{ (a}\PY{l+s+s1}{\PYZbs{}}\PY{l+s+s1}{\PYZti{}nos)\PYZdl{}}\PY{l+s+s1}{\PYZsq{}}\PY{p}{,}\PY{n}{fontsize}\PY{o}{=}\PY{l+m+mi}{15}\PY{p}{)}
\PY{n}{eje}\PY{o}{.}\PY{n}{set\PYZus{}ylabel}\PY{p}{(}\PY{l+s+sa}{r}\PY{l+s+s1}{\PYZsq{}}\PY{l+s+s1}{\PYZdl{}e\PYZdl{}}\PY{l+s+s1}{\PYZsq{}}\PY{p}{,}\PY{n}{fontsize}\PY{o}{=}\PY{l+m+mi}{15}\PY{p}{)}
\PY{n}{plt}\PY{o}{.}\PY{n}{grid}\PY{p}{(}\PY{p}{)}
\end{Verbatim}
\end{tcolorbox}

    \begin{center}
    \adjustimage{max size={0.9\linewidth}{0.9\paperheight}}{Sistema_Binario-Evolucion_Temporal_y_observaciones_HyT_files/Sistema_Binario-Evolucion_Temporal_y_observaciones_HyT_53_0.png}
    \end{center}
    { \hspace*{\fill} \\}
    
    \begin{tcolorbox}[breakable, size=fbox, boxrule=1pt, pad at break*=1mm,colback=cellbackground, colframe=cellborder]
\prompt{In}{incolor}{29}{\boxspacing}
\begin{Verbatim}[commandchars=\\\{\}]
\PY{n}{T\PYZus{}h} \PY{o}{=} \PY{n}{T\PYZus{}s}\PY{o}{/}\PY{l+m+mf}{3600.} \PY{c+c1}{\PYZsh{} periodo orbital, en horas}

\PY{n}{fig}\PY{p}{,}\PY{n}{eje} \PY{o}{=} \PY{n}{plt}\PY{o}{.}\PY{n}{subplots}\PY{p}{(}\PY{l+m+mi}{1}\PY{p}{,}\PY{l+m+mi}{1}\PY{p}{,}\PY{n}{figsize}\PY{o}{=}\PY{p}{(}\PY{l+m+mi}{5}\PY{p}{,}\PY{l+m+mi}{5}\PY{p}{)}\PY{p}{)}
\PY{n}{eje}\PY{o}{.}\PY{n}{plot}\PY{p}{(}\PY{n}{t\PYZus{}a}\PY{p}{,}\PY{n}{T\PYZus{}h}\PY{o}{\PYZhy{}}\PY{n}{T\PYZus{}h}\PY{p}{[}\PY{l+m+mi}{0}\PY{p}{]}\PY{p}{)}
\PY{n}{eje}\PY{o}{.}\PY{n}{set\PYZus{}title}\PY{p}{(}\PY{l+s+sa}{u}\PY{l+s+s1}{\PYZsq{}}\PY{l+s+s1}{Evolución del periodo orbiral}\PY{l+s+s1}{\PYZsq{}}\PY{p}{,}\PY{n}{fontsize}\PY{o}{=}\PY{l+m+mi}{15}\PY{p}{)}
\PY{n}{eje}\PY{o}{.}\PY{n}{set\PYZus{}xlabel}\PY{p}{(}\PY{l+s+sa}{u}\PY{l+s+s1}{\PYZsq{}}\PY{l+s+s1}{\PYZdl{}t\PYZdl{} (años)}\PY{l+s+s1}{\PYZsq{}}\PY{p}{,}\PY{n}{fontsize}\PY{o}{=}\PY{l+m+mi}{15}\PY{p}{)}
\PY{n}{eje}\PY{o}{.}\PY{n}{set\PYZus{}ylabel}\PY{p}{(}\PY{l+s+sa}{r}\PY{l+s+s1}{\PYZsq{}}\PY{l+s+s1}{\PYZdl{}T\PYZhy{}T\PYZus{}0\PYZdl{} (horas)}\PY{l+s+s1}{\PYZsq{}}\PY{p}{,}\PY{n}{fontsize}\PY{o}{=}\PY{l+m+mi}{15}\PY{p}{)}
\PY{n}{plt}\PY{o}{.}\PY{n}{grid}\PY{p}{(}\PY{p}{)}
\end{Verbatim}
\end{tcolorbox}

    \begin{center}
    \adjustimage{max size={0.9\linewidth}{0.9\paperheight}}{Sistema_Binario-Evolucion_Temporal_y_observaciones_HyT_files/Sistema_Binario-Evolucion_Temporal_y_observaciones_HyT_54_0.png}
    \end{center}
    { \hspace*{\fill} \\}
    
    Este comportamiento implica que en este intervalo de tiempo de
aproximadamente 30 años los valores de \(\dot{T}\), \(\dot{a}\) y
\(\dot{e}\) pueden considerarse constantes. Con el valor de \(\dot{T}\)
podemos modelar el retardo acumulado en el movimiento orbital del
sistema. Si \(\dot{T}=\) cte. entonces el tiempo transcurrido hasta
completar la \(n\)-ésima revolución es determinado por las relaciones \[
T(t)\approx T_0 + \dot{T}(t-t_0),
\] \[
t_{n+1} \approx t_n + T(t_n),
\] que al ser iteradas implican que \begin{equation}
t_n \approx t_0+nT_0+\dot{T}T_0\frac{n(n-1)}{2}+O(\dot{T}{}^2).
\end{equation} Por lo tanto, el \emph{retardo} respecto al valor
newtoniano (\(t_n^{\rm Newton}=t_0+nT_0\)), luego de \(n\) revoluciones
es dado por \begin{equation}
(\Delta t)_n \approx \dot{T}T_0\frac{n(n-1)}{2}+O(\dot{T}{}^2).
\end{equation}

El valor de \(\dot{T}\) puede ser evaluado usando la función dotx que
definimos previamente, y con la relación \begin{equation}
\dot{T}=\frac{3}{2}\frac{c}{R_\ast}\frac{T}{\tilde{a}}\frac{d\tilde{a}}{d\tilde{t}}
\end{equation}

    \begin{tcolorbox}[breakable, size=fbox, boxrule=1pt, pad at break*=1mm,colback=cellbackground, colframe=cellborder]
\prompt{In}{incolor}{30}{\boxspacing}
\begin{Verbatim}[commandchars=\\\{\}]
\PY{n}{dota} \PY{o}{=} \PY{n}{dotx}\PY{p}{(}\PY{n}{x0}\PY{p}{,}\PY{l+m+mi}{0}\PY{p}{)}\PY{p}{[}\PY{l+m+mi}{0}\PY{p}{]}
\PY{n}{dotT} \PY{o}{=} \PY{p}{(}\PY{l+m+mi}{3}\PY{o}{/}\PY{l+m+mi}{2}\PY{p}{)}\PY{o}{*}\PY{p}{(}\PY{n}{c}\PY{o}{/}\PY{n}{R\PYZus{}ast}\PY{p}{)}\PY{o}{*}\PY{p}{(}\PY{n}{T0\PYZus{}s}\PY{o}{/}\PY{n}{at0}\PY{p}{)}\PY{o}{*}\PY{n}{dota}
\PY{n+nb}{print}\PY{p}{(}\PY{l+s+s1}{\PYZsq{}}\PY{l+s+s1}{dT/dt= }\PY{l+s+s1}{\PYZsq{}} \PY{o}{+} \PY{n+nb}{str}\PY{p}{(}\PY{n}{dotT}\PY{p}{)}\PY{p}{)}
\end{Verbatim}
\end{tcolorbox}

    \begin{Verbatim}[commandchars=\\\{\}]
dT/dt= -2.40256023445e-12
    \end{Verbatim}

    Este valor concuerda con el reportado por
\href{http://dx.doi.org/10.1088/0004-637X/722/2/1030\%5D}{Weisberg, Nice
y Taylor (2010)} (http://arxiv.org/abs/1011.0718v1), ver ec. (4).

Para comparar con los datos observacionales, cargamos los valores del
retardo acumulado, obtenidos a partir del
\href{http://arxiv.org/e-print/1011.0718v1}{gráfico original de
Wiesberg, Nice y Taylor (2010)} usando
\href{http://arohatgi.info/WebPlotDigitizer/app/}{WebPlotDigitizer} para
extraer los valores.

    \begin{tcolorbox}[breakable, size=fbox, boxrule=1pt, pad at break*=1mm,colback=cellbackground, colframe=cellborder]
\prompt{In}{incolor}{31}{\boxspacing}
\begin{Verbatim}[commandchars=\\\{\}]
\PY{n}{data} \PY{o}{=} \PY{n}{np}\PY{o}{.}\PY{n}{genfromtxt}\PY{p}{(}\PY{l+s+s1}{\PYZsq{}}\PY{l+s+s1}{data\PYZhy{}HT.txt}\PY{l+s+s1}{\PYZsq{}}\PY{p}{)}
\PY{n}{t\PYZus{}exp} \PY{o}{=} \PY{n}{data}\PY{p}{[}\PY{p}{:}\PY{p}{,}\PY{l+m+mi}{0}\PY{p}{]}\PY{o}{\PYZhy{}}\PY{n}{data}\PY{p}{[}\PY{l+m+mi}{0}\PY{p}{,}\PY{l+m+mi}{0}\PY{p}{]}
\PY{n}{Delta\PYZus{}t\PYZus{}exp} \PY{o}{=} \PY{n}{data}\PY{p}{[}\PY{p}{:}\PY{p}{,}\PY{l+m+mi}{1}\PY{p}{]}
\end{Verbatim}
\end{tcolorbox}

    \begin{tcolorbox}[breakable, size=fbox, boxrule=1pt, pad at break*=1mm,colback=cellbackground, colframe=cellborder]
\prompt{In}{incolor}{32}{\boxspacing}
\begin{Verbatim}[commandchars=\\\{\}]
\PY{n}{fig}\PY{p}{,}\PY{n}{eje} \PY{o}{=} \PY{n}{plt}\PY{o}{.}\PY{n}{subplots}\PY{p}{(}\PY{l+m+mi}{1}\PY{p}{,}\PY{l+m+mi}{1}\PY{p}{,}\PY{n}{figsize}\PY{o}{=}\PY{p}{(}\PY{l+m+mi}{5}\PY{p}{,}\PY{l+m+mi}{5}\PY{p}{)}\PY{p}{)}
\PY{n}{n} \PY{o}{=} \PY{n}{np}\PY{o}{.}\PY{n}{arange}\PY{p}{(}\PY{l+m+mi}{40000}\PY{p}{)}
\PY{n}{t\PYZus{}n} \PY{o}{=} \PY{p}{(}\PY{n}{n}\PY{o}{*}\PY{n}{T0\PYZus{}s}\PY{o}{+}\PY{n}{dotT}\PY{o}{*}\PY{n}{T0\PYZus{}s}\PY{o}{*}\PY{n}{n}\PY{o}{*}\PY{p}{(}\PY{n}{n}\PY{o}{\PYZhy{}}\PY{l+m+mi}{1}\PY{p}{)}\PY{o}{/}\PY{l+m+mf}{2.}\PY{p}{)}\PY{o}{/}\PY{l+m+mf}{31557600.} \PY{c+c1}{\PYZsh{} tiempo, en años}
\PY{n}{Delta\PYZus{}t\PYZus{}n} \PY{o}{=} \PY{n}{dotT}\PY{o}{*}\PY{n}{T0\PYZus{}s}\PY{o}{*}\PY{n}{n}\PY{o}{*}\PY{p}{(}\PY{n}{n}\PY{o}{\PYZhy{}}\PY{l+m+mi}{1}\PY{p}{)}\PY{o}{/}\PY{l+m+mi}{2} \PY{c+c1}{\PYZsh{}retraso acumulado, en segundos}
\PY{n}{eje}\PY{o}{.}\PY{n}{plot}\PY{p}{(}\PY{n}{t\PYZus{}n}\PY{p}{,}\PY{n}{Delta\PYZus{}t\PYZus{}n}\PY{p}{,} \PY{n}{label}\PY{o}{=}\PY{l+s+s1}{\PYZsq{}}\PY{l+s+s1}{RG}\PY{l+s+s1}{\PYZsq{}}\PY{p}{)}
\PY{n}{eje}\PY{o}{.}\PY{n}{hlines}\PY{p}{(}\PY{l+m+mi}{0}\PY{p}{,}\PY{l+m+mi}{0}\PY{p}{,}\PY{l+m+mi}{40}\PY{p}{,} \PY{n}{color}\PY{o}{=}\PY{l+s+s1}{\PYZsq{}}\PY{l+s+s1}{red}\PY{l+s+s1}{\PYZsq{}}\PY{p}{,}\PY{n}{label}\PY{o}{=}\PY{l+s+s1}{\PYZsq{}}\PY{l+s+s1}{Newtoniano}\PY{l+s+s1}{\PYZsq{}}\PY{p}{)}
\PY{n}{eje}\PY{o}{.}\PY{n}{set\PYZus{}xlabel}\PY{p}{(}\PY{l+s+sa}{u}\PY{l+s+s1}{\PYZsq{}}\PY{l+s+s1}{Tiempo (años)}\PY{l+s+s1}{\PYZsq{}}\PY{p}{)}
\PY{n}{eje}\PY{o}{.}\PY{n}{set\PYZus{}ylabel}\PY{p}{(}\PY{l+s+sa}{r}\PY{l+s+s1}{\PYZsq{}}\PY{l+s+s1}{Retraso acumulado (s)}\PY{l+s+s1}{\PYZsq{}}\PY{p}{)}
\PY{n}{eje}\PY{o}{.}\PY{n}{set\PYZus{}xlim}\PY{p}{(}\PY{l+m+mi}{0}\PY{p}{,}\PY{l+m+mi}{35}\PY{p}{)}
\PY{n}{eje}\PY{o}{.}\PY{n}{set\PYZus{}ylim}\PY{p}{(}\PY{o}{\PYZhy{}}\PY{l+m+mi}{45}\PY{p}{,}\PY{l+m+mi}{1}\PY{p}{)}
\PY{n}{plt}\PY{o}{.}\PY{n}{plot}\PY{p}{(}\PY{n}{t\PYZus{}exp}\PY{p}{,}\PY{n}{Delta\PYZus{}t\PYZus{}exp}\PY{p}{,}\PY{l+s+s1}{\PYZsq{}}\PY{l+s+s1}{o}\PY{l+s+s1}{\PYZsq{}}\PY{p}{,}\PY{n}{label}\PY{o}{=}\PY{l+s+s1}{\PYZsq{}}\PY{l+s+s1}{Datos}\PY{l+s+s1}{\PYZsq{}}\PY{p}{)}
\PY{n}{eje}\PY{o}{.}\PY{n}{legend}\PY{p}{(}\PY{n}{loc}\PY{o}{=}\PY{l+m+mi}{3}\PY{p}{)}
\PY{n}{plt}\PY{o}{.}\PY{n}{grid}\PY{p}{(}\PY{p}{)}
\end{Verbatim}
\end{tcolorbox}

    \begin{center}
    \adjustimage{max size={0.9\linewidth}{0.9\paperheight}}{Sistema_Binario-Evolucion_Temporal_y_observaciones_HyT_files/Sistema_Binario-Evolucion_Temporal_y_observaciones_HyT_59_0.png}
    \end{center}
    { \hspace*{\fill} \\}
    
    not bad :-)

    Now it is better! n.n


    % Add a bibliography block to the postdoc
    
    
    
\end{document}
