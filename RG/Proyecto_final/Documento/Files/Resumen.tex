En el presente trabajo se analizarán las ecuaciones del movimiento rotacional de un sistema binario bajo la perspectiva
de un movimiento kepleriano haciendo uso del tensor momento de inercía del sistema, el cual será reducido al problema de una partícula
con masa reducida $\mu$ bajo un sistema descrito por una coordenada relativa $\vec{r}$ de modo que el sistema se encuentra en el plano xy, 
para así obtener las ecuaciones diferenciaes de la evolución del semieje mayor $a$ y la excentricidad $e$ en el tiempo. Con ayuda del lenguaje
python se obtuvó una solución númerica para realizar la comparación entre la solución analítica, la solución númerica y los datos observacionales reportados por 
J. M. Weisberg \cite{Weisberg2010}. Esto resulto que el uso de la relatividad general compata con los datos observacionales, y con ello, 
una demostración de que la  relatividad general es una teória que representa en una aproximación a la realidad.\\
\textbf{Palabras clave:} Sistema binario, Pulsar PSR B1913+16, excentricidad, semieje mayor, potencia radiada, kepler.