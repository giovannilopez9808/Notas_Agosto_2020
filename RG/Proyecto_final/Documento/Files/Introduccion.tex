% Puntos a tomar en cuenta para la introduccion
% 1) Punto de vista newtoniano
% 2) sistemas de cuerpos
% 2.1) cuerpos binarios
% 3) Origen de la relatividad
% 4) principios de RG
La mecánica newtoniana o mecánica vectorial es un conjunto de formulaciones de la mecánica clásica que estudia el movimiento de partículas
y sólidos en un espacio euclideo tridimensional. Cada cuerpo contiene una velocidad inicial referenciada desde un sistema de referencia inercial
donde las ecuaciones del movimiento se ven reducidas a las leyes de Newton. La mecánica newtoniana es un modelo físico que funciona para 
describir la dinámica de los cuerpos en el espacio por medio de las fuerzas que contiene cada objeto. Historicamente, la mecánica newtoniana
fue el primer modelo físico en poder representar de buena manera la dinámica de los objetos al punto de predecir acciones importantes sobre el movimiento
de los cuerpos, en donde se incluyen las trayectorias de ciertos planetas. La mecánica newtoniana es suficientemente válida para la casos en los cuales
sus aproximaciones compatan con los resultados experimentales, ya sea como el movimiento de cohetes, trayectorias de planetas, moléculas orgánicas, trayectorias de moviles, etc. Sin embargo,
existen problemas en donde el modelo newtoniano se complica matemáticamente en comparación a otras teorías como la mecánica lagrangiana o hamiltoniana, es por ello
que debemos observar con deteminiendo el sistema de estudio para decidir que teoría utilizar y resolver el problema de una manera sencilla.