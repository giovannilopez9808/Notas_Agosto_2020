\subsection{Punto de vista newtoniano}
La mecánica newtoniana o mecánica vectorial es un conjunto de formulaciones de la mecánica clásica que estudía el movimiento de partículas y sólidos en un espacio euclídeo tridimensional. Cada cuerpo contiene una velocidad inicial referenciada desde un sistema inercial donde las ecuaciones del movimiento se ven reducidas a las leyes de Newton. La mecánica newtoniana es un modelo físico que funciona para describir la dinámica de los cuerpos en el espacio por medio de las fuerzas que contiene cada objeto. Históricamente, la mecánica newtoniana fue el primer modelo físico en poder representar de buena manera la dinámica de los objetos al punto de predecir acciones importantes sobre el movimiento de los cuerpos, en donde se incluyen las trayectorias de ciertos planetas. La mecánica newtoniana es suficientemente válida para la casos en los cuales sus aproximaciones compatan con los resultados experimentales, ya sea como el movimiento de cohetes, trayectorias de planetas, moléculas orgánicas, trayectorias de móviles, etc. Sin embargo, existen problemas en donde el modelo newtoniano se complica matemáticamente en comparación a otras teorías como la mecánica lagrangiana o hamiltoniana, es por ello
que debemos observar con detenimiento el sistema de estudio para decidir que teoría utilizar y resolver el problema de una manera sencilla.
\subsection{Sistemas de cuerpos}
En física, la cuestión del problema de los n-cuerpos trata de determinar los movimientos individuales de un grupo de partículas materiales que interactúan mutuamente según las leyes de la gravitación universal de Newton. La resolución de este problema ha sido motivada por el deseo de predecir los movimientos de los cuerpos celestes. En el siglo XX, el entendimiento de la dinámica de los sistemas de cúmulos globulares de estrellas se convirtió en un importante problema de n-cuerpos.\\\\
El problema físico clásico puede plantearse de forma simplificada como:\\\\
\textit{Dadas las propiedades orbitales (masa, posición instantánea y velocidad) de un grupo de cuerpos astronómicos, determinar las fuerzas interactivas actuantes; y consiguientemente, calcular sus movimientos orbitales para cualquier instante futuro.}\\\\
Inicialmente, el problema de los n-cuerpos no fue planteado correctamente porque no se incluía el efecto de las fuerzas interactivas gravitatorias. Newton no lo expresa explícitamente, pero de sus Principia se deduce que el problema de los n-cuerpos es irresoluble debido precisamente a aquellas fuerzas interactivas gravitacionales. En sus Principia, párrafo 21, se afirma que:\\\\
\textit{Y de ahí que la fuerza atractiva se encuentre en ambos cuerpos. El Sol atrae a Júpiter y a los otros planetas, Júpiter atrae a sus satélites y de igual modo los satélites actúan unos sobre otros. Y aunque las acciones de cada par de planetas en el otro se pueden distinguir entre sí y pueden considerarse como dos acciones por las cuales cada uno atrae al otro, sin embargo, en tanto que son los mismos dos cuerpos no son dos sino una simple operación entre dos términos. Dos cuerpos pueden ser atraídos entre sí por la contracción de una cuerda entre ellos. La causa de la acción es doble, nominalmente sobre la disposición de cada uno de los dos cuerpos; la acción es igualmente doble, en la medida en que actúa sobre los dos cuerpos; pero en la medida en que está entre los dos cuerpos, es única y una .}\\\\
Newton concluyó a través de su tercera Ley que \textit{"según esta Ley, todos los cuerpos tienen que atraer cada cual a los otros."} Esta última declaración, que implica la existencia de fuerzas interactivas gravitatorias.
\subsubsection{Sistemas Binarios}
En astronomía, el término sistema binario se utiliza para referirse a dos objetos astronómicos que orbitan alrededor de un centro de masa común debido a la fuerza gravitatoria que hay dada su cercanía. Normalmente se utiliza para referirse a dos estrellas, sin embargo, el término puede aplicarse a un sistema formado por un planeta y un satélite natural, siempre y cuando este último sea excepcionalmente grande en comparación con el planeta. Algunos ejemplos de estos son:
\begin{itemize}
    \item La galaxia de Bode (M81) y la galaxia del Cigarro (M82).
    \item Dentro del sistema solar, el sistema Plutón-Caronte está formado por un planeta y un satélite.
\end{itemize}
Una estrella binaria es un sistema estelar compuesto de dos estrellas que orbitan mutuamente alrededor de un centro de masas común\cite{CanazasGaray2015}. Los sistemas múltiples, que pueden ser ternarios, cuaternarios, o inclusive de cinco o más estrellas interactuando entre sí, suelen recibir también el nombre de estrellas binarias, como es el caso de Alfa Centauri A y B y Próxima Centauri.\\\\
Dentro de los sistemas binarios en los que se da el fenómeno de acreción, existe una clase que brilla fuertemente en rayos X e involucra la presencia de un objeto compacto. A este tipo de sistemas se les llama Binarias de Rayos X, y se dividen en dos grupos principales:
\begin{enumerate}
    \item Binarias de rayos X de alta masa (HMXB por High Mass X-Ray Binary)\cite{10.1111/j.1365-2966.2011.19862.x}, y está formado por una donadora masiva (unas 20 masas solares), en donde la cesión de masa al objeto compacto se da vía un viento estelar.
    \item Binarias de rayos X de baja masa (LMXB por Low Mass X-Ray Binary)\cite{1984ApJS...54..443P}. En este caso la donadora es de baja masa (del orden de 0.5 MS) y la cesión de masa al objeto compacto se da cuando la separación entre las estrellas es lo suficientemente chica (o la donadora es lo suficientemente grande) como para que el objeto compacto arranque el material de su superficie por fuerzas gravitacionales.
 \end{enumerate}
\subsection{Origen de la Relatividad}
La teoría de la relatividad incluye tanto a la teoría de la relatividad especial como la de relatividad general, formuladas principalmente por Albert Einstein a principios del siglo XX, que pretendían resolver la incompatibilidad existente entre la mecánica newtoniana y el electromagnetismo. La teoría de la relatividad especial, publicada en 1905, trata de la física del movimiento de los cuerpos en ausencia de fuerzas gravitatorias, en el que se hacían compatibles las ecuaciones de Maxwell del electromagnetismo con una reformulación de las leyes del movimiento. En la teoría de la relatividad especial, Einstein, Lorentz y Minkowski, entre otros, unificaron los conceptos de espacio y tiempo, en un ramado tetradimensional al que se le denominó espacio-tiempo.\\\\
La teoría de la relatividad general, publicada en 1915, es una teoría de la gravedad que reemplaza a la gravedad newtoniana, aunque coincide numéricamente con ella para campos gravitatorios débiles y "pequeñas" velocidades. La teoría general se reduce a la teoría especial en presencia de campos gravitatorios. La relatividad general estudia la interacción gravitatoria como una deformación en la geometría del espacio-tiempo. En esta teoría se introducen los conceptos de la curvatura del espacio-tiempo como la causa de la interacción gravitatoria, el principio de equivalencia que dice que para todos los observadores locales inerciales las leyes de la relatividad especial son invariantes y la introducción del movimiento de una partícula por líneas geodésicas.
\subsection{Principios de Relatividad General}
Las características esenciales de la teoría de la relatividad general son las siguientes:
\begin{itemize}
    \item El principio general de covariancia: las leyes de la Física deben tomar la misma forma matemática en todos los sistemas de coordenadas.
    \item El principio de equivalencia o de invariancia local de Lorentz: las leyes de la relatividad especial (espacio plano de Minkowski) se aplican localmente para todos los observadores inerciales.
    \item La curvatura del espacio-tiempo es lo que observamos como un campo gravitatorio, en presencia de materia la geometría del espacio-tiempo no es plana sino curva, una partícula en movimiento libre inercial en el seno de un campo gravitatorio sigue una trayectoria geodésica.
\end{itemize}