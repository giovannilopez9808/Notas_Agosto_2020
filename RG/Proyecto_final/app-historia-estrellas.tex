\chapter{Enanas Blancas y Estrellas de Neutrones: Rese\~na Hist'orica}

\begin{itemize}
 \item \emph{1844, Friedrich Wilhem Bessel}: Us'o la t'ecnica de paralaje estelar, dise\~nada por 'el, para encontrar la distancia a Sirius. Pero tambi'en descubri'o perturbaciones en su 'orbita, por lo que dedujo que Sirius es un sistema binario, con periodo 50 a\~nos.
\item \emph{1862, Alvan Graham Clark}: Con el telescopio m'as potente de su 'epoca, logr'o observar la estrella compa\~nera de Sirius A, ahora llamada Sirius B (se encontraba en su apoastro). Determin'o sus par'ametros: $L_A=23.5L_{\odot}$ y  $L_B=0.03L_{\odot}$. $M_A=2.3M_{\odot}$ y $M_B =1.0M_{\odot}$.
\item \emph{1915, Walter Adams}: Mediante t'ecnicas espectrosc'opicas, descubri'o que Sirius B es una estrella blanca-azulada muy caliente, $T_B=27000\,K$, mientras que $T_A=9910\,K$. Usando la ley de Stefan-Boltzmann, $R_B=5.5\cdot10^{6}\approx0.008R_{\odot}$. !`Sirius B es una estrella con la masa del Sol confinada a un volumen m'as peque\~no que la Tierra!.$\;\;\Rightarrow\;\;\rho_B=3.0\cdot10^{9}\;kg\cdot m^{-3}$! Los dem'as astr'onomos los consideran resultados ``absurdos''.
\item \emph{1926, W.S. Adams}: Midi'o los redshifts gravitacionales de las l'ineas espectrales emitidas por Sirius B, determinando en forma independiente el radio de la estrella, sirviendo de comprobaci'on de la naturaleza compacta de este objeto y tambi'en como un test para la teor'ia de relatividad general.
\item \emph{Agosto 1926, Dirac} formula la estad'istica de Fermi-Dirac.
\item \emph{Diciembre 1926, R.H. Fowler}: Us'o la estad'istica de Fermi-Dirac para explicar las enanas blancas: la presi'on de degeneraci'on de los electrones mantiene el equilibrio en contra de la gravedad.
\item \emph{1930, S. Chandrasekhar}: Calcul'o modelos de enanas blancas tomando en cuenta efectos relativistas en la ecuaci'on de estado degenerada de los electrones, descubriendo que ninguna enana blanca puede ser m'as masiva que $\sim1.2 M_{\odot}$ (conocida como Masa de Chandrasekhar)
\item \emph{1932, L.D. Landau} Di'o una explicaci'on elemental del l'imite de Chandrasekhar.
\item \emph{1932, James Chadwick}: Descubre el neutr'on.
\item \emph{1934, Waalter Baade y Fritz Zwicky}: Proponen la existencia de estrellas de neutrones. Identifican una nueva clase de objetos astron'omicos denominados supernovas. Sugieren que una supernova puede ser creada por el colapso de una estrella normal a una estrella de neutrones.
\item \emph{1939, Oppenheimer y Volkoff}: Realizan los primeros c'alculos detallados de la estructura de estrellas de neutrones usando la teor'ia general de la relatividad. Proponen la idea que los n'ucleos de neutrones en estrellas normales pueden ser fuente de energ'ia estelar (la fusi'on termonuclear a'un no era entendida).
\item \emph{1939, Oppenheimer y Snyder}: Calculan el colapso de una esfera homog'enea de un fluido sin presi'on usando relatividad general, y descubren c'omo la esfera pierde comunicaci'on con el resto del universo. Es el primer c'alculo de c'omo un agujero negro se puede formar.
\item \emph{1942, Duyvendak, Mayall, Oort, Baade y Minkowskii}: Deducen que la nebulosa del Cangrejo es el remanente de la supernova observada por astr'onomos chinos en 1054. Observan en el centro de ella una estrella que identifican como el remanente de la estrella que explot'o en dicho a\~no.
\item \emph{1949, Kaplan}: Deriva los efectos de la teor'ia general de la relatividad sobre las curvas masa-radio para enanas blancas masivas. Deduce que la teor'ia de relatividad general induce una inestabilidad para un radio menor de $\sim1.1\cdot10^{3}\;km$.
\item \emph{1958, Schatzman, Wakano, Harrison y Wheeler}: Incorporan el decaimiento beta inverso para la ecuaci'on de estado para la materia de enanas blancas. Muestran que este efecto induce una inestabilidad din'amica para enanas blancas con masa de $\sim M_{\odot}$ y radio menor a $\sim4\cdot10^{3}\;km$.
\item \emph{1963, Hoyle and Fowler}: Proponen la idea de estrellas supermasivas, calcularon sus propiedades, y sugirieron que podr'ian estar asociadas con n'ucleos gal'acticos activos y qu'asars.
\item \emph{1963-1964, Chandrasekhar y Feynman}: Desarrollan la teor'ia general relativista de las pulsaciones estelares, y Feynman la us'o para mostrar que estrellas supermasivas, aunque newtonianas en su estructura, est'an sujetas a una inestabilidad debida a Relatividad General.
\item \emph{1967, Hewish et al.}: Descubrimientos de los pulsares.
\item \emph{1968, Gold}: Propone que los p'ulsares son estrellas de neutrones rotantes.
\item \emph{1968-1969 varios}: Descubrimientos casi simult'aneos de los pulsares del Cangrejo (la misma estrella identificada en 1942) y Vela en remanentes conocidos de supernovas. Proveen evidencia para la formaci'on de estrellas de neutrones y pulsares en explosiones de supernovas.
\item \emph{1975, Hulse y Taylor}: Descubrimiento del primer pulsar binario. Miden la masa de una estrella de neutrones y, a la vez, proveen de un test para la teor'ia general de la relatividad.
\end{itemize}