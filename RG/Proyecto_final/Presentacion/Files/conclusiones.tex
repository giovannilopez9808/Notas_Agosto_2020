\begin{frame}{Conclusiones}
    Con estos resultados, podemos llegar a las siguientes conclusiones:
\begin{itemize}
    \item A partir de las figuras \ref{fig:periodo} y \ref{fig:semieje,excentricidad} se aprecia que la tasa de cambio 
    con respecto al tiempo de la excentricidad $(e)$, periodo $(T)$ y el semieje mayor $(a)$ es constante una gran parte del 
    tiempo en el cual el sistema binario se encuentra orbitando y es por ello que podemos calcular un tiempo de retardo durante ese periodo de años.
    \item La solución númerica de las ecuaciones de evolucion del sistema binario compata con la solución analítica, esto puede verse representado
    en la figura \ref{fig:ananum}.
    \item Con la figura \ref{fig:exp} se comprueba que la teoría de la relatividad general modela con cierta exactitud a la realidad en casos donde la teória newtoniana llega a ser inválida
    , ya que, con los datos observacionales reportados por \cite{Weisberg2010} del pulsar PSR B1913+16, las ecuaciones resultantes de principios de la teoría de la relatividad general compatan con estos datos.
\end{itemize}
\end{frame}
\begin{frame}{Código}
    \begin{enumerate}
        \item \href{https://github.com/giovannilopez9808/Notas_Agosto_2020/blob/master/RG/Proyecto_final/Documento/Scripts/Pot_efec_Graphics.py}{Pot\_efec\_Graphics.py\label{cod:potefe}}\\
        Este código genera la figura \ref{fig:pot} a partir de los parámetros $G,M,\mu,L$.
        \item \href{https://github.com/giovannilopez9808/Notas_Agosto_2020/blob/master/RG/Proyecto_final/Documento/Scripts/data-HW.csv}{data-HW.csv\label{data:H}}\\
        Datos observacionales reportados por \cite{Weisberg2010}.
        \item \href{https://github.com/giovannilopez9808/Notas_Agosto_2020/blob/master/RG/Proyecto_final/Documento/Scripts/Sistema_Binario-Evolucion_Temporal_y_observaciones_HyT.py}{Sistema\_Binario-Evolucion\_Temporal\_y\_observaciones\_HyT.py\label{cod:evo}}\\
        Este código es el encargado de realizar la simulación del pulsar binario calculando la evolucion del semieje mayor $a$ y excentricidad $a$ generando las figuras \ref{fig:semieje,excentricidad}, \ref{fig:periodo}, \ref{fig:gvse}, \ref{fig:ananum} y comparando los resultados numéricos, análiticos a los datos observaciones 
        \ref{data:H} en la figura \ref{fig:exp}.
    \end{enumerate}
\end{frame}