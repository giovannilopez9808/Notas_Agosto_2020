\documentclass[12pt,letterpaper]{report}
\usepackage{graphicx}
\usepackage{array}
\usepackage{scrextend}
\usepackage{vmargin}
\usepackage{graphicx}
\usepackage{multirow}
\usepackage[utf8]{inputenc}
\usepackage[spanish]{babel}
\usepackage{multicol}
\usepackage{enumerate}
\usepackage{float}
\usepackage{hyperref}
\usepackage{amsmath, amsthm, amssymb, amsfonts}
\usepackage[usenames]{color}
\definecolor{citecolor}{rgb}{.12,.54,.11}
\definecolor{urlcolor}{RGB}{64, 145, 108}
\definecolor{1ee592}{RGB}{30,229,146}
\definecolor{1bac70}{RGB}{27,172,112}
\definecolor{f38638}{RGB}{243,134,56}
\definecolor{black}{RGB}{0,0,0}
\definecolor{gray}{RGB}{156,156,156}
\hypersetup{
    colorlinks=true,
    linkcolor=blue,
    filecolor=magenta,      
    urlcolor=urlcolor,
    citecolor=citecolor,
}
\parindent=0mm
\spanishdecimal{.}
\pagestyle{empty}
\definecolor{miorange}{rgb}{0.91, 0.43, 0.0}
\begin{document}
\setmargins{2.5cm}      
{1.5cm}                     
{2cm}  
{24cm}                    
{10pt}                          
{1cm}                          
{0pt}                             
{2cm}
\begin{titlepage}
\begin{center}
\includegraphics[scale=0.40]{../../Logos/uanl.png} 
\hspace{2.5cm}
\includegraphics[scale=0.40]{../../Logos/fcfm.png}
\end{center}
\vspace{2cm}
\begin{center}
\textbf{
UNIVERSIDAD AUTÓNOMA DE NUEVO LEÓN\\
FACULTAD DE CIENCIAS
    FÍSICO MATEMÁTICAS}\\
\vspace*{2cm}
\begin{large}
\vspace{1cm}
\large{\textbf{Relatividad General}}\\
\textbf{Preguntas del proyecto final}\\
Carlos Luna Criado\\
\end{large}
\vspace{3.5cm}
\begin{minipage}{0.6\linewidth}
\vspace{0.5cm}
\changefontsizes{14pt}
Nombre:\\
Giovanni Gamaliel López Padilla\\
\end{minipage}
\begin{minipage}{0.2\linewidth}
\changefontsizes{14pt}
Matricula:\\
1837522
\end{minipage}
\end{center}
\vspace{4cm}
\begin{flushright}
\today
\end{flushright}
\end{titlepage}
\begin{enumerate}
    \item Enuncie la diferencia entre analizar un sistema binario desde la perspectiva de la mecánica newtoniana y la relatividad general.\\
    En la mecánica newtoniana se obtiene que el sistema binario tiene un periodo constante en el tiempo, en cambio en la relatividad general este depende del tiempo dando como resultado el colapso de este en un tiempo determinado.
    \item ¿Qué son las ondas gravitacionales?\\
    Las ondas gravitacionales son perturbaciones en el campo del espacio-tiempo las cuales pueden haber sido creadas por cuerpos masivos acelerados.
    \item Mencione y explique dos líneas de investigación abiertas actualmente en la cosmología.
    \begin{enumerate}
        \item Formación de galaxias a alto redshift.\\
        En esta línea de investigación se propone el uso de simulaciones de formación de galaxias y observaciones de galaxy redshift para resolver este problema. Actualmente, se esta ejecutando un conjunto de simulaciones hidrodinámicas del proceso de formación de galaxias.
        \item Campo de densidad de masa cósmica.\\
        Esta investigación trata de describir una distribución de la masa en el universo a un instante dado.\\
        \begin{center}
        \href{http://www.udea.edu.co/wps/portal/udea/web/inicio/investigacion/grupos-investigacion/ciencias-naturales-exactas/facom/lineas-investigacion}{udea.edu.co}
        \end{center}
    \end{enumerate}
    \item ¿Qué significa que la solución de las ecuaciones de Einstein sea estática y asintóticamente plana?\\
    Que esta no puede llegar a reproducir un espacio plano y que una vez se haya definido la solución esta no cambiara en el tiempo.
    \item Escriba las componentes de la métrica de Reissner-Nordstr\"om\\
    \begin{equation*}
        ds^2 = - \left(1-\frac{2m}{r}+\frac{q^2}{r^2}\right)dt^2 + \frac{dr^2}{\left(1-\frac{2m}{r}+\frac{q^2}{r^2}\right)}
    \end{equation*}
    \item ¿Por qué no es posible formar curvas cerradas de tipo tiempo en el espacio de Minkowski?\\
    Por definición es un espacio hiperbolico, lo cual no permite le existencia de curvas cerradas.
\end{enumerate}
\end{document}