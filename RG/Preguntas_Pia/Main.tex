\documentclass[12pt,letterpaper]{report}
\usepackage{graphicx}
\usepackage{array}
\usepackage{scrextend}
\usepackage{vmargin}
\usepackage{graphicx}
\usepackage{multirow}
\usepackage[utf8]{inputenc}
\usepackage[spanish]{babel}
\usepackage{multicol}
\usepackage{enumerate}
\usepackage{float}
\usepackage{amsmath, amsthm, amssymb, amsfonts}
\usepackage[usenames]{color}
\parindent=0mm
\spanishdecimal{.}
\pagestyle{empty}
\definecolor{miorange}{rgb}{0.91, 0.43, 0.0}
\begin{document}
\setmargins{2.5cm}      
{1.5cm}                     
{2cm}  
{24cm}                    
{10pt}                          
{1cm}                          
{0pt}                             
{2cm}
\begin{titlepage}
\begin{center}
\includegraphics[scale=0.40]{../../Logos/uanl.png} 
\hspace{2.5cm}
\includegraphics[scale=0.40]{../../Logos/fcfm.png}
\end{center}
\vspace{2cm}
\begin{center}
\textbf{
UNIVERSIDAD AUTÓNOMA DE NUEVO LEÓN\\
FACULTAD DE CIENCIAS
    FÍSICO MATEMÁTICAS}\\
\vspace*{2cm}
\begin{large}
\vspace{1cm}
\large{\textbf{Relatividad General}}\\
\textbf{Preguntas del proyecto final}\\
Carlos Luna Criado\\
\end{large}
\vspace{3.5cm}
\begin{minipage}{0.6\linewidth}
\vspace{0.5cm}
\changefontsizes{14pt}
Nombre:\\
Giovanni Gamaliel López Padilla\\
\end{minipage}
\begin{minipage}{0.2\linewidth}
\changefontsizes{14pt}
Matricula:\\
1837522
\end{minipage}
\end{center}
\vspace{4cm}
\begin{flushright}
\today
\end{flushright}
\end{titlepage}
\begin{enumerate}
    \item Enuncie la diferencia entre analizar un sistema binario desde la perspectiva de la mecánica newtoniana y la relatividad general.
    \item ¿Qué son las ondas gravitacionales?
    \item Mencione y explique dos liíneas de investigacioón abiertas actualmente en la cosmología.
\end{enumerate}
    \begin{table}[H]
        \centering
        \begin{tabular}{lcc} \hline
        \textbf{Conditions} & \textbf{UV Index at Noon 21 June} & \textbf{My Notes} \\ \hline
        Clean PBL & 15.9 & \\ \hline
        Year 2000 - Aerosols AOD:&\multirow{3}{*}{9.7}& \multirow{3}{5cm}{-6.2=39\% reduction from clean} \\ 0.23 in FT, 0.7 in PBL;\\ O\textsubscript{3}=70, NO\textsubscript{2}=40, SO\textsubscript{2}=10 ppb\\ \hline
        Excluding FT aerosols & 10.8 & +1.1 \\\hline
        Excluding PBL aerosols & 12.7 & +3.0 \\\hline
        Excluding PBL O\textsubscript{3} & 10.4 & +0.7 \\\hline
        Excluding PBL NO\textsubscript{2} & 10.1 & +0.4 \\\hline
        Excluding PBL SO\textsubscript{2} & 9.9 & +0.2\\\hline
        Year 2019 - Aerosol AOD: & \multirow{3}{*}{11.1} & \multirow{3}{5cm}{+1.4 rel 2000 14\% increase; +4.8 rel clean still 30\% reduction from clean}\\
        0.23 in FT, 0.5 in PBL;\\ O\textsubscript{3}=50, NO\textsubscript{2}=20, SO\textsubscript{2}=1 ppb\\ \hline
        \end{tabular}
        \caption{TUV Model Calculation Illustrating the Effect of Pollutants on the UV Index in Mexico City.}
        \label{table:TUVmodel}
        \end{table}
\end{document}