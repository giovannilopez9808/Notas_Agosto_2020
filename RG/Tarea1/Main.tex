\documentclass[12pt,letterpaper]{report}
\usepackage{graphicx}
\usepackage{scrextend}
\usepackage{vmargin}
\usepackage{graphicx}
\usepackage{multirow}
\usepackage[utf8]{inputenc}
\usepackage[spanish]{babel}
\usepackage{multicol}
\usepackage{enumerate}
\usepackage{float}
\usepackage{amsmath, amsthm, amssymb, amsfonts}
\usepackage[usenames]{color}
\parindent=0mm
\pagestyle{empty}
\definecolor{miorange}{rgb}{0.91, 0.43, 0.0}
\begin{document}
\setmargins{2.5cm}      
{1.5cm}                     
{2cm}  
{24cm}                    
{10pt}                          
{1cm}                          
{0pt}                             
{2cm}
\begin{titlepage}
\begin{center}
\includegraphics[scale=0.40]{../../Logos/uanl.png} 
\hspace{2.5cm}
\includegraphics[scale=0.40]{../../Logos/fcfm.png}
\end{center}
\vspace{2cm}
\begin{center}
\textbf{
UNIVERSIDAD AUTÓNOMA DE NUEVO LEÓN\\
FACULTAD DE CIENCIAS
    FÍSICO MATEMÁTICAS}\\
\vspace*{2cm}
\begin{large}
\vspace{1cm}
\large{\textbf{Relatividad General}}\\
\textbf{Tranformación de Lorentz}\\
Carlos Luna Criado\\
\end{large}
\vspace{3.5cm}
\begin{minipage}{0.6\linewidth}
\vspace{0.5cm}
\changefontsizes{14pt}
Nombre:\\
Giovanni Gamaliel López Padilla\\
\end{minipage}
\begin{minipage}{0.2\linewidth}
\changefontsizes{14pt}
Matricula:\\
1837522
\end{minipage}
\end{center}
\vspace{4cm}
\begin{flushright}
\today
\end{flushright}
\end{titlepage}
\section*{Transformación por el eje x}
De esta manera podemos definir una transformación de la siguiente manera:
\begin{equation}
    \begin{bmatrix}
        {t}'\\
        x' 
    \end{bmatrix}= 
    \begin{bmatrix}
        A & B \\
        C & D\\
    \end{bmatrix} \begin{bmatrix}
        t \\
        x
    \end{bmatrix}
    \label{ctx}
\end{equation}
al realizar la multiplicación de \ref{ctx} se obtiene que:
\begin{equation}
    \begin{bmatrix}
        {t}' \\
        {x}'
    \end{bmatrix} = 
    \begin{bmatrix}
        At+Bx \\
        Ct+Dx 
    \end{bmatrix}
    \label{ctx2}
\end{equation}
cuando $x'=0$ se cumple que:
\begin{equation}
    D x + C t =0 \Rightarrow x= -\frac{C}{D} t =vt \Rightarrow -v = \frac{C}{D}
    \label{c}
\end{equation}
por lo tanto, la componente $x'$ de \ref{ctx2} es 
\begin{equation}
    x'= D(x-vt)
\end{equation}
introduciendo esto en el invariante de Lorentz, encontramos que:
\begin{align*}
    D^2(x-vt)^2 + {y^2}' + {z^2}' - c^2(At+Bx)^2 &= x^2 + y^2 +z^2 -c^2 t^2 \\
\end{align*}
de la cual separando los coeficientes de los terminos $x^2 , xt , t^2$ obtenemos el siguiente sistema de ecuaciones
\begin{align}
    \label{d2}
    D^2-c^2B^2 &= 1\\
    \label{vd^2}
    v D^2 + c^2 AB &= 0\\
    \label{v2d2}
    -v^2 D^2 +c^2 A^2&= c^2
\end{align}
despejando de \ref{d2} y \ref{v2d2} $D^2$ y $A^2$ y multiplicandolas
\begin{align*}
    A^2 B^2 c^4 &= (D^2-1)(v^2 D^2 + c^2 )\\
                &=v^2D^4+D^2 c^2 -v^2 D^2 -c^2
\end{align*}
por la ecuación \ref{vd^2} al despejar $c^2AB$ y elevar al cuadrado, la operación anterior es igual a :
\begin{align*}
    v^2D^4+D^2 c^2 -v^2 D^2 -c^2&= D^4 v^2 \\
    D^2(c^2-v^2)&=c^2\\
\end{align*}
\begin{equation}
    D = \gamma  \label{D}
\end{equation}
por lo tanto, de la ecuación \ref{v2d2},obtenemos que:
\begin{equation}
    A= \sqrt{\frac{c^2-v^2 \gamma^2}{c^2}} = \gamma
    \label{A}
\end{equation}
y de la ecuación \ref{d2}, obtenemos lo siguiente:
\begin{equation}
    B= -\sqrt{\frac{\gamma^2-1}{c^2}} = -\frac{v}{c^2} \gamma
    \label{bvalue}
\end{equation}
y de la relación encontrada en \ref{c}, encontramos que:
\begin{equation}
    C=  -v \gamma
    \label{cvalue}
\end{equation}
ya habiendo obtenido los resultados de las ecuaciones \ref{cvalue}, \ref{bvalue}, \ref{A}, \ref{D} e introduciendolas en la matriz \ref{ctx}, obetnemos que:
\begin{equation}
    \begin{bmatrix}
        {t}'\\
        x' 
    \end{bmatrix}= 
    \begin{bmatrix}
        \gamma & -\frac{v}{c^2}\gamma \\
        -v\gamma & \gamma\\
    \end{bmatrix} \begin{bmatrix}
        t \\
        x
    \end{bmatrix}
    \label{ctx3}
\end{equation}
expandiendo esta matriz de transformación a 4 dimensiones tendriamos la siguiente:
\begin{equation}
    \begin{bmatrix}
        {t}'\\
        x' \\
        y'\\
        z'
    \end{bmatrix}= 
    \begin{bmatrix}
        \gamma & -\frac{v}{c^2}\gamma & 0 & 0 \\
        -v\gamma & \gamma & 0 & 0\\
        0 & 0 & 1 & 0 \\
        0 &  0 & 0 &1
    \end{bmatrix} \begin{bmatrix}
        t \\
        x\\
        y\\
        z
    \end{bmatrix}
    \label{ctx3}
\end{equation}
\section*{Forma del Boost generalizado}
Un vector lo podemos escribir de la siguiente manera:
\begin{equation}
    \vec{r}=\vec{r}_{\perp}+\vec{r}_{\parallel}
    \label{r=rperp+rpal}
\end{equation}
realizando la comparación con un segundo vector $\vec{r}'$, se tiene lo siguiente:
\begin{equation}
    \vec{r}_{\perp}' = \vec{r}_{\perp}
    \label{rpep}
\end{equation}
\begin{equation}
    \vec{r}_{\parallel}'=\gamma (\vec{r}_{\parallel}'-\vec{v}t)
    \label{rpal}
\end{equation}
usando \ref{rpep} y \ref{rpal} en \ref{r=rperp+rpal} se obtiene que
\begin{equation}
    \vec{r}'=\vec{r}_\perp'+\gamma (\vec{r}_\parallel' - \vec{v}t)
    \label{r'conrpep}
\end{equation}
donde
\begin{equation}
    \gamma= \frac{1}{\sqrt{1-\frac{\vec{v}\cdot\vec{v}}{c^2}}} = \frac{1}{\sqrt{1-\frac{|v^2|}{c^2}}}
    \label{gamma}
\end{equation}
calculando el producto escalar entre $\vec{r}$ y $\vec{v}$ en donde $\hat{r}=\hat{v}$
\begin{align*}
    \vec{r} \cdot \vec{v}   &= \vec{r}_\parallel \cdot \vec{v} + \vec{r}_perp \cdot \vec{v} \\
                            &=\vec{r}_\parallel \cdot \vec{v} \\
                            &=|\vec{r}_\parallel| |\vec{v}|
\end{align*}
por lo tanto
\begin{equation}
    \vec{r} \cdot \vec{v} =|\vec{r}_\parallel| |\vec{v}|
    \label{r.v}
\end{equation}
en la configuración estandar se tiene que:
\begin{equation}
    t'= \gamma\left( t-\frac{\vec{r}\cdot \vec{v}}{c^2} \right)
    \label{t'}
\end{equation}
expandiento los terminos de \ref{t'}
\begin{align*}
    t'  &= \gamma t - \gamma \frac{xv_x}{c^2}- \gamma \frac{yv_y}{c^2}- \gamma \frac{zv_z}{c^2}\\
        &= \gamma ct - \gamma \frac{xv_x}{c}- \gamma \frac{yv_y}{c}- \gamma \frac{zv_z}{c}\\
        &=  \gamma ct - \gamma x \beta_x - \gamma y \beta_y - \gamma z \beta_z
\end{align*}
por lo tanto:
\begin{equation}
    t' = \gamma ct - \gamma x \beta_x - \gamma y \beta_y - \gamma z \beta_z
    \label{t'beta}
\end{equation}
usando \ref{r'conrpep} tomando en cuenta que $\vec{r_\perp}= \vec{r}-\vec{r}_\parallel$, entonces:
\begin{equation}
    \vec{r}'=\vec{r}-\vec{r}_\parallel+\gamma (\vec{r}_\parallel' - \vec{v}t)
    \label{newr}
\end{equation}
de la relacion de la ecuacion \ref{r.v}, tenemos que:
\begin{equation}
    |\vec{r}_\parallel| = \frac{\vec{r}_\parallel \cdot \vec{v}}{|\vec{v}|} =\frac{\vec{r}\cdot \vec{v}}{|\vec{v}|} 
    \label{rparalel}
\end{equation}
entonces:
\begin{align*}
\vec{r}_\parallel   &= |\vec{r}_\parallel| \frac{\vec{v}}{|\vec{v}|}\\
                    &=\left(\frac{\vec{r}\cdot \vec{v}}{|\vec{v}|} \right) \frac{\vec{v}}{|\vec{v}|}
\end{align*}
por lo tanto la ecuacion \ref{newr} se puede reescribir como lo siguiente:
\begin{align*}
    \vec{r}'&=\vec{r}-\vec{r}_\parallel+\gamma (\vec{r}_\parallel' - \vec{v}t)\\
            &=\vec{r}+(\gamma-1) \vec{r}_\parallel - \gamma\vec{v} t\\
            &=\vec{r}+(\gamma-1) \left(\frac{\vec{r}\cdot \vec{v}}{|\vec{v}|} \right)\frac{\vec{v}}{|\vec{v}|} - \gamma \vec{v} t\\
            &=-\gamma \vec{v}t + \vec{r} + (\gamma-1) \left(\frac{\vec{r}\cdot \vec{v}}{|\vec{v}|} \right)\frac{\vec{v}}{|\vec{v}|}\\
            &= - \gamma \vec{v} t + \vec{r} + (\gamma-1)(\vec{r} \cdot \vec{v}) \frac{\vec{v}}{|\vec{v}|^2}
\end{align*}
porlo tanto:
\begin{equation}
\vec{r}' = - \gamma \vec{v} t + \vec{r} + (\gamma-1)(\vec{r} \cdot \vec{v}) \frac{\vec{v}}{|\vec{v}|^2}
\end{equation}
Calculando para cada componente:
\begin{align*}
    x'&=-\gamma v_x t + x +(\gamma -1) (xv_x+yv_Y+zv_z) \frac{v_x}{|v_x|^2}\\
    y'&=-\gamma v_y t + y +(\gamma -1) (xv_x+yv_Y+zv_z) \frac{v_y}{|v_y|^2}\\
    z'&=-\gamma v_z t + z+(\gamma -1) (xv_x+yv_Y+zv_z) \frac{v_z}{|v_z|^2}
\end{align*}
utilizando la relación $\beta_i = v_i/c$ en cada una de las componentes se tiene que:
\begin{align}
    \label{xfin}
    x'&=-\gamma \beta_x ct + x +(\gamma -1) (x\beta_x+y\beta_y+z\beta_z) \frac{\beta_x}{|\beta_x|^2}\\
    \label{zfin}
    y'&=-\gamma \beta_y ct + y +(\gamma -1) (x\beta_x+y\beta_y+z\beta_z) \frac{\beta_y}{|\beta_y|^2}\\
    \label{yfin}
    z'&=-\gamma \beta_z ct + z +(\gamma -1) (x\beta_x+y\beta_y+z\beta_z) \frac{\beta_z}{|\beta_z|^2}
\end{align}
reescribiendo las ecuaciones \ref{xfin}, \ref{yfin}, \ref{zfin} y \ref{t'beta} como producto puntos de vectores se obtiene lo siguiente:
\begin{align*}
    x^0&= \Lambda^0_\nu \cdot x^\nu \\
    x^1&= \Lambda^1_\nu \cdot y^\nu \\
    x^2&= \Lambda^2_\nu \cdot y^\nu \\
    x^3&= \Lambda^3_\nu \cdot y^\nu \\
\end{align*}
de modo que la matriz $\Lambda$ que la relación entre $\vec{r}$ y $\vec{r}'$:
\begin{equation}
    \begin{bmatrix}
        ct'\\ 
        x'\\ 
        y'\\ 
        z'
        \end{bmatrix}=
        \begin{bmatrix}
        \gamma &-\gamma \beta_x & -\gamma \beta_y & -\gamma \beta_z \\ 
        -\gamma \beta_x &1+(\gamma-1)\frac{\beta_x^2}{|\beta|^2}& (\gamma-1)\frac{\beta_x \beta_y}{|\beta|^2} &(\gamma-1)\frac{\beta_x \beta_z}{|\beta|^2}\\ 
        -\gamma \beta_y &(\gamma-1)\frac{\beta_x \beta_y}{|\beta|^2}& 1+(\gamma-1)\frac{\beta_y^2}{|\beta|^2} &(\gamma-1)\frac{\beta_y \beta_z}{|\beta|^2}\\ 
        -\gamma \beta_x &(\gamma-1)\frac{\beta_x \beta_z}{|\beta|^2}& (\gamma-1)\frac{\beta_z \beta_y}{|\beta|^2} &1+(\gamma-1)\frac{\beta_z^2}{|\beta|^2}\\ 
        \end{bmatrix} \begin{bmatrix}
        ct\\ 
        x\\ 
        y\\ 
        z
        \end{bmatrix}
        \label{trans}
\end{equation}
de modo que la relación \ref{trans} se pueda escribir de manera tensorial como
\begin{equation}
    {r^\mu}' = {\Lambda^\mu}'_\nu r^\nu
\end{equation}

\end{document}