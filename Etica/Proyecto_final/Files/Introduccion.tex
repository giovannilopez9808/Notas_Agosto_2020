La investigación científica es una actividad que contribuye conocimiento para el avance en la sociedad humana. Sus aportaciones
son múltiples, esto es por que la investigación científica es la base para la generación de conociemiento sobre la naturaleza de los objetos en nuestro
alrededor y la sociedad, estas aportaciones hacen posible el desarrollo de tecnologías para ser aplicadas día a día y, mediante la divulgación
de la actividad científica fomentan sociedades más sanas, compuestas por cuidadanos responsables.\\
La integridad de una persona que desarrolla una investigación científica es un aspecto indispensable para que los frutos de 
la investigación que buscan sigan gozando del reconocimiento y la admiración de los cuidadanos. La profesionalidad y el buen hacer de los
investigadores es una condición necesaria, sin estas bases cualquier actividad pierde su valor y objetivo principal. La integridad científica
es una responsabilidad  de la que todos deben rendir cuentas, la cual es colectiva e institucional.\\\\
El aspecto ético de la investigación ha sido regido por una combinación de factores que se originan desde los códigos de conducta específicos 
que intentan regir las prácticas que preservan la situación de prestigio profesional de los investigadores científicos de las diversas ramas existentes.
El sistema científico cotemporáneo exige competir constantemente para la obtención de reconocimiento y crédito, lo cual llega a producir una lucha
continua de apoyos económicos, propiciando en algunos casos el origen de malas conductas éticas y fraudes revelados regularmente en ciertas personalidades
narcisistas que motivan la predisposición a malas prácticas. \cite{Cami2008}