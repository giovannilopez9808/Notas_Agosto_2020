\subsection{Ejemplos de malas conductas en la investigación}
En seguida se expondran algunos casos de la edad contemporanea en donde se reportaron conductas éticamente inaceptables en la investigación
, estas son definidas como \textit{violación de códigos, normas o contratos existentes} y las conductas éticamente
cuestionables como \textit{las que están fuera del marco de principios éticos esperados más no formalmente
establecidos en códigos o normas} \cite{M2013}. Es evidente que existen conductas no éticas intencionales y no intencionales
, esto es relevante ya que las consecuencias que con llevan cada acción tiene diferente nivel de gravedad en la 
investigación científica \cite{Shamoo2009}.\\\\

\begin{itemize}
    \item Caso reportado en \cite{comstock_2013}\\
    En este documento se indicaron que la conducta no ética más frecuente es la falsificación de datos de la investigación.
    \textit{Los coordinadores de los posgrados de Astrofísica y de Ciencias Matemáticas expresaron
    que hay investigadores que anuncian sus resultados con base en hechos falsos y que una de las
    causas es la enorme presión por publicar.}
    \item Caso reportado en \cite{Aluja2004}\\
    Presentan un \textit{abuso de los apoyos financieros del
    erario público por parte del estudiante durante sus estudios de posgrado}, indicando que \textit{
        cuando un estudiante acepta una beca lo hace con el compromiso de dedicar tiempo completo a sus
estudios, y se estaría incurriendo en una conducta éticamente cuestionable si esta premisa no
se cumple}
\item Casos reportados de entrevistas en la Universidad de Valencia \cite{HirschAdler2016}:
\begin{itemize}
    \item Faltas de respeto\\
    El coordinador de Letras relató que es lamentable que haya un incremento en las conductas
    negativas de algunos alumnos y que cuando una persona en una posición de autoridad los confronta, ellos se muestran sorprendidos o responden con agresión; por ejemplo con denun-
    cias en la Defensoría de los Derechos Universitarios, la Comisión de Derechos Humanos e
    incluso en el Instituto Federal de Acceso a la Información.
    \item ndividualismo y competitividad entre los estudiantes\\
    En Valencia se manifestó que las rela-
ciones entre los alumnos muchas veces no son éticas, pues no se apoyan entre sí, principalmente
por la competencia que se establece entre ellos
    \item Interesa más el título que el aprendizaje\\
    uando se les pregunta a los estudiantes las razones
por las que estudian un posgrado sobre docencia, contestan que es porque no tienen trabajo
y que no les gusta la enseñanza
\end{itemize}
\end{itemize}