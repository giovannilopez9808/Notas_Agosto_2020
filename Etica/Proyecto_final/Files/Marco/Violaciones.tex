\subsection{Violaciones de la integridad de la investigación}
La mala conducta en investigación científica consiste en el incumplimiento de las buenas prácticas
científicas por parte de los investigadores, ya sea de manera intencional o por negligencia causando 
una lesión al proceso de la investigación, degradando las relaciones entre los investigadores, sobre explotando de la confianza
y credibilidad de la investigación exponiendo así a las personas participantes en la investigación, a la sociedad en su conjunto y al medio 
ambiente en daños innecesarios.
\subsubsection{Mala conducta y otras prácticas inaceptables en investigación}
A continuación, se describen las tres formas más graves de violación de la integridad en la
investigación, ya que distorsionan por completo la naturaleza y los fines de la investigación:
\begin{itemize}
    \item La fabricación, que consiste en la invención de los resultados y en su registro como si fueran
    reales.
    \item La falsificación, que consiste en la manipulación de los materiales, del equipamiento o del proceso
    de investigación o la modificación, omisión o supresión de datos o resultados sin justificación.
    \item El plagio, que consiste en el uso de las ideas o el trabajo de otras personas sin otorgar el crédito
    suficiente a las fuentes originales, con la consiguiente violación de los derechos de los autores
    originales a su producto intelectual.
\end{itemize}
Otros ejemplos de prácticas inaceptables incluyen:
\begin{itemize}
    \item Adulterar la autoría de un trabajo y minusvalorar o no reconocer el papel de otros investigadores
    en las publicaciones.
    \item Incurrir en el autoplagio, incluyendo las traducciones a otros idiomas, sin el reconocimiento
    debido o sin citar apropiadamente los trabajos originales.
    \item Citar selectivamente determinadas referencias con objeto de resaltar los hallazgos propios o de
    complacer a los directores de las revistas, a los revisores o a los colegas.
    \item Retrasar o dificultar indebidamente el trabajo de otros investigadores.
    \item Utilizar la posición de autoridad para fomentar violaciones de la integridad científica.
\end{itemize}