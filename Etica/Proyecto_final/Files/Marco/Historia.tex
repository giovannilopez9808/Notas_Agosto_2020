\subsubsection*{Historia de los códigos éticos en investigación}
Las circunstancias históricas y los valores de los tiempos influyeron directamente en la creación y evolución 
de los códigos éticos en la investigación científica, siendo su desarrollo acelerado pr múltiples situaciones partículares, 
tanto en sus procesos como en las disciplinas de conocimiento vinculadas en su creación. McNeil en su documento \cite{McInerney2001} destaca 
que en el año 1900 Alemania fue el primer país que generó un código ético para su uso local, siendo esta normativa remitida
a los directores de las clínicas hospitalitarias con el fin de limitar los experimentos médicos y obligando a los especialistas 
a describir sus intervenciones sanitarias a los adultos competentes. A pesar de la impliementación de estos reportes en la década 
del 20 y con el auge en la industria farmaceutica alemana se desarrollaron controvertidos experimentos que fueron motivos de fuertes 
críticas sociales, provocando así un desarrollo en las regulacione y directrices para el trabajo experimental en seres humanos 
en febrero de 1931.\\
Al momento de haber establecido códigos de ética en la investigación con seres humanos se produjo en gran medida, por las revelaciones de 
los experimentos médidos llevados a cabo por médicos nazis en los campos de concenrtración alemanes durante el Tercer Reich \cite{Schuklenk2000}. 
A partir de 1945 se realizaron esfuerzos claves para plantear los principios éticos en la investigación científica a nivel mundial. En 1947, el desarrollo
del \textit{Código de Nuremberg}, el cual orienta con principios considerados fundamentales para el establecimiento de procesos de investigación
con seres humanos, algunas de las maximas investigativas son las siguientes\cite{NurembergCode}:
\begin{enumerate}
    \item El consentimiento voluntario del ser humano es esencial, debiendo la persona tener capacidad
    legal para darlo libremente y sin intervención de elementos de fuerza, fraude, engaño, coacción
    y/o coerción. Se debe tener conocimiento cabal y comprensión de las materias involucradas en la
    investigación que se participará, a fin de conocer inconvenientes y riesgos razonables de esperar
    y efectos sobre la salud que puedan derivarse de su participación.
    \item El experimento debe producir resultados provechosos para el bien de la sociedad.
    \item El experimento debe diseñarse y basarse en los resultados de la experimentación previa con
    animales, con conocimiento de la historia natural de la enfermedad u otro problema que se
    estudie, y siempre que los resultados esperados justifiquen su realización.
    \item Se debe evitar todo tipo de lesiones en el experimento, todo sufrimiento físico y/o mental
    innecesario.
    \item Ningún experimento debe llevarse a cabo cuando existan razones que puedan producir lesiones
    incapacitantes o la muerte.
    \item El grado de riesgo nunca debe exceder el determinado por la importancia humanitaria del
    problema a ser resuelto por el experimento.
    \item El experimento debe prepararse correctamente y en instalaciones adecuadas para proteger al
    participante (sujeto experimental) contra posibilidades de lesión, incapacidad o muerte.
    \item Debe realizarse el experimento sólo por personas científicamente competentes.
    \item Durante el curso del experimento, la persona (sujeto experimental) debe tener la facultad para
    finalizarlo si considera que ha alcanzado un estado físico y/o mental donde la continuación le
    cause daño.
    \item  Así mismo, el científico responsable debe estar preparado para terminar el experimento en
    cualquier momento, si cree que es probable que resulte en lesiones, discapacidad, o la muerte de
    la persona.
\end{enumerate}