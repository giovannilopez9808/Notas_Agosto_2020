\subsection{Evaluación y revisión}
En la comunidad científica, el procedimiento más frecuente para la validación de los trabajos
escritos es la revisión por pares (\textit{peer review}) o arbitraje científico, con el fin 
de medir la calidad y rigor científico de la investigación presentada.
Las revisiones, en todas sus facetas (envíos para publicación, ascensos laborales, financiación de
proyectos, nombramientos de plazas) deben estar suficientemente razonadas y ser claras, precisas
e imparciales. Los revisores deben rechazar la revisión cuando se establezca alguna relación sospechosa de
parcialidad, falta de objetividad o transparencia respecto del sujeto y objeto de la evaluación.
Asimismo, se inhibirán de participar cuando concurra cualquier causa legal de abstención o
recusación. Por último, los revisores se inhibirán cuando no estén suficientemente preparados para
la revisión. Toda evaluación, para que sea justa y experta, tiene que ser objetiva. Los evaluadores deben
esforzarse en el conocimiento individualizado de los candidatos y saber interpretar los documentos
que se presenten, todo con el fin de hacerse una idea del trabajo desarrollado y de
la capacidad de cada aspirante. De igual forma, deben valorar a los candidatos en el contexto de su
entorno científico.