\subsection{Código de buenas prácticas}
Para la práctica de una investigación científica de calidad y éticamente correcta es necesario que exista un consenso entre los propios 
investigadores, con respecto a las actitudes y procedimientos en la preparación, desarrollo y comunicación acerca de la producción 
científica. Una forma de conseguirlo consiste en regirse bajo un código de buenas prácticas (BPC) y que los investigadores puedan
recurrir a un comité para la integridad de la investigación.\\
En la comunidad científica internacional se dispone, actualmente, de un amplio consenso
con respecto a los componentes más importantes de todo aquello que constituyen unas
buenas prácticas científicas. Las dos finalidades principales de las BPC son la mejora
de la calidad de la ciencia y la prevención de problemas de integridad de la investigación.
Consisten en un conjunto de declaraciones y compromisos que van más allá de lo que se
establece en las normas jurídicas o bien amplían algunos aspectos ya recogidos en normas
específicas para la práctica de la ciencia.
\subsection{Difusión de los resultados de la investigación}
\subsubsection{Autoría}
Para poder tener la condición plena de autor de un trabajo publicado será necesario cumplir con todas las condiciones siguientes:
\begin{enumerate}
    \item Que exista una contribución sustancial a la concepción o diseño del trabajo o a la adquisición,
    análisis o a la interpretación de los datos.
    \item Que se haya participado en la redacción del trabajo o en la revisión crítica de su contenido
    intelectual.
    \item Que se haya intervenido en la aprobación de la versión final que vaya a ser publicada.
    \item Que se tenga capacidad de responder de todos los aspectos del artículo de cara a asegurar
    que las cuestiones relacionadas con la exactitud o integridad de cualquier parte del trabajo
    están adecuadamente investigadas y resueltas.
\end{enumerate}
Cualquier persona que no cumpla con los criterios de autoría descritos pero que haya colaborado
en el trabajo de alguna otra manera deberá ser reconocida en el apartado de agradecimientos.\\
El orden de los autores debe establecerse según las pautas aceptadas en la disciplina objeto del
trabajo, las cuales deberán ser conocidas previamente al inicio de la investigación por todos ellos.
Cuando la contribución de cada autor tiene un carácter diferenciado, es una práctica habitual que
el orden de la firma en las publicaciones sea el siguiente:
\begin{itemize}
    \item El primer coautor es la persona que ha hecho la contribución más importante en la
    investigación y ha preparado el primer borrador del artículo
    \item El último autor es la persona que dirige la investigación o que tiene la última
    responsabilidad en el protocolo de investigación.
    \item El resto de coautores pueden aparecer ordenados por orden de contribución y, en
    algunos casos, si la contribución de todos ellos es similar, por orden alfabético, con
    mención expresa de ello.
\end{itemize}
Junto con los autores, deben citarse las instituciones o los centros de adscripción en los que se
hubiese realizado la investigación. Las subvenciones, ayudas financieras o patrocinios económicos
recibidos para realizar la investigación deben ser declarados y agradecidos, siempre y cuando no se
hubiere declinado su mención. En todos los trabajos publicados deberá incluirse explícitamente a
los comités de ética de la investigación que hayan aprobado el protocolo de investigación.
La difusuón de ámbitos científicos de los resultados obtenidos en un proyecto de investigación se considera al comienzo de la etapa
final de la investigación. La publicación de los resultados forma parte de del proceso en el cual la comunidad científica 
sustancia y corrige los resultados obtenidos o desarrolla cambios propuestos, este proceso es el único medio estandarizado por el cual
los resultados quedan sometidos a la revisión pública con conocimientos semejantes. La no publicación de los resultados de una investigación
o la demora se considera malversación de los recursos empleados.
\subsubsection{Publicación de los resultados}
La difusión de los resultados es un deber ético de los investigadores, como contribución al incremento
y al avance del conocimiento, una parte escencial del proceso es la redención de cuentas de la utilización 
de los medios públicos para la investigación. La publicación de los resultados estará subordinada a las posibles necesidades de protección 
de la propiedad industrial e intelectual. Los investigadores deben esforzarse en publicar los resultados y las interpretaciones de su
investigación de una manera abierta, honesta, transparente y exacta, lo que incluye aquellos
resultados que no estuvieran en línea con las hipótesis planteadas. Los resultados negativos y los no
concluyentes son tan válidos como los positivos a efectos de difusión y, por tanto, también deben
publicarse. En el caso de detectarse errores en el contenido de alguna publicación, se deberán reconocer en
publicaciones del mismo nivel. La retractación de la publicación es necesaria en el caso de errores
graves. La difusión de resultados a través de los medios de comunicación debe incluir siempre una
explicación de carácter divulgativo o una parte de la presentación adaptada a públicos no
especializados. En este tipo de presentaciones públicas, el nombre de los autores ha de ir siempre
asociado al de sus instituciones y, siempre que sea posible, se deben mencionar las subvenciones y
las ayudas recibidas. Cuando se trate de artículos de opinión, se advertirá que esos juicios son
personales y no de la institución. No se considera aceptable que se comuniquen y difundan los
resultados de una investigación a los medios de comunicación antes de la publicación en revistas
científicas. En la difusión de los resultados a los medios de comunicación, al igual que en las
publicaciones mismas, tampoco se debe expresar un optimismo excesivo ni generar falsas
expectativas con relación a la investigación.