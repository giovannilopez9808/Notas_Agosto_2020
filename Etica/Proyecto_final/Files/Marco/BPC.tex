\subsection*{Código de buenas prácticas}
Para la práctica de una investigación científica de calidad y éticamente correcta es necesario que exista un consenso entre los propios 
investigadores, con respecto a las actitudes y procedimientos en la preparación, desarrollo y comunicación acerca de la producción 
científica. Una forma de conseguirlo consiste en regirse bajo un código de buenas prácticas (BPC) y que los investigadores puedan
recurrir a un comité para la integridad de la investigación.\\
En la comunidad científica internacional se dispone, actualmente, de un amplio consenso
con respecto a los componentes más importantes de todo aquello que constituyen unas
buenas prácticas científicas. Las dos finalidades principales de las BPC son la mejora
de la calidad de la ciencia y la prevención de problemas de integridad de la investigación.
Consisten en un conjunto de declaraciones y compromisos que van más allá de lo que se
establece en las normas jurídicas o bien amplían algunos aspectos ya recogidos en normas
específicas para la práctica de la ciencia.
\subsubsection*{Prácticas de publicación}
La difusuón de ámbitos científicos de los resultados obtenidos en un proyecto de investigación se considera al comienzo de la etapa
final de la investigación. La publicación de los resultados forma parte de del proceso en el cual la comunidad científica 
sustancia y corrige los resultados obtenidos o desarrolla cambios propuestos, este proceso es el único medio estandarizado por el cual
los resultados quedan sometidos a la revisión pública con conocimientos semejantes. La no publicación de los resultados de una investigación
o la demora se considera malversación de los recursos empleados.