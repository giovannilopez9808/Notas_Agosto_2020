La investigación científica es una actividad en la que el investigador expone su trabajo al mundo para
realizar aportaciones a la comunidad, de esta manera la persona expone la integridad ética de su institución
y de el mismo. Al estar sumergidos en un contexto competitivo uno puede llegar a sentirse que no logra una 
aportación a la comunidad con sus trabajos, es por ello que modificando sus datos, estar como autores en trabajos
donde no la persona no realizó una actividad o adjudicarse ideas cuando ya fueron mencionadas en trabajos publicados
son actividades recurrentes. Con esta problematica es necesario tener una conscintencia en las prácticas y 
cuestionar nuestras propias posiciones de investigación, para esto es necesario considerar nuestras identidades 
personales y profesionales, las experiencias previas y preferencias.\\
Uno de los aspectos a considerar en los procesos de reflexión es si el tener presente los códigos y normas éticas
desarrollan buenas conductas en los profesionales. La toma de decisiones que incita a la mejora en los procesos
y resultados de la investigación con respecto a la prevención de conductas no éticas, se puede aplicar en distintos
ámbitos. Uno es el internacional, en el que ya se han propuesto importantes resoluciones y documentos con respecto
al tema, que continuamente se están revisando y complementando y que son una indispensable fuente de consulta.
Es importante que estos avances se conozcan y se socialicen en la comunidad científica y en las universidades mexicanas.