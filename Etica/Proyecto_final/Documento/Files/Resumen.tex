La investigación científica es una actividad que desarrolla la persona encargada de contribuir conocimiento 
para el avance tecnologico y la sociedad humana, la honestidad y responsabilidad que se debe de llevar a cabo
al momento de realizar una investigación científica es enorme es por ello que se crearon códigos éticos y de conducta 
durante la historia, esto aplicado para los procesos que suceden antes, durante y después de la investigación.
Debido a la competencia por publicar artículos relevantes y de gran impacto en la sociedad se han llegado a 
presentar con mayor presencia conductas no éticas, algunas de estas son reportadas o expuestas por otros científicos 
refutando lo mencionado en los artículos publicados o en algunos casos reportar a las autoridades las actitudes
que tienen con respecto a sus compañeros. Se llevo a la conclusión que deben existir periodos de reflexión en los cuales
se tengan presente los códigos y normas éticas en las tomas de decisiones para incitar a una mejora en los procesos y resultados
de la investigación científica.\\\\
Scientific research is an activity carried out by the person in charge of contributing knowledge 
for the technological advance and the human society, the honesty and responsibility that must be carried out
at the time of conducting scientific research is enormous that is why codes of ethics and conduct were created 
during history, this applied to the processes that occur before, during and after the research.
Due to the competition to publish relevant articles with great impact on society, we have reached 
present more unethical behaviors, some of which are reported or exposed by other scientists 
refuting what is mentioned in the published articles or in some cases reporting to the authorities the attitudes
they have with respect to their peers. It was concluded that there should be periods of reflection in which
ethical codes and standards are taken into account in decision-making to encourage improvement in processes and results
of scientific research.