\section{Introducción}\begin{frame}{Introduction}
    \begin{minipage}{0.4\linewidth}
        Scientific research is an activity that contributes knowledge to the advancement of human society. Their contributions
        are multiple, this is because scientific research is the basis for the generation of knowledge about the nature of the objects in our
        around and society.
    \end{minipage}
    \hspace{0.4cm}
    \begin{minipage}{0.45\linewidth}
        \centering
        \includegraphics[scale=1.2]{images/ima1.jpg}
    \end{minipage}
\end{frame}
\begin{frame}{Introduction}
    \begin{minipage}{0.45\linewidth}    
        \centering
        \includegraphics[scale=0.22]{images/ima2.jpg}
    \end{minipage}
    \hspace{0.4cm}
    \begin{minipage}{0.45\linewidth}
        The ethical aspect of the research has been governed by a combination of factors originating from specific codes of conduct 
that attempt to govern practices that preserve the professional prestige of scientific researchers in the various existing branches.
    \end{minipage}
\end{frame}
\begin{frame}{Introduction}
    \begin{minipage}{0.45\linewidth}
        The contemporary scientific system requires constant competition for recognition and credit, which can lead to a struggle
of economic support, leading in some cases to the origin of ethical misconduct and fraud regularly revealed in certain personalities
narcissists who motivate the predisposition to bad practices.\\
    \end{minipage}
    \hspace{0.5cm}
    \begin{minipage}{0.4\linewidth}
        \centering
        \includegraphics[scale=0.45]{images/ima3.jpg}
    \end{minipage}
\end{frame}

\begin{frame}{Introducción}
    \begin{minipage}{0.4\linewidth}
    La investigación científica es una actividad que contribuye conocimiento para el avance en la sociedad humana. Sus aportaciones
son múltiples, esto es por que la investigación científica es la base para la generación de conociemiento sobre la naturaleza de los objetos en nuestro
alrededor y la sociedad.
    \end{minipage}
    \hspace{0.4cm}
    \begin{minipage}{0.45\linewidth}
        \centering
        \includegraphics[scale=1.2]{images/ima1.jpg}
    \end{minipage}
\end{frame}
\begin{frame}{Introducción}
    \begin{minipage}{0.45\linewidth}    
        \centering
        \includegraphics[scale=0.22]{images/ima2.jpg}
    \end{minipage}
    \hspace{0.4cm}
    \begin{minipage}{0.45\linewidth}
        El aspecto ético de la investigación ha sido regido por una combinación de factores que se originan desde los códigos de conducta específicos 
que intentan regir las prácticas que preservan la situación de prestigio profesional de los investigadores científicos de las diversas ramas existentes.
    \end{minipage}
\end{frame}
\begin{frame}{Introducción}
    \begin{minipage}{0.45\linewidth}
        El sistema científico cotemporáneo exige competir constantemente para la obtención de reconocimiento y crédito, lo cual llega a producir una lucha
continua de apoyos económicos, propiciando en algunos casos el origen de malas conductas éticas y fraudes revelados regularmente en ciertas personalidades
narcisistas que motivan la predisposición a malas prácticas.\\
    \end{minipage}
    \hspace{0.5cm}
    \begin{minipage}{0.4\linewidth}
        \centering
        \includegraphics[scale=0.45]{images/ima3.jpg}
    \end{minipage}
\end{frame}