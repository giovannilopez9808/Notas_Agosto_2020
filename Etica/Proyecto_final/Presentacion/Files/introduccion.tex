\section{Introducción}
\begin{frame}{Introducción}
    \begin{minipage}{0.4\linewidth}
    La investigación científica es una actividad que contribuye conocimiento para el avance en la sociedad humana. Sus aportaciones
son múltiples, esto es por que la investigación científica es la base para la generación de conociemiento sobre la naturaleza de los objetos en nuestro
alrededor y la sociedad.
    \end{minipage}
    \hspace{0.4cm}
    \begin{minipage}{0.45\linewidth}
        \centering
        \includegraphics[scale=1.2]{images/ima1.jpg}
    \end{minipage}
\end{frame}
\begin{frame}{Introducción}
    \begin{minipage}{0.45\linewidth}    
        \centering
        \includegraphics[scale=0.22]{images/ima2.jpg}
    \end{minipage}
    \hspace{0.4cm}
    \begin{minipage}{0.45\linewidth}
        El aspecto ético de la investigación ha sido regido por una combinación de factores que se originan desde los códigos de conducta específicos 
que intentan regir las prácticas que preservan la situación de prestigio profesional de los investigadores científicos de las diversas ramas existentes.
    \end{minipage}
\end{frame}
\begin{frame}{Introducción}
    \begin{minipage}{0.45\linewidth}
        El sistema científico cotemporáneo exige competir constantemente para la obtención de reconocimiento y crédito, lo cual llega a producir una lucha
continua de apoyos económicos, propiciando en algunos casos el origen de malas conductas éticas y fraudes revelados regularmente en ciertas personalidades
narcisistas que motivan la predisposición a malas prácticas.\\
    \end{minipage}
    \hspace{0.5cm}
    \begin{minipage}{0.4\linewidth}
        \centering
        \includegraphics[scale=0.45]{images/ima3.jpg}
    \end{minipage}
\end{frame}