\documentclass[12pt,letterpaper]{report}
\usepackage{graphicx}
\usepackage{scrextend}
\usepackage{vmargin}
\usepackage{graphicx}
\usepackage{multirow}
\usepackage[utf8]{inputenc}
\usepackage[spanish]{babel}
\usepackage{multicol}
\usepackage{enumerate}
\usepackage{float}
\usepackage{amsmath, amsthm, amssymb, amsfonts}
\usepackage[usenames]{color}
\parindent=0mm
\pagestyle{empty}
\definecolor{miorange}{rgb}{0.91, 0.43, 0.0}
\begin{document}
\setmargins{2.5cm}      
{1.5cm}                     
{2cm}  
{24cm}                    
{10pt}                          
{1cm}                          
{0pt}                             
{2cm}
\begin{titlepage}
\begin{center}
\includegraphics[scale=0.40]{../../Logos/uanl.png} 
\hspace{2.5cm}
\includegraphics[scale=0.40]{../../Logos/fcfm.png}
\end{center}
\vspace{2cm}
\begin{center}
\textbf{
UNIVERSIDAD AUTÓNOMA DE NUEVO LEÓN\\
FACULTAD DE CIENCIAS
    FÍSICO MATEMÁTICAS}\\
\vspace*{2cm}
\begin{large}
\vspace{1cm}
\large{\textbf{Tópicos de Mécanica Cuántica}}\\
\textbf{Tarea 8}\\
Enrique Valbuena Ordonez\\
\end{large}
\vspace{3.5cm}
\begin{minipage}{0.6\linewidth}
\vspace{0.5cm}
\changefontsizes{14pt}
Nombre:\\
Giovanni Gamaliel López Padilla\\
\end{minipage}
\begin{minipage}{0.2\linewidth}
\changefontsizes{14pt}
Matricula:\\
1837522
\end{minipage}
\end{center}
\vspace{4cm}
\begin{flushright}
\today
\end{flushright}
\end{titlepage}
Encontrar las ecuaciones para $\phi$ y $\phi^*$ que satisfacen las ecuaciones de Euler-Lagrange para campos.\\
Sea la lagrangiana 
\begin{equation*}
     \mathcal{L}=\left(D^\mu \phi\right)^* \left(D_\mu \phi \right) -m^2\phi^* \phi
\end{equation*}
y las ecuaciones de euler-lagrange:
\begin{equation*}
    \frac{\partial \mathcal{L}}{\partial \phi} - \partial_\mu \frac{\partial \mathcal{L}}{\partial(\partial_\mu \phi)}=0 \qquad
    \frac{\partial \mathcal{L}}{\partial \phi^*} - \partial_\mu \frac{\partial \mathcal{L}}{\partial(\partial_\mu \phi^*)}=0
\end{equation*}
Calculando $\frac{\partial \mathcal{L}}{\partial \phi}$, se tiene que:
\begin{equation*}
    \frac{\partial \mathcal{L}}{\partial \phi} = -m^2\phi^*
\end{equation*}
Calculado $\partial_\mu \frac{\partial \mathcal{L}}{\partial(\partial_\mu \phi)}$
\begin{equation*}
    \partial_\mu \frac{\partial \mathcal{L}}{\partial(\partial_\mu \phi)} = \partial_\mu \left(D^\mu \phi\right)^*
\end{equation*}
por lo tanto:
\begin{equation*}
    \partial_\mu \left(D^\mu \phi\right)^*+ m^2\phi^* =0
\end{equation*}
Calculando $\frac{\partial \mathcal{L}}{\partial \phi^*}$, se tiene que:
\begin{equation*}
    \frac{\partial \mathcal{L}}{\partial \phi} = -m^2\phi
\end{equation*}
Calculado $\partial_\mu \frac{\partial \mathcal{L}}{\partial(\partial_\mu \phi^*)}$
\begin{equation*}
    \partial_\mu \frac{\partial \mathcal{L}}{\partial(\partial_\mu \phi)} = \partial_\mu \left(D^\mu \phi\right)
\end{equation*}
por lo tanto:
\begin{equation*}
    \partial_\mu \left(D^\mu \phi\right)+ m^2\phi =0
\end{equation*}
\end{document}