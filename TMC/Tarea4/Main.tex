\documentclass[12pt,letterpaper]{report}
\usepackage{graphicx}
\usepackage{scrextend}
\usepackage{vmargin}
\usepackage{graphicx}
\usepackage{multirow}
\usepackage[utf8]{inputenc}
\usepackage[spanish]{babel}
\usepackage{multicol}
\usepackage{enumerate}
\usepackage{float}
\usepackage{amsmath, amsthm, amssymb, amsfonts}
\usepackage[usenames]{color}
\parindent=0mm
\pagestyle{empty}
\definecolor{miorange}{rgb}{0.91, 0.43, 0.0}
\begin{document}
\setmargins{2.5cm}      
{1.5cm}                     
{2cm}  
{24cm}                    
{10pt}                          
{1cm}                          
{0pt}                             
{2cm}
\begin{titlepage}
\begin{center}
\includegraphics[scale=0.40]{../../Logos/uanl.png} 
\hspace{2.5cm}
\includegraphics[scale=0.40]{../../Logos/fcfm.png}
\end{center}
\vspace{2cm}
\begin{center}
\textbf{
UNIVERSIDAD AUTÓNOMA DE NUEVO LEÓN\\
FACULTAD DE CIENCIAS
    FÍSICO MATEMÁTICAS}\\
\vspace*{2cm}
\begin{large}
\vspace{1cm}
\large{\textbf{Tópicos de Mécanica Cuántica}}\\
\textbf{Tarea 4}\\
Enrique Valbuena Ordonez\\
\end{large}
\vspace{3.5cm}
\begin{minipage}{0.6\linewidth}
\vspace{0.5cm}
\changefontsizes{14pt}
Nombre:\\
Giovanni Gamaliel López Padilla\\
\end{minipage}
\begin{minipage}{0.2\linewidth}
\changefontsizes{14pt}
Matricula:\\
1837522
\end{minipage}
\end{center}
\vspace{4cm}
\begin{flushright}
\today
\end{flushright}
\end{titlepage}
\section*{Demostrar que el operador $\hat{P}^n$ deja invariante al operador hamiltoniano $\hat{H}$}
Se tiene que:
\begin{align}
    \label{pn}
    \hat{P}^n \psi(x)=\psi((-1)^n x )\\
    \label{hamilton}
    \hat{H}= -\frac{\hbar^2 }{2m}  \frac{\partial^2}{\partial x^2} 
\end{align}
y si el operador $\hat{P}^n$ es deja invariante a $\hat{H}$, entonces:
\begin{equation}
    \label{conmu}
    \left[\hat{H},\hat{P}^n \right] \psi(x)=0
\end{equation}
calculando la ecuación \ref{conmu} utilizando \ref{pn} y \ref{hamilton}
\begin{align*}
    \left[\hat{H},\hat{P}^n \right] \psi(r) & = \hat{H}\hat{P}^n \psi(x) - \hat{P}^n\hat{H}^n \psi(x) \\
    & =-\frac{\hbar^2 }{2m}  \frac{\partial^2}{\partial x^2} \psi((-1)^n x) - \hat{P}^n \left(-\frac{\hbar}{2m} \frac{\partial^2 }{\partial x^2}\psi(x) \right)\\
    & =-\frac{\hbar^2 }{2m}  \frac{\partial^2}{\partial x^2} \psi((-1)^n x) + \frac{\hbar}{2m}\hat{P}^n \left( \frac{\partial^2 }{\partial x^2}\psi(x) \right)\\
    & =-\frac{\hbar^2 }{2m}  \frac{\partial^2}{\partial x^2} \psi((-1)^n x) + \frac{\hbar}{2m}\frac{\partial^2 }{\partial x^2}\psi((-1)^nx)\\
    &= 0
\end{align*}
entonces la ecuación \ref{conmu} se cumple, por lo tanto, el operador $\hat{P}^n$ deja invariante al operador $\hat{H}$
\end{document}