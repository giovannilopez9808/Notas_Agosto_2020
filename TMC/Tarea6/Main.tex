\documentclass[12pt,letterpaper]{report}
\usepackage{graphicx}
\usepackage{scrextend}
\usepackage{vmargin}
\usepackage{graphicx}
\usepackage{multirow}
\usepackage[utf8]{inputenc}
\usepackage[spanish]{babel}
\usepackage{multicol}
\usepackage{enumerate}
\usepackage{float}
\usepackage{amsmath, amsthm, amssymb, amsfonts}
\usepackage[usenames]{color}
\parindent=0mm
\pagestyle{empty}
\definecolor{miorange}{rgb}{0.91, 0.43, 0.0}
\begin{document}
\setmargins{2.5cm}      
{1.5cm}                     
{2cm}  
{24cm}                    
{10pt}                          
{1cm}                          
{0pt}                             
{2cm}
\begin{titlepage}
\begin{center}
\includegraphics[scale=0.40]{../../Logos/uanl.png} 
\hspace{2.5cm}
\includegraphics[scale=0.40]{../../Logos/fcfm.png}
\end{center}
\vspace{2cm}
\begin{center}
\textbf{
UNIVERSIDAD AUTÓNOMA DE NUEVO LEÓN\\
FACULTAD DE CIENCIAS
    FÍSICO MATEMÁTICAS}\\
\vspace*{2cm}
\begin{large}
\vspace{1cm}
\large{\textbf{Tópicos de Mécanica Cuántica}}\\
\textbf{Tarea 6}\\
Enrique Valbuena Ordonez\\
\end{large}
\vspace{3.5cm}
\begin{minipage}{0.6\linewidth}
\vspace{0.5cm}
\changefontsizes{14pt}
Nombre:\\
Giovanni Gamaliel López Padilla\\
\end{minipage}
\begin{minipage}{0.2\linewidth}
\changefontsizes{14pt}
Matricula:\\
1837522
\end{minipage}
\end{center}
\vspace{4cm}
\begin{flushright}
\today
\end{flushright}
\end{titlepage}
\subsection*{Demostrar que: $\oint \vec{J} \cdot d\vec{S}$ = 0}
Se tiene que la densidad de corriente de probabilidad es lo siguiente:
\begin{equation}
    \vec{J} = \frac{\hbar }{2mi}\left(\psi^* \nabla \psi-\psi \nabla \psi^* \right)
    \label{eq:j}
\end{equation}
Donde $\psi$ y $\psi^*$ forman la densidad de probabilidad de la siguiente manera:
\begin{equation}
    \rho= \psi \psi^*
\end{equation}
derivando esta expresión respecto el tiempo:
\begin{align*}
    \frac{\partial \rho}{\partial t}    &= \frac{\partial}{\partial t}\left(\psi \psi^*  \right)\\
                                        &=\psi \frac{\partial \psi^* }{\partial t} + \psi^* \frac{\partial \psi}{\partial t}
\end{align*}
como $\psi$ y $\psi^* $ son soluciones a la ecuación de Schr\"odinger, entonces:
\begin{equation}
    H\psi =i \hbar \frac{\partial \psi}{\partial t}
    \label{eq:hpsi}
\end{equation}
\begin{equation}
    H\psi^* =i \hbar \frac{\partial \psi^*}{\partial t}
    \label{eq:hpsi*}
\end{equation}
por lo tanto:
\begin{equation}
    \frac{\partial \rho}{ \partial t} = \frac{i\hbar}{2m} \left(\psi^* \nabla^2 \psi - \psi \nabla^2 \psi^* \right)
    \label{eq:partialrho}
\end{equation}
Si calculamos $\nabla \cdot \vec{J}$, encontramos lo siguiente:
\begin{align*}
    \nabla \cdot \vec{j}    &= \frac{\hbar}{2mi} (\nabla \cdot (\psi^* \nabla \psi - \psi \nabla \psi^*))\\
                            &= \frac{\hbar}{2mi} (\nabla \cdot(\psi^* \nabla \psi) - \nabla \cdot (\psi \nabla \psi^*))\\
                            &= \frac{\hbar}{2mi} (\nabla \psi \nabla \psi^* + \psi \nabla^2 \psi^* - (\nabla \psi \nabla \psi^* + \psi^* \nabla^2 \psi))\\
                            &=\frac{\hbar}{2mi}(\psi \nabla^2 \psi^* - \psi^* \nabla^2 \psi)\\
                            &= -\frac{i\hbar}{2m} (\psi \nabla^2 \psi^* - \psi^* \nabla^2 \psi)
\end{align*}
por lo tanto 
\begin{equation}
    \nabla \cdot \vec{j} = -\frac{i\hbar}{2m} (\psi \nabla^2 \psi^* - \psi^* \nabla^2 \psi)
    \label{eq:nablaj}
\end{equation}
sumando al ecuación \ref{eq:partialrho} y \ref{eq:nablaj} se tiene que:
\begin{equation}
    \frac{\partial \rho}{ \partial t}  + \nabla \cdot \vec{j} =0
    \label{eq:suma}
\end{equation}
integrando la ecuación \ref{eq:suma} sobre todo el volumen se obtiene lo siguiente:
\begin{align*}
    \int_V \left(\frac{\partial \rho}{ \partial t}  + \nabla \cdot \vec{j} \right) dV &= \int_V \frac{\partial \rho}{ \partial t}  dV+\int_V \nabla \cdot \vec{j} dV\\
    &=\frac{\partial}{\partial t}\left(\int_V \rho dV\right) +\int_V  (\nabla \cdot \vec{j} )dV\\
    &=\frac{\partial}{\partial t} \left(1 \right) + \int_V (\nabla \cdot \vec{j} )dV \\
    &= \int_V( \nabla \cdot \vec{j} )dV
\end{align*}
por lo tanto:
\begin{equation}
    \int_V( \nabla \cdot \vec{j} )dV = 0 
    \label{eq:intj}
\end{equation}
por ende:
\begin{equation}
    \int \vec{J} \cdot d\vec{S} = 0
\end{equation}
\end{document}