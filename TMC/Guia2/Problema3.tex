\section{Calcular la lagrangiana de interacción de Schr\"odinger en términos del cuadrivector de potencial.}
Se tiene que la lagrangiana de Schro\"odinger es la ecuación \ref{eq:den_lan} y tomando en consideración que:
\begin{align*}
    \partial_t &\rightarrow\partial_t -igA_0\\
    \partial_t^* &\rightarrow\partial_t +igA_0\\
    {\nabla}' & \rightarrow \nabla - ig\vec{A}\\
    {\nabla}'^* & \rightarrow \nabla + ig\vec{A}
\end{align*}
Esto para obtener los terminos de interacción dentro de la lagrangiana, introduciendo estos terminos en la ecuación \ref{eq:den_lan}, se obtiene lo siguiente:
\begin{align*}
    \mathcal{L} &= -\frac{\hbar^2}{2m}\left[ {\nabla}'\psi \cdot {\nabla}' \psi^*\right] + i\hbar \left[\psi^* {\partial}'_t \psi - \psi {\partial}'_t \psi^*\right]\\
    &= -\frac{\hbar^2}{2m}\left[ \left(\nabla - ig\vec{A}\right)\psi \cdot \left( \nabla + ig\vec{A}\right) \psi^*\right] + i\hbar \left[\psi^* \left(\partial_t -igA_0\right) \psi - \psi \left(\partial_t +igA_0\right) \psi^*\right]\\
    &= + i\hbar \left[\psi^*\partial_t \psi - \psi \partial_t \psi^*\right]\\
\end{align*}
por lo tanto la lagrangiana de interacción de Schr\"odinger en términos del cuadrivector del potencial es:
\begin{equation}
    \changefontsizes{10pt}
    \mathcal{L}= -\frac{\hbar^2}{2m}\left[ \left(\nabla - ig\vec{A}\right)\psi \left( \nabla + ig\vec{A}\right) \psi^*\right] + i\hbar \left[\psi^* \left(\partial_t -ig\vec{A}\right) \psi - \psi \left(\partial_t +ig\vec{A}\right) \psi^*\right]
    \label{eq:den_lag_a}
\end{equation}