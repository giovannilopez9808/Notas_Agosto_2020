\documentclass[12pt,letterpaper]{report}
\usepackage{graphicx}
\usepackage{scrextend}
\usepackage{vmargin}
\usepackage{graphicx}
\usepackage{multirow}
\usepackage[utf8]{inputenc}
\usepackage[spanish]{babel}
\usepackage{multicol}
\usepackage{enumerate}
\usepackage{float}
\usepackage{amsmath, amsthm, amssymb, amsfonts}
\usepackage[usenames]{color}
\parindent=0mm
\pagestyle{empty}
\definecolor{miorange}{rgb}{0.91, 0.43, 0.0}
\begin{document}
\setmargins{2.5cm}      
{1.5cm}                     
{2cm}  
{24cm}                    
{10pt}                          
{1cm}                          
{0pt}                             
{2cm}
\begin{titlepage}
\begin{center}
\includegraphics[scale=0.40]{../../Logos/uanl.png} 
\hspace{2.5cm}
\includegraphics[scale=0.40]{../../Logos/fcfm.png}
\end{center}
\vspace{2cm}
\begin{center}
\textbf{
UNIVERSIDAD AUTÓNOMA DE NUEVO LEÓN\\
FACULTAD DE CIENCIAS
    FÍSICO MATEMÁTICAS}\\
\vspace*{2cm}
\begin{large}
\vspace{1cm}
\large{\textbf{Tópicos de Mécanica Cuántica}}\\
\textbf{Tarea 9}\\
Enrique Valbuena Ordonez\\
\end{large}
\vspace{3.5cm}
\begin{minipage}{0.6\linewidth}
\vspace{0.5cm}
\changefontsizes{14pt}
Nombre:\\
Giovanni Gamaliel López Padilla\\
\end{minipage}
\begin{minipage}{0.2\linewidth}
\changefontsizes{14pt}
Matricula:\\
1837522
\end{minipage}
\end{center}
\vspace{4cm}
\begin{flushright}
\today
\end{flushright}
\end{titlepage}
Encontrar la eecuaciones de Maxwell a partir del lagrangiano con la interacción electromagnética y la ecuación de euler-lagrange para campos.\\
Sea el lagrangiano:
\begin{equation*}
    \mathcal{L}=-\frac{1}{4\mu_0}F_{\mu \nu}F^{\mu\nu}, \qquad
    F_{\mu \nu}=\partial_\mu A_\nu - \partial_\nu A_\mu
\end{equation*}
y las ecuaciones de euler-lagrange para campos:
\begin{equation*}
    \partial_\nu \frac{\partial \mathcal{L}}{\partial (\partial_\nu A_\mu)}  - \frac{\partial \mathcal{L}}{\partial A_\mu} = 0
\end{equation*}
Calculando $\frac{\partial \mathcal{L}}{\partial (\partial_\nu A_\mu)} $
\begin{align*}
    \frac{\partial \mathcal{L}}{\partial (\partial_\nu A_\mu)}  &= \frac{\partial }{\partial (\partial_\nu A_\mu)} \left(-\frac{1}{4\mu_0}F_{\mu \nu}F^{\mu\nu}\right)\\
    &= \frac{1}{4\mu_0}\frac{\partial }{\partial (\partial_\nu A_\mu)} \left(\partial_\nu A_\mu-\partial_\mu A_\nu \right) F^{\mu \nu}\\
    &= \frac{1}{4\mu_0} \frac{\partial }{\partial (\partial_\nu A_\mu)} \left(\partial_\nu A_\mu F^{\mu \nu}\right) \\
    & = \frac{1}{4\mu_0} F^{\mu \nu}
\end{align*}
entonces:
\begin{equation*}
    \frac{\partial \mathcal{L}}{\partial (\partial_\nu A_\mu)}  = \frac{1}{4\mu_0} F^{\mu \nu}
\end{equation*}
como la lagrangiana no depende el término $A_\mu$, entonces
\begin{equation*}
    \frac{\partial \mathcal{L}}{\partial A_\mu} =0
\end{equation*}
por lo que tenemos el siguiente resultado para la ecuación de euler-lagrange
\begin{equation*}
    \frac{1}{4\mu_0}\partial_\nu F^{\mu \nu }=0
\end{equation*}
donde 
\begin{equation*}
    F^{\mu \nu} = \left(\begin{matrix}
        0 & B_x & B_y & B_z \\
        -B_x & 0 & \frac{E_z}{c} & -\frac{E_y}{c} \\
        -B_y & -\frac{E_z}{c}& 0 & \frac{E_x}{c} \\
        -B_z & \frac{E_y}{c} & -\frac{E_x}{c} & 0
    \end{matrix}\right) \qquad 
    *F^{\mu \nu} = \left(\begin{matrix}
        0 & \frac{E_x}{c} & \frac{E_y}{c} & \frac{E_z}{c} \\
        -\frac{E_x}{c} & 0 & B_z & -B_y \\
        -\frac{E_y}{c} & -B_z& 0 & B_x \\
        -\frac{E_z}{c} & B_y & -B_x & 0
    \end{matrix}\right)
\end{equation*}
tomando a $F^{\mu \nu}$, se tiene que:
\begin{equation*}
    \partial_\nu F^{0 \nu} = 0 \qquad \partial_\nu F^{1 \nu} = 0 \qquad \partial_\nu F^{2 \nu} = 0 \qquad \partial_\nu F^{3 \nu} = 0
\end{equation*}
de $\partial_\nu F^{0 \nu} = 0$, se obtiene la siguiente ecuación de Maxwell:
\begin{equation*}
    \nabla \cdot E = 0
\end{equation*}
realizando la suma de $\partial_\nu F^{1 \nu} +\partial_\nu F^{2 \nu}+\partial_\nu F^{3 \nu}=0$, se obtiene que:
\begin{equation*}
    \nabla \times B = \frac{1}{c^2}\partial_t E
\end{equation*}
tomando a $*F^{\mu \nu}$, se tiene que:
\begin{equation*}
    \partial_\nu F^{0 \nu} = 0 \qquad \partial_\nu F^{1 \nu} = 0 \qquad \partial_\nu F^{2 \nu} = 0 \qquad \partial_\nu F^{3 \nu} = 0
\end{equation*}
de $\partial_\nu F^{0 \nu} = 0$, se obtiene la siguiente ecuación de Maxwell:
\begin{equation*}
    \nabla \cdot B = 0
\end{equation*}
realizando la suma de $\partial_\nu F^{1 \nu} +\partial_\nu F^{2 \nu}+\partial_\nu F^{3 \nu}=0$, se obtiene que:
\begin{equation*}
    \nabla \times E = -\partial_t B
\end{equation*}
por lo que llegamos a obtener las cuatro ecuaciones de Maxwell
\begin{equation*}
    \nabla \cdot E = 0  \qquad 
    \nabla \times B = \frac{1}{c^2}\partial_t E  \qquad
    \nabla \cdot B = 0 \qquad
    \nabla \times E = -\partial_t B
\end{equation*}
\end{document}