\textbf{1. Demostrar que la lagrangiana de interacción de Klein-Gordon es invariante de norma, posee simetria U(1) y obtener
las corrientes de Noether asociadas a las respectivas transformaciones}
\begin{itemize}
    \item Invariante de norma.\\
    Se tiene que:
    \begin{equation*}
        \mathcal{L}= \left(\partial_\mu \phi \right)^* \left(\partial^\mu \phi\right) - m^2\phi^*\phi
    \end{equation*}
    aplicando la transformaciónes:
    \begin{equation*}
        \partial^\mu \rightarrow \partial^\mu-igA^\mu 
    \end{equation*}
    se tiene lo siguiente:
    \begin{equation*}
        \mathcal{L}= \left(\partial_\mu\phi^*+igA_\mu  \phi^*\right)
        \left(\partial^\mu \phi-igA^\mu \phi \right)-m^2 \phi^* \phi .
    \end{equation*}
    Realizando las operaciones se tiene que:
    \begin{equation*}
        {\mathcal{L}}'= \partial_\mu \phi^* \partial^\mu \phi -m^2 \phi^* \phi -ig A_\mu \phi \partial^\mu \phi +ig A^\mu \phi \partial_\mu \phi^* + g^2 A^\mu A_\mu \phi^* \phi
    \end{equation*}
    nombrando así al lagrangiano de interacción como:
    \begin{equation*}
        \mathcal{L}_{interaction}=-ig A_\mu \phi^* \partial^\mu \phi +ig A^\mu \phi \partial_\mu \phi^* + g^2 A^\mu A_\mu \phi^* \phi
    \end{equation*}
    tomando en cuenta las transformaciones de norma, tales que:
    \begin{equation*}
        \phi \rightarrow e^{i\theta} \phi \qquad A_\mu \rightarrow A_\mu- \frac{1}{g}\partial_\mu \theta
    \end{equation*}
    se tiene que
    \begin{equation*}
        {\mathcal{L}}'_{interaction}=-ig \left(A_\mu- \partial_\mu \theta\right) {\phi}'^* \partial^\mu {\phi}' +ig \left(A^\mu- \partial^\mu \theta\right) {\phi}'^* \partial_\mu {\phi}' + g^2 \left(A^\mu- \frac{1}{g}\partial^\mu \theta\right) \left(A_\mu- \frac{1}{g}\partial_\mu \theta\right) {\phi}'^* {\phi}'
    \end{equation*}
    calculando $-ig \left(A_\mu- \frac{1}{g}\partial_\mu \theta\right) {\phi}'^* \partial^\mu {\phi}'$
    \begin{align*}
        -ig \left(A_\mu- g^{-1}\partial_\mu \theta\right) {\phi}'^* \partial^\mu {\phi}' =& -igA_\mu {\phi}'^* \partial^\mu {\phi}'+i {\phi}'^* \partial^\mu {\phi}' \partial_\mu \theta \\
        =& -ig A_\mu e^{-i\theta} \phi \left(e^{i\theta} \partial^\mu \phi +i \phi e^{i\theta} \partial^\mu \theta \right) +ie^{-i\theta}\phi \left(e^{i\theta} \partial^\mu \phi +i \phi e^{i\theta} \partial^\mu \theta \right) \partial_\mu \theta\\
        =& -igA_\mu\phi^* \left(\partial^\mu \phi + i\phi\partial^\mu \theta\right) + i \phi^* \left(\partial^\mu \phi + i\phi\partial^\mu \theta\right) \partial_\mu \theta\\
    \end{align*}
    teniendo asi:
    \begin{equation*}
        -ig \left(A_\mu- g^{-1}\partial_\mu \theta\right) {\phi}' \partial^\mu {\phi}'^*= 
        -igA_\mu \phi^* \partial^\mu \phi
         + g\phi \phi A_\mu \partial^\mu \theta 
         + i \phi \partial^\mu \phi \partial_\mu \theta^*
         - \phi \phi^* \partial^\mu \theta \partial_\mu \theta
    \end{equation*}
    calculando $ig \left(A^\mu- \frac{1}{g}\partial^\mu \theta\right) {\phi}' \partial_\mu {\phi}'^*$
    \begin{align*}
        ig \left(A^\mu- \frac{1}{g}\partial^\mu \theta\right) {\phi}' \partial_\mu {\phi}'^*=&ig \left(A^\mu- \frac{1}{g}\partial^\mu \theta\right) e^{i\theta}{\phi} \left(e^{-i\theta}\partial_\mu \phi^* - i\phi^* e^{-i\theta} \partial_\mu \theta\right)\\
        =& ig \left(A^\mu- \frac{1}{g}\partial^\mu \theta\right) {\phi}^* \left(\partial_\mu \phi - i \phi \partial_\mu \theta \right)\\
        =&ig A^\mu \phi \left(\partial_\mu \phi^* - i \phi^* \partial_\mu \theta \right) - i\partial^\mu \theta \phi \left(\partial_\mu \phi^* - i\phi^* \partial_\mu \theta \right)
    \end{align*}
    teniendo asi:
    \begin{equation*}
        \mathcal{L}_,=ig \left(A^\mu- \frac{1}{g}\partial^\mu \theta\right) {\phi}'^* \partial_\mu {\phi}' = igA^\mu \phi \partial_\mu \phi^* + g\phi^* \phi A^\mu \partial_\mu \theta -i \phi \partial^\mu \theta\partial_\mu \phi^* - \phi^* \phi \partial^\mu \theta\partial_\mu \theta
    \end{equation*}
    por lo que
    \begin{align*}
        -ig \left(A_\mu- g^{-1}\partial_\mu \theta\right) {\phi}' \partial^\mu {\phi}'^* +ig \left(A^\mu- \frac{1}{g}\partial^\mu \theta\right) {\phi}'^* \partial_\mu {\phi}' &= \\
        &-igA_\mu \phi^* \partial^\mu \phi \\
        &+igA^\mu \phi \partial_\mu \phi^* \\
        &+ g\phi \phi^* A_\mu \partial^\mu \theta\\
        &+ g\phi \phi^* A^\mu \partial_\mu \theta\\
        &-2\phi \phi^* \partial^\mu \theta \partial_\mu \theta
    \end{align*}
    calculando $g^2 \left(A^\mu- \frac{1}{g}\partial^\mu \theta\right) \left(A_\mu- \frac{1}{g}\partial_\mu \theta\right) {\phi}'^* {\phi}'$
    \begin{align*}
        g^2 \left(A^\mu- \frac{1}{g}\partial^\mu \theta\right) \left(A_\mu- \frac{1}{g}\partial_\mu \theta\right) {\phi}'^* {\phi}' =& g^2 \left(A^\mu A_\mu - \frac{1}{g}A^\mu \partial_\mu \theta - \frac{1}{g}A_\mu \partial^\mu \theta + \frac{1}{g^2}\partial^\mu \theta \partial_\mu \theta \right) \phi^* \phi \\
        =&\left(g^2A^\mu A_\mu - gA^\mu \partial_\mu \theta - gA_\mu \partial^\mu \theta + \partial^\mu \theta \partial_\mu \theta \right) \phi^* \phi 
    \end{align*}
    por lo tanto:
    \begin{equation*}
        \mathcal{L}_,+g^2 \left(A^\mu- \frac{1}{g}\partial^\mu \theta\right) \left(A_\mu- \frac{1}{g}\partial_\mu \theta\right) {\phi}'^* {\phi}' = -igA_\mu \phi^* \partial^\mu \phi +igA^\mu \phi \partial_\mu \phi^* +g^2A^\mu A_\mu\phi \phi^*
    \end{equation*}
    por lo que, se puede que observar que la lagrangiana de interacción es invarinate bajo transformaciones de norma.
    \begin{equation*}
        \mathcal{L}_{interaction}={\mathcal{L}}'_{interaction}
    \end{equation*}
    \item Corrientes de Noether.\\
    Se tiene que:
    \begin{equation*}
        \delta \mathcal{L}=0
    \end{equation*}
    entonces:
    \begin{equation*}
        \delta \mathcal{L} = \frac{\partial \mathcal{L}}{\partial \phi} \delta \phi \frac{\partial \mathcal{L}}{\partial \phi^*} \delta \phi^*+ \frac{\partial \mathcal{L}}{\partial (\partial^\mu \phi)} \delta (\partial^\mu \phi)  +\frac{\partial \mathcal{L}}{\partial (\partial^\mu {\phi}^*)} \delta (\partial^\mu {\phi}^*)
    \end{equation*}
    donde \begin{equation*}
        \delta \phi = i\alpha \phi \qquad \delta (\partial^\mu\phi) = i\alpha \partial_\mu  \phi
    \end{equation*}
    \begin{equation*}
        \delta \phi = -i\alpha \phi \qquad \delta (\partial^\mu\phi) = -i\alpha \partial_\mu  \phi
    \end{equation*}
    entonces\begin{align*}
        \delta \mathcal{L} =& i\alpha \frac{\partial \mathcal{L}}{\partial \phi} \phi-i\alpha \frac{\partial \mathcal{L}}{\partial \phi^*} \phi^* + i\alpha \frac{\partial \mathcal{L}}{\partial (\partial^\mu \phi)} \partial_\mu \phi - i\alpha \frac{\partial \mathcal{L}}{\partial (\partial^\mu {\phi}^*)} \partial_\mu {\phi}^*\\
        =& i\alpha \frac{\partial \mathcal{L}}{\partial \phi } \phi -i\alpha \frac{\partial \mathcal{L}}{\partial \phi^* } \phi^* -i\alpha \partial_\mu \left(\frac{\partial \mathcal{L}}{\partial (\partial^\mu \phi)}\right) \phi + i\alpha \partial^\mu \left( \frac{\partial \mathcal{L}}{\partial (\partial^\mu \phi)} \phi \right) - i\alpha\partial^\mu \left( \frac{\partial \mathcal{L}}{\partial (\partial^\mu {\phi}^*)} {\phi}^* \right) \\
        =& i\alpha \left[ \frac{\partial \mathcal{L}}{\partial \phi} - \partial_\mu \left(\frac{\partial \mathcal{L}}{\partial (\partial^\mu \phi)}\right) \right] \phi-i\alpha \left[ \frac{\partial \mathcal{L}}{\partial \phi^*} - \partial_\mu \left(\frac{\partial \mathcal{L}}{\partial (\partial^\mu \phi^*)}\right) \right] \phi^*  + i\alpha \partial^\mu \left[ \frac{\partial \mathcal{L}}{\partial (\partial^\mu \phi)} \phi  - \frac{\partial \mathcal{L}}{\partial (\partial^\mu {\phi}^*)} {\phi}^* \right]\\
        =& \partial^\mu \left(i\alpha\left[ \frac{\partial \mathcal{L}}{\partial (\partial^\mu \phi)} \phi  - \frac{\partial \mathcal{L}}{\partial (\partial^\mu {\phi}^*)} {\phi}^* \right]\right)
    \end{align*}
    con lo cual obtenemos que:
    \begin{equation*}
        \partial^\mu \left(i\alpha\left[ \frac{\partial \mathcal{L}}{\partial (\partial^\mu \phi)} \phi  - \frac{\partial \mathcal{L}}{\partial (\partial^\mu {\phi}^*)} {\phi}^* \right]\right) = 0
    \end{equation*}
    donde 
     \begin{equation*}
         j_\mu = i\alpha\left[ \frac{\partial \mathcal{L}}{\partial (\partial^\mu \phi)} \phi  - \frac{\partial \mathcal{L}}{\partial (\partial^\mu {\phi}^*)} {\phi}^* \right]
     \end{equation*}
     calculando las derivadas parciales se tiene que:
     \begin{equation*}
        \frac{\partial \mathcal{L}}{\partial (\partial^\mu \phi)} = -igA_\mu \phi^*  \qquad \frac{\partial \mathcal{L}}{\partial (\partial^\mu \phi^*)} = i g A^\mu \phi
     \end{equation*}
     por lo tanto:
     \begin{equation*}
         j_\mu = g\alpha \left[A_\mu +A^\mu \right] \phi^*\phi
     \end{equation*}
\end{itemize}