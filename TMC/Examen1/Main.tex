\documentclass[12pt,letterpaper]{article}
\usepackage{scrextend}
\usepackage{vmargin}
\usepackage{graphicx}
\usepackage{multirow}
\usepackage[utf8]{inputenc}
\usepackage[spanish]{babel}
\usepackage{multicol}
\usepackage{enumerate}
\usepackage{float}
\usepackage{amsmath, amsthm, amssymb, amsfonts}
\usepackage[usenames]{color}
\parindent=0mm
\pagestyle{empty}
\definecolor{miorange}{rgb}{0.91, 0.43, 0.0}
\begin{document}
\setmargins{2.5cm}      
{1.5cm}                     
{2cm}  
{24cm}                    
{10pt}                          
{1cm}                          
{0pt}                             
{2cm}
\begin{titlepage}
\begin{center}
\includegraphics[scale=0.40]{../../Logos/uanl.png} 
\hspace{2.5cm}
\includegraphics[scale=0.40]{../../Logos/fcfm.png}
\end{center}
\vspace{2cm}
\begin{center}
\textbf{
UNIVERSIDAD AUTÓNOMA DE NUEVO LEÓN\\
FACULTAD DE CIENCIAS
    FÍSICO MATEMÁTICAS}\\
\vspace*{2cm}
\begin{large}
\vspace{1cm}
\large{\textbf{Tópicos de Mécanica Cuántica}}\\
\textbf{Examen primer parcial}\\
Enrique Valbuena Ordonez\\
\end{large}
\vspace{3.5cm}
\begin{minipage}{0.6\linewidth}
\vspace{0.5cm}
\changefontsizes{14pt}
Nombre:\\
Giovanni Gamaliel López Padilla\\
\end{minipage}
\begin{minipage}{0.2\linewidth}
\changefontsizes{14pt}
Matricula:\\
1837522
\end{minipage}
\end{center}
\vspace{4cm}
\begin{flushright}
\today
\end{flushright}
\end{titlepage}
\section*{Calcular los siguientes conmutadores e interpretar físicamente el resultado}
\begin{enumerate}
    \item[1)] $[\hat{H},\hat{x}]$\\
    Se sabe que el conmutador es igual a:
    \begin{equation}
        \label{eq:op_Hx}
        [\hat{H},\hat{x}]=(\hat{H}\hat{x}-\hat{x}\hat{H})\psi
    \end{equation}
    donde \\
    \begin{minipage}{0.5\linewidth}
        \begin{equation}
            \label{eq:op_H}
            \hat{H} = \frac{\hat{P}^2}{2m} +V(\vec{x})
        \end{equation}
    \end{minipage}
    \begin{minipage}{0.5\linewidth}
        \begin{equation}
            \label{eq:op_X}
            \hat{x} = x
        \end{equation}
    \end{minipage}
    calculando la parte izquierda de la ecuación \ref{eq:op_Hx} con \ref{eq:op_H} y \ref{eq:op_X}, se tiene que:
    \begin{align*}
        \hat{H}\hat{x}\psi &= \left(\frac{\hat{P}^2}{2m} +V(\vec{r}) \right) (x\psi)\\ 
        &= \frac{\hat{P}^2 }{2m} x\psi + V(\vec{r})x\psi \\
        &= \left(\frac{\hat{P}^2_x+\hat{P}^2_y+\hat{P}^2_z }{2m}\right) x\psi + V(\vec{r})x\psi \\
        &= -\frac{\hbar^2}{2m} \left(\nabla^2(x\psi)  \right)+ V(\vec{r})x\psi \\
        &= -\frac{\hbar^2}{2m} \left(x\nabla^2(\psi) + 2 \nabla \psi \right)+ V(\vec{r})x\psi \\
    \end{align*}
    por lo tanto:
    \begin{equation}
        \label{eq:Hx_izq}
        \hat{H}\hat{x}\psi =  -\frac{\hbar^2}{2m} \left(x\nabla^2(\psi) + 2 \nabla \psi \right)+ V(\vec{r})x\psi
    \end{equation}
    calculando la parte derecha de la ecuación \ref{eq:op_Hx} con \ref{eq:op_H} y \ref{eq:op_X}, se tiene que:
    \begin{align}
        \label{eq:Hx_der}
        \hat{x}\hat{H}\psi &= x\left(\frac{\hat{P}^2}{2m} +V(\vec{r}) \right) (\psi)\\ 
        &= x\left( \frac{\hat{P}^2 }{2m} \psi + V(\vec{r})\psi \right) \\
        &= x\left(\frac{\hat{P}^2_x+\hat{P}^2_y+\hat{P}^2_z }{2m}\right) \psi + xV(\vec{r})\psi \\
        &= -\frac{\hbar^2}{2m} x\left(\nabla^2(\psi)\right)+ xV(\vec{r})\psi \\
    \end{align}
    juntando por lo tanto la ecuación \ref{eq:op_Hx} es:
    \begin{align*}
        [\hat{H},\hat{x}]&=(\hat{H}\hat{x}-\hat{x}\hat{H})\psi \\
        &=-\frac{\hbar^2}{2m} \left(x\nabla^2(\psi) + 2 \nabla \psi \right)+ V(\vec{r})x\psi+\frac{\hbar^2}{2m} x\left(\nabla^2(\psi)\right)- xV(\vec{r})\psi \\
        &=-\frac{\hbar^2}{m} \nabla \psi
    \end{align*}
    por lo tanto:
    \begin{equation*}
        [\hat{H},\hat{x}]=-\frac{\hbar^2}{m} \nabla \psi
    \end{equation*}
    La interpretación física es que no se puede realizar a la energía y posición de la partícula simultaneamente, ya que la medición que no se realice su valor se vera afectado.
    \item[4)] $[\hat{x},\hat{P}_x]$\\
    Se sabe que el conmutador es igual a:
    \begin{equation}
        \label{eq:conmu_xpx}
        [\hat{x},\hat{P}_x]= (\hat{x}\hat{P}-\hat{x}\hat{P}_x)\psi
    \end{equation}
    calculando la parte izquierda de la ecuación \ref{eq:conmu_xpx}, se tiene que:
    \begin{align*}
        \hat{x}\hat{P}_x &= x\left( - i\hbar \frac{\partial}{\partial x} \psi\right)\\
        &= -i \hbar x \frac{\partial}{\partial x} \psi
    \end{align*}
    por lo que: 
    \begin{equation}
        \label{eq:xpx_izq}
        \hat{x}\hat{P}_x= -i \hbar x \frac{\partial}{\partial x} \psi
    \end{equation}
    calculando la parte derecha de la ecuación \ref{eq:conmu_xpx}, se tiene que:
    \begin{align*}
        \hat{P}_x\hat{x} &= - i\hbar \frac{\partial}{\partial x} (x\psi)\\
        &= -i \hbar x \frac{\partial}{\partial x} \psi - i\hbar \psi
    \end{align*}
    por lo que:
    \begin{equation}
        \label{eq:xpx_der}
        \hat{P}_x\hat{x} = -i \hbar x \frac{\partial}{\partial x} \psi - i\hbar \psi
    \end{equation}
    por lo que juntando la ecuación \ref{eq:xpx_izq} y \ref{eq:xpx_der} en la ecuación \ref{eq:conmu_xpx}, lo cual se obtiene que:
    \begin{align*}
        [\hat{x},\hat{P}_x] &= i\hbar \psi + i \hbar x \frac{\partial}{\partial x} \psi -i \hbar x \frac{\partial}{\partial x} \psi\\
        &= i\hbar \psi
    \end{align*}
    por lo tanto:
    \begin{equation}
        [\hat{x},\hat{P}_x] = i\hbar
    \end{equation}
    La interpretación física que le daria es que cuando sucede alguna alteración la posición o al momento lineal en x inmediatamente sucede un cambio en el estado no afectado inicialmente.
    \item[7)] $[\hat{L_y},\hat{L_z}]$ \\
    Se sabe que 
    \begin{equation}
        \label{eq:conm_lylz}
        [\hat{L_y},\hat{L_z}]=(\hat{L_y}\hat{L_z}-\hat{L_z}\hat{L_y})
    \end{equation}
    \begin{align}
        \label{ly}
        L_y&= \frac{\hbar}{i} \left( z\frac{\partial}{\partial x}- x \frac{\partial}{\partial z}\right)\\
        L_z&= \frac{\hbar}{i} \left( x\frac{\partial}{\partial y}- y \frac{\partial}{\partial x}\right)
        \label{lz}
    \end{align}
    Calculando la parte izquierda de la ecuación \ref{eq:conm_lylz} con \ref{ly} y \ref{lz}, se tiene que:
    \begin{align*}
        \hat{L}_y\hat{L}_z&=\hbar^2 \left(z\frac{\partial}{\partial x} \left(x\frac{\partial}{\partial y}\right)-
        z\frac{\partial}{\partial x} \left(y\frac{\partial}{\partial x}\right)  -
        x\frac{\partial}{\partial z} \left(x\frac{\partial}{\partial y}\right) +
        x\frac{\partial}{\partial z} \left(y\frac{\partial}{\partial x}\right) \right)\\
        &=-\hbar^2 \left(z \frac{\partial}{\partial z}\frac{\partial}{\partial x}x+zx \frac{\partial}{\partial y}\frac{\partial}{\partial x}
        -zy \frac{\partial}{\partial x}\frac{\partial}{\partial x}-
        z \frac{\partial}{\partial x}\frac{\partial}{\partial x}y\right.\\
        &\left. -x^2 \frac{\partial}{\partial z}\frac{\partial}{\partial y}
        -x \frac{\partial}{\partial y}\frac{\partial}{\partial z}x 
        +xy \frac{\partial}{\partial z}\frac{\partial}{\partial x}
        +x \frac{\partial}{\partial z}\frac{\partial}{\partial x}y\right)\\
        &=-\hbar^2 \left(z\frac{\partial}{\partial y}+zx \frac{\partial^2}{ \partial x\partial y}-
        +zy \frac{\partial^2}{ \partial x^2}-x^2 \frac{\partial^2}{ \partial z\partial y}+xy \frac{\partial^2}{ \partial z\partial x} \right)
    \end{align*}
    por lo tanto
    \begin{equation}
        \label{eq:conm_lylz_izq}
        \hat{L}_y\hat{L}_z=-\hbar^2 \left(z\frac{\partial}{\partial y}+zx \frac{\partial^2}{ \partial x\partial y}-
        +zy \frac{\partial^2}{ \partial x^2}-x^2 \frac{\partial^2}{ \partial z\partial y}+xy \frac{\partial^2}{ \partial z\partial x} \right)
    \end{equation}
    Calculando la parte izquierda de la ecuación \ref{eq:conm_lylz} con \ref{ly} y \ref{lz}, se tiene que:
    \begin{align*}
        \hat{L}_y\hat{L}_z&=\hbar^2 \left(x\frac{\partial}{\partial y} \left(z\frac{\partial}{\partial x}\right)-
        y\frac{\partial}{\partial x} \left(z\frac{\partial}{\partial x}\right)  -
        x\frac{\partial}{\partial y} \left(x\frac{\partial}{\partial z}\right) +
        y\frac{\partial}{\partial x} \left(x\frac{\partial}{\partial z}\right) \right)\\
        &=-\hbar^2 \left(xz \frac{\partial}{\partial y}\frac{\partial}{\partial x}-x^2 \frac{\partial}{\partial y}\frac{\partial}{\partial z}
        -zy \frac{\partial}{\partial x}\frac{\partial}{\partial x}-
        yx \frac{\partial}{\partial x}\frac{\partial}{\partial x}+y\frac{\partial}{\partial z}\right)\\
    \end{align*}
    por lo tanto:
    \begin{equation}
        \label{eq:conmu_lylz_der}
        \hat{L}_y\hat{L}_z=-\hbar^2 \left(xz \frac{\partial}{\partial y}\frac{\partial}{\partial x}-x^2 \frac{\partial}{\partial y}\frac{\partial}{\partial z}
        -zy \frac{\partial}{\partial x}\frac{\partial}{\partial x}-
        yx \frac{\partial}{\partial x}\frac{\partial}{\partial x}+y\frac{\partial}{\partial z}\right)
    \end{equation}
    sustitutendo las ecuaciones \ref{eq:conm_lylz_izq} y \ref{eq:conmu_lylz_der} en \ref{eq:conm_lylz}, se obtiene lo siguiente:
    \begin{align*}
        [\hat{L_y},\hat{L_z}]&=-\hbar^2 \left( z\frac{\partial}{\partial y}- y\frac{\partial}{\partial z}\right)\\
        &=i\hbar L_x
    \end{align*}
    por lo tanto:
    \begin{equation}
        [\hat{L_y},\hat{L_z}]=i\hbar L_x
    \end{equation}
    La interpretación física de esto, es que los estados del momento angular x,y y z estan relacionados de forma en que si uno cambia, los otros lo haran.
\end{enumerate}
\pagebreak
\section*{Demostrar que los siguientes estados satisfacen la ecuación de Schr\"odinger}
\begin{itemize}
    \item $\hat{P}_x \psi$
\end{itemize}
Sea \begin{equation}
    \label{eq:psip}
    \psi ' = \hat{P}_x \psi
\end{equation}
sustituyendo esta función en la ecuación de Schr\"odinger se obtiene lo siguiente:
\begin{align*}
    -\frac{\hbar^2}{2m} \nabla^2 {\psi}' + V(r) {\psi}' &= i\hbar\frac{\partial}{\partial t} {\psi}'\\
    - \frac{\hbar^2}{2m} \left(\frac{\partial^2}{\partial x^2} +\frac{\partial^2}{\partial y^2} +\frac{\partial^2}{\partial z^2}  \right) {\psi}' +V(r){\psi'} &= i\hbar\frac{\partial}{\partial t} {\psi}'\\
    - \frac{\hbar^2}{2m} \left(\frac{\partial^2}{\partial x^2} +\frac{\partial^2}{\partial y^2} +\frac{\partial^2}{\partial z^2}  \right) \left( \frac{\hbar}{i} \frac{\partial}{\partial x} \psi\right) +V(r)\left( \frac{\hbar}{i} \frac{\partial}{\partial x} \psi\right)&= i\hbar\frac{\partial}{\partial t} \left( \frac{\hbar}{i} \frac{\partial}{\partial x} \psi\right)\\
    \left(\frac{\hbar}{i}  \right) \left[- \frac{\hbar^2}{2m} \left(\frac{\partial^2}{\partial x^2} +\frac{\partial^2}{\partial y^2} +\frac{\partial^2}{\partial z^2}  \right) \left(\frac{\partial}{\partial x} \psi\right) +\left(\frac{\partial}{\partial x} \psi\right)V(r)\right]&= \left(\frac{\hbar}{i} \right)i\hbar\frac{\partial}{\partial t} \left(\frac{\partial}{\partial x} \psi\right)\\
    - \frac{\hbar^2}{2m} \left(\frac{\partial^2}{\partial x^2}\frac{\partial}{\partial x} +\frac{\partial^2}{\partial y^2}\frac{\partial}{\partial x} +\frac{\partial^2}{\partial z^2}\frac{\partial}{\partial x}  \right) \left(\psi\right) +\left(\frac{\partial}{\partial x} \psi\right)V(r)&=i\hbar\frac{\partial}{\partial t}\frac{\partial}{\partial x} \psi\\
    - \frac{\hbar^2}{2m} \left(\frac{\partial}{\partial x}\frac{\partial^2}{\partial x^2} +\frac{\partial}{\partial x}\frac{\partial^2}{\partial y^2} +\frac{\partial}{\partial x}\frac{\partial^2}{\partial z^2} \right) \left(\psi\right) +\left(\frac{\partial}{\partial x} \psi\right)V(r)&=i\hbar\frac{\partial}{\partial x}\frac{\partial}{\partial t} \psi\\
    \left[\frac{\partial}{\partial x} \right]\left(- \frac{\hbar^2}{2m} \left(\frac{\partial^2}{\partial x^2} +\frac{\partial^2}{\partial y^2} +\frac{\partial^2}{\partial z^2} \right) \left(\psi\right) +\psi V(r)\right)&=\left(\frac{\partial}{\partial x}\right) i\hbar\frac{\partial}{\partial t} \psi\\ 
    \left[\frac{\partial}{\partial x} \right]\left(- \frac{\hbar^2}{2m} \nabla^2 \psi +V(r)\psi \right)&=\left[\frac{\partial}{\partial x}\right]\left(i\hbar\frac{\partial}{\partial t} \psi \right)\\
\end{align*}
y como la función $\psi$ es solución a la ecuación de Schr\"odinger, entonces la función ${\psi}'$ definida en \ref{eq:psip} es solución a la ecuación de Schr\"odinger
\pagebreak
\section*{Demostrar que las siguientes transformaciones discretas dejan invariante a la ecuación de
Schrödinger en 1D y especificar qué condiciones son necesarias para que esto ocurra; usar una
solución separable}
\begin{itemize}
    \item $\Pi \psi(t) = \psi(-t)$
\end{itemize}
Sea el operador de paridad
\begin{equation}
    \Pi\psi(t) = \psi(-t)
    \label{parity}
\end{equation}
Para demostrar que se deja invariante a la ecuación de Schr\"odinger se tiene que comprobar que:
\begin{equation}
    [\Pi,\hat{H}]=0
    \label{conmu}
\end{equation}
ya que con esto podemos decir que el operador paridad es constante del movimiento, de modo que si en el instante inical el estado de la partícula tiene paridad, en cualquier instante posterior seguirá teniendo la misma paridad.\\
Calculando la ecuación \ref{conmu} tomando en cuenta que la funció de onda es separable se tiene que:
\begin{equation*}
[\Pi,\hat{H}]= \Pi\hat{H} \psi(x,t) - \hat{H}\Pi \psi(x,t)
\end{equation*}
La parte izquierda es igual a:
\begin{align*}
\hat{H}\Pi \psi(x,t)    &= \hat{H}\Pi\psi(x)\psi(t)\\
                        &= \hat{H}\psi(x)\Pi\psi(t)\\
                        &= \hat{H}\psi(x)\psi(-t)\\
                        &-\frac{\hbar}{2m}\frac{\partial^2}{\partial x^2} \psi(x)\psi(-t) +V(x)\psi(x)\psi(-t)\\
                        &-\frac{\hbar}{2m}\psi(-t)\frac{\partial^2}{\partial x^2} \psi(x) +\psi(-t)V(x)\psi(x)
\end{align*}
\begin{equation}
\hat{H}\Pi \psi(x,t)= -\frac{\hbar}{2m}\psi(-t)\frac{\partial^2}{\partial x^2} \psi(x) +\psi(-t)V(x)\psi(x)
\label{izq}
\end{equation}
La parte derecha es igual a:
\begin{align*}
\Pi\hat{H}\psi(x,t) & =\Pi\left(-\frac{\hbar}{2m}\frac{\partial^2}{\partial x^2} \psi(x,t) +V(x)\psi(x,t)\right)\\
                    &=\Pi\left(-\frac{\hbar}{2m}\frac{\partial^2}{\partial x^2} \psi(x)\psi(t) +V(x)\psi(x)\psi(t)\right)\\
                    &=\Pi\left(-\frac{\hbar}{2m}\psi(t)\frac{\partial^2}{\partial x^2} \psi(x) + \psi(t)V(x)\psi(x)\right)\\
                    &=-\frac{\hbar}{2m}\Pi\psi(t)\frac{\partial^2}{\partial x^2} \psi(x) + \Pi\psi(t)V(x)\psi(x)\\
                    &=-\frac{\hbar}{2m}\psi(-t)\frac{\partial^2}{\partial x^2} \psi(x) + \psi(-t)V(x)\psi(x)\\
    \end{align*}
\begin{equation}
\Pi\hat{H} \psi(x,t)=-\frac{\hbar}{2m}\psi(-t)\frac{\partial^2}{\partial x^2} \psi(x) + \psi(-t)V(x)\psi(x)\\
\label{der}
\end{equation}
como la ecuacion \ref{der} y \ref{izq} son lo mismo, entonces la ecuación \ref{conmu} es cierta, por lo tanto el operador paridad deja invariante a la ecuación de Schr\"odinger.\\
Las condiciones encontradas para que esto se cumpla es que la transformación de paridad sea una transformación lineal y que afecte unicamente al tiempo, ya que si induce algun cambio en el espacio esta sería diferente, otra condición es que este solamente puede dejar invariante ante potenciales que tiene paridad en el tiempo, o en su caso no dependa de esta variable.
\end{document}