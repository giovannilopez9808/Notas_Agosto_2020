\section*{3. Calcular la lagrangiana de interacción de Schr\"odinger en términos del cuadrivector de potencial.}
Se tiene que la lagrangiana de Schro\"odinger es:
\begin{equation}
    \changefontsizes{11pt}
    \mathcal{L}\left(\psi,\psi^*,\frac{\partial \psi}{\partial x},\frac{\partial \psi^*}{\partial x},\frac{\partial \psi}{\partial t},\frac{\partial \psi^*}{\partial t},x\right) = -\frac{\hbar^2}{2m} \frac{\partial \psi}{\partial x} \frac{\partial \psi^*}{\partial x}+i\hbar \left[ \psi^* \frac{\partial \psi}{\partial t} - \psi \frac{\partial \psi^*}{\partial t} \right] -V(x)\psi \psi^*.
    \label{eq:den_lan}
\end{equation}
y tomando en consideración que:
\begin{align*}
    \partial_t &\rightarrow\partial_t -igA_0\\
    \partial_t^* &\rightarrow\partial_t +igA_0\\
    {\nabla}' & \rightarrow \nabla - ig\vec{A}\\
    {\nabla}'^* & \rightarrow \nabla + ig\vec{A}
\end{align*}
Esto para obtener los terminos de interacción dentro de la lagrangiana, introduciendo estos terminos en la ecuación \ref{eq:den_lan}, se obtiene lo siguiente:
\begin{align*}
    &\mathcal{L} = -\frac{\hbar^2}{2m}\left[ {\nabla}'\psi \cdot {\nabla}' \psi^*\right] + i\hbar \left[\psi^* {\partial}'_t \psi - \psi {\partial}'_t \psi^*\right]\\
    &= -\frac{\hbar^2}{2m}\left[ \left(\nabla - ig\vec{A}\right)\psi \cdot \left( \nabla + ig\vec{A}\right) \psi^*\right] + i\hbar \left[\psi^* \left(\partial_t -igA_0\right) \psi - \psi \left(\partial_t +igA_0\right) \psi^*\right]\\
    &=- \frac{\hbar^2}{2m} \nabla \psi \cdot \nabla \psi^* -\frac{\hbar^2}{2m}\left[\nabla \psi \cdot ig\vec{A} \psi^* - ig \vec{A}\psi \cdot \nabla \psi^* + g^2 \vec{A}\psi \cdot \vec{A} \psi^* \right]\\
    &+ i\hbar \left[\psi^*\partial_t \psi - \psi \partial_t \psi^* -2igA_0 \psi \psi^* \right]\\
    &=- \frac{\hbar^2}{2m} \nabla \psi \cdot \nabla \psi^* + i\hbar \left[\psi^*\partial_t \psi - \psi \partial_t \psi^* \right]\\
    &-\frac{\hbar^2}{2m}\left[\nabla \psi \cdot ig\vec{A} \psi^* - ig \vec{A}\psi \cdot \nabla \psi^* + g^2 \vec{A}\psi \cdot \vec{A} \psi^* \right]+ 2\hbar gA_0 \psi \psi^*\\
    & = \mathcal{L}_f -\frac{\hbar^2g}{2m}\left[\nabla \psi \cdot i\vec{A} \psi^* - i\vec{A}\psi \cdot \nabla \psi^* + g \vec{A}\psi \cdot \vec{A} \psi^* \right]+ 2\hbar gA_0 \psi \psi^*\\
    &= \mathcal{L}_f - \frac{\hbar^2}{2m}igA \cdot \left[\psi^* \nabla \psi - \psi \nabla \psi^* + \frac{1}{i}Ag \psi \psi^*\right] +2 g \hbar  A_0 \psi \psi^*\\
    &= \mathcal{L}_f + \hbar A \cdot \left[\frac{\hbar}{2mi}g \left(\psi^* \nabla \psi - \psi \nabla \psi^* \right)+ \frac{\hbar}{2m} g^2 A |\psi|^2 \right] + \hbar g A_0 |\psi|^2\\
    &= \mathcal{L}_f + g\hbar J^\alpha A_\alpha
\end{align*}
por lo tanto la lagrangiana de interacción de Schr\"odinger en términos del cuadrivector del potencial es:
\begin{equation}
    \changefontsizes{10pt}
    L_A=g\hbar J^\alpha A_\alpha
    \label{eq:den_lag_a}
\end{equation}
donde:
\begin{equation*}
    J^\alpha = \frac{\hbar}{2mi} \left[\left(\psi^* \nabla \psi - \psi \nabla \psi^* \right)-gA^\alpha |\psi|^2\right]
\end{equation*}
Agregando el termino del lagrangiano de interacción con el campo electromagnético, el lagrangiano de interacción se transforma en lo siguiente:
\begin{equation}
    \mathcal{L}_i= g\hbar J^\alpha A_\alpha + \square A^\mu A_\mu - \frac{1}{4\mu_0} F_{\mu \nu}F^{\mu \nu}
\end{equation}