\documentclass[12pt,letterpaper]{report}
\usepackage{graphicx}
\usepackage{scrextend}
\usepackage{vmargin}
\usepackage{graphicx}
\usepackage{multirow}
\usepackage[utf8]{inputenc}
\usepackage[spanish]{babel}
\usepackage{multicol}
\usepackage{enumerate}
\usepackage{float}
\usepackage{amsmath, amsthm, amssymb, amsfonts}
\usepackage[usenames]{color}
\parindent=0mm
\pagestyle{empty}
\definecolor{miorange}{rgb}{0.91, 0.43, 0.0}
\begin{document}
\setmargins{2.5cm}      
{1.5cm}                     
{2cm}  
{24cm}                    
{10pt}                          
{1cm}                          
{0pt}                             
{2cm}
\begin{titlepage}
\begin{center}
\includegraphics[scale=0.40]{../../Logos/uanl.png} 
\hspace{2.5cm}
\includegraphics[scale=0.40]{../../Logos/fcfm.png}
\end{center}
\vspace{2cm}
\begin{center}
\textbf{
UNIVERSIDAD AUTÓNOMA DE NUEVO LEÓN\\
FACULTAD DE CIENCIAS
    FÍSICO MATEMÁTICAS}\\
\vspace*{2cm}
\begin{large}
\vspace{1cm}
\large{\textbf{Tópicos de Mécanica Cuántica}}\\
\textbf{Tarea 7}\\
Enrique Valbuena Ordonez\\
\end{large}
\vspace{3.5cm}
\begin{minipage}{0.6\linewidth}
\vspace{0.5cm}
\changefontsizes{14pt}
Nombre:\\
Giovanni Gamaliel López Padilla\\
\end{minipage}
\begin{minipage}{0.2\linewidth}
\changefontsizes{14pt}
Matricula:\\
1837522
\end{minipage}
\end{center}
\vspace{4cm}
\begin{flushright}
\today
\end{flushright}
\end{titlepage}
Un espacio de Hilber $\mathcal{H}$ es un conjunto de elementos vectoriales y escalares
que satisfacen la siguientes propiedades:
\begin{enumerate}
    \item $\mathcal{H}$ es un espacio lineal.
    \item $\mathcal{H}$ tiene un producto escalar definido que es estrictamente positivo.\\
    El producto escalar de un elemento $\varphi$ con otro elemento $\phi$ es en general un número complejo, denotado producto
    $(\varphi,\phi)$. El producto escalar satisface las siguientes propiedades:
    \begin{itemize}
        \item El producto escalar de $\varphi$ con $\phi$ es igual al complejo conjugado de $\phi$ por $\varphi$
        \begin{equation*}
            (\varphi,\phi)=(\phi,\varphi)^*
        \end{equation*}
        \item El producto escalar de $\phi$ con $\varphi$ es lineal con respecto al segundo factor si $\varphi=a\varphi_1+b\varphi_2$
        \begin{equation*}
            (\phi,a\varphi_1+b\varphi_2) =a (\phi,\varphi_1)+b(\phi,\varphi_2)
        \end{equation*}
        y antilineal con respecto al primer factor si $\phi=a\phi_1+b\phi_2$
        \begin{equation*}
            (\phi=a\phi_1+b\phi_2)=a^*(\phi,\varphi)+b^* (\phi,\varphi)
        \end{equation*}
        \item El producto escalar de un vector $\varphi$ consigo mismo es un número real y positivo.
        \begin{equation*}
            (\varphi,\varphi)=|\varphi| \geq 0
                \end{equation*}
        donde la igualdad sostiene sólo para $\varphi=0$.
    \end{itemize}
    \item $\mathcal{H}$ es separable.\\
    Existe una secuencia Cauchy $\varphi_n \epsilon \mathcal{H}(n=1,2,\cdots)$ tal que para cada $\varphi$ de $\mathcal{H}$ y $\epsilon > 0$, existe
    al menos una $\varphi_n$ para la cual:
    \begin{equation*}
        |\varphi-\varphi|<\epsilon
    \end{equation*}
    \item $\mathcal{H}$ es completo.\\
    Toda secuencia de Cauchy $\varphi_n \epsilon \mathcal{H}$ converge a un elemento de $\mathcal{H}$. Esto es, para cualquier $\varphi_n$, la relación 
    \begin{equation*}
        \lim_{n,m\rightarrow \infty} |\varphi_m-\varphi_n|=0,
    \end{equation*}
    define un límite único $\varphi$ de $\mathcal{H}$ tal que
    \begin{equation*}
        \lim_{n\rightarrow \infty} |\varphi-\varphi_n|=0
    \end{equation*}
\end{enumerate}
\end{document}