\documentclass[12pt,letterpaper]{report}
\usepackage{graphicx}
\usepackage{scrextend}
\usepackage{vmargin}
\usepackage{graphicx}
\usepackage{multirow}
\usepackage[utf8]{inputenc}
\usepackage[spanish]{babel}
\usepackage{multicol}
\usepackage{enumerate}
\usepackage{float}
\usepackage{amsmath, amsthm, amssymb, amsfonts}
\usepackage[usenames]{color}
\parindent=0mm
\pagestyle{empty}
\definecolor{miorange}{rgb}{0.91, 0.43, 0.0}
\begin{document}
\setmargins{2.5cm}      
{1.5cm}                     
{2cm}  
{24cm}                    
{10pt}                          
{1cm}                          
{0pt}                             
{2cm}
\begin{titlepage}
\begin{center}
\includegraphics[scale=0.40]{../../Logos/uanl.png} 
\hspace{2.5cm}
\includegraphics[scale=0.40]{../../Logos/fcfm.png}
\end{center}
\vspace{2cm}
\begin{center}
\textbf{
UNIVERSIDAD AUTÓNOMA DE NUEVO LEÓN\\
FACULTAD DE CIENCIAS
    FÍSICO MATEMÁTICAS}\\
\vspace*{2cm}
\begin{large}
\vspace{1cm}
\large{\textbf{Tópicos de Mécanica Cuántica}}\\
\textbf{Tarea 2}\\
Enrique Valbuena Ordonez\\
\end{large}
\vspace{3.5cm}
\begin{minipage}{0.6\linewidth}
\vspace{0.5cm}
\changefontsizes{14pt}
Nombre:\\
Giovanni Gamaliel López Padilla\\
\end{minipage}
\begin{minipage}{0.2\linewidth}
\changefontsizes{14pt}
Matricula:\\
1837522
\end{minipage}
\end{center}
\vspace{4cm}
\begin{flushright}
\today
\end{flushright}
\end{titlepage}
\subsection*{Demostrar que la función de onda para la particula libre sigue siendo solución a pesar de que le hayamos aplicado el operador P$_z$}
Se sabe que la función de onda para la partícula libre es
\begin{equation}
    \psi(\vec{r},t)=Ae^{i(\vec{k}\cdot\vec{r}-\omega t)}
    \label{psi}
\end{equation}
y el operador P$_z$ es
\begin{equation}
    P_z= \frac{\hbar}{i} \frac{\partial}{\partial z}
    \label{pz}
\end{equation}
Aplicando \ref{pz} en \ref{psi} se obtiene lo siguiente:
\begin{align*}
\psi'(\vec{r},t)&= \frac{\hbar}{i} A  \frac{\partial}{\partial z} e^{i(\vec{k}\cdot\vec{r}-\omega t)}\\
                &=\frac{A\hbar}{i} (iK_z) e^{i(\vec{k}\cdot\vec{r}-\omega t)}\\
                &= A\hbar K_z e^{i(\vec{k}\cdot\vec{r}-\omega t)}
\end{align*}
por lo tanto, la función de onda es la siguiente:
\begin{equation}
\psi'(\vec{r},t)=A\hbar K_z e^{i(\vec{k}\cdot\vec{r}-\omega t)}
\label{psi2}
\end{equation}
para comprobar que esta función sigue siendo solucion a la ecuación de Schr\"odinger tenemos que sustituirla en la siguiente:
\begin{equation}
    \label{eqschr}
    -\frac{\hbar^2}{2m} \nabla^2 \psi' = i\hbar \frac{\partial}{\partial t} \psi'
\end{equation}
sustituyendo \ref{psi2} en \ref{eqschr}:\\
calculando la parte izquierda:
\begin{align*}
    -\frac{\hbar^2}{2m} \nabla^2 \psi'  &= -\frac{\hbar^2}{2m} \left(\frac{\partial^2}{\partial x^2} +\frac{\partial^2}{\partial y^2}+\frac{\partial^2}{\partial z^2}\right)A\hbar K_z e^{i(\vec{k}\cdot \vec{r}-\omega t)}\\
                                        &= \frac{\hbar^3 K_z A}{2m}  (K_x^2+K_y^2+K_z^2) e^{i(\vec{k}\cdot \vec{r}-\omega t)}\\
                                        &= \frac{\hbar^3 K_z A}{2m} K^2 e^{i(\vec{k}\cdot \vec{r}-\omega t)}\\
                                        &= E A\hbar e^{i(\vec{k}\cdot \vec{r}-\omega t)}\\
\end{align*}
por lo tanto:
\begin{equation}
    -\frac{\hbar^2}{2m} \nabla^2 \psi'=E A\hbar e^{i(\vec{k}\cdot \vec{r}-\omega t)}
    \label{izq}
\end{equation}
calculando la parte derecha: 
\begin{align*}
i\hbar \frac{\partial }{\partial t} \psi'   &= i\hbar^2 AK_z  \frac{\partial }{\partial t}e^{i(\vec{k}\cdot \vec{r}-\omega t)}\\
                                            &= \hbar^2 A K_z \omega e^{i(\vec{k}\cdot \vec{r}-\omega t)}\\
                                            &= E A\hbar e^{i(\vec{k}\cdot \vec{r}-\omega t)}\\
\end{align*}
por lo tanto
\begin{equation}
    i\hbar \frac{\partial }{\partial t} \psi' =  E A\hbar e^{i(\vec{k}\cdot \vec{r}-\omega t)}
    \label{der}
\end{equation}
como \ref{izq} y \ref{der} son iguales, la funcion de onda $\psi'$ cumple la ecuación de Schr\"odinger
\end{document}