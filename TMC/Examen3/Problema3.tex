\textbf{Obtener la expresión cuántica para el campo electromagnético a partir de la lagrangiana
electromagnética clásica.}\\
Se tiene que la lagrangiana del electromagnétismo clásico es
\begin{equation*}
    \mathcal{L}=-\frac{1}{4\mu_0}F_{\mu\nu}F^{\mu\nu}
\end{equation*}
donde
\begin{equation*}
    F^{\mu\nu}=F_{\mu\nu} = \left(\begin{matrix}
        0 & -\frac{Ex}{c} & -\frac{Ey}{c}   & -\frac{Ez}{c} \\
        \frac{E_x}{c} &  0  & -B_z & B_y \\
        \frac{E_y}{c} & B_z & 0 &-B_x \\
        \frac{E_z}{c} & -B_y & B_x & 0 
    \end{matrix}\right)
\end{equation*}
por lo que el lagrangiano puede ser descrito de la siguiente manera:
\begin{equation*}
    \mathcal{L}=\frac{1}{2\mu_0}\left(\frac{E^2}{c^2}+B^2\right)= \frac{1}{2}\left(\epsilon_0E^2+\frac{B^2}{\mu_0}\right)
\end{equation*}
Se tiene que el campo eléctrico y magnético puedes obtenerse a partir de las siguientes operaciones:
\begin{equation*}
    E=-\nabla V-\frac{1}{c}\frac{\partial \vec{A}}{\partial t} \qquad \vec{B}=  \left(\vec{\nabla}\times\vec{A}\right)
\end{equation*}
sustituyendo el campo eléctrico y magnético en la lagrangiana se obtiene lo siguiente:
\begin{equation*}
    \mathcal{L}=\frac{1}{2}\left(\epsilon_0 \left(\left(\nabla V\right)^2 +\frac{2}{c}\nabla V \frac{\partial \vec{A}}{\partial t}+\frac{1}{c^2}\left(\frac{\partial \vec{A}}{\partial t}\right)^2\right)+\frac{1}{\mu_0}\left(\nabla \times \vec{A}\right)^2\right)
\end{equation*}
donde se puede obtener las coordenasas generalizadas y momento generalizado es
\begin{equation*}
    q= \left\langle v,\vec{A} \right\rangle \qquad p=\left\langle 0,-\frac{\epsilon_0}{c}\vec{E}\right\rangle
\end{equation*}
calculando el hamiltoniano se tiene que
\begin{align*}
    \mathcal{H}=&\sum p_i\dot{q}_i - \mathcal{L}\\
    =& -\frac{\epsilon_0}{c}\vec{E}\cdot \frac{\partial \vec{A}}{\partial t}+ \frac{B^2}{\mu_0}- \epsilon E^2\\
    =& -\frac{\epsilon_0}{c}\vec{E}\cdot \left[-c \left[\vec{E}+\nabla V\right]\right]+\frac{B^2}{\mu_0}-\epsilon_0 E^2   \\
    =& \epsilon_0 E^2 + \epsilon \vec{E}\cdot \nabla V + \frac{B^2}{2\mu_0}-\frac{\epsilon_0}{2}E^2\\   
    =& \epsilon_0 \vec{E}\cdot \nabla V + \frac{1}{2}\left(\frac{B^2}{\mu_0}+\epsilon_0 \vec{E}^2\right)
\end{align*}
usando la norma de Coulomb, que son las siguientes
\begin{equation*}
    V=0 \qquad \nabla \cdot \vec{A} = 0
\end{equation*}
por lo que el hamiltoniano del sistema es
\begin{equation*}
    \mathcal{H}= \frac{1}{2}\left(\epsilon_0 E^2 + \frac{B^2}{\mu_0}\right)
\end{equation*}
y los campos eléctricos y magnético pasan a tener la siguiente forma
\begin{equation*}
    E=-\frac{1}{c}\frac{\partial \vec{A}}{\partial t} \qquad \vec{B}=  \left(\vec{\nabla}\times\vec{A}\right)
\end{equation*}
de las expresión del campo eléctrico podemos obtener que
\begin{equation*}
    \dot{q}= \frac{\partial \vec{A}}{\partial t}= -cE
\end{equation*}
y de la expresión para p, se puede obtener que
\begin{equation*}
    \frac{c^2}{\epsilon_0} p= -c \vec{E}
\end{equation*}
por lo tanto
\begin{equation*}
    \dot{q}= \frac{c^2}{\epsilon_0} p
\end{equation*}
de otra manera:
\begin{align*}
    p=& -\frac{\epsilon_0}{c} \vec{E} \\
    =& -\frac{\epsilon_0}{c}\left(-\frac{1}{c}\frac{\partial \vec{A}}{\partial t}\right)\\
    =& \frac{\epsilon_0}{c^2}\frac{\partial \vec{A}}{\partial t}
\end{align*}
por lo tanto
\begin{equation*}
    \frac{\partial p}{\partial t} = \frac{\epsilon_0}{c^2} \frac{\partial^2 \vec{A}}{\partial t^2}
\end{equation*}
donde
\begin{equation*}
    \dot{p}= \epsilon_0\ddot{q} = \epsilon_0\nabla^2 A    
\end{equation*}
al igualar estas dos expresiones obtenemos la siguiente ecuación
\begin{equation*}
    \nabla^2 \vec{A} - \frac{1}{c^2} \frac{\partial^2 \vec{A}}{\partial t^2}=0
\end{equation*}
la cual es la ecuación diferencial de una onda, es por ello que se propone que el potencial vector $\vec{A}$, puede escribirse como una serie de Fourier tal que:
\begin{equation*}
    A(\vec{r},t) =\sum_k\left[\vec{a}_{p} e^{i\left(\vec{k}\cdot \vec{r}-\omega_k t \right)}+\vec{a}_{p}^* e^{-i\left(\vec{k}\cdot \vec{r}-\omega_k t \right)}\right]
\end{equation*}
donde $\vec{a}_{p}$ es un operador tal que contiene la infomación de la polarización y depende unicamente del tiempo, $\vec{k}$ es el número de onda asociado al momento, $\omega_k$, es la frecuencia 
que contiene cada onda asociada a la energía de De Broglie. Reescribiendo esta expresión podemos separar la dependencia temporal en la forma
\begin{equation*}
    A(\vec{r},t) =\sum_k \left[ \vec{\epsilon}_{k} e^{-i\left(\vec{k}\cdot \vec{r}\right)}+\vec{\epsilon}_{k}^* e^{i\left(\vec{k}\cdot \vec{r}\right)}\right]
\end{equation*}
donde 
\begin{equation*}
    \vec{\epsilon}_k= \vec{a}_{p}  e^{i\omega_k t} \qquad {\vec{\epsilon}_k}^*= \vec{a}^*_{p}  e^{-i\omega_k t}
\end{equation*}
por lo mismo, las expresiones del campo eléctrico y magnético las podemos escribir de la siguiente manera:
\begin{align*}
    \vec{E}&=\sum_k \left(\vec{E}_k(t)e^{i\vec{k}\cdot \vec{r}}+\vec{E}_k^*(t)e^{-i\vec{k}\cdot \vec{r}}\right)\\
    \vec{B}&=\sum_k \left(\vec{B}_k(t)e^{i\vec{k}\cdot \vec{r}}+\vec{B}_k^*(t)e^{-i\vec{k}\cdot \vec{r}}\right)
\end{align*}
calculando $\partial \vec{A}/\partial t$, se tiene que
\begin{align*}
    \frac{1}{c}\frac{\partial \vec{A}}{\partial t} =&\frac{1}{c}\frac{\partial}{\partial t}\sum_k\left[\vec{\epsilon}_k e^{-i\left(\vec{k}\cdot \vec{r}\right)}+\vec{\epsilon}_{p}^* e^{i\left(\vec{k}\cdot \vec{r}\right)}\right]\\
    =& \frac{1}{c}\sum_k\left[ e^{-i\left(\vec{k}\cdot \vec{r}\right)}\frac{\partial \vec{\epsilon_k}}{\partial t}+ e^{i\left(\vec{k}\cdot \vec{r}\right) }\frac{\partial \vec{\epsilon}_{p}^*}{\partial t}\right]\\
    =& \frac{1}{c}\sum_k\left[ e^{-i\left(\vec{k}\cdot \vec{r}\right)}(i\omega_k)\vec{\epsilon_k}+ e^{i\left(\vec{k}\cdot \vec{r}\right) }(i\omega_k \vec{\epsilon}_k)^*\right]\\
    =&\sum_k\left[ e^{-i\left(\vec{k}\cdot \vec{r}\right)}(i|\vec{k}|)\vec{\epsilon_k}+ e^{i\left(\vec{k}\cdot \vec{r}\right) }(i|\vec{k}| \vec{\epsilon}_k)^*\right]\\
\end{align*}
por lo tanto:
\begin{equation*}
    \sum_k \left(\vec{E}_k(t)e^{i\vec{k}\cdot \vec{r}}+\vec{E}_k^*(t)e^{-i\vec{k}\cdot \vec{r}}\right)=\sum_k\left[ e^{-i\left(\vec{k}\cdot \vec{r}\right)}(ik)\vec{\epsilon_k}+ e^{i\left(\vec{k}\cdot \vec{r}\right) }(ik \vec{\epsilon}_k)^*\right]
\end{equation*}
donde se apreia que:
\begin{equation*}
    \vec{E}_k = i|\vec{k}|\vec{\epsilon}_k
\end{equation*}
calculando $\vec{\nabla}\times \vec{A}$ se tiene que:
\begin{align*}
    \left(\vec{\nabla}\times\vec{A}\right)_k =&\epsilon_{xyz} \partial_y A_z\\
    =& i \epsilon_{xyz} k_y A_z \\
    =& i\vec{k}\times\vec{A}_k
\end{align*}
por lo tanto:
\begin{equation*}
    \vec{B}_k=i\vec{k}\times\vec{A}_k
\end{equation*}
por lo tanto, el campo electromagnético es
\begin{equation*}
    EM=\sum_k i\left(|\vec{k}|\vec{\epsilon}_k+\vec{k}\times\vec{A}_k\right) e^{i\vec{k}\cdot \vec{r}}-i\left(|\vec{k}|\vec{\epsilon}_k^*(t)+\vec{k}\times\vec{A}_k^*\right)e^{-i\vec{k}\cdot \vec{r}}
\end{equation*}