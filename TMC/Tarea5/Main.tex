\documentclass[12pt,letterpaper]{report}
\usepackage{graphicx}
\usepackage{scrextend}
\usepackage{vmargin}
\usepackage{graphicx}
\usepackage{multirow}
\usepackage[utf8]{inputenc}
\usepackage[spanish]{babel}
\usepackage{multicol}
\usepackage{enumerate}
\usepackage{float}
\usepackage{amsmath, amsthm, amssymb, amsfonts}
\usepackage[usenames]{color}
\parindent=0mm
\pagestyle{empty}
\definecolor{miorange}{rgb}{0.91, 0.43, 0.0}
\begin{document}
\setmargins{2.5cm}      
{1.5cm}                     
{2cm}  
{24cm}                    
{10pt}                          
{1cm}                          
{0pt}                             
{2cm}
\begin{titlepage}
\begin{center}
\includegraphics[scale=0.40]{../../Logos/uanl.png} 
\hspace{2.5cm}
\includegraphics[scale=0.40]{../../Logos/fcfm.png}
\end{center}
\vspace{2cm}
\begin{center}
\textbf{
UNIVERSIDAD AUTÓNOMA DE NUEVO LEÓN\\
FACULTAD DE CIENCIAS
    FÍSICO MATEMÁTICAS}\\
\vspace*{2cm}
\begin{large}
\vspace{1cm}
\large{\textbf{Tópicos de Mécanica Cuántica}}\\
\textbf{Tarea 5}\\
Enrique Valbuena Ordonez\\
\end{large}
\vspace{3.5cm}
\begin{minipage}{0.6\linewidth}
\vspace{0.5cm}
\changefontsizes{14pt}
Nombre:\\
Giovanni Gamaliel López Padilla\\
\end{minipage}
\begin{minipage}{0.2\linewidth}
\changefontsizes{14pt}
Matricula:\\
1837522
\end{minipage}
\end{center}
\vspace{4cm}
\begin{flushright}
\today
\end{flushright}
\end{titlepage}
\subsection*{¿Qué es una constante del movimiento?}
Una constante de movimeinto es un objeto o cantidad que se conserva a lo largo de un desplazamiento, estas son mayormente consecuencias de las ecuaciones de movimiento, en lugar de ser una restricción impuesta.
\subsection*{¿Qué es una corriente conservada?}
Al tener nosotros un campo y someterlo a variaciones infinitedecimales como se muestra en la ecuación \ref{eq:delta}
\begin{equation}
    \label{eq:delta}
    \phi \rightarrow \phi+ \delta \phi
\end{equation}
esto provocara una variación infinitedecimal en el lagrangiano del sistema.
\begin{equation*}
    \mathcal{L} \rightarrow \mathcal{L} + \delta \mathcal{L}
\end{equation*}
si la transformación presenta una simetria, entonces la variación que habiamos realizado en el lagrangiano es nula, por ende:
\begin{equation}
    \label{eq:0}
    \delta \mathcal{L} =0
\end{equation}
como el lagrangiano puede ser descrito a partir de una función de campo y sus derivadas, esta puede ser reescriba como lo siguiente:
\begin{equation}
    \label{eq:newlagrangiano}
    \delta \mathcal{L} = \frac{\partial \mathcal{L}}{\partial \phi} \delta \phi + \frac{\partial \mathcal{L}}{\partial(\partial_\mu \phi)} \partial_\mu (\delta \phi )
\end{equation}
con el lagrangiano podemos calcular sus ecuaciones de movimiento asociadas, por lo que el lagrangiano podemos usarlo dentro de la ecuación de Euler-Lagrange, por lo tanto:
\begin{equation}
    \label{eq:euler}
    \frac{\partial \mathcal{L}}{\partial \phi} = \partial_\mu \left(\frac{\partial \mathcal{L}}{\partial(\partial_mu \phi)}\right)
\end{equation}
al momento de sustituir la ecuación \ref{eq:euler} en \ref{eq:newlagrangiano}, se obtiene lo siguiente:
\begin{align*}
    \delta \mathcal{L}  &= \left(\partial_\mu \left(\frac{\partial \mathcal{L}}{\partial(\partial_mu \phi)}\right)\right) \delta \phi + \frac{\partial \mathcal{L}}{\partial(\partial_\mu \phi)} \partial_\mu (\delta \phi )\\
                        &= \partial_\mu \left(\frac{\partial \mathcal{L}}{\partial(\partial_\mu \phi)} \delta \phi \right)
\end{align*}
y por la ecuación \ref{eq:0}, se tiene que :
\begin{equation}
    \label{eq:delta}
    \partial_\mu \left(\frac{\partial \mathcal{L}}{\partial(\partial_\mu \phi)} \delta \phi \right) = 0
\end{equation}
como la derivada es cero, debe existir algún objeto que sea constante, en este caso lo llamaremos $J^{\mu}$, la cual es la corriente conservada, por lo tanto:
\begin{equation}
    J^{\mu}=\frac{\partial \mathcal{L}}{\partial(\partial_\mu \phi)} \delta \phi
\end{equation}
este objeto matemático representa al flujo de una magitud física.
\end{document}