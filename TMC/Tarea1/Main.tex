\documentclass[12pt,letterpaper]{report}
\usepackage{graphicx}
\usepackage{scrextend}
\usepackage{vmargin}
\usepackage{graphicx}
\usepackage{multirow}
\usepackage[utf8]{inputenc}
\usepackage[spanish]{babel}
\usepackage{multicol}
\usepackage{enumerate}
\usepackage{float}
\usepackage{amsmath, amsthm, amssymb, amsfonts}
\usepackage[usenames]{color}
\parindent=0mm
\pagestyle{empty}
\definecolor{miorange}{rgb}{0.91, 0.43, 0.0}
\begin{document}
\setmargins{2.5cm}      
{1.5cm}                     
{2cm}  
{24cm}                    
{10pt}                          
{1cm}                          
{0pt}                             
{2cm}
\begin{titlepage}
\begin{center}
\includegraphics[scale=0.40]{../../Logos/uanl.png} 
\hspace{2.5cm}
\includegraphics[scale=0.40]{../../Logos/fcfm.png}
\end{center}
\vspace{2cm}
\begin{center}
\textbf{
UNIVERSIDAD AUTÓNOMA DE NUEVO LEÓN\\
FACULTAD DE CIENCIAS
    FÍSICO MATEMÁTICAS}\\
\vspace*{2cm}
\begin{large}
\vspace{1cm}
\large{\textbf{Tópicos de Mécanica Cuántica}}\\
\textbf{Tarea 1: Operador P$_{z}$ sobre el Hamiltoniano}\\
Enrique Valbuena Ordonez\\
\end{large}
\vspace{3.5cm}
\begin{minipage}{0.6\linewidth}
\vspace{0.5cm}
\changefontsizes{14pt}
Nombre:\\
Giovanni Gamaliel López Padilla\\
\end{minipage}
\begin{minipage}{0.2\linewidth}
\changefontsizes{14pt}
Matricula:\\
1837522
\end{minipage}
\end{center}
\vspace{4cm}
\begin{flushright}
\today
\end{flushright}
\end{titlepage}
\subsection*{Demostrar que el operador P$_{z}$ es invariante al Hamiltoniano}
Se tiene que el operador P$_{z}$ es el siguiente:
\begin{equation*}
    P_{z}=\frac{\hbar}{i} \frac{\partial}{\partial z}
\end{equation*}
Si se llega a aplicar el operador $P_z$ a un estado de forma que:
\begin{equation*}
    |\psi'(t)\rangle = P_z |\psi(t) \rangle
\end{equation*}
Teniendo esto en la ecuación de Schr\"odinger de la siguiente manera:
\begin{equation*}
    i\hbar \frac{d |\psi'(t)\rangle } {dt} = \hat{H}|\psi'(t)\rangle \Rightarrow i\hbar \frac{d P_z |\psi'(t)\rangle } {dt} = \hat{H}P_z |\psi(t)\rangle=
\end{equation*}
Donde se nota que se tiene que cumpler la conmutación entre el operador Hamiltoniano y el operador $P_{z}$:
\begin{equation*}
    [\hat{H},P_z]=0
\end{equation*}
\begin{equation*}
    [\hat{H},P_z]=\hat{H}P_Z-P_z\hat{H}
\end{equation*}
Calculando $\hat{H}P_Z$:
\begin{align*}
\hat{H}P_z  &=\left(i\hbar \frac{d}{dt} \right) \left(\frac{\hbar}{i}\frac{\partial}{\partial z} \right)\\
            &=\hbar^2 \frac{d}{dt} \frac{\partial}{\partial z}
\end{align*}
Calculando $P_z \hat{H}$:
\begin{align*}
    \hat{H}P_z  &=\left(\frac{\hbar}{i}\frac{\partial}{\partial z} \right) \left(i\hbar \frac{d}{dt} \right)\\
                &=\hbar^2 \frac{\partial}{\partial z} \frac{d}{dt} \frac{\partial}{\partial z}\\
                &=\hbar^2 \frac{d}{dt} \frac{\partial}{\partial z}
    \end{align*}
Por lo que al momento de realizar el comuntador, se obtiene lo siguiente:
\begin{equation*}
    [\hat{H},P_z]=\hbar^2 \frac{d}{dt} \frac{\partial}{\partial z}-\hbar^2 \frac{d}{dt} \frac{\partial}{\partial z}=0
\end{equation*}
Por lo tanto, el operador H es invariante bajo el operador $P_z$
\end{document}